% \iffalse meta-comment
%
% Copyright (C) 2018 Kevin W. Hamlen
%
% This program is free software; you can redistribute it and/or
% modify it under the terms of the GNU General Public License
% as published by the Free Software Foundation; either version 2
% of the License, or (at your option) any later version.
%
% This program is distributed in the hope that it will be useful,
% but WITHOUT ANY WARRANTY; without even the implied warranty of
% MERCHANTABILITY or FITNESS FOR A PARTICULAR PURPOSE.  See the
% GNU General Public License for more details.
%
% You should have received a copy of the GNU General Public License
% along with this program; if not, write to the Free Software
% Foundation, Inc., 51 Franklin Street, 5th Floor, Boston,
% MA 02110-1301, USA.
%
% The latest version of this program can be obtained from
% http://songs.sourceforge.net.
%
% \fi
%
% \iffalse
%<*driver>
\ProvidesFile{songs.dtx}
%</driver>
%<package>\NeedsTeXFormat{LaTeX2e}
%<package>\ProvidesPackage{songs}
%<*package>
  [2018/09/12 v3.1 Songs package]
%</package>
%
%<*driver>
\documentclass{ltxdoc}

% This documentation compiles as part of the Songs self-installer, so it
% needs to work even on tiny LaTeX installations with few packages.  We
% therefore try to be especially robust to missing package failures.
\newcommand\trypackage[4][]{\IfFileExists{#2.sty}{\usepackage[#1]{#2}#3}{#4}}
\trypackage{microtype}{\linepenalty=20 \parfillskip=0pt plus\textwidth\relax}{}
\trypackage{lmodern}{\usepackage[T1]{fontenc}}{}
\trypackage{ifpdf}{}{\expandafter\newif\csname ifpdf\endcsname\pdffalse}
\trypackage{color}{}{}
\ifpdf
  \trypackage[bookmarks,linkbordercolor={.6 0 0}]{hyperref}{}{}
\fi
\usepackage[nopdfindex]{songs}

% Provide back-up hyperlinking and color macros that do nothing,
% in case the hyperref and/or color packages are absent.
\providecommand\href[2]{#2}
\providecommand\url[1]{#1}
\providecommand\pdfbookmark[3][]{}
\providecommand\hyperdef[3]{}
\providecommand\hyperlink[2]{#2}
\providecommand\texorpdfstring[2]{#1}
\providecommand\definecolor[3]{}
\providecommand\color[1]{}
\providecommand\textcolor[2]{#2}
\providecommand\microtypesetup[1]{}

% Configure the document:
\let\oldSE\StopEventually
\EnableCrossrefs
\CodelineIndex
\RecordChanges
%\OnlyDescription

% Create the \Songs logo, if the musix13 font is available:
\newcommand\Songs{\texorpdfstring{{\sffamily songs}}{songs}}
\newcount\imode
\imode=\interactionmode
\batchmode
\newfont\musicfont{musix13}
\interactionmode=\imode
\ifx\musicfont\nullfont\else
  \newdimen\msize
  \newcommand\wholenote{%
    \msize4ex
    \expandafter\ifx\csname musicfont\the\msize\endcsname\relax
      \expandafter\newfont\csname musicfont\the\msize\endcsname
        {musix13 at \the\msize}%
    \fi
    \kern.13ex
    \raise.5ex\hbox{{\csname musicfont\the\msize\endcsname\symbol9}}%
    \kern1.53ex
  }
  \renewcommand\Songs{\texorpdfstring{{\sffamily s\wholenote ngs}}{songs}}
\fi

% Create the logo for Christopher Rath's Songbook package:
\newcommand{\Rath}{{\sffamily Song$\flat$ook}}

% Colors
\definecolor{myred}{rgb}{0.6,0,0}
\definecolor{myblu}{rgb}{0,0,0.6}
\definecolor{mygra}{gray}{0.33}
\def\PrintDescribeMacro#1{\strut{\MacroFont\color{myred}\string#1}\ }
\let\PrintMacroName\PrintDescribeMacro
\def\PrintDescribeEnv#1{\strut{\MacroFont\color{myred}#1}\ }
\let\PrintEnvName\PrintDescribeEnv
{\makeatletter\gdef\verbatim@font{\normalfont\color{myblu}\ttfamily}}

% For some reason, doc.sty removes the \verbatim@font customization hook
% for \verb|...|.  (Why??)  That's not nice, so we put it back:
{\makeatletter
 \gdef\verb {\relax\ifmmode\hbox\else\leavevmode\null\fi
   \bgroup \let\do\do@noligs \verbatim@nolig@list
   \verbatim@font \verb@eol@error \let\do\@makeother \dospecials
   \@ifstar{\@sverb}{\@vobeyspaces \frenchspacing \@sverb}}
}

% Define some environments to simulate the interior of a verse,
% for showing samples in the documentation:
{\makeatletter
 \gdef\likeverse{%
   \SB@insongtrue\SB@inversetrue
   \SB@loadactives
   \global\SB@ctail\SB@cr@
 }
 \gdef\chordheight{\SB@setbaselineskip}
}

% Typeset a block of LaTeX code:
\newcommand\pfs{\parfillskip0pt plus1fil\relax}
\newenvironment{codeblock}{%
  \medskip
  \begingroup
    \ifdim\parindent=0pt \parindent=20pt\fi
    \indent\vbox\bgroup
      \hsize\linewidth
      \advance\hsize-\parindent
      \rightskip=0pt plus1fil\pfs
      \parindent=0pt
      \color{myblu}%
      \obeylines
}{%
  \egroup\endgroup
  \medskip
}

% Typeset a sample song or scripture quotation:
\songcolumns{0}
\newenvironment{sample}{%
  \medskip
  \noindent\hfil
  \vbox\bgroup
    \hsize.5\hsize
    \advance\hsize-.5\columnsep
    \versesep=5pt
    \pfs
}{%
  \egroup{\pfs\par}\medskip
}

% Typeset a sample lyric book fragment:
\newenvironment{lyrics}{%
  \medskip
  \noindent\hfil
  \vbox\bgroup\begingroup
    \hsize=.7\textwidth
    \leftskip=20pt\rightskip=0pt plus1fil\pfs
    \parindent=-20pt
    \likeverse\obeylines
}{%
  \par\endgroup\egroup
  \hfil{\pfs\par}
  \medskip
}

% Typeset a sample chord book fragment:
\newenvironment{chorded}{%
  \medskip
  \noindent\hfil
  \vbox\bgroup\begingroup
    \hsize=.7\textwidth
    \leftskip=20pt\rightskip=0pt plus1fil\pfs
    \parindent=-20pt
    \likeverse\obeylines
    \versesep=5pt\chordheight
}{%
  \par\endgroup\egroup
  \hfil{\pfs\par}
  \medskip
}

% Typeset a "<code> produces <text>" example:
\newlength\prodlen
\setlength{\prodlen}{2.7in}
\newbox\prodbox
\newcommand{\example}{%
  \medskip\setbox\prodbox\hbox\bgroup\begingroup\color{myblu}%
}
\newcommand{\produces}{%
  \endgroup\egroup
  \indent
  \vbox{\hbox to\prodlen{\unhbox\prodbox\hfil}}
  \ {\it produces}\quad
  \afterassignment\prodprefix
  \setbox\prodbox\hbox
}
\newcommand{\prodprefix}{%
  \likeverse\chordheight
  \aftergroup\prodsuffix
}
\newcommand{\prodsuffix}{\unhbox\prodbox{\pfs\par}\medskip}
\newbox\vcbox
\newdimen\vcadjust
\newcommand{\vcenterbox}[1]{%
  \setbox\vcbox\vbox{\hbox{#1}}%
  \vcadjust=.5\ht\vcbox
  \advance\vcadjust by-6pt
  \lower\vcadjust\box\vcbox
}

% Recode \DescribeMacro and \DescribeEnv to make nice pdfbookmark entries.
% Also create some \MainImpl macros that make pdfbookmarks to help the reader
% find the "real" implementations of important macros.
\newcount\seclevel
\newcommand\mybookmark[2]{%
  \ifnum\value{subsection}=0 \seclevel=2
  \else\ifnum\value{subsubsection}=0 \seclevel=3
  \else\seclevel=4 \fi\fi
  \setcounter{tocdepth}{4}%
  \pdfbookmark[\the\seclevel]{#1}{#2}%
  \setcounter{tocdepth}{3}%
}
{\makeatletter
 \xdef\bschar{\@backslashchar}
 \global\let\For\@for}
\newcommand\DescMacro[1]{%
  \ifhmode\unskip\fi
  \mybookmark{\bschar\bschar#1}{macdef-#1}%
  \expandafter\DescribeMacro\expandafter{\csname#1\endcsname}%
  \hyperdef{macro}{#1}{}\unpenalty
  \ignorespaces
}
\newcommand\DescMacroGroup[3]{%
  \ifhmode\unskip\fi
  \mybookmark{\bschar\bschar#2}{macdef-#1}%
  \expandafter\DescribeMacro\expandafter{\csname#2\endcsname}%
  \For\temp:=#3\do{\hyperdef{macro}{\temp}{}}\unpenalty
  \ignorespaces
}
\newcommand\MainImpl[1]{%
  \pdfbookmark[3]{\bschar\bschar#1}{mimpl-#1}%
}
\newcommand{\DescEnv}[1]{%
  \ifhmode\unskip\fi
  \mybookmark{#1}{envdef-#1}%
  \DescribeEnv{#1}%
  \hyperdef{env}{#1}{}\unpenalty
  \ignorespaces
}
\newcommand{\MainEnvImpl}[1]{%
  \mybookmark{#1}{eimpl-#1}%
}

% Create macros to hyperlink macro and environment names to their
% documentation points.
\newcommand{\mac}[1]{{\tt\hyperlink{macro.#1}{\textcolor{myblu}{\char92 #1}}}}
\newcommand{\env}[1]{{\tt\hyperlink{env.#1}{\textcolor{myblu}{#1}}}}

% Defining bookmarks for definitions of active characters is a little trickier
% because many of these characters have special meanings either to TeX or to
% PDF.  The only reliable way is to insert an "\ooo" escape sequence into the
% bookmark text, where ooo is the ascii number of the character expressed in
% octal.  To achieve this, we use `\string<symbol> to obtain the decimal
% ascii number d of the symbol, then do some math (implemented in \octal)
% to compute and tokenize each octal digit of d into the bookmark text.
\newcount\cnta
\newcount\cntb
\newcommand\ooo{}
\newcommand\octal{%
  \cntb\cnta
  \divide\cnta8
  \multiply\cnta-8
  \advance\cntb\cnta
  \edef\ooo{\the\cntb\ooo}%
  \divide\cnta-8
}
\newcommand{\DescChar}[2]{%
  \ifhmode\unskip\fi
  \expandafter\let\csname string#1\expandafter\endcsname
    \expandafter=\string#2%
  \cnta\expandafter`\string#2%
  \def\ooo{}\octal\octal\octal
  \mybookmark{\bschar\ooo}{#1def}%
  {\def\SpecialUsageIndex##1{}%
   \expandafter\DescribeMacro\string#2}%
  \hyperdef{env}{#1}{}\unpenalty
  \ignorespaces
}
\newcommand{\refchar}[1]{{\tt\hyperlink{env.#1}{\csname string#1\endcsname}}}

% At the end of the implementation section we'll have a code line index of
% macro definitions and usages. To make it look a bit less ragged than the
% default index style and to conserve some space, we'll customize a few of
% the parameters:

\IndexPrologue{%
  \subsection{Codeline Index}%
  Underlined numbers refer to the code line where the corresponding entry
  is defined; other numbers refer to the code lines where the entry
  is used.}

{\makeatletter

 \gdef\IndexParms{%
   \sfcode`,=1750
   \parindent0pt
   \columnsep15pt
   \parskip0pt plus1pt
   \rightskip0pt
   \mathsurround0pt
   \parfillskip0pt
   \small
   \microtypesetup{protrusion=false}%
   \def\@idxitem{\par\hangindent15pt}%
   \def\subitem{\@idxitem\hspace*{15pt}}%
   \def\subsubitem{\@idxitem\hspace*{25pt}}%
   \def\indexspace{\par\vspace{10pt plus 2pt minus 3pt}}}

 \gdef\SpecialMainOptIndex#1{%
   \@bsphack
   \special@index{%
     #1\actualchar{\string\ttfamily\space#1} (option)\encapchar main}%
   \special@index{%
     options:\levelchar
     #1\actualchar{\string\ttfamily\space#1}\encapchar main}%
   \@esphack}
}

% Create a conditional that typesets its first argument if we're including
% the implementation section, otherwise typesets its second argument.
\newcommand\ImplOrDesc[2]{%
  \ifx\StopEventually\oldSE#1\else#2\fi
}

% Hyphenating the word "choruses" looks weird. No "ruses" please!
\hyphenation{choruses white-space}

% An environment for describing the implementation of a package option:
\let\oldsmei\SpecialMainEnvIndex
\newenvironment{option}[1]{%
  \let\SpecialMainEnvIndex\SpecialMainOptIndex
  \begin{environment}{#1}%
    \let\SpecialMainEnvIndex\oldsmei
}{%
  \end{environment}%
  \let\SpecialMainEnvIndex\oldsmei
}

% Describe the default setting for an option:
\newcommand\optdef[1]{\noindent{\it(Default: #1)}\hspace{.5cm}}

% Typeset a chord name:
\newcommand{\chord}[1]{{\sffamily\slshape#1}}

% Here are a few macros to produce nice syntax parameters:
\newcommand\Meta[1]{{\color{mygra}\meta{#1}}}
\newcommand\argp[1]{\Meta{arg#1}}
\newcommand\Metarm[1]{\textrm{\Meta{#1}}}
\newcommand\OR{\,$\mid$\,}
\newcommand\SPC{\char`\ }

% Now let's quell those annoying "Marginpar has moved" warning messages.
{\makeatletter
 \global\let\oldamp=\@addmarginpar
 \global\let\oldlwnl=\@latex@warning@no@line
 \gdef\@addmarginpar{%
   \let\@latex@warning@no@line\@gobble
   \oldamp
   \let\@latex@warning@no@line\oldlwnl
 }
}

% The \eat macro just gobbles its argument. I use it to appease my syntax
% highlighter when it gets confused.
\newcommand{\eat}[1]{}

\begin{document}
  \DocInput{songs.dtx}
\end{document}
%</driver>
% \fi
%
% \CheckSum{8935}
%
% \CharacterTable
%  {Upper-case    \A\B\C\D\E\F\G\H\I\J\K\L\M\N\O\P\Q\R\S\T\U\V\W\X\Y\Z
%   Lower-case    \a\b\c\d\e\f\g\h\i\j\k\l\m\n\o\p\q\r\s\t\u\v\w\x\y\z
%   Digits        \0\1\2\3\4\5\6\7\8\9
%   Exclamation   \!     Double quote  \"     Hash (number) \#
%   Dollar        \$     Percent       \%     Ampersand     \&
%   Acute accent  \'     Left paren    \(     Right paren   \)
%   Asterisk      \*     Plus          \+     Comma         \,
%   Minus         \-     Point         \.     Solidus       \/
%   Colon         \:     Semicolon     \;     Less than     \<
%   Equals        \=     Greater than  \>     Question mark \?
%   Commercial at \@     Left bracket  \[     Backslash     \\
%   Right bracket \]     Circumflex    \^     Underscore    \_
%   Grave accent  \`     Left brace    \{     Vertical bar  \|
%   Right brace   \}     Tilde         \~}
%
% \changes{v1.0}{2001/12/01}{Initial version}
% \changes{v1.1}{2005/04/03}{Change log introduced and first release of this documentation}
% \changes{v1.17}{2005/09/24}{Transformed the source from a class to a package}
% \changes{v1.18}{2005/09/29}{Verse numbering added}
% \changes{v2.0}{2007/06/20}{Keyval syntax and chord-replay system added}
%
% \iffalse
% Here we list all the macros that should not be indexed because they are:
% (a) too common and therefore the index would be too large if we listed them,
% (b) not useful in an index because they are predefined TeX macros, or
% (c) not really macros but rather control sequence names given to \string.
% \fi
% \DoNotIndex{\@M,\@depth,\@empty,\@firstofone,\@firstoftwo,\@gobble,\@gobbletwo,\@height,\@m,\@minus,\@ne,\@plus,\@secondoftwo,\@width,\m@ne,\p@,\thr@@,\tw@,\voidb@x,\@xpt,\z@,\z@skip}
% \DoNotIndex{\advance,\char,\count,\divide,\font,\fontdimen,\maxdimen,\multiply,\setcounter,\setlength,\settoheight,\settowidth,\stepcounter}
% \DoNotIndex{\begin,\begingroup,\bgroup,\egroup,\end,\endgroup}
% \DoNotIndex{\box,\copy,\dp,\hbox,\ht,\leavevmode,\lower,\null,\prevdepth,\raise,\rlap,\setbox,\unhbox,\unhcopy,\unpenalty,\unskip,\unvbox,\unvcopy,\vbox,\vtop,\wd}
% \DoNotIndex{\csname,\def,\edef,\endcsname,\futurelet,\gdef,\global,\let,\long,\mathchardef,\newcommand,\renewcommand,\renewenvironment,\xdef}
% \DoNotIndex{\@ifundefined,\@for,\do,\else,\fi,\ifcase,\ifcat,\ifdim,\iffalse,\ifhmode,\ifmmode,\ifinner,\ifnum,\ifodd,\ifvbox,\ifvmode,\ifvoid,\ifx,\loop,\or,\repeat,\undefined}
% \DoNotIndex{\afterassignment,\aftergroup,\expandafter,\ignorespaces,\immediate,\noexpand,\protect,\relax,\space,\string,\the}
% \DoNotIndex{\hfil,\hfilneg,\hskip,\hss,\indent,\kern,\nobreak,\noindent,\nointerlineskip,\offinterlineskip,\par,\penalty,\strut,\thinspace,\vadjust,\vfil,\vfilneg,\vphantom,\vskip}
% \DoNotIndex{\@octets,\four,\three,\two,\UTFviii@,\UTFviii@zero@octets,\0,\1,\2,\3,\4,\5,\6,\7,\8,\9,\X,\O}
%
% \GetFileInfo{songs.dtx}
%
% \title{The \Songs{} package\thanks{This manual documents
%    \textsf{songs}~\fileversion, dated~\filedate,
%    \copyright~2018 Kevin W.~Hamlen, and
%    distributed under version~2 the GNU General Public License
%    as published by the Free Software Foundation.}}
% \author{Kevin W. Hamlen}
%
% \maketitle
%
% \begin{abstract}
% The \Songs{} package produces songbooks that contain lyrics and chords
% (but not full sheet music).
% It allows lyric books, chord books, overhead slides, and digital projector
% slides to all be maintained and generated from a single \LaTeX{} source
% document.
% Automatic transposition, guitar tablature diagrams, handouts, and
% a variety of specialized song indexes are supported.
% \end{abstract}
%
% \section{Introduction}
%
% The \Songs{} \LaTeX{} package produces books of songs that contain lyrics and
% (optionally) chords.
% A single source document yields a lyric book for singers, a chord book for
% musicians, and overhead or digital projector slides for corporate singing.
%
% The software is especially well suited for churches and religious
% fellowships desiring to create their own books of worship songs.
% Rather than purchasing a fixed hymnal of songs, the \Songs{} package allows
% worship coordinators to maintain a constantly evolving repertoire of music
% to which they can add and remove songs over time.
% As the book content changes, the indexes, spacing, and other formatting
% details automatically adjust to stay consistent.
% Songs can also be quickly selected and arranged for specific events or
% services through the use of scripture indexes, automatic transposition,
% and handout and slide set creation features.
%
% \section{Terms of Use}
%
% \noindent
% The \Songs{} package is free software; you can redistribute it and/or
% modify it under the terms of the GNU General Public License
% as published by the Free Software Foundation; either version~2
% of the License, or (at your option) any later version.
% A copy of the license can be found in \S\ref{sec:license}.
%
% \medskip
%
% \noindent
% This program is distributed in the hope that it will be useful,
% but {\sc without any warranty}; without even the implied warranty of
% {\sc merchantability} or {\sc fitness for a particular purpose}. See the
% GNU General Public License in \S\ref{sec:license} for details.
% A copy of the license can also be obtained by writing to the
% Free Software Foundation, Inc., 51 Franklin Street, 5th Floor,
% Boston, MA 02110-1301, USA.
%
% \medskip
%
% \noindent
% This software is copyright \copyright~2018 Kevin W.~Hamlen.
% For contact information or the latest version, see the project webpage at:
%
% \vskip1.5ex
% \begingroup\centering\noindent
% \href{http://songs.sourceforge.net}{{\tt http://songs.sourceforge.net}}\par
% \endgroup
%
% \section{Sample Document}
%
% For those who would like to start making song books quickly, the
% following is a sample document that yields a simple song book with
% one song.
% Starting from this template, you can begin to add songs and customizations
% to create a larger book.
% Instructions for compiling this sample song book follow the listing.
%
% \begingroup\color{myblu}
% \begin{verbatim}
% \documentclass{article}
% \usepackage[chorded]{songs}
%
% \noversenumbers
%
% \begin{document}
%
% \songsection{Worship Songs}
%
% \begin{songs}{}
% \beginsong{Doxology}[by={Louis Bourgeois and Thomas Ken},
%                      sr={Revelation 5:13},
%                      cr={Public domain.}]
% \beginverse
% \[G]Praise God, \[D]from \[Em]Whom \[Bm]all \[Em]bless\[D]ings \[G]flow;
% \[G]Praise Him, all \[D]crea\[Em]tures \[C]here \[G]be\[D]low;
% \[Em]Praise \[D]Him \[G]a\[D]bove, \[G]ye \[C]heav'n\[D]ly \[Em]host;
% \[G]Praise Fa\[Em]ther, \[D]Son, \[Am]and \[G/B G/C]Ho\[D]ly \[G]Ghost.
% \[C]A\[G]men.
% \endverse
% \endsong
% \end{songs}
%
% \end{document}
% \end{verbatim}
% \endgroup\nointerlineskip\vskip-6pt plus0pt minus0pt
%
% To compile this book, run \LaTeX{} (|pdflatex| is recommended):
%
% \begin{codeblock}
% |pdflatex mybook.tex|
% \end{codeblock}
%
% \noindent
% (where |mybook.tex| is the name of the source document above).
% The final document is named |mybook.pdf| if you use |pdflatex| or
% |mybook.dvi| if you use regular |latex|.
%
% Note that compiling a document that includes indexes requires extra steps.
% See \S\ref{sec:compiling} for details.
%
% \begin{figure}
% \noindent\vbox{\begingroup\hsize=352pt
%   \versesep=12pt\columnsep=7pt\parindent=20pt
%   \def\colbotglue{0pt}\def\lastcolglue{0pt}
%   \baselineadj=-1pt\relax
%   \noversenumbers
%   \setlength\textwidth{352pt}
%   \setlength\textheight{498pt}
%   \songcolumns{2}
%   \vskip-3.5ex plus-1ex minus-.2ex
%   \nointerlineskip\null\nointerlineskip
%   \songsection*{Worship Songs}
%   \begin{songs}{}
%   \beginsong{Doxology}[
%      by={Louis Bourgeois and Thomas Ken},
%      sr={Revelation 5:13},
%      cr={Public domain.}]
%   \beginverse
%   \[G]Praise God, \[D]from \[Em]Whom \[Bm]all \[Em]bless\[D]ings \[G]flow;
%   \[G]Praise Him, all \[D]crea\[Em]tures \[C]here \[G]be\[D]low;
%   \[Em]Praise \[D]Him \[G]a\[D]bove, \[G]ye \[C]heav'n\[D]ly \[Em]host;
%   \[G]Praise Fa\[Em]ther, \[D]Son, \[Am]and \[G/B G/C]Ho\[D]ly \[G]Ghost.
%   \[C]A\[G]men.
%   \endverse
%   \endsong\eat\]
%   \parindent=15pt
%   \beginscripture{Psalm 18:2-6}
%   \Acolon The LORD is my rock and my fortress and my deliverer,
%   \Bcolon my God, my rock, in whom I take refuge,
%   \Bcolon my shield, and the horn of my salvation, my stronghold.
%   \Acolon I call upon the LORD, who is worthy to be praised,
%   \Bcolon and I am saved from my enemies.
%   \strophe
%   \Acolon The cords of death encompassed me;
%   \Bcolon the torrents of destruction assailed me;
%   \Acolon the cords of Sheol entangled me;
%   \Bcolon the snares of death confronted me.
%   \strophe
%   \Acolon In my distress I called upon the LORD;
%   \Bcolon to my God I cried for help.
%   \Acolon From his temple he heard my voice,
%   \Bcolon and my cry to him reached his ears.
%   \endscripture
%   \parindent=20pt
%   \beginsong{A Mighty Fortress Is Our God}[
%     by={Martin Luther},
%     cr={Public Domain.}]
%   \beginverse
%   A \[A]mighty \[C#m]Fortress \[B7]is our \[E]God,
%   A \[D]bulwark \[A]never \[E7]fail\[A]ing.
%   Our helper \[C#m]He, a\[B7]mid the \[E]flood
%   Of \[D]mortal \[A]ills pre\[E7]vail\[A]ing.
%   For still our \[B7sus4]an\[B7]cient \[E]foe
%   Doth \[A]seek to \[E/G#]work us \[F#m]woe;
%   His craft and \[B7]pow'r are \[E]great,
%   And, \[Bm]armed with cruel \[C#]hate,
%   On \[D]earth is \[A]not his \[E7]e\[A]qual.
%   \endverse
%   \beginverse
%   Did ^we in ^our own ^strength con^fide,
%   Our ^striving ^would be ^los^ing.
%   Were not the ^right Man ^on our ^side,
%   The ^Man of ^God's own ^choos^ing.
%   Dost ask who ^that ^may ^be?
%   Christ ^Jesus, ^it is ^He;
%   Lord Saba^oth His ^Name,
%   From ^age to age the ^same;
%   And ^He must ^win the ^bat^tle.
%   \endverse
%   \endsong\eat\]
%   \end{songs}
% \endgroup}\par
% \caption{Sample page from a chord book}\label{fig:sample}
% \end{figure}
% A copy of the first page of a sample song section is shown in
% Figure~\ref{fig:sample}.
% The page shown in that figure is from a chorded version of the book.
% When generating a lyric version, the chords are omitted.
% See \S\ref{sec:options} for information on how to generate different
% versions of the same book.
%
% \section{Initialization and Options}\label{sec:options}
%
% Each \LaTeX{} document that uses the \Songs{} package should contain a
% line like the following near the top of the document:
%
% \begin{codeblock}
% |\usepackage[|\Meta{options}|]{songs}|
% \end{codeblock}
%
% \noindent
% Supported \Meta{options} include the following:
%
% \paragraph{Output Type.}
% \DescEnv{lyric}
% \DescEnv{chorded}
% \DescEnv{slides}
% \DescEnv{rawtext}
% The \Songs{} package can produce four kinds of books: lyric books, chord
% books, books of overhead slides, and raw text output.
% You can specify which kind of book is to be produced by specifying one of
% |lyric|, |chorded|, |slides|, or |rawtext| as an option.
% The |slides| and |chorded| options can be used together to create chorded
% slides.
% If no output options are specified, |chorded| is the default.
%
% Lyric books omit all chords, whereas chord books include chords and
% additional information for musicians (specified using \mac{musicnote}).
% Books of overhead slides typeset one song per page in a large font, centered.
%
% Raw text output yields an ascii text file named \Meta{jobname}|.txt|
% (where \Meta{jobname} is the root filename) containing lyrics without chords.
% This can be useful for importing song books into another program, such as a
% spell-checker.
%
% \DescMacro{chordson}
% \DescMacro{chordsoff}
% Chords can be turned on or off in the middle of the document
% by using the |\chordson| or |\chordsoff| macros.
%
% \DescMacro{slides}
% Slides mode can be activated in the middle of the document by using the
% |\slides| macro.
% For best results, this should typically only be done in the document
% preamble or at the beginning of a fresh page.
%
% \paragraph{Measure Bars.}
% \DescEnv{nomeasures}
% \DescEnv{showmeasures}
% \DescMacro{measureson}
% \DescMacro{measuresoff}
% The \Songs{} package includes a facility for placing measure bars in chord
% books (see \S\ref{sec:measures}).
% To omit these measure bars, use the |nomeasures| option;
% to display them, use the |showmeasures| option (the default).
% Measure bars can also be turned on or off in the middle of the document by
% using the |\measureson| or |\measuresoff| macros.
%
% \paragraph{Transposition.}
% \DescEnv{transposecapos}
% The |transposecapos| option changes the effect of the \mac{capo} macro.
% Normally, using |\capo{|\Meta{n}|}| within a song environment produces a
% textual note in chord books that suggests the use of a guitar capo on fret
% \Meta{n}.
% However, when the |transposecapos| option is active, these textual notes
% are omitted and instead the effect of |\capo{|\Meta{n}|}| is the
% same as for \mac{transpose}|{|\Meta{n}|}|.
% That is, chords between the \mac{capo} macro and the end of the song are
% automatically transposed up by \Meta{n} half-steps.
% This can be useful for adapting a chord book for guitarists to one that can
% be used by pianists, who don't have the luxury of capos.
% See \S\ref{sec:notes} and \S\ref{sec:transpose} for more information on the
% \mac{capo} and \mac{transpose} macros.
% 
% \paragraph{Indexes.}
% \DescEnv{noindexes}
% \DescMacro{indexeson}
% \DescMacro{indexesoff}
% The |noindexes| option suppresses the typesetting of any in-document indexes.
% Display of indexes can also be turned on or off using the |\indexeson| and
% |\indexesoff| macros.
%
% \DescEnv{nopdfindex}
% PDF bookmark entries and hyperlinks can be suppressed with the |nopdfindex|
% option.
% For finer control of PDF indexes, see \S\ref{sec:idxcust}.
%
% \paragraph{Scripture Quotations.}
% \DescEnv{noscripture}
% \DescMacro{scriptureon}
% \DescMacro{scriptureoff}
% The |noscripture| option omits scripture quotations (see
% \S\ref{sec:scripture}) from the output.
% You can also turn scripture quotations on or off in the middle of the
% document by using |\scriptureon| or |\scriptureoff|, respectively.
%
% \paragraph{Shaded Boxes.}
% \DescEnv{noshading}
% The |noshading| option causes all shaded boxes, such as those that surround
% song numbers and textual notes, to be omitted.
% You might want to use this option if printing such shaded boxes causes
% problems for your printer or uses too much ink.
%
% \paragraph{Partial Song Sets.}
% \DescMacro{includeonlysongs}
% Often it is useful to be able to extract a subset of songs from the master
% document---e.g.~to create a handout or set of overhead slides for a specific
% worship service.
% To do this, you can type |\includeonlysongs{|\Meta{songlist}|}| in the
% document preamble (i.e., before the |\begin{document}| line), where
% \Meta{songlist} is a comma-separated list of the song numbers to include.
% For example, 
%
% \begin{codeblock}
% |\includeonlysongs{37,50,2}|
% \end{codeblock}
%
% \noindent
% creates a document consisting only of songs 37, 50, and 2, in that order.
%
% Partial books generated with |\includeonlysongs| omit all scripture
% quotations (\S\ref{sec:scripture}), and ignore uses of
% \mac{nextcol}, \mac{brk}, \mac{sclearpage}, and \mac{scleardpage}
% between songs unless they are followed by a star (e.g., \mac{nextcol}|*|).
% To force a column- or page-break at a specific point in a partial book,
% add the word |nextcol|, |brk|, |sclearpage|, or |scleardpage| at the
% corresponding point in the \Meta{songlist}.
%
% The |\includeonlysongs| macro only reorders songs within each
% \env{songs} environment (see \S\ref{sec:songs}), not between different
% \env{songs} environments.
% It also cannot be used in conjunction with the \env{rawtext} option.
%
% \section{Songs}\label{sec:songs}
%
% \subsection{Beginning a Song}
%
% \paragraph{Song Sets.}
% \DescEnv{songs}
% Songs are contained within |songs| environments.
% Each |songs| environment begins and ends with:
%
% \begin{codeblock}
% |\begin{songs}{|\Meta{indexes}|}|
% $\vdots$
% |\end{songs}|
% \end{codeblock}
%
% \noindent
% \Meta{indexes} is a comma-separated list of index \Meta{id}'s
% (see \S\ref{sec:indexes})---one identifier for each index that is to
% include songs in this song set.
% Between the |\begin{songs}| and |\end{songs}| lines of
% a song section only songs (see below) or inter-song environments
% (see \S\ref{sec:between}) may appear.
% No text in a |songs| environment may appear outside of these environments.
%
% \paragraph{Songs.}
% \DescMacro{beginsong}
% \DescMacro{endsong}
% A song begins and ends with:
%
% \begin{codeblock}
% |\beginsong{|\Meta{titles}|}[|\Meta{otherinfo}|]|
% $\vdots$
% |\endsong|
% \end{codeblock}
%
% \noindent
% Songs should appear only within \env{songs} environments (see above)
% unless you are supplying your own page-builder (see \S\ref{sec:layout}).
%
% In the \mac{beginsong} line, \Meta{titles} is one or more song titles
% separated by |\\|.
% If multiple titles are provided, the first is typeset normally
% atop the song and the rest are each typeset in parentheses on
% separate lines.
%
% The |[|\Meta{otherinfo}|]| part is an optional comma-separated list of
% key-value pairs (keyvals) of the form \Meta{key}|=|\Meta{value}.
% The possible keys and their values are:
%
% \medskip
% \noindent\hfil\vbox{\halign{#\hfil&\kern2em{\it#}\hfil\cr
%   |by={|\Meta{authors}|}|&authors, composers, and other contributors\cr
%   |cr={|\Meta{copyright}|}|&copyright information\cr
%   |li={|\Meta{license}|}|&licensing information\cr
%   |sr={|\Meta{refs}|}|&related scripture references\cr
%   |index={|\Meta{lyrics}|}|&an extra index entry for a line of lyrics\cr
%   |ititle={|\Meta{title}|}|&an extra index entry for a hidden title\cr}}
% {\pfs\par}\medskip
%
% \noindent
% For example, a song that begins and ends with
%
% \begin{codeblock}
% |\beginsong{Title1 \\ Title2}[by={Joe Smith}, sr={Job 3},|
% |  cr={\copyright~|\unskip\expandafter|\the\year|| XYZ.}, li={Used with permission.}]|
% |\endsong|
% \end{codeblock}
%
% \noindent looks like
%
% \begin{sample}
%  \setcounter{songnum}{1}%
%  \vskip1pt%
%  \beginsong{Title1 \\ Title2}[by={Joe Smith}, sr={Job 3},
%     cr={\copyright~\the\year{} XYZ.}, li={Used with permission.}]
%  \endsong
% \end{sample}
%
% The four keyvals used in the above example are described in detail in the
% remainder of this section;
% the final two are documented in \S\ref{sec:ientry}.
% You can also create your own keyvals (see \S\ref{sec:newkey}).
%
% \paragraph{Song Authors.}
% \DescEnv{by=}
% The |by={|\Meta{authors}|}| keyval lists one or more authors,
% composers, translators, etc.
% An entry is added to each author index associated with the current
% \env{songs} environment for each contributor listed.
% Contributors are expected to be separated by commas, semicolons, or the
% word |and|.
% For example:
%
% \begin{codeblock}
% |by={Fred Smith, John Doe, and Billy Bob}|
% \end{codeblock}
%
% \noindent
% Words separated by a macro-space (\verb*@\ @\eat*) or tie (|~|)
% instead of a regular space are treated as single words by the indexer.
% For example, \verb*@The Vienna Boys' Choir@\eat* is indexed as
% ``Choir, The Vienna Boys'\thinspace'' but
% \verb*@The Vienna\ Boys'\ Choir@\eat* is indexed as
% ``Vienna Boys' Choir, The''.
%
% \paragraph{Copyright Info.}
% \DescEnv{cr=}
% The |cr={|\Meta{copyright}|}| keyval specifies the copyright-holder of the
% song, if any.
% For example:
%
% \begin{codeblock}
% |cr={\copyright~2000 ABC Songs, Inc.}|
% \end{codeblock}
%
% \noindent
% Copyright information is typeset in fine print at the bottom of the song.
%
% \paragraph{Licensing Info.}
% \DescEnv{li=}
% \DescMacro{setlicense}
% Licensing information is provided by |li={|\Meta{license}|}|, where
% \Meta{license} is any text.
% Licensing information is displayed in fine print under the song just
% after the copyright information (if any).
% Alternatively, writing |\setlicense{|\Meta{license}|}| anywhere between
% the \mac{beginsong} and \mac{endsong} lines is equivalent to using
% |li={|\Meta{license}|}| in the \mac{beginsong} line.
%
% When many songs in a book are covered by a common license, it is
% usually convenient to create a macro to abbreviate the licensing
% information.
% For example, if your organization has a music license from Christian
% Copyright Licensing International with license number 1234567, you might
% define a macro like
%
% \begin{codeblock}
% |\newcommand{\CCLI}{(CCLI \#1234567)}|
% \end{codeblock}
%
% \noindent
% Then you could write |li=\CCLI| in the \mac{beginsong} line of each song
% covered by CCLI.
%
% \paragraph{Scripture References.}
% \DescEnv{sr=}
% The \Songs{} package has extensive support for scripture citations and
% indexes of scripture citations.
% To cite scripture references for the song, use the keyval
% |sr={|\Meta{refs}|}|, where \Meta{refs} is a list of scripture
% references.
% Index entries are added to all scripture indexes associated
% with the current \env{songs} environment for each such reference.
% The |songidx| index generation script (see \S\ref{sec:compiling}) expects
% \Meta{refs} to be a list of references in which semicolons are used to
% separate references to different books, and commas are used to separate
% references to to different chapters and verses within the same book.
% For example, one valid scripture citation is
%
% \begin{codeblock}
% |sr={John 3:16,17, 4:1-5; Jude 3}|
% \end{codeblock}
%
% The full formal syntax of a valid \Meta{refs} argument is given in
% Figure~\ref{fig:srsyntax}.
% \begin{figure}
% \noindent\hfil\vbox{\advance\baselineskip2pt
% \halign{\hfil{\tt#}\,$\longrightarrow$\,&{\tt#}\hfil\cr
%   \Metarm{refs}&\Metarm{nothing}\OR\Metarm{ref};\SPC\Metarm{ref};$\ldots$;\SPC\Metarm{ref}\cr
%   \Metarm{ref}&\Metarm{many-chptr-book}\SPC\Metarm{chapters}\OR\Metarm{one-chptr-book}\SPC\Metarm{verses}\cr
%   \Metarm{many-chptr-book}&Genesis\OR Exodus\OR Leviticus\OR Numbers\OR $\ldots$\cr
%   \Metarm{one-chptr-book}&Obadiah\OR Philemon\OR 2 John\OR 3 John\OR Jude\cr
%   \Metarm{chapters}&\Metarm{chref},\SPC\Metarm{chref},$\ldots$,\SPC\Metarm{chref}\cr
%   \Metarm{chref}&\Metarm{chapter}\OR\Metarm{chapter}-\Metarm{chapter}\OR\Metarm{chapter}:\Metarm{verses}\OR\cr
%   \omit&\quad\Metarm{chapter}:\Metarm{verse}-\Metarm{chapter}:\Metarm{verse}\cr
%   \Metarm{verses}&\Metarm{vref},\Metarm{vref},$\ldots$,\Metarm{vref}\cr
%   \Metarm{vref}&\Metarm{verse}\OR\Metarm{verse}-\Metarm{verse}\cr}}
% \caption{Formal syntax rules for song scripture references}\label{fig:srsyntax}
% \end{figure}
% In those syntax rules, \Meta{chapter} and \Meta{verse} stand for arabic
% numbers denoting a valid chapter number for the given book, and a valid
% verse number for the given chapter, respectively.
% Note that when referencing a book that has only one chapter,
% one should list only its verses after the book name
% (rather than |1:|\Meta{verses}).
%
% \subsection{Verses and Choruses}
%
% \paragraph{Starting A Verse Or Chorus.}
% \DescMacro{beginverse}
% \DescMacro{endverse}
% \DescMacro{beginchorus}
% \DescMacro{endchorus}
% Between the \mac{beginsong} and \mac{endsong} lines of a song can
% appear any number of verses and choruses.
% A verse begins and ends with:
%
% \begin{codeblock}
% |\beginverse|
% $\vdots$
% |\endverse|
% \end{codeblock}
%
% \noindent and a chorus begins and ends with:
%
% \begin{codeblock}
% |\beginchorus|
% $\vdots$
% |\endchorus|
% \end{codeblock}
%
% \noindent
% Verses are numbered (unless \mac{noversenumbers} has been used to
% suppress verse numbering) whereas choruses have a vertical line placed to
% their left.
%
% To create an unnumbered verse, begin the verse with |\beginverse*| instead.
% This can be used for things that aren't really verses but should be
% typeset like a verse (e.g.~intros, endings, and the like).
% A verse that starts with |\beginverse*| should still end with |\endverse|
% (not |\endverse*|).
%
% Within a verse or chorus you should enter one line of text for
% each line of lyrics.
% Each line of the source document produces a separate line in the resulting
% document (like \LaTeX's |\obeylines| macro).
% Lines that are too long to fit are wrapped with hanging indentation
% of width |\parindent|.
%
% \subsection{Chords}\label{sec:chords}
%
% \DescMacro{[}
% \DescChar{hash}{#}
% \DescChar{amp}{&}
% Between the \mac{beginverse} and \mac{endverse} lines, or between
% the \mac{beginchorus} and \mac{endchorus} lines,
% chords can be produced using the macro |\[|\Meta{chordname}|]|\eat\].
% Chords only appear in chord books; they are omitted from lyric books.
% The \Meta{chordname} may consist of arbitrary text.
% To produce sharp and flat symbols, use |#| and |&| respectively.
%
% Any text that immediately follows the |\[]|\eat\] macro with no
% intervening whitespace is assumed to be the word or syllable
% that is to be sung as the chord is struck, and is therefore
% typeset directly under the chord.
% For example:
%
% \example|\[E&]peace and \[Am]joy|\produces{\[E&]peace and \[Am]joy}
% \eat\]
%
% \noindent
% If whitespace (a space or \Meta{return}) immediately follows,
% then the chord name be typeset without any lyric text
% below it, indicating that the chord is to be struck between
% any surrounding words.
% For example:
%
% \example|\[E&]peace and \[Am] joy|\produces{\[E&]peace and \[Am] joy}
% \eat\]
%
% If the lyric text that immediately follows the chord contains
% another chord, and if the width of the chord name exceeds the
% width of the lyric text, then hyphenation is added automatically.
% For example:
%
% \example|\[F#sus4]e\[A]ternal|\produces{\[F#sus4]e\[A]ternal}
% \eat\]
%
% Sequences of chords that sit above a single word can be written
% back-to-back with no intervening space, or as a single chord:
%
% \example|\[A]\[B]\[Em]joy|\produces{\[A]\[B]\[Em]joy}
% \example|\[A B Em]joy|\produces{\[A B Em]joy}
% \eat\]
%
% \noindent
% The only difference between the two examples above is that the chords
% in the first example can later be replayed separately (see
% \S\ref{sec:replay}) whereas the chords in the second example can only be
% replayed as a group.
%
% You can explicitly dictate how much of the text following a
% chord macro is to appear under the chord name by using braces.
% To exclude text that would normally be drawn under the chord,
% use a pair of braces that includes the chord macro.
% For example:
%
% \example|{\[G A]e}ternal|\produces{{\[G A]e}ternal}
% \eat\]
%
% \noindent
% (Without the braces, the syllables ``ternal'' would not be
% pushed out away from the chord.)
% This might be used to indicate that the chord transition occurs
% on the first syllable rather than as the second syllable is
% sung.
%
% Contrastingly, braces that do not include the chord itself can
% be used to include text under a chord that would otherwise be
% excluded.
% For example:
%
% \example|\[Gmaj7sus4]{th' eternal}|\produces{\[Gmaj7sus4]{th' eternal}}
% \eat\]
%
% \noindent
% Without the braces, the word ``eternal'' would be pushed out away
% from the chord so that the chord would appear only over
% the partial word ``th'\thinspace''.
%
% \paragraph{Chords Without Lyrics.}
% \DescMacro{nolyrics}
% Sometimes you may want to write a line of chords with no lyrics in it at all,
% such as for an instrumental intro or solo.
% To make the chords in such a line sit on the baseline instead of raised above
% it, use the |\nolyrics| macro.
% For example:
%
% \example|{\nolyrics Intro: \[G] \[A] \[D]}|\produces{\nolyrics Intro: \[G] \[A] \[D]}
% \eat\]
%
% \noindent
% Note the enclosing braces that determine how long the effect should last.
% Multiple lines can be included in the braces.
% Instrumental solos should typically not appear in lyric books, so such
% lines should usually also be surrounded by \mac{ifchorded} and |\fi|
% (see \S\ref{sec:conditionals}).
%
% \paragraph{Symbols Under Chords.}
% \DescMacro{DeclareLyricChar}
% If you are typesetting songs in a language whose alphabet contains symbols
% that \LaTeX{} treats as punctuation, you can use the |\DeclareLyricChar|
% macro to instruct the \Songs{} package to treat the symbol as
% non-chord-ending, so that it is included under chords by default just
% like an alphabetic character.
%
% \begin{codeblock}
% |\DeclareLyricChar{|\Meta{token}|}|
% \end{codeblock}
%
% \noindent
% Here, \Meta{token} must be a single \TeX{} macro control sequence,
% active character, letter (something \TeX{} assigns catcode 11), or
% punctuation symbol (something \TeX{} assigns catcode 12).
% For example, by default,
%
% \example|\[Fmaj7]s\dag range|\produces{\[Fmaj7]s\dag range}
% \eat\]
%
% \noindent
% because |\dag| is not recognized as an alphabetic symbol;
% but if you first type,
%
% \begin{codeblock}
% |\DeclareLyricChar{\dag}|
% \end{codeblock}
%
% \noindent
% then instead you will get:
%
% \DeclareLyricChar{\dag}
% \example|\[Fmaj7]s\dag range|\produces{\[Fmaj7]s\dag range}
% \eat\]
%
% \noindent
% \DescMacro{DeclareNonLyric}
% Likewise, you can type
%
% \begin{codeblock}
% |\DeclareNonLyric{|\Meta{token}|}|
% \end{codeblock}
%
% \noindent
% to reverse the above effect and force a token to be lyric-ending.
% Such tokens are pushed out away from long chord names so that they
% never fall under a chord, and hyphenation is added to the resulting gap.
%
% \DescMacro{DeclareNoHyphen}
% To declare a token to be lyric-ending but without the added hyphenation,
% use |\DeclareNoHyphen{|\Meta{token}|}| instead.
% Such tokens are pushed out away from long chord names so that they never
% fall under the chord, but hyphenation is not added to the resulting gap.
%
% \paragraph{Extending Chords Over Adjacent Words.}
% \DescMacro{MultiwordChords}
% The |\MultiwordChords| macro forces multiple words to be squeezed under one
% chord by default.
% Normally a long chord atop a short lyric pushes subsequent
% lyrics away to make room for the chord:
%
% \example|\[Gmaj7sus4]my life|\produces{\[Gmaj7sus4]my life}
% \eat\]
%
% \noindent
% But if you first type |\MultiwordChords|, then instead you get the more
% compact:
%
% \begingroup\MultiwordChords
% \example|\[Gmaj7sus4]my life|\produces{\[Gmaj7sus4]my life}
% \eat\]
% \endgroup
%
% \noindent
% Authors should exercise caution when using |\MultiwordChords| because
% including many words under a single chord can often produce output that
% is ambiguous or misleading to musicians.
% For example,
%
% \begingroup\MultiwordChords
% \example|\[F G Am]me free|\produces{\[F G Am]me free}\par
% \eat\]
% \endgroup
%
% \noindent
% This might be misleading to musicians if all three chords are intended
% to be played while singing the word ``me.''
% Liberal use of braces is therefore required to make |\MultiwordChords|
% produce good results, which is why it isn't the default.
%
% \paragraph{Accidentals Outside Chords.}
% \DescMacro{shrp}
% \DescMacro{flt}
% Sharp and flat symbols can be produced with |#| and |&| when they appear
% in chord macros, but if you wish to produce those symbols in
% other parts of the document, you must use the |\shrp| and |\flt| macros.
% For example, to define a macro that produces a \chord{C\shrp} chord, use:
%
% \begin{codeblock}
% |\newcommand{\Csharp}{C\shrp}|
% \end{codeblock}
%
% \subsection{Replaying Chords and Choruses}\label{sec:replay}
%
% \DescChar{hat}{^}
% Many songs consist of multiple verses that use the same chords.
% The \Songs{} package simplifies this common case by providing a means to
% replay the chord sequence of a previous verse without having to retype
% all the chords.
% To replay a chord from a previous verse, type a hat symbol (|^|) anywhere
% you would otherwise use a chord macro (|\[]|\eat\]).
% For example,
%
% \begin{codeblock}
% \mac{beginverse}
% |\[G]This is the \[C]first \[G]verse.|\eat\]
% \mac{endverse}
% \mac{beginverse}
% |The ^second verse ^ has the same ^chords.|
% \mac{endverse}
% \end{codeblock}
%
% \noindent produces
%
% \begin{chorded}\memorize
%   \[G]This is the \[C]first \[G]verse.\eat\]
% \end{chorded}
% \begin{chorded}
%   The ^second verse ^ has the same ^chords.
% \end{chorded}
%
% Normal chords can appear amidst replayed chords without disrupting the
% sequence of chords being replayed.
% Thus, a third verse could say,
%
% \begin{codeblock}
% \mac{beginverse}
% |The ^third verse ^has a \[Cm]new ^chord.|\eat\]
% \mac{endverse}
% \end{codeblock}
%
% \noindent to produce
%
% \begin{chorded}
%   The ^third verse ^has a \[Cm]new ^chord.\eat\]
% \end{chorded}
%
% Replaying can be used in combination with automatic transposition to produce
% modulated verses.
% See \S\ref{sec:transpose} for an example.
%
% \DescMacro{memorize}
% By default, chords are replayed from the current song's first verse, but
% you can replay the chords of a different verse or chorus by saying
% |\memorize| at the beginning of any verse or chorus whose chords you want
% to later replay.
% Subsequent verses or choruses that use \refchar{hat} replay chords
% from the most recently memorized verse or chorus.
%
% \paragraph{Selective Memorization.}
% It is also possible to inject unmemorized chords into a memorized verse
% so that they are not later replayed.
% To suppress memorization of a chord, begin the chord's name with a hat
% symbol.
% For example,
%
% \begin{codeblock}
% \mac{beginverse}\mac{memorize}
% |The \[G]third \[C]chord will \[^Cm]not be re\[G]played.|\eat\]
% \mac{endverse}
% \mac{beginverse}
% |When ^replaying, the ^unmemorized chord is ^skipped.|
% \mac{endverse}
% \end{codeblock}
%
% \noindent produces
%
% \begin{chorded}\memorize
%   The \[G]third \[C]chord will \[^Cm]not be re\[G]played.\eat\]
%   \vskip5pt%
%   When ^replaying, the ^unmemorized chord is ^skipped.
% \end{chorded}
%
% \noindent
% This is useful when the first verse of a song has something unique,
% like an intro that won't be repeated in subsequent verses, but has
% other chords that you wish to replay.
%
% \paragraph{Memorizing Multiple Chord Sequences.}
% By default, the \Songs{} package only memorizes one sequence of chords
% at a time and \refchar{hat} replays chords from that most recently
% memorized sequence.
% However, you can memorize and replay multiple independent sequences
% using the macros described in the following paragraphs.
%
% \DescMacro{newchords}
% Memorized or replayed chord sequences are stored in chord-replay registers.
% To declare a new chord-replay register, type
%
% \begin{codeblock}
% |\newchords{|\Meta{regname}|}|
% \end{codeblock}
%
% \noindent
% where \Meta{regname} is any unique alphabetic name.
%
% Once you've declared a register, you can memorize into that register
% by providing the \Meta{regname} as an optional argument to
% \mac{memorize}:
%
% \begin{codeblock}
% \mac{memorize}|[|\Meta{regname}|]|
% \end{codeblock}
%
% \noindent
% Memorizing into a non-empty register replaces the contents of that
% register with the new chord sequence.
%
% \DescMacro{replay}
% To replay chords from a particular register, type
%
% \begin{codeblock}
% |\replay[|\Meta{regname}|]|
% \end{codeblock}
%
% \noindent
% Subsequent uses of \refchar{hat} reproduce chords from the sequence
% stored in register \Meta{regname}.
%
% Register contents are global, so you can memorize a chord sequence from one
% song and replay it in others.
% You can also use |\replay| multiple times in the same verse or chorus to
% replay a sequence more than once.
%
% \paragraph{Replaying Choruses.}
% \DescMacro{repchoruses}
% When making overhead slides, it is often convenient to repeat the song's
% chorus after the first verse on each page, so that the projector-operator
% need not flip back to the first slide each time the chorus is to be sung.
% You can say |\repchoruses| to automate this process.
% This causes the first chorus in each subsequent song to be automatically
% repeated after the first verse on each subsequent page of the song (unless
% that verse is already immediately followed by a chorus).
% If the first chorus is part of a set of two or more consecutive choruses,
% then the whole set of choruses is repeated.
% (A set of choruses is assumed to consist of things like pre-choruses that
% should always be repeated along with the chorus.)
% Choruses are not automatically inserted immediately after unnumbered
% verses (i.e., verses that begin with \mac{beginverse}|*|).
% Unnumbered verses are assumed to be bridges or endings that aren't
% followed by a chorus.
%
% \DescMacro{norepchoruses}
% Writing |\norepchoruses| turns off chorus repetition for subsequent songs.
%
% If you need finer control over where replayed choruses appear, use the
% conditional macros covered in \S\ref{sec:conditionals} instead of
% |\repchoruses|.
% For example, to manually insert a chorus into only slide books at a
% particular point (without affecting other versions of your book),
% you could write:
%
% \begin{codeblock}
% \mac{ifslides}
% \mac{beginchorus}
% $\vdots$
% \mac{endchorus}
% |\fi|
% \end{codeblock}
%
% \noindent
% and copy and paste the desired chorus into the middle.
%
% \subsection{Line and Column Breaks}
%
% \paragraph{Line Breaking.}
% \DescMacro{brk}
% To cause a long line of lyrics to be broken in a particular place, put the
% |\brk| macro at that point in the line.
% This does not affect lines short enough to fit without breaking.
% For example,
%
% \begin{codeblock}
% |\beginverse|
% {\tt\frenchspacing This is a |\brk| short line.
%  But this is a particularly long line of lyrics |\brk| that will need to be wrapped.
% } |\endverse|
% \end{codeblock}
%
% \noindent produces
%
% \begin{lyrics}
%   This is a \brk short line.
%   But this is a particularly long line of lyrics \brk that will need to be wrapped.
% \end{lyrics}
%
% \paragraph{Column Breaks Within Songs.}
% To suggest a column break within a verse or chorus too long to fit in a
% single column, use |\brk| on a line by itself.
% If there are no |\brk| lines in a long verse, it is broken
% somewhere that a line does not wrap.
% (A wrapped line is never divided by a column break.)
% If there are no |\brk| lines in a long chorus, it overflows the column,
% yielding an overfull vbox warning.
%
% \paragraph{Column Breaks Between Songs.}
% \DescMacro{nextcol}
% \DescMacro{sclearpage}
% \DescMacro{scleardpage}
% To force a column break between songs, use |\nextcol|, |\brk|, |\sclearpage|,
% or |\scleardpage| between songs.
% The |\nextcol| macro ends the column by leaving blank space at the bottom.
% The |\brk| macro ends the current column in lyric books by stretching
% the preceeding text so that the column ends flush with the bottom
% of the page.
% (In non-lyric books |\brk| is identical to |\nextcol|.)
% The |\sclearpage| macro is like |\nextcol| except that it shifts to the next
% blank page if the current page is nonempty.
% The |\scleardpage| macro is like |\sclearpage| except that it shifts to the
% next blank even-numbered page in two-sided documents.
% Column breaks usually need to be in different places in different book types.
% To achieve this, use a conditional block from \S\ref{sec:conditionals}.
% For example,
%
% \begin{codeblock}
% \mac{ifchorded}|\else|\mac{ifslides}|\else\brk\fi\fi|
% \end{codeblock}
%
% \noindent
% forces a column break only in lyric books but does not affect chord books
% or books of overhead slides.
%
% When a partial list of songs is being extracted with \mac{includeonlysongs},
% |\brk|, |\nextcol|, |\clearpage|, and |\cleardpage| macros between songs
% must be followed by a star to have any effect.
% To force a column-break at a specific point in a partial book, add the
% word |nextcol|, |brk|, |clearpage|, or |cleardpage| at the corresponding
% point in the argument to \mac{includeonlysongs}.
%
% \subsection{Echoes and Repeats}
%
% \paragraph{Echo Parts.}
% \DescMacro{echo}
% To typeset an echo part, use |\echo{|\Meta{lyrics and chords}|}|.
% Echo parts are parenthesized and italicized.
% For example,
%
% \example|Alle\[G]luia! \echo{Alle\[A]luia!}|\produces{Alle\[G]luia! \echo{Alle\[A]luia!}}
% \eat\]
%
% \paragraph{Repeated Lines.}
% \DescMacro{rep}
% To indicate that a line should be sung multiple times by all singers, put
% |\rep{|\Meta{n}|}| at the end of the line.
% For example,
%
% \example|Alleluia! \rep{4}|\produces{Alleluia! \rep{4}}
%
% \DescMacro{lrep}
% \DescMacro{rrep}
% To indicate exactly where repeated parts begin and end, use |\lrep| and
% |\rrep| to create begin- and end-repeat signs.
% For example,
%
% \example|\lrep \[G]Alleluia!\rrep \rep{4}|\produces{\lrep \[G]Alleluia!\rrep \rep{4}}
% \eat\]
%
% \subsection{Measure Bars}\label{sec:measures}
%
% \DescMacro{measurebar}
% \DescChar{pipe}{|}
% Measure bars can be added to chord books in order to help musicians
% keep time when playing unfamiliar songs.
% To insert a measure bar, type either |\measurebar| or type the
% vertical pipe symbol (``\verb@|@'').
% For example,
%
% \example\verb@Alle|\[G]luia@\produces{Alle\meter{}{}\measurebar\[G]luia}
% \eat\]
%
% \noindent
% In order for measure bars to be displayed, the \env{showmeasures}
% option must be enabled.
% Measure bars are only displayed by default in chord books.
%
% \DescMacro{meter}
% The first measure bar in a song has meter numbers placed above
% it to indicate the time signature of the piece.
% By default, these numbers are 4/4, denoting four quarter notes
% per measure.
% To change the default, type |\meter{|\Meta{n}|}{|\Meta{d}|}|
% somewhere after the \mac{beginsong} line of the song but before the
% first measure bar, to declare a time signature of \Meta{n} \Meta{d}th
% notes per measure.
%
% \DescMacro{mbar}
% You can also change meters mid-song either by using |\meter| in the
% middle of the song or by typing |\mbar{|\Meta{n}|}{|\Meta{d}|}|
% to produce a measure bar with a time signature of \Meta{n}/\Meta{d}.
% For example,
%
% \begin{codeblock}
% |\meter{6}{8}|
% |\beginverse|
% \verb@|Sing to the |heavens, ye \mbar{4}{4}saints of |old!@
% |\endverse|
% \end{codeblock}
%
% \noindent produces
%
% \begin{chorded}
%   \meter{6}{8}%
%   \measurebar Sing to the \measurebar heavens,  ye \mbar{4}{4}saints of \measurebar old!
% \end{chorded}
%
% \subsection{Textual Notes}\label{sec:notes}
%
% \DescMacro{textnote}
% \DescMacro{musicnote}
% Aside from verses and choruses, songs can also contain textual notes
% that provide instructions to singers and musicians.
% To create a textual note that is displayed in both lyric books
% and chord books, use:
%
% \begin{codeblock}
% |\textnote{|\Meta{text}|}|
% \end{codeblock}
%
% \noindent
% To create a textual note that is displayed only in chord books, use:
%
% \begin{codeblock}
% |\musicnote{|\Meta{text}|}|
% \end{codeblock}
%
% \noindent
% Both of these create a shaded box containing \Meta{text}.
% For example,
%
% \begin{codeblock}
% |\textnote{Sing as a two-part round.}|
% \end{codeblock}
%
% \noindent produces
%
% \begin{lyrics}
%   \textnote{Sing as a two-part round.}
% \end{lyrics}
%
% \noindent
% Textual notes can be placed anywhere within a song, either within
% verses and choruses or between them.
%
% \paragraph{Guitar Capos.}
% \DescMacro{capo}
% One special kind of textual note suggests to guitarists a fret on which
% they should put their capos.
% Macro |\capo{|\Meta{n}|}| should be used for this purpose.
% It normally has the same effect as \mac{musicnote}|{capo |\Meta{n}|}|;
% however, if the \env{transposecapos} option is active then it
% instead has the effect of \mac{transpose}|{|\Meta{n}|}|.
% See \S\ref{sec:transpose} for more information on automatic chord
% transposition.
%
% \subsection{Chords in Ligatures}
%
% This subsection covers an advanced topic and can probably be
% skipped by those creating song books for non-professional use.
%
% The \mac{[\eat]} macro is the normal means by which chords should be inserted
% into a song; however, a special case occurs when a chord falls within a
% ligature.
% Ligatures are combinations of letters or symbols that \TeX{} normally
% typesets as a single font character so as to produce cleaner-looking
% output.
% The only ligatures in English are: ff, fi, fl, ffi, and ffl.
% Other languages have additional ligatures like \ae{} and \oe.
% Notice that in each of these cases, the letters are ``squished''
% together to form a single composite symbol.
%
% \DescMacro{ch}
% When a chord macro falls inside a ligature, \LaTeX{} fails to compact
% the ligature into a single font character even in non-chorded versions of
% the book.
% To avoid this minor typographical error, use the |\ch| macro to typeset
% the chord:
%
% \begin{codeblock}
% |\ch{|\Meta{chord}|}{|\Meta{pre}|}{|\Meta{post}|}{|\Meta{full}|}|
% \end{codeblock}
%
% \noindent
% where \Meta{chord} is the chord text, \Meta{pre} is the text to
% appear before the hyphen if the ligature is broken by auto-hyphenation,
% \Meta{post} is the text to appear after the hyphen if the ligature
% is broken by auto-hyphenation, and \Meta{full} is the full ligature
% if it is not broken by hyphenation.
% For example, to correctly typeset |\[Gsus4]dif\[G]ficult|\eat\],
% in which the \chord{G} chord falls in the middle of the ``ffi''
% ligature, one should use:
%
% \example|di\ch{G}{f}{fi}{ffi}cult|\produces{di\ch{G}{f}{fi}{ffi}cult}
%
% \noindent
% This causes the ``ffi'' ligature to appear intact yet still correctly
% places the \chord{G} chord over the second f.
% To use the |\ch| macro with a replayed chord name (see \S\ref{sec:replay}),
% use |^| as the \Meta{chord}.
%
% \DescMacro{mch}
% The |\mch| macro is exactly like the \mac{ch} macro except that it
% also places a measure bar into the ligature along with the chord.
% For example,
%
% \example|di\mch{G}{f}{fi}{ffi}cult|\produces{di\mch{G}{f}{fi}{ffi}cult}
%
% \noindent
% places both a measure bar and a \chord{G} chord after the first ``f''
% in ``difficult'', yet correctly produces an unbroken ``ffi'' ligature
% in copies of the book in which measure bars are not displayed.
%
% In the unusual case that a meter change is required within a
% ligature, this can be achieved with a construction like:
%
% \example|\meter{6}{8}di\mch{G}{f}{fi}{ffi}cult|\produces{\meter{6}{8}di\mch{G}{f}{fi}{ffi}cult}
%
% \noindent
% The \mac{meter} macro sets the new time signature, which appears
% above the next measure bar---in this case the measure bar
% produced by the |\mch| macro.
%
% Chords and measure bars produced with \refchar{hat} or
% \refchar{pipe} are safe to use in ligatures.
% Thus, |dif|\refchar{pipe}\refchar{hat}|ficult| requires
% no special treatment; it leaves the ``ffi'' ligature intact when measure
% bars are not being displayed.
%
% \section{Guitar Tablatures}\label{sec:tablatures}
%
% \DescMacro{gtab}
% Guitar tablature diagrams can be created by using the construct
%
% \begin{codeblock}
% |\gtab{|\Meta{chord}|}{|\Meta{fret}|:|\Meta{strings}|:|\Meta{fingering}|}|
% \end{codeblock}
%
% \noindent
% where the \Meta{fret} and \Meta{fingering} parts are both optional
% (and you may omit any colon that borders an omitted argument).
%
% \Meta{chord} is a chord name to be placed above the diagram.
%
% \Meta{fret} is an optional digit from 2 to 9 placed to the
% left of the diagram.
%
% \Meta{strings} should be a series of symbols, one for each string
% of the guitar from lowest pitch to highest.
% Each symbol should be one of:
% |X| if that string is not to be played,
% |0| (zero or the letter O) if that string is to be played open, or
% one of |1| through |9| if that string is to be played on the given
% numbered fret.
%
% \Meta{fingering} is an optional series of digits, one for each
% string of the guitar from lowest pitch to highest.
% Each digit should be one of:
% |0| if no fingering information is to be displayed for that string (e.g., if
% the string is not being played or is being played open), or
% one of |1| through |4| to indicate that the given numbered finger is to be
% used to hold down that string.
%
% Here are some examples to illustrate:
%
% \example|\gtab{A}{X02220:001230}|\produces{\vcenterbox{\gtab{A}{{\hphantom{4}}:X02220:001230}}}
% \example|\gtab{C#sus4}{4:XX3341}|\produces{\vcenterbox{\gtab{C\shrp sus4}{4:XX3341}}}
% \example|\gtab{B&}{X13331}|\produces{\vcenterbox{\gtab{B\flt}{{\hphantom{4}}:X13331:}}}
%
% To create a barre chord in which one finger is extended across multiple
% strings, use parentheses |()| or brackets |[]| in the \Meta{strings}
% argument to group the barred strings.
% Each such group will draw a barre on the lowest numbered fret it contains.
% For example:
%
% \example|\gtab{C7}{X(3535X):013140}|\produces{\vcenterbox{\gtab{C7}{{\hphantom{4}}:X(3535X):013140}}}
%
% \DescMacro{minfrets}
% By default, tablature diagrams always consist of at least 4 fret rows
% (more if the \Meta{strings} argument contains a number larger than 4).
% To change the minimum number of fret rows, change the value of |\minfrets|.
% For example, typing
%
% \begin{codeblock}
% |\minfrets=1|
% \end{codeblock}
%
% \noindent
% causes tablature diagrams to have only as many rows are required to
% accommodate the largest digit appearing in the \Meta{strings} argument.
%
% \paragraph{Tablatures Within Macros}
% Macros that produce tablatures must not bury the colons that separate the
% \Meta{fret}, \Meta{strings}, and \Meta{fingering} arguments within other
% macros, and it's safest to always include both colons to avoid ambiguities
% related to optional argument parsing.
% For example,
%
% \begin{codeblock}
% |\newcommand{\mystrings}{X4412X}|
% |\newcommand{\myfingers}{X3412X}|
% |\newcommand{\mychord}{|\mac{gtab}|{C|\mac{shrp}|}{:\mystrings:\myfingers}}|
% \end{codeblock}
%
% \noindent
% works as expected.
% But omitting the colon before |\mystrings| in the definition of |\mychord|
% confuses \mac{gtab} into thinking |\mystrings| is the \Meta{fret}
% argument, and writing code like |\gtab{C\shrp}{\allargs}| with |\allargs|
% defined to something with colons results in an error, because it confuses
% \mac{gtab} into thinking that |\allargs| is only the \Meta{strings} argument.
%
% \section{Automatic Transposition}\label{sec:transpose}
%
% \DescMacro{transpose}
% You can automatically transpose some or all of the chords in a song up by
% \Meta{n} half-steps by adding the line
%
% \begin{codeblock}
% |\transpose{|\Meta{n}|}|
% \end{codeblock}
%
% \noindent
% somewhere between the song's \mac{beginsong} line and the first chord to
% be transposed.
% For example, if a song's first chord is |\[D]|\eat\], and the line
% |\transpose{2}| appears before it, then the chord appears as an
% \chord{E} in the resulting document.
% Specifying a negative number for \Meta{n} transposes subsequent chords
% down instead of up.
%
% The |\transpose| macro affects all chords appearing after it until the
% \mac{endsong} line.
% If two |\transpose| macros appear in the same song, their effects are
% cumulative.
%
% When the \env{transposecapos} option is active, the \mac{capo}
% macro acts like |\transpose|.
% See \S\ref{sec:notes} for more information.
%
% \paragraph{Enharmonics.}
% \DescMacro{preferflats}
% \DescMacro{prefersharps}
% When using \mac{transpose} to automatically transpose the chords of a song,
% the \Songs{} package code chooses between enharmonically equivalent
% names for ``black key'' notes based on the first chord of the song.
% For example, if |\transpose{1}| is used, and if the first chord of the
% song is an \chord{E}, then all \chord{A} chords that appear in
% the song are transcribed as \chord{B\flt} chords rather than
% \chord{A\shrp} chords, since the key of \chord{F}-major (\chord{E}
% transposed up by one half-step) has a flatted key signature.
% Usually this guess produces correct results, but if not, you can use
% either |\preferflats| or |\prefersharps| after the \mac{transpose} line
% to force all transcription to use flatted names or sharped names
% respectively, when resolving enharmonic equivalents.
%
% \paragraph{Modulated Verses.}
% Automatic transposition can be used in conjunction with chord-replaying
% (see \S\ref{sec:chords}) to produce modulated verses.
% For example,
%
% \begin{codeblock}
% \mac{beginverse}\mac{memorize}
% |\[F#]This is a \[B/F#]memorized \[F#]verse. \[E&7]|\eat\]
% \mac{endverse}
% \mac{transpose}|{2}|
% \mac{beginverse}
% |^This verse is ^modulated up two ^half-steps.|
% \mac{endverse}
% \end{codeblock}
%
% \noindent produces
%
% \begin{chorded}\memorize
%   \[F#]This is a \[B/F#]memorized \[F#]verse. \[E&7]\eat\]
%   \vskip5pt\replay\transpose{2}%
%   ^This verse is ^modulated up two ^half-steps.
%   \transpose{-2}%
% \end{chorded}
%
% \paragraph{Both Keys.}
% \DescMacro{trchordformat}
% By default, when chords are automatically transposed using \mac{transpose},
% only the transposed chords are printed.
% However, in some cases you may wish to print the old chords and the
% transposed chords together so that musicians playing transposing and
% non-transposing instruments can play from the same piece of music.
% This can be achieved by redefining the |\trchordformat| macro, which
% receives two arguments---the original chord name and the transposed chord
% name, respectively.
% For example, to print the old chord above the new chord above each lyric,
% define
%
% \begin{codeblock}
% |\renewcommand{\trchordformat}[2]{\vbox{\hbox{#1}\hbox{#2}}}|
% \end{codeblock}
%
% \paragraph{Changing Note Names.}
% \DescMacro{solfedge}
% \DescMacro{alphascale}
% In many countries it is common to use the solfedge names for the notes of
% the scale (\chord{LA, SI, DO, RE, MI, FA, SOL\/}) instead of the
% alphabetic names (\chord{A, B, C, D, E, F, G\/}).
% By default, the transposition logic only understands alphabetic names, but
% you can tell it to look for solfedge names by typing |\solfedge|.
% To return to alphabetic names, type |\alphascale|.
%
% \DescMacro{notenames}
% You can use other note names as well.
% To define your own note names, type
%
% \begin{codeblock}
% |\notenames{|\Meta{nameA}|}{|\Meta{nameB}|}|$\ldots$|{|\Meta{nameG}|}|
% \end{codeblock}
%
% \noindent
% where each of \Meta{nameA} through \Meta{nameG} must consist entirely of
% a sequence of one or more \emph{uppercase} letters.
% For example, some solfedge musicians use \chord{TI} instead of \chord{SI}
% for the second note of the scale.
% To automatically transpose such music, use:
%
% \begin{codeblock}
% |\notenames{LA}{TI}{DO}{RE}{MI}{FA}{SOL}|
% \end{codeblock}
%
% \DescMacro{notenamesin}
% \DescMacro{notenamesout}
% The \Songs{} package can also automatically convert one set of note names
% to another.
% For example, suppose you have a large song book in which chords have been
% typed using alphabetic note names, but you wish to produce a book that
% uses the equivalent solfedge names.
% You could achieve this by using the |\notenamesin| macro to tell the
% \Songs{} package which note names you typed in the input file, and then
% using |\notenamesout| to tell the \Songs{} package how you want it to
% typeset each note name in the output file.
% The final code looks like this:
%
% \begin{codeblock}
% |\notenamesin{A}{B}{C}{D}{E}{F}{G}|
% |\notenamesout{LA}{SI}{DO}{RE}{MI}{FA}{SOL}|
% \end{codeblock}
%
% \noindent
% The syntaxes of |\notenamesin| and |\notenamesout| are identical to that
% of \mac{notenames} (see above), except that the arguments of |\notenamesout|
% can consist of any \LaTeX{} code that is legal in horizontal mode, not just
% uppercase letters.
%
% To stop converting between note names, use \mac{alphascale}, \mac{solfedge},
% or \mac{notenames} to reset all note names back to identical input and
% output scales.
%
% \paragraph{Transposing Chords In Macros.}
% \DescMacro{transposehere}
% The automatic transposition logic does not find chord names that are hidden
% inside macro bodies.
% For example, if you abbreviate a chord by typing,
%
% \begin{codeblock}
% |\newcommand{\mychord}{F|\mac{shrp}| sus4/C|\mac{shrp}|}|
% \mac{transpose}|{4}|
% |\[\mychord]|\eat\]
% \end{codeblock}
%
% \noindent
% then the \mac{transpose} macro fails to transpose it; the
% resulting chord is still an \chord{F\shrp sus4/C\shrp} chord.
% To fix the problem, you can use |\transposehere| in your macros to
% explicitly invoke the transposition logic on chord names embedded in
% macro bodies.
% The above example could be corrected by instead defining:
%
% \begin{codeblock}
% |\newcommand{\mychord}{\transposehere{F|\mac{shrp}| sus4/C|\mac{shrp}|}}|
% \end{codeblock}
%
% \DescMacro{notrans}
% Transposition can be suppressed within material that would otherwise be
% transposed by using the |\notrans| macro.
% For example, writing
%
% \begin{codeblock}
% \mac{transposehere}|{G = \notrans{G}}|
% \end{codeblock}
%
% \noindent
% would typeset a transposed \chord{G} followed by a non-transposed
% \chord{G} chord.
% This does not suppress note name conversion (see \mac{notenames}).
% To suppress both transposition and note name conversion, just use
% braces (e.g., |{G}| instead of |\notrans{G}|).
%
% \paragraph{Transposing Guitar Tablatures.}
% \DescMacro{gtabtrans}
% The songs package cannot automatically transpose tablature diagrams
% (see \S\ref{sec:tablatures}).
% Therefore, when automatic transposition is taking place, only the chord
% names of \mac{gtab} macros are displayed (and transposed); the diagrams
% are omitted.
% To change this default, redefine the |\gtabtrans| macro, whose two
% arguments are the two arguments to \mac{gtab}.
% For example, to display original tablatures without transposing them even
% when transposition has been turned on, write
%
% \begin{codeblock}
% |\renewcommand{\gtabtrans}[2]{|\mac{gtab}|{|\mac{notrans}|{#1}}{#2}}|
% \end{codeblock}
%
% \noindent
% To transpose the chord name but not the diagram under it, replace
% \mac{notrans}|{#1}| with simply |#1| in the above.
% To restore the default behavior, write
%
% \begin{codeblock}
% |\renewcommand{\gtabtrans}[2]{|\mac{transposehere}|{#1}}|
% \end{codeblock}
%
% \section{Between Songs}\label{sec:between}
%
% Never put any material directly into the top level of a \env{songs}
% environment.
% Doing so will disrupt the page-builder, usually producing strange page
% breaks and blank pages.
% To safely put material between songs, use one of the environments
% described in this section.
%
% \subsection{Intersong Displays}\label{sec:intersong}
%
% \DescEnv{intersong}
% To put column-width material between the songs in a \env{songs} environment,
% use an |intersong| environment:
%
% \begin{codeblock}
% |\begin{intersong}|
% $\vdots$
% |\end{intersong}|
% \end{codeblock}
%
% \noindent
% Material contributed in an |intersong| environment is subject to the same
% column-breaking rules as songs (see \S\ref{sec:layout}), but all other
% formatting is up to you.
% By default, \LaTeX{} inserts interline glue below the last line of an
% |intersong| environment.
% To suppress this, end the |intersong| content with |\par\nointerlineskip|.
%
% \DescEnv{intersong*}
% To instead put page-width material above a song, use an |intersong*|
% environment:
%
% \begin{codeblock}
% |\begin{intersong*}|
% $\vdots$
% |\end{intersong*}|
% \end{codeblock}
%
% \noindent
% This starts a new page if the current page already has column-width
% material in it.
%
% \DescEnv{songgroup}
% By default, all intersong displays are omitted when generating a partial
% book with \mac{includeonlysongs}.
% You can force them to be included whenever a particular song is included
% by using a |songgroup| environment:
%
% \begin{codeblock}
% |\begin{songgroup}|
% $\vdots$
% |\end{songgroup}|
% \end{codeblock}
%
% \noindent
% Each |songgroup| environment may include any number of \env{intersong},
% \env{intersong*}, or scripture quotations (see \S\ref{sec:scripture}),
% but must include exactly one song.
% When using \mac{includeonlysongs}, the entire group is included in the
% book if the enclosed song is included; otherwise the entire group is
% omitted.
%
% \subsection{Scripture Quotations}\label{sec:scripture}
%
% \paragraph{Starting a Scripture Quotation.}
% \DescMacro{beginscripture}
% \DescMacro{endscripture}
% A special form of intersong block typesets a scripture quotation.
% Scripture quotations begin and end with
%
% \begin{codeblock}
% |\beginscripture{|\Meta{ref}|}|
% $\vdots$
% |\endscripture|
% \end{codeblock}
%
% \noindent
% where \Meta{ref} is a scripture reference that is
% typeset at the end of the quotation.
% The \Meta{ref} argument should conform to the same syntax
% rules as for the \Meta{ref} arguments passed to \mac{beginsong}
% macros (see \S\ref{sec:songs}).
%
% The text of the scripture quotation between the |\beginscripture| and
% |\endscripture| lines are parsed in normal paragraph mode.
% For example:
%
% \begin{codeblock}
% |\beginscripture{James 5:13}|
% {\tt\frenchspacing%
%   Is any one of you in trouble? He should pray. Is anyone happy? Let him sing songs of praise.
% } |\endscripture|
% \end{codeblock}
%
% \noindent produces
%
% \begin{sample}
%   \beginscripture{James 5:13}
%   Is any one of you in trouble? He should pray. Is anyone happy? Let him sing songs of praise.
%   \endscripture
% \end{sample}
%
% \paragraph{Tuplets.}
% \DescMacro{Acolon}
% \DescMacro{Bcolon}
% To typeset biblical poetry the way it appears in most bibles, begin each
% line with either |\Acolon| or |\Bcolon|.
% A-colons are typeset flush with the left margin, while B-colons are
% indented.
% Any lines too long to fit are wrapped with double-width hanging indentation.
% For example,
%
% \begin{codeblock}
% |\beginscripture{Psalm 1:1}|
% {\tt\frenchspacing%
%   |\Acolon| Blessed is the man
%   |\Bcolon| who does not walk in the counsel of the wicked
%   |\Acolon| or stand in the way of sinners
%   |\Bcolon| or sit in the seat of mockers.
% } |\endscripture|
% \end{codeblock}
%
% \noindent produces
%
% \begin{sample}
%   \beginscripture{Psalm 1:1}
%   \Acolon Blessed is the man
%   \Bcolon who does not walk in the counsel of the wicked
%   \Acolon or stand in the way of sinners
%   \Bcolon or sit in the seat of mockers.
%   \endscripture
% \end{sample}
%
% \paragraph{Stanzas.}
% \DescMacro{strophe}
% Biblical poetry is often grouped into stanzas or ``strophes'',
% each of which is separated from the next by a small vertical
% space.
% You can create that vertical space by typing |\strophe|.
% For example,
%
% \begin{codeblock}
% |\beginscripture{Psalm 88:2-3}|
% {\tt\frenchspacing%
%   |\Acolon| May my prayer come before you;
%   |\Bcolon| turn your ear to my cry.
%   |\strophe|
%   |\Acolon| For my soul is full of trouble
%   |\Bcolon| and my life draws near the grave.
% } |\endscripture|
% \end{codeblock}
%
% \noindent produces
%
% \begin{sample}
%   \beginscripture{Psalm 88:2-3}
%   \Acolon May my prayer come before you;\par
%   \Bcolon turn your ear to my cry.\par
%   \strophe
%   \Acolon For my soul is full of trouble\par
%   \Bcolon and my life draws near the grave.
%   \endscripture
% \end{sample}
%
% \paragraph{Indented Blocks.}
% \DescMacro{scripindent}
% \DescMacro{scripoutdent}
% Some bible passages, such as those that mix prose and poetry, contain
% indented blocks of text.
% You can increase the indentation level within a scripture quotation
% by using |\scripindent| and decrease it by using |\scripoutdent|.
% For example,
%
% \begin{codeblock}
% |\beginscripture{Hebrews 10:17-18}|
% {\tt\frenchspacing%
%   Then he adds:
%   |\scripindent|
%   |\Acolon ``|Their sins and lawless acts
%   |\Bcolon| I will remember no more.|''|
%   |\scripoutdent|
%   And where these have been forgiven, there is no longer any sacrifice for sin.
% } |\endscripture|
% \end{codeblock}
%
% \noindent produces
%
% \begin{sample}
%   \beginscripture{Hebrews 10:17-18}
%   Then he adds:\par
%   \scripindent
%   \Acolon ``Their sins and lawless acts\par
%   \Bcolon I will remember no more.''\par
%   \scripoutdent
%   And where these have been forgiven, there is no longer any sacrifice for sin.
%   \endscripture
% \end{sample}
%
% \section{Chapters and Sections}\label{sec:sectioning}
%
% \DescMacro{songsection}
% \DescMacro{songchapter}
% Song books can be divided into chapters and sections using all the usual
% macros provided by \LaTeX{} (e.g., |\chapter|, |\section|, etc.) and by
% other macro packages.
% In addition, the \Songs{} package provides two helpful built-in sectioning
% macros:
%
% \begin{codeblock}
% |\songchapter{|\Meta{title}|}|
% |\songsection{|\Meta{title}|}|
% \end{codeblock}
%
% \noindent
% which act like \LaTeX's |\chapter| and |\section| commands except that they
% center the \Meta{title} text in sans serif font and omit the chapter/section
% number.
% The |\songchapter| macro only works in document classes that support
% |\chapter| (e.g., the |book| class).
%
% \section{Indexes}
%
% \subsection{Index Creation}\label{sec:indexes}
%
% \DescMacro{newindex}
% \DescMacro{newauthorindex}
% \DescMacro{newscripindex}
% The \Songs{} package supports three kinds of indexes: indexes by title and/or
% notable lyrics, indexes by author, and indexes by scripture reference.
% To generate an index, first declare the index in the document preamble
% (i.e., before the |\begin{document}| line) with one of the following:
%
% \begin{codeblock}
% |\newindex{|\Meta{id}|}{|\Meta{filename}|}|
% |\newauthorindex{|\Meta{id}|}{|\Meta{filename}|}|
% |\newscripindex{|\Meta{id}|}{|\Meta{filename}|}|
% \end{codeblock}
%
% \noindent
% The \Meta{id} should be an alphabetic identifier that will be used to
% identify the index in other macros that reference it.
% The \Meta{filename} should be a string that, when appended with an
% extension, constitutes a valid filename on the system.
% Auxiliary files named \Meta{filename}|.sxd| and \Meta{filename}|.sbx|
% are generated during the automatic index generation process.
% For example:
%
% \begin{codeblock}
% |\newindex{mainindex}{idxfile}|
% \end{codeblock}
%
% \noindent
% creates a title index named ``|mainindex|'' whose data is
% stored in files named |idxfile.sxd| and |idxfile.sbx|.
%
% \DescMacro{showindex}
% To display the index in the document, use:
%
% \begin{codeblock}
% |\showindex[|\Meta{columns}|]{|\Meta{title}|}{|\Meta{id}|}|
% \end{codeblock}
%
% \noindent
% where \Meta{id} is the same identifier used in the \mac{newindex},
% \mac{newauthorindex}, or \mac{newscripindex} command, and where
% the \Meta{title} is the title of the index, which should consist only of
% simple text (no font or formatting macros, since those cannot be used in
% pdf bookmark indexes).
% The |[|\Meta{columns}|]| part is optional; if specified it dictates the
% number of columns if the index can't fit in a single column.
% For example, for a 2-column title index, write:
%
% \begin{codeblock}
% |\showindex[2]{Index of Song Titles}{mainindex}|
% \end{codeblock}
%
% \subsection{Index Entries}\label{sec:ientry}
%
% Every song automatically gets entries in the current \env{songs}
% environment's list of title index(es) (see \S\ref{sec:songs}).
% However, you can also add extra index entries for a song to any index.
%
% \paragraph{Indexing Lyrics.}
% \DescEnv{index=}
% For example, title indexes often have entries for memorable lines
% of lyrics in a song in addition to the song's title.
% You can add an index entry for the current song to the section's
% title index(es) by adding |index={|\Meta{lyrics}|}| to the song's
% \mac{beginsong} line.
% For example,
%
% \begin{codeblock}
% \mac{beginsong}|{Doxology}|
% |          [index={Praise God from Whom all blessings flow}]|
% \end{codeblock}
%
% \noindent
% causes the song to be indexed both as ``\textit{Doxology}'' and as
% ``Praise God from Whom all blessings flow'' in the section's title index(es).
% You can use |index=| multiple times in a \mac{beginsong} line to produce
% multiple additional index entries.
% Index entries produced with |index={|\Meta{lyrics}|}| are
% typeset in an upright font instead of in italics to distinguish
% them from song titles.
%
% \paragraph{Indexing Extra Song Titles.}
% \DescEnv{ititle=}
% To add a regular index entry typeset in italics to the title
% index(es), use:
%
% \begin{codeblock}
% |ititle={|\Meta{title}|}|
% \end{codeblock}
%
% \noindent
% in the \mac{beginsong} line instead.
% Like \env{index=} keyvals, |ititle=| can be used multiple times to produce
% multiple additional index entries.
%
% \DescMacro{indexentry}
% \DescMacro{indextitleentry}
% You can also create index entries by saying
% |\indexentry[|\Meta{indexes}|]{|\Meta{lyrics}|}| (which creates an
% entry like \env{index=}) or
% |\indextitleentry[|\Meta{indexes}|]{|\Meta{title}|}| (which
% creates an entry like \env{ititle=}).
% These two macros can be used anywhere between the song's \mac{beginsong}
% and \mac{endsong} lines, and can be used multiple times to produce
% multiple entries.
% If specified, \Meta{indexes} is a comma-separated list of the identifiers
% of indexes to which the entry should be added.
% Otherwise the new entry is added to all of the title indexes for the current
% \env{songs} environment.
%
% \subsection{Compiling}\label{sec:compiling}
%
% As with a typical \LaTeX{} document, compiling a song book document with
% indexes requires three steps.
% First, use \LaTeX{} (|pdflatex| is recommended) to generate auxiliary files
% from the |.tex| file:
%
% \begin{codeblock}
% |pdflatex mybook.tex|
% \end{codeblock}
%
% Second, use the |songidx.lua| script to generate an index for each index that
% you declared with \mac{newindex}, \mac{newauthorindex}, or
% \mac{newscripindex}.
% The script can be launched using Lua\TeX, using the following syntax:
%
% \begin{codeblock}
% |texlua songidx.lua |\textcolor{black}{[}|-b| \Meta{canon}|.can|\textcolor{black}{]} \Meta{filename}|.sxd| \Meta{filename}|.sbx|
% \end{codeblock}
%
% \noindent
% where \Meta{filename} is the same \Meta{filename} that was used in the
% \mac{newindex}, \mac{newauthorindex}, or \mac{newscripindex} macro.
% If the index was declared with \mac{newscripindex}, then the |-b| option
% is used to specify which version of the bible you wish to use as a basis
% for sorting your scripture index.
% The \Meta{canon} part can be any of the |.can| files provided with
% the |songidx| distribution.
% If you are using a Protestant, Catholic, or Greek Orthodox Christian bible
% with book names in English, then the |bible.can| canon file should work
% well.
% For other bibles, you should create your own |.can| file by copying and
% modifying one of the existing |.can| files.
%
% For example, if your song book |.tex| file contains the lines
%
% \begin{codeblock}
% \mac{newindex}|{titleidx}{titlfile}|
% \mac{newauthorindex}|{authidx}{authfile}|
% \mac{newscripindex}|{scripidx}{scrpfile}|
% \end{codeblock}
%
% \noindent
% then to generate indexes sorted according to a Christian English bible,
% execute:
%
% \begin{codeblock}
% |texlua songidx.lua titlfile.sxd titlfile.sbx|
% |texlua songidx.lua authfile.sxd authfile.sbx|
% |texlua songidx.lua -b bible.can scrpfile.sxd scrpfile.sbx|
% \end{codeblock}
%
% Once the indexes are generated, generate the final book by invoking
% \LaTeX{} one more time:
%
% \begin{codeblock}
% |pdflatex mybook.tex|
% \end{codeblock}
%
% \section{Customizing the Book}
%
% \subsection{Song and Verse Numbering}\label{sec:numbering}
%
% \paragraph{Song Numbering.}
% \DescEnv{songnum}
% The |songnum| counter defines the next song's number.
% It is set to 1 at the beginning of a \env{songs} environment and is
% increased by 1 after each \mac{endsong}.
% It can be redefined anywhere except within a song.
% For example,
%
% \begin{codeblock}
% |\setcounter{songnum}{3}|
% \end{codeblock}
%
% \noindent sets the next song's number to be 3.
%
% \DescMacro{thesongnum}
% You can change the song numbering style for a song section by redefining
% |\thesongnum|.
% For example, to cause songs to be numbered A1, A2, etc., in the current
% song section, type
%
% \begin{codeblock}
% |\renewcommand{\thesongnum}{A\arabic{songnum}}|
% \end{codeblock}
%
% \noindent
% The expansion of |\thesongnum| must always produce plain text with no
% font formatting or unexpandable macro tokens, since its text is
% exported to auxiliary index generation files where it is sorted.
%
% \DescMacro{printsongnum}
% To change the formatting of song numbers as they appear at the beginning
% of each song, redefine the |\printsongnum| macro,
% which expects the text yielded by \mac{thesongnum} as its only argument.
% For example, to typeset song numbers in italics atop each song, define
%
% \begin{codeblock}
% |\renewcommand{\printsongnum}[1]{\it\LARGE#1}|
% \end{codeblock}
%
% \DescMacro{songnumwidth}
% The |\songnumwidth| length defines the width of the shaded boxes that contain
% song numbers at the beginning of each song.
% For example, to make each such box 2 centimeters wide, you could define
%
% \begin{codeblock}
% |\setlength{\songnumwidth}{2cm}|
% \end{codeblock}
%
% \noindent
% If |\songnumwidth| is set to zero, song numbers are not shown at all.
%
% \DescMacro{nosongnumbers}
% To turn off song numbering entirely, type |\nosongnumbers|.
% This inhibits the display of the song number atop each song
% (but song numbers are still be displayed elsewhere, such as in indexes).
% The same effect can be achieved by setting \mac{songnumwidth} to zero.
%
% \paragraph{Verse Numbering.}
% \DescEnv{versenum}
% The |versenum| counter defines the next verse's number.
% It is set to 1 after each \mac{beginsong} line and is increased by 1 after
% each \mac{endverse} (except if the verse begins with \mac{beginverse}|*|).
% The |versenum| counter can be redefined anywhere within a song.
% For example,
%
% \begin{codeblock}
% |\setcounter{versenum}{3}|
% \end{codeblock}
%
% \noindent sets the next verse's number to be 3.
%
% \DescMacro{theversenum}
% You can change the verse numbering style by redefining |\theversenum|.
% For example, to cause verses to be numbered in uppercase roman numerals,
% define
%
% \begin{codeblock}
% |\renewcommand{\theversenum}{\Roman{versenum}}|
% \end{codeblock}
%
% \DescMacro{printversenum}
% To change the formatting of verse numbers as they appear at the beginning
% of each verse, redefine the |\printversenum| macro,
% which expects the text yielded by \mac{theversenum} as its only argument.
% For example, to typeset verse numbers in italics, define
%
% \begin{codeblock}
% |\renewcommand{\printversenum}[1]{\it\LARGE#1.\ }|
% \end{codeblock}
%
% \DescMacro{versenumwidth}
% The |\versenumwidth| length defines the horizontal space reserved for verse
% numbers to the left of each verse text.
% Verse text is shifted right by this amount.
% For example, to reserve half a centimeter of space for verse numbers, define
%
% \begin{codeblock}
% |\setlength{\versenumwidth}{0.5cm}|
% \end{codeblock}
%
% Verse numbers whose widths exceed |\versenumwidth| indent the first
% line of the verse an additional amount to make room, but subsequent lines
% of the verse are only indented by |\versenumwidth|.
%
% \DescMacro{noversenumbers}
% To turn off verse numbering entirely, use |\noversenumbers|.
% This is equivalent to saying
%
% \begin{codeblock}
% |\renewcommand{|\mac{printversenum}|}[1]{}|
% |\setlength{\versenumwidth}{0pt}|
% \end{codeblock}
%
% \DescMacro{placeversenum}
% The horizontal placement of verse numbers within the first line of each
% verse is controlled by the |\placeversenum| macro.
% By default, each verse number is placed flush-left.
% \ImplOrDesc
%   {Authors interested in changing the placement of verse numbers should
%    consult \S\ref{sec:impparams} of the implementation section for more
%    information on this macro.}
%   {For more information on this macro, recompile this documentation with
%    the implementation section included.}
%
% \subsection{Song Appearance}
%
% \paragraph{Font Selection.}
% \DescMacro{lyricfont}
% By default, lyrics are typeset using the document-default font
% (|\normalfont|) and with the document-default point size (|\normalsize|).
% You can change these defaults by redefining |\lyricfont|.
% For example, to cause lyrics to be typeset in small sans serif font,
% you could define
%
% \begin{codeblock}
% |\renewcommand{\lyricfont}{\sffamily\small}|
% \end{codeblock}
%
% \DescMacro{stitlefont}
% Song titles are typeset in a sans-serif, slanted font by default
% (sans-serif, upright if producing slides), with minimal line spacing.
% You can change this default by redefining |\stitlefont|.
% For example, to cause titles to be typeset in a roman font with lines
% spaced 20 points apart, you could define
%
% \begin{codeblock}
% |\renewcommand{\stitlefont}{|
% |  \rmfont\Large\baselineskip=20pt\lineskiplimit=0pt|
% |}|
% \end{codeblock}
%
% \DescMacro{versefont}
% \DescMacro{chorusfont}
% \DescMacro{meterfont}
% \DescMacro{echofont}
% \DescMacro{notefont}
% You can apply additional font changes to verses, choruses, meter numbers,
% echo parts produced with \mac{echo}, and textual notes produced with
% \mac{textnote} and \mac{musicnote}, by redefining |\versefont|,
% |\chorusfont|, |\meterfont|, |\echofont|, and |\notefont|, respectively.
% For example, to typeset choruses in italics, you could define
%
% \begin{codeblock}
% |\renewcommand{\chorusfont}{\it}|
% \end{codeblock}
%
% \DescMacro{notebgcolor}
% \DescMacro{snumbgcolor}
% The colors of shaded boxes containing textual notes and song numbers
% can be changed by redefining the |\notebgcolor| and |\snumbgcolor|
% macros.
% For example:
%
% \begin{codeblock}
% |\renewcommand{\notebgcolor}{red}|
% \end{codeblock}
%
% \DescMacro{printchord}
% By default, chords are typeset in sans serif oblique (slanted) font.
% You can customize chord appearance by redefining |\printchord|, which
% accepts the chord text as its argument.
% For example, to cause chords to be printed in roman boldface font,
% you could define
%
% \begin{codeblock}
% |\renewcommand{\printchord}[1]{\rmfamily\bf#1}|
% \end{codeblock}
%
% \paragraph{Accidental Symbols.}
% \DescMacro{sharpsymbol}
% \DescMacro{flatsymbol}
% By default, sharp and flat symbols are typeset using \LaTeX's
% |\#| ($\#$) and |\flat| ($\flat$) macros.
% Users can change this by redefining |\sharpsymbol| and |\flatsymbol|.
% For example, to use |\sharp| ($\sharp$) instead of $\#$, one could
% redefine |\sharpsymbol| as follows.
%
% \begin{codeblock}
% |\renewcommand{\sharpsymbol}{\ensuremath{^\sharp}}|
% \end{codeblock}
%
% \paragraph{Verse and Chorus Titles.}
% \DescMacro{everyverse}
% \DescMacro{everychorus}
% The |\everyverse| macro is executed at the beginning of each verse, and
% |\everychorus| is executed at the beginning of each chorus.
% Thus, to begin each chorus with the word ``Chorus:'' one could type,
%
% \begin{codeblock}
% |\renewcommand{\everychorus}{|\mac{textnote}|{Chorus:}}|
% \end{codeblock}
%
% \paragraph{Spacing Options.}
% \DescMacro{versesep}
% The vertical distance between song verses and song choruses is defined by
% the skip register |\versesep|.
% For example, to put 12 points of space between each pair of verses and
% choruses, with a flexibility of plus or minus 2 points, you could define
%
% \begin{codeblock}
% |\versesep=12pt plus 2pt minus 2pt|
% \end{codeblock}
%
% \DescMacro{afterpreludeskip}
% \DescMacro{beforepostludeskip}
% The vertical distance between the song's body and its prelude and postlude
% material is controlled by skips |\afterpreludeskip| and
% |\beforepostludeskip|.
% This glue can be made stretchable for centering effects.
% For example, to cause each song body to be centered on the page with one
% song per page, you could write:
%
% \begin{codeblock}
% \mac{songcolumns}|{1}|
% \mac{spenalty}|=-10000|
% |\afterpreludeskip=2pt plus 1fil|
% |\beforepostludeskip=2pt plus 1fil|
% \end{codeblock}
%
% \DescMacro{baselineadj}
% The vertical distance between the baselines of consecutive lines of
% lyrics is computed by the \Songs{} package based on several factors
% including the lyric font size, the chord font size (if in \env{chorded}
% mode), and whether \env{slides} mode is currently active.
% You can adjust the results of this computation by redefining skip
% register |\baselineadj|.
% For example, to reduce the natural distance between baselines by 1 point
% but allow an additional 1 point of stretching when attempting to balance
% columns, you could define
%
% \begin{codeblock}
% |\baselineadj=-1pt plus 1pt minus 0pt|
% \end{codeblock}
%
% \DescMacro{clineparams}
% To change the vertical distance between chords and the lyrics below them,
% redefine the |\clineparams| macro with a definition that adjusts the
% \LaTeX{} parameters |\baselineskip|, |\lineskiplimit|, and |\lineskip|.
% For example, to cause the baselines of chords and their lyrics to be
% 12 points apart with at least 1 point of space between the bottom of the
% chord and the top of the lyric, you could write:
%
% \begin{codeblock}
% |\renewcommand{\clineparams}{|
% |  \baselineskip=12pt|
% |  \lineskiplimit=1pt|
% |  \lineskip=1pt|
% |}|
% \end{codeblock}
%
% \DescMacro{cbarwidth}
% The width of the vertical line that appears to the left of choruses is
% controlled by the |\cbarwidth| length.
% To eliminate the line entirely (and the spacing around it), you can set
% |\cbarwidth| to |0pt|:
%
% \begin{codeblock}
% |\setlength{\cbarwidth}{0pt}|
% \end{codeblock}
%
% \DescMacro{sbarheight}
% The height of the horizontal line that appears between each pair of songs
% is controlled by the |\sbarheight| length.
% To eliminate the line entirely (and the spacing around it), you can set
% |\sbarheight| to |0pt|:
%
% \begin{codeblock}
% |\setlength{\sbarheight}{0pt}|
% \end{codeblock}
%
% \paragraph{Song Top and Bottom Material.}
% You can adjust the header and footer material that precedes and concludes
% each song by redefining |\extendprelude| and |\extendpostlude|.
%
% \DescMacro{extendprelude}
% \DescMacro{showauthors}
% \DescMacro{showrefs}
% By default, |\extendprelude| displays the song's authors and scripture
% references using the macros |\showauthors| and |\showrefs|.
% The following definition changes it to also print copyright info:
%
% \begin{codeblock}
% |\renewcommand{\extendprelude}{|
% |  |\mac{showrefs}\mac{showauthors}
% |  {\bfseries|\mac{songcopyright}|\par}|
% |}|
% \end{codeblock}
%
% \DescMacro{extendpostlude}
% By default, |\extendpostlude| prints the song's copyright and licensing
% information as a single paragraph using \mac{songcopyright} and
% \mac{songlicense}.
% The following definition changes it to also print the words
% ``Used with permission'' at the end of every song's footer information:
%
% \begin{codeblock}
% |\renewcommand{\extendpostlude}{|
% |  |\mac{songcopyright}|\ |\mac{songlicense}|\unskip|
% |  \ Used with permission.|
% |}|
% \end{codeblock}
%
% In general, any macro documented in \S\ref{sec:songinfo} can be used
% in |\extendprelude| and |\extendpostlude| to print song information, such
% as \mac{songauthors}, \mac{songrefs}, \mac{songcopyright}, and
% \mac{songlicense}.
% For convenience, the \mac{showauthors} and \mac{showrefs} macros display
% author and scripture reference information as a pre-formatted paragraph
% the way it appears in the default song header blocks.
%
% See \S\ref{sec:newkey} for how to define new \mac{beginsong} keyvals and
% use them in |\extendprelude|.
%
% \DescMacro{makeprelude}
% \DescMacro{makepostlude}
% For complete control over the appearance of the header and footer material
% that precedes and concludes each song, you can redefine the macros
% |\makeprelude| and |\makepostlude|.
% When typesetting a song, the \Songs{} package code invokes both of these
% macros once (after processing all the material between the \mac{beginsong}
% and \mac{endsong} lines), placing the results within vboxes.
% The resulting vboxes are placed atop and below the song content.
% By default, |\makeprelude| displays the song's titles, authors, and scripture
% references to the right of a shaded box containing the song's number; and
% |\makepostlude| displays the song's copyright and licensing information in
% fine print.
%
% As a simple example, the following causes each song to start with its
% number and title(s), centered, in a large, boldface font, and then centers
% the rest of the prelude material (e.g., references and authors) below that
% (using \mac{extendprelude}).
%
% \begin{codeblock}
% |\renewcommand\makeprelude{%|
% |  |\mac{resettitles}
% |  \centering|
% |  {\Large\bfseries|\mac{thesongnum}|. |\mac{songtitle}|\par|
% |   |\mac{nexttitle}\mac{foreachtitle}|{(|\mac{songtitle}|)\par}}%|
% |  |\mac{extendprelude}
% |}|
% \end{codeblock}
%
% \paragraph{Page- and Column-breaking.}
% \DescMacro{vvpenalty}
% \DescMacro{ccpenalty}
% \DescMacro{vcpenalty}
% \DescMacro{cvpenalty}
% \DescMacro{brkpenalty}
% Page-breaking and column-breaking within songs that are too large to fit
% in a single column/page is influenced by the values of several penalties.
% Penalties of value |\interlinepenalty| are inserted between consecutive
% lines of each verse and chorus;
% penalties of value |\vvpenalty|, |\ccpenalty|, |\vcpenalty|, and |\cvpenalty|
% are inserted into each song between consecutive verses, between consecutive
% choruses, after a verse followed by a chorus, and after a chorus followed by
% a verse, respectively;
% and penalties of value |\brkpenalty| are inserted wherever \mac{brk} is
% used on a line by itself.
% The higher the penalty, the less likely \TeX{} is to place a
% page- or column-break at that site.
% If any are set to $-10000$ or lower, breaks are forced there.
% By default, |\interlinepenalty| is set to 1000 and the rest are set to 200
% so that breaks between verses and choruses are preferred over breaks within
% choruses and verses, but are not forced.
%
% \DescMacro{sepverses}
% Saying |\sepverses| sets all of the above penalties to $-10000$ except for
% |\ccpenalty| which is set to 100.
% This is useful in \env{slides} mode because it forces each verse and
% chorus to be typeset on a separate slide, except for consecutive choruses,
% which remain together when possible.
% (This default reflects an expectation that consecutive choruses typically
% consist of a pre-chorus and chorus that are always sung together.)
%
% These defaults can be changed by changing the relevant penalty register
% directly.
% For example, to force a page- or column-break between consecutive choruses,
% type
%
% \begin{codeblock}
% |\ccpenalty=-10000|
% \end{codeblock}
%
% \paragraph{Text Justification.}
% \DescMacro{versejustify}
% \DescMacro{chorusjustify}
% \DescMacro{justifyleft}
% \DescMacro{justifycenter}
% To left-justify or center the lines of verses or choruses, redefine
% |\versejustify| or |\chorusjustify| to |\justifyleft| or |\justifycenter|,
% respectively.
% For example, to cause choruses to be centered, one could type:
%
% \begin{codeblock}
% |\renewcommand{|\mac{chorusjustify}|}{\justifycenter}|
% \end{codeblock}
%
% \DescMacro{notejustify}
% Justification of textual notes too long to fit on a single line
% is controlled by the |\notejustify| macro.
% By default, it sets up an environment that fully justifies the note
% (i.e., all but the last line of each paragraph extends all the way from
% the left to the right margin).
% \ImplOrDesc
%   {Authors interested in changing this behavior should consult
%    \S\ref{sec:impparams} of the implementation section for more
%    information about this macro.}
%   {For more information, recompile this documentation with the
%    implementation section included.}
%
% \DescMacro{placenote}
% A textual note that is shorter than a single line is placed flush-left by
% default, or is centered when in slides mode.
% This placement of textual notes is controlled by |\placenote|.
% \ImplOrDesc
%   {Authors interested in changing this behavior should consult
%    \S\ref{sec:impparams} of the implementation section for more
%    information about this macro.}
%   {For more information, recompile this documentation with the
%    implementation section included.}
%
% \subsection{Scripture Appearance}
%
% \DescMacro{scripturefont}
% By default, scripture quotations are typeset in Zaph Chancery font
% with the document-default point size (|\normalsize|).
% You can change these defaults by redefining |\scripturefont|.
% For example, to cause scripture quotations to be typeset in sans serif
% italics, define:
%
% \begin{codeblock}
% |\renewcommand{\scripturefont}{\sffamily\it}|
% \end{codeblock}
%
% \DescMacro{printscrcite}
% By default, the citation at the end of a scripture quotation is
% typeset in sans serif font at the document-default point size
% (|\normalsize|).
% You can customize the appearance of the citation by redefining
% |\printscrcite|, which accepts the citation text as its argument.
% For example, to cause citations to be printed in roman italics font, define:
%
% \begin{codeblock}
% |\renewcommand{\printscrcite}[1]{\rmfamily\it#1}|
% \end{codeblock}
%
% \subsection{Conditional Blocks}\label{sec:conditionals}
%
% Conditional macros allow certain material to be included in some books but
% not others.
% For example, a musician's chord book might include extra verses with
% alternate chordings.
%
% \DescMacroGroup{if}{if...}{ifchorded,iflyric,ifslides,ifpartiallist,ifsongindexes,ifmeasures,ifrawtext,iftranscapos,ifnolyrics,ifpagepreludes,ifvnumbered}
% A conditional block begins with a macro named |\if|\Meta{type}, where
% \Meta{type} is one of the types listed in the first column of
% Table~\ref{tab:conditionals}.
% \begin{table}
% \newcommand\tablerule{\noalign{\hrule}}
% \newlength\oldbaselineskip \oldbaselineskip\baselineskip
% \newlength\oldlineskip \oldlineskip\lineskip
% \newdimen\oldlineskiplimit \oldlineskiplimit\lineskiplimit
% \newcommand\oninterlineskip{%
%   \baselineskip\oldbaselineskip
%   \lineskip\oldlineskip
%   \lineskiplimit\oldlineskiplimit}
% \vbox{\offinterlineskip\hrule
% \halign{&\vrule#&\strut\quad#\hfil\quad&\vrule#&\quad\vtop{\oninterlineskip\hsize3.5in\leftskip0.25in\parindent-0.25in\indent\vrule height\ht\strutbox width0pt depth0pt#\vrule height0pt width0pt depth\dp\strutbox\par}\quad\cr
% height2pt&\omit&&\omit&\cr
% &\hfil{\large\strut Type}&&\hfil{\large\strut Processed only if\kern1pt$\ldots$}&\cr\tablerule
% height2pt&\omit&&\omit&\cr
% &|chorded|&&the \env{chorded} option is active&\cr\tablerule
% &|lyric|&&the \env{chorded} option is not active&\cr\tablerule
% &|slides|&&the \env{slides} option is active&\cr\tablerule
% &|partiallist|&&the \mac{includeonlysongs} macro is being used to extract
%   a partial list of songs&\cr\tablerule
% &|songindexes|&&the \env{noindexes} option is not active&\cr\tablerule
% &|measures|&&the \env{nomeasures} option is not active&\cr\tablerule
% &|rawtext|&&the \env{rawtext} option is active&\cr\tablerule
% &|transcapos|&&the \env{transposecapos} option is active&\cr\tablerule
% &|nolyrics|&&the \mac{nolyrics} macro is in effect&\cr\tablerule
% &|pagepreludes|&&the \mac{pagepreludes} macro is in effect&\cr\tablerule
% &|vnumbered|&&the current verse is numbered (i.e., it was started
%   with \mac{beginverse} instead of \mac{beginverse}|*|)&\cr}
% \hrule}
% \caption{Conditional macros}\label{tab:conditionals}
% \end{table}
% The conditional block concludes with the macro |\fi|.
% Between the |\if|\Meta{type} and the |\fi| may also appear an |\else|.
% For example, in the construction
%
% \begin{codeblock}
% |\ifchorded|
% \quad\Meta{A}
% |\else|
% \quad\Meta{B}
% |\fi|
% \end{codeblock}
%
% \noindent
% material \Meta{A} is only included if the \env{chorded} option is active,
% and material \Meta{B} is only included if the \env{chorded} option is not
% active.
%
% \subsection{Page Layout}\label{sec:layout}
%
% \DescMacro{songcolumns}
% The number of columns per page can be set with |\songcolumns|.
% For example, to create 3 columns per page, write
%
% \begin{codeblock}
% |\songcolumns{3}|
% \end{codeblock}
%
% \noindent
% The number of columns should only be changed outside of \env{songs}
% environments.
%
% Setting the number of columns to zero disables the page-building algorithm
% entirely.
% This can be useful if you want to use an external package, such as
% |multicol| or \LaTeX's built-in |\twocolumn| macro, to build pages.
% For example, the following sets up an environment that is suitable for
% a lyric book that uses |\twocolumn|:
%
% \begin{codeblock}
% |\songcolumns{0}|
% |\flushbottom|
% |\twocolumn[\LARGE\centering My Songs]|
% |\begin{|\env{songs}|}{}|
% $\vdots$
% |\end{|\env{songs}|}|
% \end{codeblock}
%
% \noindent
% When disabling the page-builder, please note the following potential
% issues:
%
% \begin{itemize}
% \item The \mac{repchoruses} feature does not work when the page-builder
% is disabled because the page-builder is responsible for inserting
% repeated choruses as new columns are formed.
% \item External page-building packages tend to allow column- and
% page-breaks within songs because they have no mechanism for moving an
% entire song to the next column or page to avoid such a break
% (see \mac{songpos} below).
% \item Indexes produced with \mac{showindex} are typeset to the width of
% the enclosing environment.
% Thus, you should be sure to reset \LaTeX{} back to one column (via
% |\onecolumn|) before executing \mac{showindex}.
% \end{itemize}
%
% \DescMacro{pagepreludes}
% Song preludes (i.e., the material atop each song, including the title) are
% typeset by default at column width.
% Writing |\pagepreludes| typesets subsequent preludes at page width atop
% fresh pages, with the rest of the song in multiple columns beneath its title.
% (To prohibit separation of songs from their preludes, it also sets
% \mac{songpos} to 0.)
%
% \DescMacro{columnsep}
% The horizontal distance between consecutive columns is controlled by
% the |\columnsep| dimension.
% For example, to separate columns by 1 centimeter of space, write
%
% \begin{codeblock}
% |\columnsep=1cm|
% \end{codeblock}
%
% \DescMacro{colbotglue}
% When \LaTeX{} ends each column it inserts glue equal to |\colbotglue|.
% In lyric books this macro is set to |0pt| so that each column ends flush with
% the bottom of the page.
% In other books that have ragged bottoms, it is set to stretchable
% glue so that columns end at whatever vertical position is convenient.
% The recommended setting for typsetting columns with ragged bottoms is:
%
% \begin{codeblock}
% |\renewcommand{\colbotglue}{0pt plus .5\textheight minus 0pt}|
% \end{codeblock}
%
% \DescMacro{lastcolglue}
% The last column in a \env{songs} environment gets |\lastcolglue| appended
% to it instead.
% By default it is infinitely stretchable so that the last column ends
% at its natural height.
% By setting it to |0pt|, you can force the last column to be flush with
% the bottom of the page:
%
% \begin{codeblock}
% |\renewcommand{\lastcolglue}{0pt}|
% \end{codeblock}
%
% \DescMacro{songpos}
% The \Songs{} package uses a song-positioning algorithm that
% moves songs to the next column or page in order to avoid column- or
% page-breaks within songs.
% The algorithm has four levels of aggressiveness, numbered from 0 to 3.
% You can change the aggressiveness level by typing
%
% \begin{codeblock}
% |\songpos{|\Meta{level}|}|
% \end{codeblock}
%
% \noindent
% The default level is 3, which avoids column-breaks, page-breaks,
% and page-turns within songs whenever possible.
% (Page-turns are page-breaks after odd-numbered pages in two-sided documents,
% or after all pages in one-sided documents.)
% Level 2 avoids page-breaks and page-turns but allows column-breaks within
% songs.
% Level 1 avoids only page-turns within songs.
% Level 0 turns off the song-positioning algorithm entirely.
% This causes songs to be positioned wherever \TeX{} thinks is best
% based on penalty settings (see \mac{vvpenalty} and \mac{spenalty}).
%
% \DescMacro{spenalty}
% The value of |\spenalty| controls the undesirability of column breaks
% at song boundaries.
% Usually it should be set to a value between 0 and \mac{vvpenalty} so that
% breaks between songs are preferable to breaks between verses within a song.
% By default it is set to 100.
% When it is $-10000$ or less, breaks between songs are required, so that
% each song always begins a fresh column.
%
% \subsection{Indexes}\label{sec:idxcust}
%
% \subsubsection{Index Appearance}
%
% \paragraph{Index Titles.}
% To customize the appearance of index titles, redefine the \mac{songsection}
% and/or \mac{songchapter} macros from \S\ref{sec:sectioning}.
% For example, to use \LaTeX's built-in |\section| and |\chapter| macros
% instead, you could write:
%
% \begin{codeblock}
% |\renewcommand{|\mac{songchapter}|}{\chapter}|
% |\renewcommand{|\mac{songsection}|}{\section}|
% \end{codeblock}
%
% \paragraph{Layout and page divisions.}
% \DescMacro{sepindexestrue}
% \DescMacro{sepindexesfalse}
% Indexes are by default typeset on separate pages, and when an index is
% sufficiently small, it is centered on the page in one column.
% To disable these defaults, write |\sepindexesfalse|.
% This causes indexes to avoid using unnecessary vertical space or
% starting unnecessary new pages.
% To re-enable the defaults, use |\sepindexestrue|.
%
% \DescMacro{idxheadwidth}
% The |\idxheadwidth| length defines the width of the shaded boxes that
% begin each alphabetic block of a large title index.
% Setting it to 0pt suppresses the boxes entirely.
% For example, to set the width of those boxes to 1 centimeter, you could
% define
%
% \begin{codeblock}
% |\setlength{\idxheadwidth}{1cm}|
% \end{codeblock}
%
% \paragraph{Fonts and colors.}
% \DescMacro{idxrefsfont}
% To control the formatting of the list of references on the right-hand side
% of index entries, redefine |\idxrefsfont|.
% For example, to typeset each list in boldface, write
%
% \begin{codeblock}
% |\renewcommand{\idxrefsfont}{\bfseries}|
% \end{codeblock}
%
% \DescMacro{idxtitlefont}
% \DescMacro{idxlyricfont}
% Title indexes contain entries for song titles and also entries for notable
% lines of lyrics.
% The fonts for these entries are controlled by |\idxtitlefont| and
% |\idxlyricfont|, respectively.
% For example, to show title entries in boldface sans-serif and lyric entries
% in regular roman font, one could define:
%
% \begin{codeblock}
% |\renewcommand{\idxtitlefont}{\sffamily\bfseries}|
% |\renewcommand{\idxlyricfont}{\rmfamily\mdseries}|
% \end{codeblock}
%
% \DescMacro{idxheadfont}
% To change the font used to typeset the capital letters that start each
% alphabetic section of a large title index, redefine |\idxheadfont|.
% For example, to typeset those letters in italics instead of boldface, type
%
% \begin{codeblock}
% |\renewcommand{\idxheadfont}{\sffamily\it\LARGE}|
% \end{codeblock}
%
% \DescMacro{idxbgcolor}
% To change the background color of the shaded boxes that contain the
% capital letters that start each alphabetic sectino of a large title
% index, redefine |\idxbgcolor|.
% For example:
%
% \begin{codeblock}
% |\renewcommand{\idxbgcolor}{red}|
% \end{codeblock}
%
% \DescMacro{idxauthfont}
% The font used to typeset entries of an author index is controlled by
% |\idxauthfont|.
% For example, to typeset such entries in italics instead of boldface, type
%
% \begin{codeblock}
% |\renewcommand{\idxauthfont}{\small\it}|
% \end{codeblock}
%
% \DescMacro{idxscripfont}
% The font used to typeset entries of a scripture index is controlled by
% |\idxscripfont|.
% For example, to typeset such entries in boldface instead of italics, type
%
% \begin{codeblock}
% |\renewcommand{\idxscripfont}{\sffamily\small\bfseries}|
% \end{codeblock}
%
% \DescMacro{idxbook}
% To control the formatting of the lines that start each new book of the
% bible in a scripture index, redefine |\idxbook|, which accepts the book
% name as its single argument.
% For example, to typeset each book name in a box, one could define
%
% \begin{codeblock}
% |\renewcommand{\idxbook}[1]{\framebox{\small\bfseries#1}}|
% \end{codeblock}
%
% \DescMacro{idxcont}
% In a scripture index, when a column break separates a block of entries
% devoted to a book of the bible, the new column is titled
% ``\Meta{bookname} (continued)'' by default.
% You can change this default by redefining the |\idxcont| macro, which
% receives the \Meta{bookname} as its single argument.
% For example, to typeset an index in German, one might define
%
% \begin{codeblock}
% |\renewcommand{\idxcont}[1]{\small\textbf{#1} (fortgefahren)}|
% \end{codeblock}
%
% \subsubsection{Entry References}
%
% \DescMacro{indexsongsas}
% By default, the right-hand side of each index entry contains a list of
% one or more song numbers.
% To instead list page numbers, use the |\indexsongsas| macro:
%
% \begin{codeblock}
% |\indexsongsas{|\Meta{id}|}{\thepage}|
% \end{codeblock}
%
% \noindent
% where \Meta{id} is the same identifier used in the \mac{newindex},
% \mac{newauthorindex}, or \mac{newscripindex} macro that created the index.
% The second argument must always be something that expands into raw text
% without any formatting, since this text gets output to auxiliary files that
% are lexographically sorted by the index-generation program.
% To go back to indexing songs by song number, use \mac{thesongnum} in place
% of |\thepage| in the above.
%
% \subsubsection{PDF Bookmarks and Links}
%
% \DescMacro{songtarget}
% \DescMacro{songlink}
% Each \mac{beginsong} environment adds a PDF bookmark (if generating a PDF)
% and hyperlink target (if using the |hyperref| package) for the
% song by invoking |\songtarget| with two arguments:
% (1) a suggested PDF bookmark level, and (2) a link target name.
% Links in indexes to these targets are created by |\songlink|, which
% also gets two arguments:
% (1) the link target name (same as the second argument to \mac{songtarget}),
% and (2) the text to be linked.
%
% Redefine these macros to customize or suppress these bookmarks, targets,
% and links.
% For example, to enable both bookmarks and links (the default behavior) use:
%
% \begin{codeblock}
% |\renewcommand{\songtarget}[2]|
% |  {\pdfbookmark[#1]{|\mac{thesongnum}|. |\mac{songtitle}|}{#2}}|
% |\renewcommand{\songlink}[2]{\hyperlink{#1}{#2}}|
% \end{codeblock}
%
% \noindent
% To enable links but not bookmarks, use:
%
% \begin{codeblock}
% |\renewcommand{\songtarget}[2]{\hypertarget{#2}{\relax}}|
% |\renewcommand{\songlink}[2]{\hyperlink{#1}{#2}}|
% \end{codeblock}
%
% \noindent
% To disable both bookmarks and links, use:
%
% \begin{codeblock}
% |\renewcommand{\songtarget}[2]{}|
% |\renewcommand{\songlink}[2]{#2}|
% \end{codeblock}
%
% \subsubsection{Sort Order}
%
% The alphabetic ordering of entries in title and author indexes is dictated
% by the computer system on which the \Songs{} software is installed.
% Different languages and regions have different sorting conventions, so the
% |songidx| Lua script delegates decisions about order to your operating system.
% If the default ordering proves inadequate, you can modify it by changing your
% operating system's \emph{locale} (see your system's local help files).
% Alternatively, you can explicitly tell the |songidx| program which locale to
% use in one of three ways:
%
% \begin{itemize}
% \item \emph{Windows:}
% Edit the |generate.bat| file in the |Sample| folder (or your working folder)
% with any plain text editor (e.g., Vim or Notepad).
% Near the top, find the line that says |SET locale=|.
% After the |=|, type any valid locale name.
% For a list of valid locale names on Windows, please see the ``Language name
% abbreviation'' column of Microsoft's online National Language Support (NLS)
% API Reference:
%
% {\centering\url{http://msdn.microsoft.com/en-us/goglobal/bb896001.aspx}\par}
%
% \item \emph{Unix:}
% Create an environment variable named |SONGIDX_LOCALE| and set it equal to
% the desired locale name.
% The command |locale -a| lists all valid locale names on most Unix systems.
%
% \item \emph{Command-line:}
% If you are executing the |songidx| script manually, use the |-l| option to
% specify the locale:
%
% \begin{codeblock}
% |texlua songidx -l sv_SE myindex.sxd myindex.sbx|
% \end{codeblock}
%
% \end{itemize}
%
% \subsubsection{Special Words In Song Info}
%
% The following macros control how certain keywords are treated when parsing
% and sorting index entries.
% They only affect indexes that have already been declared, so put them
% strictly after all your index creation commands (see \S\ref{sec:indexes}).
%
% \DescMacro{titleprefixword}
% In English, when a title begins with ``The'' or ``A'', it is traditional to
% move these words to the end of the title and sort the entry by the following
% word.
% So for example, ``The Song Title'' is typically indexed as
% ``Song Title, The''.
% To change this default behavior, you can use |\titleprefixword| in the
% document preamble to identify each word to be moved to the end whenever
% it appears as the first word of a title index entry.
% For example, to cause the word ``I'' to be moved to the end of title index
% entries, one could say,
%
% \begin{codeblock}
% |\titleprefixword{I}|
% \end{codeblock}
%
% \noindent
% The first use of |\titleprefixword| overrides the defaults, so if you also
% want to continue to move ``The'' and ``A'' to the end of entries, you must
% also say |\titleprefixword{The}| and |\titleprefixword{A}| explicitly.
% This macro may only be used in the document preamble but may be used
% multiple times to declare multiple prefix words.
%
% \DescMacro{authsepword}
% When parsing author index entries, the word ``and'' is recognized by the
% |songidx| script as a conjunctive that separates author names.
% To override this default and specify a different conjunctive, use the
% |\authsepword| macro one or more times in the document preamble.
% For example, to instead treat ``und'' as a conjunctive, you could say,
%
% \begin{codeblock}
% |\authsepword{und}|
% \end{codeblock}
%
% \noindent
% The first use of |\authsepword| and each of the following macros overrides
% the default, so if you also want to continue to treat ``and'' as a
% conjunctive, you must also say |\authsepword{and}| explicitly.
%
% \DescMacro{authbyword}
% When parsing author index entries, the word ``by'' is recognized as a
% keyword signaling that the index entry should only include material
% in the current list item that follows the word ``by''.
% So for example, ``Music by J.S. Bach'' is indexed as ``Bach, J.S.''
% rather than ``Bach, Music by J.S.''
% To recognize a different word instead of ``by'', you can use |\authbyword|
% in the document preamble.
% For example, to recognize ``durch'' instead, you could say
%
% \begin{codeblock}
% |\authbyword{durch}|
% \end{codeblock}
%
% \DescMacro{authignoreword}
% When parsing author index entries, if a list item contains the word
% ``unknown'', that item is ignored and is not indexed.
% This prevents items like ``Composer unknown'' from being indexed as names.
% To cause the indexer to recognize and ignore a different word, you can
% use the |\authignoreword| macro in the document preamble.
% For example, to ignore author index entries containing the word
% ``unbekannt'', you could say,
%
% \begin{codeblock}
% |\authignoreword{unbekannt}|
% \end{codeblock}
%
% \subsection{Page Headers and Footers}
%
% In \LaTeX, page headers and footers are defined using a system of
% invisible \emph{marks} that get inserted into the document at the
% beginning of each logical unit of the document (e.g., each section, song,
% verse, and chorus).
% The headers and footers are then defined so as to refer to the first
% and/or last invisible mark that ends up on each page once the document
% is divided into pages.
% This section describes the marks made available by the \Songs{} package.
% For more detailed information about the marks already provided by
% \LaTeX{} and how to use them, consult any \LaTeX{} user manual.
%
% \DescMacro{songmark}
% \DescMacro{versemark}
% \DescMacro{chorusmark}
% To add song information to page headings and footers, redefine |\songmark|,
% |\versemark|, or |\chorusmark| to add the necessary \TeX{} marks to the
% current page whenever a new song, verse, or chorus begins.
% These macros expect no arguments; to access the current song's
% information including titles, use the macros documented in
% \S\ref{sec:songinfo}.
% To access the current song's number or the current verse's number, use
% \mac{thesongnum} or \mac{theversenum} (see \S\ref{sec:numbering}).
% For example, to include the song number in the page headings produced by
% \LaTeX's |\pagestyle{myheadings}| feature, you could redefine |\songmark|
% as follows:
%
% \begin{codeblock}
% |\renewcommand{\songmark}{\markboth{|\mac{thesongnum}|}{|\mac{thesongnum}|}}|
% \end{codeblock}
%
% \subsection{Defining New Beginsong Keyvals}\label{sec:newkey}
%
% \DescMacro{newsongkey}
% The \mac{beginsong} macro supports several optional keyval parameters for
% declaring song information, including \env{by=}, \env{sr=}, and \env{cr=}.
% Users can define their own additional keyvals as well.
% To do so, use the |\newsongkey| macro, which has the syntax
%
% \begin{codeblock}
% |\newsongkey{|\Meta{keyname}|}{|\Meta{initcode}|}[|\Meta{default}|]{|\Meta{setcode}|}|
% \end{codeblock}
%
% \noindent
% Here, \Meta{keyname} is the name of the new key for the keyval,
% \Meta{initcode} is \LaTeX{} code that is executed at the start of each
% \mac{beginsong} line before the \mac{beginsong} arguments are processed,
% \Meta{default} (if specified) is the default value used for the keyval when
% \Meta{keyname} appears in \mac{beginsong} without a value,
% and \Meta{setcode} is macro code that is executed whenever
% \Meta{key} is parsed as part of the \mac{beginsong} keyval arguments.
% In \Meta{setcode}, |#1| expands to the value given by the user for the
% keyval (or to \Meta{default} if no value was given).
%
% For example, to define a new song key called |arr| which stores its
% value in a macro called |\arranger|, one could write:
%
% \begin{codeblock}
% |\newcommand{\arranger}{}|
% |\newsongkey{arr}{\def\arranger{}}|
% |                {\def\arranger{Arranged by #1\par}}|
% \end{codeblock}
%
% \noindent
% Then one could redefine \mac{extendprelude} to print the arranger below the
% other song header information:
%
% \begin{codeblock}
% |\renewcommand{\extendprelude}{|
% |  |\mac{showrefs}\mac{showauthors}
% |  {\bfseries\arranger}|
% |}|
% \end{codeblock}
%
% \noindent
% A \mac{beginsong} line could then specify the song's arranger as follows:
%
% \begin{codeblock}
% \mac{beginsong}|{The Title}[arr={R. Ranger}]|
% $\vdots$
% \mac{endsong}
% \end{codeblock}
%
% \noindent This produces
%
% \begin{sample}
%  \setcounter{songnum}{1}%
%  \vskip1pt%
%  \newcommand\arranger{}%
%  \newsongkey{arr}{\def\arranger{}}%
%                  {\def\arranger{Arranged by #1\par}}%
%  \renewcommand{\extendprelude}{
%    \showrefs\showauthors
%    {\bfseries\arranger}
%  }%
%  \beginsong{The Title}[arr={R. Ranger}]
%  \endsong
%  \renewcommand{\extendprelude}{}%
% \end{sample}
%
% For more detailed information about keyvals and how they work, consult the
% documentation for David Carlisle's |keyval| package, which comes standard
% with most \LaTeXe{} installations.
%
% \subsection{Font Kerning Corrections}
%
% \paragraph{Chord Overstriking.}
% In order to conserve space and keep songs readable, the \Songs{} package
% pushes chords down very close to the lyrics with which they are paired.
% Unfortunately, this can sometimes cause low-hanging characters in chord
% names to overstrike the lyrics they sit above.
% For example,
%
% \example|\[(Gsus4/D)]Overstrike|\produces{\[(Gsus4/D)]Overstrike}
% \eat\]
%
% \noindent
% Note that the parentheses and slash symbols in the chord name have
% invaded the lyric that sits beneath them.
%
% \DescMacro{chordlocals}
% The best solution to this problem is to use a font for chord names that
% minimizes low-hanging symbols; but if you lack such a font, then the
% following trick works pretty well.
% Somewhere in the preamble of your document, you can write the following
% \LaTeX{} code:
%
% \begin{codeblock}
% |\renewcommand{\chordlocals}{\catcode`(\active|
% |                            \catcode`)\active|
% |                            \catcode`/\active}|
% |\newcommand{\smraise}[1]{\raise2pt\hbox{\small#1}}|
% |\newcommand{\myslash}{\smraise/}|
% |\newcommand{\myopenparen}{\smraise(|\eat)|}|
% |\newcommand{\mycloseparen}{\smraise)}|
% |{\chordlocals|
% | \global\let(\myopenparen|
% | \global\let)\mycloseparen|
% | \global\let/\myslash}|
% \end{codeblock}
%
% \noindent
% This sets the |/|, |(|, and |)| symbols as active characters whenever they
% appear within chord names.
% \ImplOrDesc
%   {(See \S\ref{sec:chordlocals} for documentation of the
%    \texttt{\string\chordlocals} hook.)}
%   {(Recompile this documentation to include the implementation section
%    for more information about the \texttt{\string\chordlocals} macro.)}
% Each active character is defined so that it produces a smaller, raised
% version of the original symbol.
% The result is as follows:
%
% \renewcommand{\chordlocals}{\catcode`(\active
%                             \catcode`)\active
%                             \catcode`/\active}
% \newcommand{\smraise}[1]{\raise2pt\hbox{\small#1}}
% \newcommand{\myslash}{\smraise/}
% \newcommand{\myopenparen}{\smraise(}
% \newcommand{\mycloseparen}{\smraise)}
% {\chordlocals
%  \global\let(\myopenparen
%  \global\let)\mycloseparen
%  \global\let/\myslash}
% 
% \example|\[(Gsus4/D)]Overstrike (fixed)|\produces{\[(Gsus4/D)]Overstrike (fixed)}
% \eat\]
%
% \renewcommand\chordlocals{}
%
% \noindent
% As you can see, the low-hanging symbols have been elevated so that they
% sit above the baseline, correcting the overstrike problem.
%
% \paragraph{Scripture Font Quotation Marks.}
% \DescMacro{shiftdblquotes}
% The \Songs{} package compensates for a kerning problem in the Zaph Chancery
% font (used to typeset scripture quotations) by redefining the |``| and |''|
% token sequences to be active characters that yield double-quotes shifted
% 1.1 points and 2 points left, respectively, of their normal positions.
% If you use a different font size for scripture quotations, then you can use
% the |\shiftdblquotes| macro when redefining \mac{scripturefont} to change
% this kerning correction.
% For example,
%
% \begin{codeblock}
% |\renewcommand{|\mac{scripturefont}|}{|
% |  \usefont{OT1}{pzc}{mb}{it}|
% |  \shiftdblquotes{-1pt}{-2pt}{-3pt}{-4pt}|
% |}|
% \end{codeblock}
%
% \noindent
% removes 1 point of space to the left and 2 points of space to the
% right of left-double-quote characters, and 3 points to the left and 4 points
% to the right of right-double-quotes, within scripture quotations.
%
% \section{Informational Macros}\label{sec:songinfo}
%
% The macros described in this section can be used to retrieve information
% about the current song.
% This can be used when redefining \mac{extendprelude}, \mac{extendpostlude},
% \mac{makeprelude}, \mac{makepostlude}, \mac{songmark}, \mac{versemark}, or
% \mac{chorusmark}, or any other macros that might typeset this information.
%
% \DescMacro{songauthors}
% To get the current song's list of authors (if any) use |\songauthors|.
% This yields the value of the \env{by=} key used in the \mac{beginsong}
% line.
%
% \DescMacro{songrefs}
% To get the current song's list of scripture references (if any) use
% |\songrefs|.
% This yields the value of the \env{sr=} key used in the \mac{beginsong}
% line, but modified with hyphens changed to en-dashes and spaces falling
% within a list of verse numbers changed to thin spaces for better
% typesetting.
% In addition, various penalties have been added to inhibit line breaks
% in strange places and encourage line breaks in others.
%
% \DescMacro{songcopyright}
% To get the current song's copyright info (if any), use |\songcopyright|.
% This yields the value of the \env{cr=} key used in the \mac{beginsong} line.
%
% \DescMacro{songlicense}
% To get the current song's licensing information (if any), use
% |\songlicense|.
% This yields the value of the \env{li=} key used in the \mac{beginsong}
% line, or whatever text was declared with \mac{setlicense}.
%
% \DescMacro{songtitle}
% The |\songtitle| macro yields the current song's title.
% By default this is the first title provided in the \mac{beginsong} line.
% The \mac{nexttitle} and \mac{foreachtitle} macros (see below) cause it
% to be set to the current song's other titles, if any.
%
% \DescMacro{resettitles}
% To get the current song's primary title (i.e., the first title specified
% in the song's \mac{beginsong} line), execute |\resettitles|.
% This sets the |\songtitle| macro to be the song's primary title.
%
% \DescMacro{nexttitle}
% To get the song's next title, execute |\nexttitle|, which
% sets |\songtitle| to be the next title in the song's list of titles
% (or sets |\songtitle| to |\relax| if there are no more titles).
%
% \DescMacro{foreachtitle}
% The |\foreachtitle| macro accepts \LaTeX{} code as its single
% argument and executes it once for each (remaining) song title.
% Within the provided code, use |\songtitle| to get the current title.
% For example, the following code generates a comma-separated list of all
% of the current song's titles:
%
% \begin{codeblock}
% \mac{resettitles}
% \mac{songtitle}
% \mac{nexttitle}
% |\foreachtitle{, |\mac{songtitle}|}|
% \end{codeblock}
%
% \DescMacro{songlist}
% When \mac{includeonlysongs} is used to extract a partial list of songs, the
% |\songlist| macro expands to the comma-separated list of songs that is being
% extracted.
% Redefining |\songlist| within the document preamble alters the list of
% songs to be extracted.
% Redefining it after the preamble may have unpredictable results.
%
% \section{Other Resources}
%
% There are a number of other \LaTeX{} packages available for typesetting
% songs, tablature diagrams, or song books.
% Probably the best of these is the \Rath{} package by Christopher Rath
% (\href{http://rath.ca/Misc/Songbook/}{{\tt http://rath.ca/Misc/Songbook/}}).
% Most of the differences between other packages and this one are intentional;
% the following is a summary of where I've adopted various differing design
% decisions and why.
%
% \bigskip
%
% \paragraph{Ease of Song Entry.}
% Much of the \Songs{} package programming is devoted to easing the burden of
% typing chords.
% With most \LaTeX{} song book packages the user types chords using a standard
% \LaTeX{} macro syntax like |\chord{|\Meta{chord}|}{|\Meta{lyric}|}|.
% The \Songs{} package uses a less conventional
% |\[|\Meta{chord}|]|\Meta{lyric}\eat\] syntax for several
% reasons detailed below.
%
% First, macros in the standard \LaTeX{} syntax require more key-presses
% than macros in the \Songs{} package's syntax.
% This can become become very taxing when typing up a large book.
% Chords often appear as frequently as one per syllable, especially in hymns,
% so keeping the syntax as brief as possible is desirable.
%
% Second, the standard \LaTeX{} macro syntax requires the user to estimate how
% much of the \Meta{lyric} will lie below the chord (because the \Meta{lyric}
% part must be enclosed in braces) whereas the \Songs{} package's syntax does
% not.
% Estimating this accurately can be quite difficult, since in many cases the
% \Meta{lyric} part must include punctuation or multiple words to get proper
% results.
% The \Songs{} package automates this for the user, significantly easing the
% task of chord-entry.
%
% Third, unlike the standard \LaTeX{} chord syntax, the \Songs{} package's
% syntax handles all hyphenation of chorded lyrics fully automatically.
% Extra hyphenation must be introduced in chord books wherever a chord
% is wider than the syllable it sits above.
% With the standard \LaTeX{} chord syntax such hyphenation must be
% introduced manually by the user (usually via a special hyphenation macro),
% but the \Songs{} package does this automatically.
%
% Fourth and finally, some other packages allow the user to use ``|b|''
% in a \Meta{chord} to produce a flat symbol, whereas the \Songs{} package
% requires an ``|&|'' instead.
% Using ``|b|'' is more intuitive but prevents the
% use of ``|b|'' for any other purpose within a \Meta{chord}, such as to
% produce a literal ``b'' or to type another macro name like |\hbox| that
% contains a ``b''.
% Consequently, the \Songs{} package uses the less obvious ``|&|'' symbol to
% produce flat symbols.
%
% \paragraph{Song Structure.}
% The \Songs{} package provides a relatively small number of macros for
% typesetting high-level song structure, including verses, choruses,
% textual comments, and conditional macros that indicate that certain sections
% should go in chord books but not lyric books.
% These can be combined to typeset more sophisticated structures such as
% intros, bridges, brackets, endings, and the like.
% This is done in lieu of providing a specific macro for each of these
% structures since it results in greater flexibility and fewer macros for
% users to learn.
%
% \paragraph{Multiple columns.}
% The \Songs{} package was designed from the ground up to produce song books
% with many songs per page, arranged in multiple columns.
% As a result, it includes elaborate support for many features not found in
% most other packages, such as automatic column balancing, completely
% customizable song header and song footer blocks, and facilities for adding
% beautiful scripture quotations to fill in gaps between songs.
%
% \paragraph{Indexes.}
% Another major feature of the \Songs{} package is its support for a variety
% of different index types, most notably indexes arranged by scripture
% reference.
% Scripture indexes can be invaluable for planning services around particular
% sermons or topics.
% The \Songs{} package allows book authors to specify the names and preferred
% ordering of books of the bible, and automatically handles complex issues
% like overlapping verse ranges to produce an easy-to-read, compact, and
% well-ordered index.
% Other supported indexes include those sorted by author, by title, and by
% notable lines of lyrics.
%
% \paragraph{Automatic Transposition.}
% The \Songs{} package has a facility for automatically transposing songs, and
% even generating chord books that print the chords in multiple keys (e.g., so
% that a pianist and guitarist using a capo can play together from the same
% book).
%
% \bigskip
%
% The \Songs{} package was developed entirely independently of all other
% \LaTeX{} song book packages.
% I originally developed the set of \LaTeX{} macros that eventually became
% the \Songs{} package in order to typeset a song book for the Graduate
% Christian Fellowship (GCF) at Cornell University, and the Cornell
% International Christian Fellowship (CICF).
% Once I had fine-tuned my package to be sufficiently versatile, I decided
% to release it for public use.
% At that time I noticed the \Rath{} package and others, and wrote this
% summary of the most prominent differences.
%
% For information on more song-typesetting resources for \LaTeX, I recommend
% consulting the documentation provided with the \Rath{} package.
% It includes an excellent list of other resources that might be of interest
% to creators of song books.
%
% \section{GNU General Public License}\label{sec:license}
%
% \begingroup\small
%
% \begin{center}
% {\large\sc Terms and Conditions For \\ Copying, Distribution and Modification}
% \end{center}
%
% \begin{enumerate}\addtocounter{enumi}{-1}
%
% \item 
%
% This License applies to any program or other work which contains a notice
% placed by the copyright holder saying it may be distributed under the
% terms of this General Public License.  The ``Program'', below, refers to
% any such program or work, and a ``work based on the Program'' means either
% the Program or any derivative work under copyright law: that is to say, a
% work containing the Program or a portion of it, either verbatim or with
% modifications and/or translated into another language.  (Hereinafter,
% translation is included without limitation in the term ``modification''.)
% Each licensee is addressed as ``you''.
%
% Activities other than copying, distribution and modification are not
% covered by this License; they are outside its scope.  The act of
% running the Program is not restricted, and the output from the Program
% is covered only if its contents constitute a work based on the
% Program (independent of having been made by running the Program).
% Whether that is true depends on what the Program does.
%
% \item You may copy and distribute verbatim copies of the Program's source
%   code as you receive it, in any medium, provided that you conspicuously
%   and appropriately publish on each copy an appropriate copyright notice
%   and disclaimer of warranty; keep intact all the notices that refer to
%   this License and to the absence of any warranty; and give any other
%   recipients of the Program a copy of this License along with the Program.
%
% You may charge a fee for the physical act of transferring a copy, and you
% may at your option offer warranty protection in exchange for a fee.
%
% \item
%
% You may modify your copy or copies of the Program or any portion
% of it, thus forming a work based on the Program, and copy and
% distribute such modifications or work under the terms of Section~1
% above, provided that you also meet all of these conditions:
%
% \begin{enumerate}
%
% \item 
%
% You must cause the modified files to carry prominent notices stating that
% you changed the files and the date of any change.
%
% \item
%
% You must cause any work that you distribute or publish, that in
% whole or in part contains or is derived from the Program or any
% part thereof, to be licensed as a whole at no charge to all third
% parties under the terms of this License.
%
% \item
% If the modified program normally reads commands interactively
% when run, you must cause it, when started running for such
% interactive use in the most ordinary way, to print or display an
% announcement including an appropriate copyright notice and a
% notice that there is no warranty (or else, saying that you provide
% a warranty) and that users may redistribute the program under
% these conditions, and telling the user how to view a copy of this
% License.  (Exception: if the Program itself is interactive but
% does not normally print such an announcement, your work based on
% the Program is not required to print an announcement.)
%
% \end{enumerate}
%
% These requirements apply to the modified work as a whole.  If
% identifiable sections of that work are not derived from the Program,
% and can be reasonably considered independent and separate works in
% themselves, then this License, and its terms, do not apply to those
% sections when you distribute them as separate works.  But when you
% distribute the same sections as part of a whole which is a work based
% on the Program, the distribution of the whole must be on the terms of
% this License, whose permissions for other licensees extend to the
% entire whole, and thus to each and every part regardless of who wrote it.
%
% Thus, it is not the intent of this section to claim rights or contest
% your rights to work written entirely by you; rather, the intent is to
% exercise the right to control the distribution of derivative or
% collective works based on the Program.
%
% In addition, mere aggregation of another work not based on the Program
% with the Program (or with a work based on the Program) on a volume of
% a storage or distribution medium does not bring the other work under
% the scope of this License.
%
% \item
% You may copy and distribute the Program (or a work based on it,
% under Section~2) in object code or executable form under the terms of
% Sections~1 and~2 above provided that you also do one of the following:
%
% \begin{enumerate}
%
% \item
%
% Accompany it with the complete corresponding machine-readable
% source code, which must be distributed under the terms of Sections~1
% and~2 above on a medium customarily used for software interchange; or,
%
% \item
%
% Accompany it with a written offer, valid for at least three
% years, to give any third party, for a charge no more than your
% cost of physically performing source distribution, a complete
% machine-readable copy of the corresponding source code, to be
% distributed under the terms of Sections~1 and~2 above on a medium
% customarily used for software interchange; or,
%
% \item
%
% Accompany it with the information you received as to the offer
% to distribute corresponding source code.  (This alternative is
% allowed only for noncommercial distribution and only if you
% received the program in object code or executable form with such
% an offer, in accord with Subsection~b above.)
%
% \end{enumerate}
%
% The source code for a work means the preferred form of the work for
% making modifications to it.  For an executable work, complete source
% code means all the source code for all modules it contains, plus any
% associated interface definition files, plus the scripts used to
% control compilation and installation of the executable.  However, as a
% special exception, the source code distributed need not include
% anything that is normally distributed (in either source or binary
% form) with the major components (compiler, kernel, and so on) of the
% operating system on which the executable runs, unless that component
% itself accompanies the executable.
%
% If distribution of executable or object code is made by offering
% access to copy from a designated place, then offering equivalent
% access to copy the source code from the same place counts as
% distribution of the source code, even though third parties are not
% compelled to copy the source along with the object code.
%
% \item
% You may not copy, modify, sublicense, or distribute the Program
% except as expressly provided under this License.  Any attempt
% otherwise to copy, modify, sublicense or distribute the Program is
% void, and will automatically terminate your rights under this License.
% However, parties who have received copies, or rights, from you under
% this License will not have their licenses terminated so long as such
% parties remain in full compliance.
%
% \item
% You are not required to accept this License, since you have not
% signed it.  However, nothing else grants you permission to modify or
% distribute the Program or its derivative works.  These actions are
% prohibited by law if you do not accept this License.  Therefore, by
% modifying or distributing the Program (or any work based on the
% Program), you indicate your acceptance of this License to do so, and
% all its terms and conditions for copying, distributing or modifying
% the Program or works based on it.
%
% \item
% Each time you redistribute the Program (or any work based on the
% Program), the recipient automatically receives a license from the
% original licensor to copy, distribute or modify the Program subject to
% these terms and conditions.  You may not impose any further
% restrictions on the recipients' exercise of the rights granted herein.
% You are not responsible for enforcing compliance by third parties to
% this License.
%
% \item
% If, as a consequence of a court judgment or allegation of patent
% infringement or for any other reason (not limited to patent issues),
% conditions are imposed on you (whether by court order, agreement or
% otherwise) that contradict the conditions of this License, they do not
% excuse you from the conditions of this License.  If you cannot
% distribute so as to satisfy simultaneously your obligations under this
% License and any other pertinent obligations, then as a consequence you
% may not distribute the Program at all.  For example, if a patent
% license would not permit royalty-free redistribution of the Program by
% all those who receive copies directly or indirectly through you, then
% the only way you could satisfy both it and this License would be to
% refrain entirely from distribution of the Program.
%
% If any portion of this section is held invalid or unenforceable under
% any particular circumstance, the balance of the section is intended to
% apply and the section as a whole is intended to apply in other
% circumstances.
%
% It is not the purpose of this section to induce you to infringe any
% patents or other property right claims or to contest validity of any
% such claims; this section has the sole purpose of protecting the
% integrity of the free software distribution system, which is
% implemented by public license practices.  Many people have made
% generous contributions to the wide range of software distributed
% through that system in reliance on consistent application of that
% system; it is up to the author/donor to decide if he or she is willing
% to distribute software through any other system and a licensee cannot
% impose that choice.
%
% This section is intended to make thoroughly clear what is believed to
% be a consequence of the rest of this License.
%
% \item
% If the distribution and/or use of the Program is restricted in
% certain countries either by patents or by copyrighted interfaces, the
% original copyright holder who places the Program under this License
% may add an explicit geographical distribution limitation excluding
% those countries, so that distribution is permitted only in or among
% countries not thus excluded.  In such case, this License incorporates
% the limitation as if written in the body of this License.
%
% \item
% The Free Software Foundation may publish revised and/or new versions
% of the General Public License from time to time.  Such new versions will
% be similar in spirit to the present version, but may differ in detail to
% address new problems or concerns.
%
% Each version is given a distinguishing version number.  If the Program
% specifies a version number of this License which applies to it and ``any
% later version'', you have the option of following the terms and conditions
% either of that version or of any later version published by the Free
% Software Foundation.  If the Program does not specify a version number of
% this License, you may choose any version ever published by the Free Software
% Foundation.
%
% \item
% If you wish to incorporate parts of the Program into other free
% programs whose distribution conditions are different, write to the author
% to ask for permission.  For software which is copyrighted by the Free
% Software Foundation, write to the Free Software Foundation; we sometimes
% make exceptions for this.  Our decision will be guided by the two goals
% of preserving the free status of all derivatives of our free software and
% of promoting the sharing and reuse of software generally.
%
% \begin{center}
% {\large\sc No Warranty}
% \end{center}
%
% \item
% {\sc Because the program is licensed free of charge, there is no warranty
% for the program, to the extent permitted by applicable law.  Except when
% otherwise stated in writing the copyright holders and/or other parties
% provide the program ``as is'' without warranty of any kind, either expressed
% or implied, including, but not limited to, the implied warranties of
% merchantability and fitness for a particular purpose.  The entire risk as
% to the quality and performance of the program is with you.  Should the
% program prove defective, you assume the cost of all necessary servicing,
% repair or correction.}
%
% \item
% {\sc In no event unless required by applicable law or agreed to in writing
% will any copyright holder, or any other party who may modify and/or
% redistribute the program as permitted above, be liable to you for damages,
% including any general, special, incidental or consequential damages arising
% out of the use or inability to use the program (including but not limited
% to loss of data or data being rendered inaccurate or losses sustained by
% you or third parties or a failure of the program to operate with any other
% programs), even if such holder or other party has been advised of the
% possibility of such damages.}
%
% \end{enumerate}
%
% \endgroup
%
% \StopEventually{\PrintIndex}
%
% \clearpage
% \section{Implementation}
%
% The following provides the verbatim implementation of the \Songs{} \LaTeX{} 
% package, along with commentary on how it works.
% In general, macro names that contain a |@| symbol are not intended to be
% directly accessible by the outside world; they are for purely internal use.
% All other macros are intended to be used or redefined by document authors.
%
% Most of the macros likely to be of real interest to song book authors can
% be found in \S\ref{sec:impparams}.
% To find the implementation of any particular macro, the index at the end
% of this document should prove helpful.
%
% The unwary \TeX er may wonder at the rather large size of the
% implementation.
% The volume and complexity of the code stems mainly from the following
% challenging features:
% \begin{itemize}
% \item Putting chords above lyrics fully automatically requires building an
% entire lyric-parser in \LaTeX{} (see \S\ref{sec:lyricscan}).
% \item Avoiding page-turns within songs without prohibiting column-breaks
% requires building a completely new page-breaking algorithm
% (see \S\ref{sec:pagebuilder}).
% \item The package must be able to generate a daunting number of document
% variants from a common source: lyric-only books, chorded books, digital
% slides, transparency slides, selected song subsets, transposed songs, and
% combinations of the above.
% This is like putting six or more packages into one.
% \item Song book indexes are far more complex than those for a prose book.
% See \S\ref{sec:indexgen} for some of the difficulties involved.
% \end{itemize}
%
% \subsection{Initialization}
%
% The code in this section detects any \TeX{} versioning or configuration
% settings that are relevant to the rest of the song book code.
%
% \begin{macro}{\ifSB@etex}
% Numerous enhancements are possible when using an $\varepsilon$-\TeX{}
% compatible version of \LaTeX.
% We start by checking to see whether $\varepsilon$-\TeX{} primitives are
% available.
%    \begin{macrocode}
\newif\ifSB@etex
\ifx\eTeXversion\undefined\else
  \ifx\eTeXversion\relax\else
    \SB@etextrue
    \ifx\e@alloc\@undefined
      \IfFileExists{etex.sty}{\RequirePackage{etex}}{}
    \fi
  \fi
\fi
%    \end{macrocode}
% \end{macro}
%
% \begin{macro}{\ifSB@pdf}
% Detect whether we're generating a pdf file, since this affects the
% treatment of hyperlinks and bookmark indexes.
%    \begin{macrocode}
\newif\ifSB@pdf\SB@pdffalse
\IfFileExists{ifpdf.sty}{\RequirePackage{ifpdf}\ifpdf\SB@pdftrue\fi}{
  \ifx\pdfoutput\undefined\else
    \ifx\pdfoutput\relax\else
      \ifnum\pdfoutput<\@ne\else
        \SB@pdftrue
      \fi
    \fi
  \fi
}
%    \end{macrocode}
% \end{macro}
%
% \begin{macro}{\ifSB@preamble}
% Some macros have different effects depending on when they're used in the
% preamble or in the document body, so we need a conditional that remembers
% whether we're still in the preamble.
% It gets initialized to true and later changed to false once the body begins.
%    \begin{macrocode}
\newif\ifSB@preamble
\SB@preambletrue
%    \end{macrocode}
% \end{macro}
%
% \begin{macro}{\ifSB@test}
% \begin{macro}{\ifSB@testii}
% \begin{macro}{\SB@temp}
% \begin{macro}{\SB@tempii}
% \begin{macro}{\SB@tempiii}
% \begin{macro}{\SB@tempiv}
% \begin{macro}{\SB@tempv}
% Reserve some control sequence names for scratch use.
%    \begin{macrocode}
\newif\ifSB@test
\newif\ifSB@testii
\newcommand\SB@temp{}
\newcommand\SB@tempii{}
\newcommand\SB@tempiii{}
\newcommand\SB@tempiv{}
\newcommand\SB@tempv{}
%    \end{macrocode}
% \end{macro}
% \end{macro}
% \end{macro}
% \end{macro}
% \end{macro}
% \end{macro}
% \end{macro}
%
% \begin{macro}{\SB@newcount}
% \begin{macro}{\SB@newdimen}
% \begin{macro}{\SB@newbox}
% \begin{macro}{\SB@newtoks}
% \begin{macro}{\SB@newwrite}
% Create macros for safely allocating count, dimen, box, token, and write
% registers with detection for name-clashes.
% For some reason, the default allocation macros provided by the \LaTeX{}
% kernel do not detect name-clashes(!), which means that packages that use them
% might accidentally overwrite our registers, causing all sorts of problems.
% But at least we can do our best to avoid overwriting their registers.
%    \begin{macrocode}
\newcommand\SB@newcount[1]{\@ifdefinable#1{\newcount#1}}
\newcommand\SB@newdimen[1]{\@ifdefinable#1{\newdimen#1}}
\newcommand\SB@newbox[1]{\@ifdefinable#1{\newbox#1}}
\newcommand\SB@newtoks[1]{\@ifdefinable#1{\newtoks#1}}
\newcommand\SB@newwrite[1]{\@ifdefinable#1{\newwrite#1}}
%    \end{macrocode}
% \end{macro}
% \end{macro}
% \end{macro}
% \end{macro}
% \end{macro}
%
% \begin{macro}{\SB@dimen}
% \begin{macro}{\SB@dimenii}
% \begin{macro}{\SB@dimeniii}
% \begin{macro}{\SB@dimeniv}
% \begin{macro}{\SB@box}
% \begin{macro}{\SB@boxii}
% \begin{macro}{\SB@boxiii}
% \begin{macro}{\SB@toks}
% \begin{macro}{\SB@cnt}
% \begin{macro}{\SB@cntii}
% \begin{macro}{\SB@skip}
% Reserve some temp registers for various purposes.
%    \begin{macrocode}
\SB@newdimen\SB@dimen
\SB@newdimen\SB@dimenii
\SB@newdimen\SB@dimeniii
\SB@newdimen\SB@dimeniv
\SB@newbox\SB@box
\SB@newbox\SB@boxii
\SB@newbox\SB@boxiii
\SB@newtoks\SB@toks
\SB@newcount\SB@cnt
\SB@newcount\SB@cntii
\newlength\SB@skip
%    \end{macrocode}
% \end{macro}
% \end{macro}
% \end{macro}
% \end{macro}
% \end{macro}
% \end{macro}
% \end{macro}
% \end{macro}
% \end{macro}
% \end{macro}
% \end{macro}
%
% \begin{macro}{\SB@envbox}
% Also reserve a slightly less volatile box register for per-environment use.
% In scripture environments it holds the scripture citation.
% In indexes it holds the index title text.
%    \begin{macrocode}
\SB@newbox\SB@envbox
%    \end{macrocode}
% \end{macro}
%
% Load David Carlisle's |keyval| package for processing
% \Meta{key}=\Meta{value} style macro arguments.
%    \begin{macrocode}
\RequirePackage{keyval}
%    \end{macrocode}
%
% \begin{macro}{\SB@app}
% Utility macro: Append some text to the definition of another macro.
%    \begin{macrocode}
\newcommand\SB@app[3]{%
  \expandafter#1\expandafter#2\expandafter{#2#3}%
}
%    \end{macrocode}
% \end{macro}
%
% \subsection{Default Parameters}\label{sec:impparams}
%
% This section defines macros and lengths that will typically be executed or
% redefined by the user in the document preamble to initialize the document.
% (Not all of these are restricted to preamble usage, however. Many can be used
% throughout the document to switch styles for different sections or different
% songs.)
%
% \begin{macro}{\lyricfont}\MainImpl{lyricfont}
% Define the font style to use for formatting song lyrics.
%    \begin{macrocode}
\newcommand\lyricfont{\normalfont\normalsize}
%    \end{macrocode}
% \end{macro}
%
% \begin{macro}{\stitlefont}\MainImpl{stitlefont}
% Define the font style to use for formatting song titles.
%    \begin{macrocode}
\newcommand\stitlefont{%
  \sffamily\ifslides\Huge\else\slshape\Large\fi%
}
%    \end{macrocode}
% \end{macro}
%
% \begin{macro}{\versefont}\MainImpl{versefont}
% \begin{macro}{\chorusfont}\MainImpl{chorusfont}
% \begin{macro}{\notefont}\MainImpl{notefont}
% \begin{macro}{\meterfont}\MainImpl{meterfont}
% \changes{v2.1}{2007/08/02}{Added}
% By default, verses, choruses, and textual notes just allow the |\lyricfont|
% style to continue.
% Meter numbers are in tiny, sans-serif, upright font.
% Echo parts toggle slanted and upright fonts.
%    \begin{macrocode}
\newcommand\versefont{}
\newcommand\chorusfont{}
\newcommand\notefont{}
\newcommand\meterfont{\tiny\sffamily\upshape}
%    \end{macrocode}
% \end{macro}
% \end{macro}
% \end{macro}
% \end{macro}
%
% \begin{macro}{\echofont}\MainImpl{echofont}
% \changes{v2.18}{2014/06/28}{Added}
% Echo parts toggle between oblique and upright shapes like |\emph|, but we
% use |\slshape| instead of |\itshape| because it tends to look nicer with the
% larger fonts used in slides mode.
%    \begin{macrocode}
\newcommand\echofont{%
  \ifdim\fontdimen\@ne\font>\z@\upshape\else\slshape\fi%
}
%    \end{macrocode}
% \end{macro}
%
% \begin{macro}{\scripturefont}\MainImpl{scripturefont}
% \changes{v1.13}{2005/05/12}{Added kerning correction for double-quote ligatures}
% Define the font style to use for formatting scripture quotations
% (defaults to Zapf Chancery).
%    \begin{macrocode}
\newcommand\scripturefont{%
  \usefont{OT1}{pzc}{mb}{it}%
  \shiftdblquotes{-1.1\p@}\z@{-2\p@}\z@%
}
%    \end{macrocode}
% \end{macro}
%
% \begin{macro}{\printscrcite}\MainImpl{printscrcite}
% Define the printing style for the citation at the end of a scripture
% quotation.
%    \begin{macrocode}
\newcommand\printscrcite[1]{\sffamily\small#1}
%    \end{macrocode}
% \end{macro}
%
% \begin{macro}{\snumbgcolor}\MainImpl{snumbgcolor}
% \begin{macro}{\notebgcolor}\MainImpl{notebgcolor}
% \begin{macro}{\idxbgcolor}\MainImpl{idxbgcolor}
% Define the background color used for shaded boxes containing
% song numbers, textual notes, and index section headers, respectively.
% To turn off all shading for a box type, use |\def|\Meta{macroname}|{}|.
%    \begin{macrocode}
\newcommand\snumbgcolor{SongbookShade}
\newcommand\notebgcolor{SongbookShade}
\newcommand\idxbgcolor{SongbookShade}
%    \end{macrocode}
% \end{macro}
% \end{macro}
% \end{macro}
%
% \begin{macro}{\versejustify}\MainImpl{versejustify}
% \begin{macro}{\chorusjustify}\MainImpl{chorusjustify}
% \changes{v2.1}{2007/08/02}{Added}
% Verses and choruses are both left-justified with hanging indentation equal
% to |\parindent|.
%    \begin{macrocode}
\newcommand\versejustify{\justifyleft}
\newcommand\chorusjustify{\justifyleft}
%    \end{macrocode}
% \end{macro}
% \end{macro}
%
% \begin{macro}{\notejustify}\MainImpl{notejustify}
% \changes{v2.1}{2007/08/02}{Added}
% Textual notes are fully justified when they are too long to fit in
% a single line.
%    \begin{macrocode}
\newcommand\notejustify{%
  \advance\baselineskip\p@\relax%
  \leftskip\z@skip\rightskip\z@skip%
  \parfillskip\@flushglue\parindent\z@%
}
%    \end{macrocode}
% \end{macro}
%
% \begin{macro}{\placenote}\MainImpl{placenote}
% \changes{v2.1}{2007/08/02}{Added}
% Textual notes are placed flush-left.
% The single argument to this macro is horizontal material that comprises the
% note.
% Usually it will consist of various hboxes and specials that were produced
% by |\colorbox|.
%    \begin{macrocode}
\newcommand\placenote[1]{%
  \leftskip\z@skip\rightskip\@flushglue\SB@cbarshift%
  \noindent#1\par%
}
%    \end{macrocode}
% \end{macro}
%
% These counters define the current song number and verse number.
% They can be redefined by the user at any time.
%    \begin{macrocode}
\newcounter{songnum}
\newcounter{versenum}
%    \end{macrocode}
%
% \begin{macro}{\thesongnum}\MainImpl{thesongnum}
% \begin{macro}{\songnumstyle}
% By default, the song numbering style will simply be an arabic number.
% Redefine |\thesongnum| to change it.
% (The |\songnumstyle| macro is obsolete and exists only for backward
% compatibility.)
%    \begin{macrocode}
\renewcommand\thesongnum{\songnumstyle{songnum}}
\newcommand\songnumstyle{}
\let\songnumstyle\arabic
%    \end{macrocode}
% \end{macro}
% \end{macro}
%
% \begin{macro}{\theversenum}\MainImpl{theversenum}
% \begin{macro}{\versenumstyle}
% By default, the verse numbering style will simply be an arabic number.
% Redefine |\theversenum| to change it.
% (The |\versenumstyle| macro is obsolete and exists only for backward
% compatibility.)
%    \begin{macrocode}
\renewcommand\theversenum{\versenumstyle{versenum}}
\newcommand\versenumstyle{}
\let\versenumstyle\arabic
%    \end{macrocode}
% \end{macro}
% \end{macro}
%
% \begin{macro}{\printsongnum}\MainImpl{printsongnum}
% Define the printing style for the large, boxed song numbers starting each
% song.
%    \begin{macrocode}
\newcommand\printsongnum[1]{\sffamily\bfseries\LARGE#1}
%    \end{macrocode}
% \end{macro}
%
% \begin{macro}{\printversenum}\MainImpl{printversenum}
% Define the printing style for the verse numbers to the left of each verse.
%    \begin{macrocode}
\newcommand\printversenum[1]{\lyricfont#1.\ }
%    \end{macrocode}
% \end{macro}
%
% \begin{macro}{\placeversenum}\MainImpl{placeversenum}
% \changes{v2.1}{2007/08/02}{Added}
% Verse numbers are placed flush-left.
% This is achieved by inserting horizontal glue that reverses both the
% |\leftskip| and the |\parindent|.
% The single argument to this macro is an hbox containing the verse number.
%    \begin{macrocode}
\newcommand\placeversenum[1]{%
  \hskip-\leftskip\hskip-\parindent\relax%
  \box#1%
}
%    \end{macrocode}
% \end{macro}
%
% \begin{macro}{\everyverse}\MainImpl{everyverse}
% \begin{macro}{\everychorus}\MainImpl{everychorus}
% \changes{v2.1}{2007/08/02}{Added}
% The following hooks allow users to insert material at the head of each
% verse or chorus.
%    \begin{macrocode}
\newcommand\everyverse{}
\newcommand\everychorus{}
%    \end{macrocode}
% \end{macro}
% \end{macro}
%
% \begin{macro}{\printchord}\MainImpl{printchord}
% Define the printing style for chords.
%    \begin{macrocode}
\newcommand\printchord[1]{\sffamily\slshape\large#1}
%    \end{macrocode}
% \end{macro}
%
% \begin{macro}{\chordlocals}\MainImpl{chordlocals}
% \label{sec:chordlocals}
% This hook is expanded at the start of the scoping group that surrounds
% every chord name.
% Thus, it can be used to set any catcodes or definitions that should be
% local to chord names.
%    \begin{macrocode}
\newcommand\chordlocals{}
%    \end{macrocode}
% \end{macro}
%
% \begin{macro}{\versesep}\MainImpl{versesep}
% Specify the vertical distance between song verses.
% This gets set to a sentinel value by default; if the user doesn't redefine
% it by the end of the document preamble, it gets redefined to something
% sensible based on other settings.
%    \begin{macrocode}
\newlength\versesep
\versesep123456789sp\relax
%    \end{macrocode}
% \end{macro}
%
% \begin{macro}{\afterpreludeskip}\MainImpl{afterpreludeskip}
% \begin{macro}{\beforepostludeskip}\MainImpl{beforepostludeskip}
% Users can specify the amount of vertical space that separates song prelude
% and postlude material from the body of the song by adjusting the following
% two macros.
%    \begin{macrocode}
\newlength\afterpreludeskip
\afterpreludeskip=2\p@\@plus4\p@
\newlength\beforepostludeskip
\beforepostludeskip=2\p@\@plus4\p@
%    \end{macrocode}
% \end{macro}
% \end{macro}
%
% \begin{macro}{\baselineadj}\MainImpl{baselineadj}
% Define an adjustment factor for the vertical distance between consecutive
% lyric baselines.
% Setting this to zero accepts the default baseline distance computed by the
% songs package.
%    \begin{macrocode}
\newlength\baselineadj
\baselineadj\z@skip
%    \end{macrocode}
% \end{macro}
%
% \begin{macro}{\clineparams}\MainImpl{clineparams}
% The spacing between chords and the lyrics below them can be adjusted
% by changing the values of |\baselineskip|, |\lineskiplimit|, and
% |\lineskip| within the following macro.
% By default, |\baselineskip| is set to 2 points smaller than the height
% of the current (lyric) font, and |\lineskiplimit| and |\lineskip| are
% set so that chords intrude at most 2 points into the lyric below them.
% This helps to keep chords tight with lyrics.
%    \begin{macrocode}
\newcommand\clineparams{%
  \baselineskip\f@size\p@%
  \advance\baselineskip-2\p@%
  \lineskiplimit-2\p@%
  \lineskip-2\p@%
}
%    \end{macrocode}
% \end{macro}
%
% \begin{macro}{\parindent}
% The |\parindent| length controls how far broken lyric lines are
% indented from the left margin.
%    \begin{macrocode}
\parindent.25in
%    \end{macrocode}
% \end{macro}
%
% \begin{macro}{\idxheadwidth}\MainImpl{idxheadwidth}
% Specify the width of the head-boxes in a large index.
%    \begin{macrocode}
\newlength\idxheadwidth
\setlength\idxheadwidth{1.5cm}
%    \end{macrocode}
% \end{macro}
%
% \begin{macro}{\songnumwidth}\MainImpl{songnumwidth}
% Set the width of the song number boxes that begin each song.
% We guess a suitable width by typesetting the text ``999.''
%    \begin{macrocode}
\newlength\songnumwidth
\settowidth\songnumwidth{\printsongnum{999.}}
%    \end{macrocode}
% \end{macro}
%
% \begin{macro}{\versenumwidth}\MainImpl{versenumwidth}
% Set the width that is reserved for normal-sized verse numbers.
% (Verse numbers wider than this will indent the first line of lyrics.)
%    \begin{macrocode}
\newlength\versenumwidth
\settowidth\versenumwidth{\printversenum{9\kern1em}}
%    \end{macrocode}
% \end{macro}
%
% \begin{macro}{\cbarwidth}
% This dictates the width of the vertical line placed to the left of
% choruses.
% Setting it to |0pt| eliminates the line entirely.
%    \begin{macrocode}
\newlength\cbarwidth
\setlength\cbarwidth\p@
%    \end{macrocode}
% \end{macro}
%
% \begin{macro}{\sbarheight}
% This dictates the height of the horizontal line placed between each pair
% of songs.
% Setting it to |0pt| eliminates the line entirely.
%    \begin{macrocode}
\newlength\sbarheight
\setlength\sbarheight\p@
%    \end{macrocode}
% \end{macro}
%
% Column- and page-breaks should typically not occur within a verse or chorus
% unless they are unavoidable.
% Thus, we set the |\interlinepenalty| to a high number (1000).
%    \begin{macrocode}
\interlinepenalty\@m
%    \end{macrocode}
%
% \begin{macro}{\vvpenalty}\MainImpl{vvpenalty}
% \begin{macro}{\ccpenalty}\MainImpl{ccpenalty}
% \begin{macro}{\vcpenalty}\MainImpl{vcpenalty}
% \begin{macro}{\cvpenalty}\MainImpl{cvpenalty}
% \begin{macro}{\brkpenalty}\MainImpl{brkpenalty}
% \changes{v2.1}{2007/08/02}{Added.}
% The following count registers define the line-breaking penalties inserted
% between verses, between choruses, after a verse followed by a chorus, after
% a chorus followed by a verse, and at |\brk| macros, respectively.
%
% The default value of 200 was chosen based on the following logic:
% Chord books should not yield underfull vbox warnings no matter how short
% their columns are.
% However, we still want to put as much material in each column as possible
% while avoiding intra-song column-breaks when they can be avoided.
% Chorded mode therefore sets |\colbotglue| with glue whose stretchability
% is half of the |\textheight|.
% Such glue will stretch at most twice its stretchability, yielding a
% badness of 800 in the worst case.
% The default |\vbadness| setting starts issuing warnings at badness 1000,
% so we set the penalties below to $1000-800=200$.
%    \begin{macrocode}
\SB@newcount\vvpenalty\vvpenalty200
\SB@newcount\ccpenalty\ccpenalty200
\SB@newcount\vcpenalty\vcpenalty200
\SB@newcount\cvpenalty\cvpenalty200
\SB@newcount\brkpenalty\brkpenalty200
%    \end{macrocode}
% \end{macro}
% \end{macro}
% \end{macro}
% \end{macro}
% \end{macro}
%
% \begin{macro}{\spenalty}\MainImpl{spenalty}
% \changes{v2.1}{2007/08/02}{Added.}
% The following penalty gets inserted between songs.
% Setting it to a proper value is a somewhat delicate balancing act.
% It should typically be something between 0 and the default penalties above,
% so for now it defaults to 100.
% To start each song on a fresh column/page, set it to $-10000$ or below.
%    \begin{macrocode}
\SB@newcount\spenalty\spenalty100
%    \end{macrocode}
% \end{macro}
%
% \begin{macro}{\songmark}\MainImpl{songmark}
% \changes{v1.17}{2005/09/24}{Added.}
% \begin{macro}{\versemark}\MainImpl{versemark}
% \begin{macro}{\chorusmark}\MainImpl{chorusmark}
% \changes{v2.1}{2007/08/02}{Added.}
% The user can redefine the following macros to add \TeX{} marks for each
% song, each verse, or each chorus.
% Such marks are used by \LaTeX{} to define page headers and footers.
%    \begin{macrocode}
\newcommand\songmark{}
\newcommand\versemark{}
\newcommand\chorusmark{}
%    \end{macrocode}
% \end{macro}
% \end{macro}
% \end{macro}
%
% \begin{macro}{\extendprelude}\MainImpl{extendprelude}
% \begin{macro}{\extendpostlude}\MainImpl{extendpostlude}
% \changes{v2.0}{2007/06/18}{Added.}
% To just add some fields to the existing |\makeprelude| or |\makepostlude|
% without having to redefine them entirely, users can redefine
% |\extendprelude| or |\extendpostlude|.
% By default, the prelude has the scripture references followed by the
% authors, and the postlude has the copyright info followed by the licensing
% info.
%    \begin{macrocode}
\newcommand\extendprelude{\showrefs\showauthors}
\newcommand\extendpostlude{\songcopyright\ \songlicense\unskip}
%    \end{macrocode}
% \end{macro}
% \end{macro}
%
% \begin{macro}{\idxheadfont}\MainImpl{idxheadfont}
% \changes{v2.8}{2009/02/22}{Added.}
% Users can redefine |\idxheadfont| to affect the font in which each capital
% letter that heads a section of a title index is rendered.
%    \begin{macrocode}
\newcommand\idxheadfont{\sffamily\bfseries\LARGE}
%    \end{macrocode}
% \end{macro}
%
% \begin{macro}{\idxtitlefont}\MainImpl{idxtitlefont}
% \changes{v2.8}{2009/02/22}{Added.}
% Users can redefine |\idxtitlefont| to affect the font in which song title
% index entries are rendered.
%    \begin{macrocode}
\newcommand\idxtitlefont{\sffamily\slshape}
%    \end{macrocode}
% \end{macro}
%
% \begin{macro}{\idxlyricfont}\MainImpl{idxlyricfont}
% \changes{v2.8}{2009/02/22}{Added.}
% Users can redefine |\idxlyricfont| to affect the font in which notable lines
% of lyrics are rendered in a title index.
%    \begin{macrocode}
\newcommand\idxlyricfont{\rmfamily}
%    \end{macrocode}
% \end{macro}
%
% \begin{macro}{\idxscripfont}\MainImpl{idxscripfont}
% \changes{v2.8}{2009/02/22}{Added.}
% Users can redefine |\idxscripfont| to affect the font in which scripture
% references are rendered in a scripture index.
%    \begin{macrocode}
\newcommand\idxscripfont{\sffamily\small\slshape}
%    \end{macrocode}
% \end{macro}
%
% \begin{macro}{\idxauthfont}\MainImpl{idxauthfont}
% \changes{v2.8}{2009/02/22}{Added.}
% Users can redefine |\idxauthfont| to affect the font in which contributor
% names are rendered in an author index.
%    \begin{macrocode}
\newcommand\idxauthfont{\small\bfseries}
%    \end{macrocode}
% \end{macro}
%
% \begin{macro}{\idxrefsfont}\MainImpl{idxrefsfont}
% \changes{v2.8}{2009/02/22}{Added.}
% Users can redefine |\idxrefsfont| to affect the font in which the list of
% song references on the right-hand-side of an index entry is typeset.
%    \begin{macrocode}
\newcommand\idxrefsfont{\normalfont\normalsize}
%    \end{macrocode}
% \end{macro}
%
% \begin{macro}{\idxbook}\MainImpl{idxbook}
% \changes{v2.8}{2009/02/22}{Added.}
% Users can redefine |\idxbook| to dictate the book name header in a
% scripture index that begins each book of the bible.
%    \begin{macrocode}
\newcommand\idxbook[1]{\small\bfseries#1}
%    \end{macrocode}
% \end{macro}
%
% \begin{macro}{\idxcont}\MainImpl{idxcont}
% \changes{v2.0}{2007/06/18}{Added.}
% Users can redefine |\idxcont| to dictate the column header in a scripture
% index after a column break falls within a book of the bible.
%    \begin{macrocode}
\newcommand\idxcont[1]{\small\textbf{#1} (continued)}
%    \end{macrocode}
% \end{macro}
%
% \begin{macro}{\colbotglue}
% Glue of size |\colbotglue| is inserted at the bottom of each column.
% We use a macro instead of a glue register so that this can be redefined
% in terms of variable quantities such as |\textheight|.
%    \begin{macrocode}
\newcommand\colbotglue{}
\let\colbotglue\z@skip
%    \end{macrocode}
% \end{macro}
%
% \begin{macro}{\lastcolglue}
% Glue of size |\lastcolglue| is inserted at the bottom of the last column.
%    \begin{macrocode}
\newcommand\lastcolglue{}
\let\lastcolglue\@flushglue
%    \end{macrocode}
% \end{macro}
%
% \begin{macro}{\minfrets}
% Define the minimum number of fret rows that should appear in tablature
% diagrams.
%    \begin{macrocode}
\SB@newcount\minfrets\minfrets4
%    \end{macrocode}
% \end{macro}
%
% \begin{macro}{\SB@colwidth}
% Define a length to store the computed width of each column in a
% multi-column song page.
% The user shouldn't set this one directly, but some users might want to
% refer to it in calculations.
%    \begin{macrocode}
\SB@newdimen\SB@colwidth
%    \end{macrocode}
% \end{macro}
%
% \subsection{Package Options}
%
% This section defines code associated with the various option
% settings that can be specified on the |\usepackage| line.
% Many of these options can also be turned on or off subsequent to the
% |\usepackage| line, so macros for doing that are also located here.
% The options are not actually processed until \S\ref{sec:optproc} because
% some of the macros defined here refer to macros that have not yet been
% defined.
%
% \begin{option}{slides}\MainEnvImpl{slides}
% \begin{macro}{\slides}\MainImpl{slides}
% \optdef{off}
% Turning this option on generates a book of overhead slides---one for each
% song.
% It really just amounts to changing various parameter settings.
% Elsewhere in the code we also consult |\ifslides| to determine a few default
% parameter settings and to use a different song preamble structure.
% All the parameter changes below are local to the current scope; so to
% undo slides mode, just put |\slides| within a group and end the group
% wherever you want the slides settings to end.
%    \begin{macrocode}
\DeclareOption{slides}{\slides}
\newcommand\slides{%
  \slidestrue%
  \def\lyricfont{\normalfont\huge}%
  \def\chorusfont{\slshape}%
  \def\versejustify{\justifycenter}%
  \let\chorusjustify\versejustify
  \def\placenote##1{\justifycenter\noindent##1\par}%
  \scriptureoff%
  \onesongcolumn%
  \ifSB@preamble\ifSB@chordedspec\else\SB@chordsoff\fi\fi%
  \spenalty-\@M%
  \let\colbotglue\@flushglue%
  \setlength\cbarwidth\z@%
  \setlength\sbarheight\z@%
}
%    \end{macrocode}
% \end{macro}
% \end{option}
%
% \begin{macro}{\justifyleft}
% \changes{v2.1}{2007/08/02}{Added}
% The |\justifyleft| macro sets up an environment in which lyrics are
% left-justified with hanging indentation equal to |\parindent|.
% It reserves spaces for verse numbers if used in a verse, and reserves
% space for the vertical bar left of choruses if used in a chorus.
%    \begin{macrocode}
\newcommand\justifyleft{%
  \leftskip\parindent%
  \ifSB@inverse\advance\leftskip\versenumwidth\fi%
  \SB@cbarshift%
  \parindent-\parindent%
}
%    \end{macrocode}
% \end{macro}
%
% \begin{macro}{\justifycenter}
% \changes{v2.1}{2007/08/02}{Added}
% The |\justifycenter| macro sets up an environment in which lyrics are
% centered on each line.
% Verse numbers continue to be placed flush-left, but |\placeversenum|
% is temporarily redefined to keep the rest of the line containing a
% verse number centered.
%    \begin{macrocode}
\newcommand\justifycenter{%
  \centering\SB@cbarshift\rightskip\leftskip%
  \def\placeversenum##1{%
    \hskip-\leftskip\hskip-\parindent\relax%
    \hangindent-\wd##1\hangafter\m@ne%
    \box##1\hfil%
  }%
}
%    \end{macrocode}
% \end{macro}
%
% \begin{option}{unouter}\MainEnvImpl{unouter}
% \begin{macro}{\SB@outer}
% \optdef{off}
% Several macros provided by the \Songs{} package are, by default, declared
% |\outer| to aid in debugging.
% However, unusual documents may need to use these macros within larger
% constructs.
% To do so, use the |unouter| option to prevent any of the macros supplied
% by this package from being declared |\outer|.
%    \begin{macrocode}
\newcommand\SB@outer{\outer}
\DeclareOption{unouter}{\let\SB@outer\relax}
%    \end{macrocode}
% \end{macro}
% \end{option}
%
% \begin{option}{rawtext}\MainEnvImpl{rawtext}
% \optdef{off}
% Instead of generating a document, this dumps a text version of the song book
% to a file. This option can only be set in the |\usepackage| line because
% it dictates many top-level macro definitions. Turning rawtext on turns off
% the indexes by default, but this can be overridden by explicitly setting
% index options. (Note: Using rawtext with indexes turned on doesn't actually
% work yet, but might be added in a future revision.)
%    \begin{macrocode}
\DeclareOption{rawtext}{\rawtexttrue\indexesoff}
%    \end{macrocode}
% \end{option}
%
% \begin{option}{noshading}\MainEnvImpl{noshading}
% \optdef{off}
% Inhibit all shaded boxes (e.g., if the color package is unavailable).
% This option can only be set in the |\usepackage| line because the color
% package must be loaded in the preamble if at all. (Note: In a future release
% this might be extended to be modifiable throughout the preamble.)
%    \begin{macrocode}
\DeclareOption{noshading}{\SB@colorboxesfalse}
%    \end{macrocode}
% \end{option}
%
% \begin{option}{noindexes}\MainEnvImpl{noindexes}
% \begin{macro}{\indexeson}\MainImpl{indexeson}
% \begin{macro}{\indexesoff}\MainImpl{indexesoff}
% \optdef{off}
% Suppress generation of index files and displaying of in-document indexes.
% The |\indexeson| and |\indexesoff| macros can be used elsewhere to toggle
% display of indexes.
% Index-regeneration will occur if indexes are turned on by the end of the
% document.
%    \begin{macrocode}
\DeclareOption{noindexes}{\indexesoff}
\newcommand\indexeson{\songindexestrue}
\newcommand\indexesoff{\songindexesfalse}
%    \end{macrocode}
% \end{macro}
% \end{macro}
% \end{option}
%
% \begin{option}{nopdfindex}\MainEnvImpl{nopdfindex}
% \optdef{off}
% Suppress creation of PDF bookmark entries and hyperlinks.
%    \begin{macrocode}
\DeclareOption{nopdfindex}{%
  \let\songtarget\@gobbletwo%
  \let\songlink\@secondoftwo%
}
%    \end{macrocode}
% \end{option}
%
% \begin{macro}{\ifSB@measurespec}
% \begin{macro}{\ifSB@chordedspec}
% The |showmeasures| and |chorded| options interact in the sense that by
% default, switching one of them on or off switches the other on or off as
% well.
% However, if the user explicitly says that one should be on or off, then
% switching the other shouldn't affect it.
% To produce this behavior, we need two extra conditionals to remember whether
% each of these options has been explicitly specified by the user or whether
% it is still in a default state.
%    \begin{macrocode}
\newif\ifSB@measurespec
\newif\ifSB@chordedspec
%    \end{macrocode}
% \end{macro}
% \end{macro}
%
% \begin{option}{chorded}\MainEnvImpl{chorded}
% \begin{option}{lyric}\MainEnvImpl{lyric}
% \begin{macro}{\chordson}\MainImpl{chordson}
% \begin{macro}{\chordsoff}\MainImpl{chordsoff}
% \begin{macro}{\SB@chordson}
% \begin{macro}{\SB@chordsoff}
% \changes{v1.22}{2007/05/15}{Update \cs{baselineskip} when in songs.}
% \optdef{chorded}
% Determines whether chords should be shown.
% This option can be set in the |\usepackage| line or toggled elsewhere
% with the |\chordson| and |\chordsoff| macros.
% Chords cannot be turned on in conjunction with the |rawtext| option.
% If chords are turned on by the end of the preamble, no attempt will be made
% to balance columns on each page.
%    \begin{macrocode}
\DeclareOption{chorded}{\chordson}
\DeclareOption{lyric}{\chordsoff}
\newcommand\chordson{\SB@chordedspectrue\SB@chordson}
\newcommand\chordsoff{\SB@chordedspectrue\SB@chordsoff}
\newcommand\SB@chordson{%
  \ifrawtext%
    \SB@errrtopt%
  \else%
    \chordedtrue\lyricfalse%
    \let\SB@bracket\SB@chord%
    \let\SB@rechord\SB@@rechord%
    \let\SB@ch\SB@ch@on%
    \ifSB@measurespec%
      \ifmeasures\SB@measureson\else\SB@measuresoff\fi%
    \else%
      \SB@measureson%
    \fi%
    \ifSB@preamble\def\colbotglue{\z@\@plus.5\textheight}\fi%
    \SB@setbaselineskip%
  \fi%
}
\newcommand\SB@chordsoff{%
  \chordedfalse\lyrictrue%
  \def\SB@bracket##1]{\ignorespaces}%
  \let\SB@rechord\relax%
  \let\SB@ch\SB@ch@off%
  \ifSB@measurespec%
    \ifmeasures\SB@measureson\else\SB@measuresoff\fi%
  \else%
    \SB@measuresoff%
  \fi%
  \ifSB@preamble\let\colbotglue\z@skip\fi%
  \SB@setbaselineskip%
}
%    \end{macrocode}
% \eat\]
% \end{macro}
% \end{macro}
% \end{macro}
% \end{macro}
% \end{option}
% \end{option}
%
% \begin{option}{showmeasures}\MainEnvImpl{showmeasures}
% \begin{option}{nomeasures}\MainEnvImpl{nomeasures}
% \begin{macro}{\measureson}\MainImpl{measureson}
% \begin{macro}{\measuresoff}\MainImpl{measuresoff}
% \begin{macro}{\SB@measureson}
% \begin{macro}{\SB@measuresoff}
% \optdef{showmeasures if chorded, nomeasures otherwise}
% Determines whether measure bars and meter notes should be shown.
% Option can be set in the |\usepackage| line or toggled elsewhere with the
% |\measureson| and |\measuresoff| macros.
%    \begin{macrocode}
\DeclareOption{showmeasures}{\measureson}
\DeclareOption{nomeasures}{\measuresoff}
\newcommand\measureson{\SB@measurespectrue\SB@measureson}
\newcommand\measuresoff{\SB@measurespectrue\SB@measuresoff}
\newcommand\SB@measureson{%
  \measurestrue%
  \let\SB@mbar\SB@makembar%
  \ifchorded%
    \let\SB@mch\SB@mch@on%
  \else%
    \let\SB@mch\SB@mch@m%
  \fi%
  \ifSB@inverse\SB@loadactives\fi%
  \ifSB@inchorus\SB@loadactives\fi%
}
\newcommand\SB@measuresoff{%
  \measuresfalse%
  \let\SB@mbar\@gobbletwo%
  \ifchorded%
    \let\SB@mch\SB@ch@on%
  \else%
    \let\SB@mch\SB@ch@off%
  \fi%
  \ifSB@inverse\SB@loadactives\fi%
  \ifSB@inchorus\SB@loadactives\fi%
}
%    \end{macrocode}
% \end{macro}
% \end{macro}
% \end{macro}
% \end{macro}
% \end{option}
% \end{option}
%
% \begin{option}{transposecapos}\MainEnvImpl{transposecapos}
% \optdef{off}
% If set, the |\capo| macro transposes the song instead of printing a note
% to use a capo. Use this option to generate a chord book for pianists who
% have trouble transposing or guitarists who don't have capos.
%    \begin{macrocode}
\DeclareOption{transposecapos}{\transcapostrue}
%    \end{macrocode}
% \end{option}
%
% \begin{option}{noscripture}\MainEnvImpl{noscripture}
% \begin{macro}{\scriptureon}\MainImpl{scriptureon}
% \begin{macro}{\scriptureoff}\MainImpl{scriptureoff}
% \optdef{off}
% Inhibits the display of scripture quotes.
% This option can also be toggled on and off anywhere with the |\sciptureon|
% and |\scriptureoff| macros.
%    \begin{macrocode}
\DeclareOption{noscripture}{\SB@omitscriptrue}
\newcommand\scriptureon{\SB@omitscripfalse}
\newcommand\scriptureoff{\SB@omitscriptrue}
%    \end{macrocode}
% \end{macro}
% \end{macro}
% \end{option}
%
% \begin{option}{onesongcolumn}\MainEnvImpl{onesongcolumn}
% \begin{option}{twosongcolumns}\MainEnvImpl{twosongcolumns}
% \begin{macro}{\onesongcolumn}\MainImpl{onesongcolumn}
% \begin{macro}{\twosongcolumns}\MainImpl{twosongcolumns}
% \begin{macro}{\songcolumns}\MainImpl{songcolumns}
% \optdef{onesongcolumn is the default if generating slides or rawtext, twosongcolumns otherwise}
% The number of columns per page is specified using the following package
% options and macros.
% In \env{rawtext} mode it must remain set to one column per page.
% The entire page-making system can be turned off by setting the number of
% columns to zero.
% This will cause each song to be contributed to the current vertical list
% without any attempt to form columns; the enclosing environment must handle
% the page layout.
% Probably this means that |\repchoruses| will not work, since an external
% package won't know to insert repeated choruses when building pages.
%    \begin{macrocode}
\DeclareOption{twosongcolumns}{\SB@numcols\tw@}
\DeclareOption{onesongcolumn}{\SB@numcols\@ne}
\newcommand\songcolumns[1]{%
  \SB@cnt#1\relax%
  \ifnum\SB@cnt=\SB@numcols\else%
    \ifSB@preamble\else{\SB@clearpage}\fi%
  \fi%
  \SB@numcols\SB@cnt%
  \ifnum\SB@numcols>\z@%
    \SB@colwidth-\columnsep%
    \multiply\SB@colwidth\SB@numcols%
    \advance\SB@colwidth\columnsep%
    \advance\SB@colwidth\textwidth%
    \divide\SB@colwidth\SB@numcols%
  \else%
    \ifrepchorus\SB@warnrc\fi%
  \fi%
}
\newcommand\onesongcolumn{\songcolumns\@ne}
\newcommand\twosongcolumns{\songcolumns\tw@}
%    \end{macrocode}
% \end{macro}
% \end{macro}
% \end{macro}
% \end{option}
% \end{option}
%
% \begin{macro}{\includeonlysongs}\MainImpl{includeonlysongs}
% \begin{macro}{\songlist}
% Display only a select list of songs and ignore the rest.
%    \begin{macrocode}
\newcommand\songlist{}
\newcommand\includeonlysongs[1]{%
  \ifSB@songsenv\SB@errpl\else%
    \partiallisttrue%
    \renewcommand\songlist{#1}%
  \fi%
}
%    \end{macrocode}
% \end{macro}
% \end{macro}
%
% \begin{macro}{\nosongnumbers}\MainImpl{nosongnumbers}
% \changes{v2.9}{2009/04/01}{Added.}
% The user can turn off song numbering with the following macro.
%    \begin{macrocode}
\newcommand\nosongnumbers{\setlength\songnumwidth\z@}
%    \end{macrocode}
% \end{macro}
%
% \begin{macro}{\noversenumbers}\MainImpl{noversenumbers}
% \changes{v1.20}{2006/03/12}{Added.}
% The user can turn off verse numbering with the following macro.
%    \begin{macrocode}
\newcommand\noversenumbers{%
  \renewcommand\printversenum[1]{}%
  \setlength\versenumwidth\z@%
}
%    \end{macrocode}
% \end{macro}
%
% \begin{macro}{\repchoruses}\MainImpl{repchoruses}
% \begin{macro}{\norepchoruses}\MainImpl{norepchoruses}
% \changes{v2.1}{2007/08/02}{Added.}
% Using |\repchoruses| causes choruses to be automatically repeated on
% subsequent pages of the song.
% The feature requires $\varepsilon$-\TeX{} because the supporting code needs
% an extended mark register class.
%    \begin{macrocode}
\ifSB@etex
  \newcommand\repchoruses{%
    \ifnum\SB@numcols<\@ne\SB@warnrc\fi%
    \repchorustrue%
  }
\else
  \newcommand\repchoruses{\SB@erretex}
\fi
\newcommand\norepchoruses{\repchorusfalse}
%    \end{macrocode}
% \end{macro}
% \end{macro}
%
% \begin{macro}{\sepverses}
% The following penalty settings cause verses and choruses to be separated
% onto different slides when in slides mode, except that consecutive choruses
% remain together when they fit.
%    \begin{macrocode}
\newcommand\sepverses{%
  \vvpenalty-\@M%
  \ccpenalty100 %
  \vcpenalty\vvpenalty%
  \cvpenalty\vvpenalty%
  \let\colbotglue\@flushglue%
}
%    \end{macrocode}
% \end{macro}
%
% Some option settings, margins, and other lengths are finalized at the end of
% the preamble.
% That code is below.
%
%    \begin{macrocode}
\AtBeginDocument{
%    \end{macrocode}
%
% If the user hasn't set the |\versesep|, set it to the default.
%    \begin{macrocode}
  \SB@setversesep
%    \end{macrocode}
%
% Initialize page layout algorithm.
%    \begin{macrocode}
  \songcolumns\SB@numcols
%    \end{macrocode}
%
% Macros used after this point occur outside the preamble.
%    \begin{macrocode}
  \SB@preamblefalse
}
%    \end{macrocode}
%
% \subsection{Page-builder}
% \label{sec:pagebuilder}
%
% The following macros handle the building of pages that contain songs.
% They compute where best to place each song (e.g., whether to place it in the
% current column or move to the next column or page).
% The output routines for generating a partial list of songs in a specified
% order also can be found here.
%
% \begin{macro}{\SB@songbox}
% The most recently processed song (or scripture quotation) is stored in this
% box.
%    \begin{macrocode}
\SB@newbox\SB@songbox
%    \end{macrocode}
% \end{macro}
%
% \begin{macro}{\SB@numcols}
% \begin{macro}{\SB@colnum}
% Reserve two count registers to hold the total number of columns and the
% current column number, respectively.
%    \begin{macrocode}
\SB@newcount\SB@numcols\SB@numcols\tw@
\SB@newcount\SB@colnum
%    \end{macrocode}
% \end{macro}
% \end{macro}
%
% \begin{macro}{\SB@colbox}
% Reserve a box register to hold the current column in progress.
%    \begin{macrocode}
\SB@newbox\SB@colbox
%    \end{macrocode}
% \end{macro}
%
% \begin{macro}{\SB@pgbox}
% Reserve a box register to hold the current page in progress.
%    \begin{macrocode}
\SB@newbox\SB@pgbox
%    \end{macrocode}
% \end{macro}
%
% \begin{macro}{\SB@mrkbox}
% Reserve a box register to hold marks that migrate out of songs as they
% get split into columns and pages.
%    \begin{macrocode}
\SB@newbox\SB@mrkbox
%    \end{macrocode}
% \end{macro}
%
% \begin{macro}{\SB@maxmin}
% The following helper macro takes the max or min of two dimensions.
% If \argp{2}=``|<|'', it sets \argp{1} to the maximum of \argp{1} and
% \argp{3}.
% If \argp{2}=``|>|'', it sets \argp{1} to the minimum of \argp{1} and
% \argp{3}.
%    \begin{macrocode}
\newcommand\SB@maxmin[3]{\ifdim#1#2#3#1#3\fi}
%    \end{macrocode}
% \end{macro}
%
% \begin{macro}{\SB@mkpage}
% The following macro is the heart of the page-building engine.
% It splits the contents of a box into a page of columns.
% If |\repchoruses| is active, the contents of |\SB@chorusbox|
% are additionally inserted into fresh columns created during the spitting
% process.
% The macro arguments are:
% \begin{enumerate}
% \item an integer (positive or zero) indicating whether box $b$ should be
% fully emptied and committed as columns (if positive), or whether its
% final less-than-column-height remainder should be reserved as an in-progress
% column (if zero);
% \item the box $b$ to split;
% \item a count register $i$ equaling the column index (zero or greater)
% where the content of $b$ is to begin; and
% \item the desired column height.
% \end{enumerate}
% Box $b$ is split and $i$ is incremented until $i$ reaches
% |\SB@numcols| or $b$ is emptied.
% If $b$ is emptied and the first argument is 0, the final column is \emph{not}
% contributed; instead it is left in $b$ and $i$ is left equal to the index
% of the column that would have been added if $b$ had been emptied.
% This allows the next call to reconsider whether to end the
% current column here or add some or all of the next contribution to it.
% Otherwise, if $b$ is emptied and the first argument is positive, the final
% column is contributed and $i$ is set to one greater than the index of that
% column.
% (If $i$ reaches |\SB@numcols| before $b$ is emptied, the first argument is
% ignored.)
%
% Box $b$ and count register $i$ are globally modified.
% If |\SB@updatepage| is not redefined, boxes |\SB@pgbox| and |\SB@mrkbox|
% are also globally modified based on the results of the split.
%
% The implementation takes two special steps to avoid pre-committing
% in-progress columns (when the first macro argument is zero):
% First, the final split that empties box $b$ is ``undone'' by reverting to a
% backup copy made before each split.
% Second, any underfull box warnings for this final split are suppressed by
% temporarily adding infinite-stretch |\vfil| glue to the bottom of the box.
% This strategy preserves underfull and overfull box warnings for the columns
% that are actually committed, but suppresses faux warnings for the last split
% that is undone.
%    \begin{macrocode}
\newcommand\SB@mkpage[4]{%
  \ifvoid#2\else\begingroup%
    \edef\SB@temp{\ifnum#2=\SB@box\SB@boxii\else\SB@box\fi}%
    \edef\SB@tempii{\ifnum#2=\SB@boxiii\SB@boxii\else\SB@boxiii\fi}%
    \splitmaxdepth\maxdepth\splittopskip\z@skip%
    \ifnum#1=\z@\global\setbox#2\vbox{\unvbox#2\vfil}\fi%
    \loop\ifnum#3<\SB@numcols%
      \ifnum#1=\z@\setbox\SB@tempii\copy#2\fi%
      \setbox\SB@temp\vsplit#2to#4\relax%
      \ifvoid#2%
        \ifnum#1=\z@%
          \global\setbox#2\box\SB@tempii%
        \else%
          \SB@updatepage%
          \global\advance#3\@ne%
        \fi%
        #3\SB@numcols%
      \else%
        \SB@updatepage%
        \global\advance#3\@ne%
        \ifrepchorus\ifvoid\SB@chorusbox\else%
          \SB@insertchorus#2%
        \fi\fi%
      \fi%
    \repeat%
    \ifnum#1=\z@\global\setbox#2\vbox{\unvbox#2\unskip}\fi%
  \endgroup\fi%
}
%    \end{macrocode}
% \end{macro}
%
% \begin{macro}{\SB@migrate}
% Migrate a mark out of a recently split vertical list, but do not insert
% superfluous empty marks that may override previous marks.
%    \begin{macrocode}
\newcommand\SB@migrate[1]{%
  \SB@toks\expandafter{#1}%
  \edef\SB@temp{\the\SB@toks}%
  \ifx\SB@temp\@empty\else\mark{\the\SB@toks}\fi%
}
%    \end{macrocode}
% \end{macro}
%
% \begin{macro}{\SB@updatepage}
% Update boxes |\SB@pgbox| and |\SB@mrkbox| immediately after splitting
% the contents of |\SB@colbox|.
%    \begin{macrocode}
\newcommand\SB@updatepage{%
  \global\setbox\SB@mrkbox\vbox{%
    \unvbox\SB@mrkbox%
    \SB@migrate\splitfirstmark%
    \SB@migrate\splitbotmark%
  }%
  \global\setbox\SB@pgbox\hbox{%
    \SB@dimen\SB@colwidth%
    \advance\SB@dimen\columnsep%
    \multiply\SB@dimen\SB@colnum%
    \advance\SB@dimen-\wd\SB@pgbox%
    \unhbox\SB@pgbox%
    \ifdim\SB@dimen=\z@\else\hskip\SB@dimen\relax\fi%
    \box\SB@temp%
  }%
}
%    \end{macrocode}
% \end{macro}
%
% \begin{macro}{\SB@droppage}
% This alternate definition of |\SB@updatepage| drops the just-created
% page instead of contributing it.
% This allows |\SB@mkpage| to be called by the song-positioning algorithm
% as a trial run without outputting anything.
%    \begin{macrocode}
\newcommand\SB@droppage{\setbox\SB@temp\box\voidb@x}
%    \end{macrocode}
% \end{macro}
%
% \begin{macro}{\SB@output}
% This is the main output routine for the page-builder.
% It repeatedly calls |\SB@mkpage|, emitting pages as they are completed,
% until the remaining content of box |\SB@colbox| is not enough to fill a
% column.
% If the macro argument is 0, this final, in-progress column is left
% unfinished, pending future contributions.
% If the argument is positive, the final material is committed as a column.
% If the argument is two or greater, the entire in-progress page is also
% committed and the column number reset.
%    \begin{macrocode}
\newcommand\SB@output[1]{%
  \ifnum\SB@numcols>\z@\begingroup%
    \loop%
      \SB@dimen\textheight%
      \ifinner\else\advance\SB@dimen-\pagetotal\fi%
      \SB@mkpage#1\SB@colbox\SB@colnum\SB@dimen%
      \SB@testfalse\SB@testiitrue%
      \ifnum#1>\@ne\ifvoid\SB@colbox\ifnum\SB@colnum>\z@%
        \SB@testtrue\SB@testiifalse%
      \fi\fi\fi%
      \ifnum\SB@colnum<\SB@numcols\SB@testiifalse\else\SB@testtrue\fi%
      \ifSB@test%
        \unvbox\SB@mrkbox%
        \ifinner\else\kern\z@\fi%
        \box\SB@pgbox%
        \ifinner\else\vfil\break\vskip\vsize\relax\fi%
        \global\SB@colnum\z@%
      \fi%
    \ifSB@testii\repeat%
  \endgroup\else%
    \unvbox\SB@colbox\unskip%
  \fi%
}
%    \end{macrocode}
% \end{macro}
%
% \begin{macro}{\SB@putboxes}
% Create a vertical list consisting of the already committed contents of the
% current column plus the most recently submitted song box.
% The \LaTeX{} primitive that should be used to contribute each box is
% specified in the first argument.
%    \begin{macrocode}
\newcommand\SB@putboxes[1]{%
  \SB@dimen\ifnum\SB@numcols>\z@\ht\SB@colbox\else\p@\fi%
  #1\SB@colbox%
  \ifdim\SB@dimen>\z@%
    \SB@breakpoint\spenalty%
    \ifdim\sbarheight>\z@%
      \vskip-\sbarheight\relax%
    \fi%
  \fi%
  #1\SB@songbox%
}
%    \end{macrocode}
% \end{macro}
%
% \begin{macro}{\SB@nextcol}
% Force $n$ column breaks, where $n$ is given by the first argument.
% The first created column is finished with the glue specified in the
% second argument.
% When the second argument is |\@flushglue|, this forces a break that leaves
% whitespace at the bottom of the column.
% When it's |\colbotglue|, it acts like a natural column break chosen by
% the page-breaker.
% However, if the current column is empty, |\@flushglue| is always used so
% that an empty column will result.
%    \begin{macrocode}
\newcommand\SB@nextcol[2]{%
  \ifnum#1>\z@%
    \ifnum\SB@numcols>\z@%
      \global\setbox\SB@colbox\vbox{%
        \SB@cnt#1\relax%
        \SB@dimen\ht\SB@colbox%
        \unvbox\SB@colbox%
        \unskip%
        \ifdim\SB@dimen>\z@%
          \vskip#2\relax%
          \break%
          \advance\SB@cnt\m@ne%
        \fi%
        \loop\ifnum\SB@cnt>\z@%
          \nointerlineskip%
          \null%
          \vfil%
          \break%
          \advance\SB@cnt\m@ne%
        \repeat%
      }%
      \SB@output1%
    \else%
      \ifnum\lastpenalty=-\@M\null\fi%
      \break%
    \fi%
  \fi%
}
%    \end{macrocode}
% \end{macro}
%
% \begin{macro}{\SB@selectcol}
% This is the entrypoint to the song-positioning algorithm.
% It gets defined by |\songpos| to either |\SB@@selectcol| (below) or
% |\relax| (when song-positioning is turned off).
%    \begin{macrocode}
\newcommand\SB@selectcol{}
%    \end{macrocode}
% \end{macro}
%
% \begin{macro}{\SB@@selectcol}
% \changes{v2.1}{2007/08/02}{Rewritten to better handle glue}
% \changes{v2.9}{2009/07/30}{Rewritten to handle repeated choruses}
% Songs should be squeezed in wherever they fit, but breaking a column or page
% within a song should be avoided.
% The following macro outputs zero or more column breaks to select a good
% place for |\SB@songbox| to be contributed to the current (or the next) page.
% The number of column breaks is determined by temporarily setting
% |\SB@updatepage| to |\SB@droppage| and then calling the |\SB@mkpage|
% algorithm under various conditions to see how many columns it would
% contribute if we start the current song at various positions.
%    \begin{macrocode}
\newcommand\SB@@selectcol{%
  \begingroup%
    \SB@cnt\z@%
    \vbadness\@M\vfuzz\maxdimen%
    \let\SB@updatepage\SB@droppage%
    \SB@dimen\textheight%
    \ifinner\else\advance\SB@dimen-\pagetotal\fi%
    \setbox\SB@boxii\vbox{\SB@putboxes\unvcopy}%
    \SB@cntii\SB@colnum%
    \SB@mkpage0\SB@boxii\SB@cntii\SB@dimen%
    \SB@spos%
    \global\SB@cnt\SB@cnt%
  \endgroup%
  \SB@nextcol\SB@cnt\colbotglue%
}
%    \end{macrocode}
% \end{macro}
%
% \begin{macro}{\SB@spbegnew}
% Begin a trial typesetting of the current song on a fresh page to see if
% it fits within a page.
%    \begin{macrocode}
\newcommand\SB@spbegnew{%
  \setbox\SB@boxiii\copy\SB@songbox%
  \SB@cntii\z@%
  \SB@mkpage0\SB@boxiii\SB@cntii\textheight%
}
%    \end{macrocode}
% \end{macro}
%
% \begin{macro}{\SB@spextold}
% Tentatively extend the song previously typeset on the current even page to
% the next odd page to see whether it fits on a double-page.
% If the current page is odd-numbered, do nothing since extending the song
% to the next page would introduce a page-turn.
%    \begin{macrocode}
\newcommand\SB@spextold{%
  \ifodd\c@page\else%
    \SB@cntii\z@%
    \SB@mkpage0\SB@boxii\SB@cntii\textheight%
  \fi%
}
%    \end{macrocode}
% \end{macro}
%
% \begin{macro}{\SB@spextnew}
% Extend the trial typesetting started with |\SB@spbegnew| to a second
% page to see whether the song fits on a fresh double-page.
%    \begin{macrocode}
\newcommand\SB@spextnew{%
  \SB@cntii\z@%
  \SB@mkpage0\SB@boxiii\SB@cntii\textheight%
}
%    \end{macrocode}
% \end{macro}
%
% \begin{macro}{\SB@spdblpg}
% Compute the number of column breaks required to shift the current song
% to the next double-page if the result of the last test run fits within
% its page (as indicated by counter |\SB@cntii|).
% Otherwise leave the requested number of column breaks set to zero.
%    \begin{macrocode}
\newcommand\SB@spdblpg{%
  \ifnum\SB@cntii<\SB@numcols%
    \SB@cnt\SB@numcols%
    \advance\SB@cnt-\SB@colnum%
    \if@twoside\ifodd\c@page\else%
      \advance\SB@cnt\SB@numcols%
    \fi\fi%
  \fi%
}
%    \end{macrocode}
% \end{macro}
%
% \begin{macro}{\SB@sposi}
% This is the level-1 song positioning algorithm.
% It moves songs to the next double-page only if doing so would avoid a
% page-turn that would otherwise appear within the song.
%    \begin{macrocode}
\newcommand\SB@sposi{%
  \ifnum\SB@cntii<\SB@numcols\else\if@twoside%
    \SB@spextold%
  \fi\fi%
  \ifnum\SB@cntii<\SB@numcols\else%
    \SB@spbegnew%
    \ifnum\SB@cntii<\SB@numcols\else\if@twoside%
      \SB@spextnew%
    \fi\fi%
    \SB@spdblpg%
  \fi%
}
%    \end{macrocode}
% \end{macro}
%
% \begin{macro}{\SB@sposii}
% This is the level-2 song-positioning algorithm.
% It moves songs to the next page or double-page if doing so avoids a
% page-break or page-turn that would otherwise appear within the song.
%    \begin{macrocode}
\newcommand\SB@sposii{%
  \ifnum\SB@cntii<\SB@numcols\else%
    \SB@spbegnew%
    \ifnum\SB@cntii<\SB@numcols%
      \SB@cnt\SB@numcols%
      \advance\SB@cnt-\SB@colnum%
    \else%
      \if@twoside%
        \SB@spextold%
        \ifnum\SB@cntii<\SB@numcols\else%
          \SB@spextnew%
          \SB@spdblpg%
        \fi%
      \fi%
    \fi%
  \fi%
}
%    \end{macrocode}
% \end{macro}
%
% \begin{macro}{\SB@sposiii}
% This is the level-3 song-positioning algorithm.
% It moves songs to the next column, the next page, or the next double-page
% if doing so avoids a column-break, page-break, or page-turn that would
% otherwise appear within the song.
%    \begin{macrocode}
\newcommand\SB@sposiii{%
  \ifnum\SB@cntii>\SB@colnum%
    \SB@cnt\SB@colnum%
    \advance\SB@cnt\@ne%
    \ifnum\SB@cnt<\SB@numcols%
      \setbox\SB@boxiii\copy\SB@songbox%
      \SB@mkpage0\SB@boxiii\SB@cnt\SB@dimen%
      \advance\SB@cnt\m@ne%
    \fi%
    \ifnum\SB@cnt>\SB@colnum%
      \SB@cnt\z@%
      \SB@sposii%
    \else%
      \SB@cnt\@ne%
    \fi%
  \fi%
}
%    \end{macrocode}
% \end{macro}
%
% \begin{macro}{\songpos}
% This is the macro by which the user adjusts the aggressiveness level of the
% song-positioning algorithm.
% See the macros above for what each level does.
%    \begin{macrocode}
\newcommand\songpos[1]{%
  \ifcase#1%
    \let\SB@selectcol\relax%
    \let\SB@spos\relax%
  \or%
    \let\SB@selectcol\SB@@selectcol%
    \let\SB@spos\SB@sposi%
  \or%
    \let\SB@selectcol\SB@@selectcol%
    \let\SB@spos\SB@sposii%
  \or%
    \let\SB@selectcol\SB@@selectcol%
    \let\SB@spos\SB@sposiii%
  \else%
    \SB@errspos%
  \fi%
}
%    \end{macrocode}
% \end{macro}
%
% \begin{macro}{\SB@spos}
% The |\SB@spos| macro gets redefined by |\songpos| above depending on the
% current song-positioning aggressiveness level.
% By default it is set to level 3.
%    \begin{macrocode}
\newcommand\SB@spos{}
\songpos\thr@@
%    \end{macrocode}
% \end{macro}
%
% \begin{macro}{\SB@clearpage}
% Output all contributed material as a new page unless there is no contributed
% material. In that case do nothing (i.e., don't produce a blank page).
% The |\SB@colbox| is tested for zero height and depth rather than voidness,
% since sometimes it contains zero-length |\splittopskip| glue.
%    \begin{macrocode}
\newcommand\SB@clearpage{%
  \SB@testtrue%
  \ifvoid\SB@pgbox%
    \ifdim\ht\SB@colbox=\z@\ifdim\dp\SB@colbox=\z@%
      \SB@testfalse%
    \fi\fi%
  \fi%
  \ifSB@test%
    \SB@cnt\SB@numcols%
    \advance\SB@cnt-\SB@colnum%
    \SB@nextcol\SB@cnt\lastcolglue%
    \SB@output2%
  \fi%
}
%    \end{macrocode}
% \end{macro}
%
% \begin{macro}{\SB@cleardpage}
% Like |\SB@clearpage| but shift to a fresh \emph{even-numbered} page in
% two-sided documents.
% Note that this differs from \LaTeX's |\cleardoublepage|, which shifts to
% odd-numbered pages.
% Song books prefer starting things on even-numbered pages because this
% maximizes the distance until the next page-turn.
%    \begin{macrocode}
\newcommand\SB@cleardpage{%
  \SB@clearpage%
  \if@twoside\ifodd\c@page%
    \SB@nextcol\SB@numcols\@flushglue%
    \SB@output2%
  \fi\fi%
}
%    \end{macrocode}
% \end{macro}
%
% \begin{macro}{\SB@stype}
% There are two song content submission types: column- and page-submissions.
% Page-submissions are page-width and go atop fresh pages unless the current
% page has only page-width material so far.
% Column-submissions are column-width and start a new page only when the
% current page is full.
% This macro gets set to the desired type for the current submission.
% Mostly it stays set to the default column-submission type.
%    \begin{macrocode}
\newcommand\SB@stype{\SB@stypcol}
%    \end{macrocode}
% \end{macro}
%
% \begin{macro}{\SB@stypcol}
% \changes{v2.1}{2007/08/02}{Rewritten to better handle glue}
% Column-submissions contribute the contents of |\SB@songbox| to either the
% current column or the next column or page, depending on where it best fits.
%    \begin{macrocode}
\newcommand\SB@stypcol{%
  \ifnum\SB@numcols>\z@%
    \SB@selectcol%
    \global\setbox\SB@colbox\vbox{\SB@putboxes\unvbox}%
    \SB@output0%
  \else%
    \unvbox\voidb@x%
    \SB@breakpoint\spenalty%
    \ifdim\sbarheight>\z@%
      \vskip-\sbarheight\relax%
    \fi%
    \unvbox\SB@songbox%
  \fi%
}
%    \end{macrocode}
% \end{macro}
%
% \begin{macro}{\SB@styppage}
% Page-submissions go directly to the top of the nearest fresh page unless
% the current page has all page-width material so far.
%
% Implementation notes:
% The |\null| is needed because the page builder consults |\pagetotal|,
% which isn't updated by \TeX{} until a box is contributed (|\unvbox| doesn't
% count).
% Both |\nointerlineskip|s are needed because |\unvbox| fails to update
% |\prevdepth|, and it doesn't make sense to inherit its value from whatever
% preceeded this contribution.
% Authors who want interline glue must therefore insert it explicitly at the
% bottom of their contributed text.
%    \begin{macrocode}
\newcommand\SB@styppage{%
  \ifnum\SB@numcols>\z@%
    \SB@clearpage%
    \unvbox\SB@songbox%
    \nointerlineskip\null%
  \else%
    \unvbox\SB@songbox%
  \fi%
  \nointerlineskip%
}
%    \end{macrocode}
% \end{macro}
%
% \begin{macro}{\SB@sgroup}
% This macro controls whether songs submitted to the
% page-builder are actually contributed to the final document when
% using |\includeonlysongs| to generate a partial list.
% If |\SB@sgroup| is empty, then the song is silently dropped.
% Otherwise it is contributed only if |\SB@sgroup| is a member of
% |\songlist|.
%    \begin{macrocode}
\newcommand\SB@sgroup{}
\let\SB@sgroup\@empty
%    \end{macrocode}
% \end{macro}
%
% \begin{macro}{\SB@groupcnt}
% This counter assigns a unique integer to each item of a group.
% Environments that come before the group's song are numbered decreasingly
% from $-1$.
% The song itself has number 0.
% Environments that come after the song are numbered increasingly from 1.
%    \begin{macrocode}
\SB@newcount\SB@groupcnt
%    \end{macrocode}
% \end{macro}
%
% \begin{macro}{\SB@clearpboxes}
% This dynamically constructed macro clears the content of all boxes created
% by the workings of |\includeonlysongs|.
%    \begin{macrocode}
\newcommand\SB@clearpboxes{}
%    \end{macrocode}
% \end{macro}
%
% \begin{macro}{\SB@partbox}
% Save a box of full-song or chorus material for later output when producing
% a partial list using |\includeonlysongs|.
%    \begin{macrocode}
\newcommand\SB@partbox[1]{%
  \SB@newbox#1%
  \SB@app\gdef\SB@clearpboxes{\setbox#1\box\voidb@x}%
  \global\setbox#1\box%
}
%    \end{macrocode}
% \end{macro}
%
% \begin{macro}{\SB@submitpart}
% When a song completes and we're generating a partial list, save the song
% in a box so that it can be submitted at the end of the section in the
% order specified by |\includeonlysongs|.
%    \begin{macrocode}
\newcommand\SB@submitpart{%
  \ifx\SB@sgroup\@empty\else%
    \SB@testfalse
    \@for\SB@temp:=\songlist\do{\ifx\SB@temp\SB@sgroup\SB@testtrue\fi}%
    \ifSB@test%
      \edef\SB@tempii{\SB@sgroup @\the\SB@groupcnt}%
      \expandafter\SB@partbox
        \csname songbox@\SB@tempii\endcsname\SB@songbox%
      \global\expandafter\let%
        \csname stype@\SB@tempii\endcsname\SB@stype%
      \ifrepchorus\ifvoid\SB@chorusbox\else%
        \expandafter\SB@partbox
	  \csname chbox@\SB@tempii\endcsname\SB@chorusbox%
      \fi\fi%
    \fi%
    \global\advance\SB@groupcnt%
      \ifnum\SB@groupcnt<\z@\m@ne\else\@ne\fi%
  \fi%
  \setbox\SB@songbox\box\voidb@x%
  \setbox\SB@chorusbox\box\voidb@x%
}
%    \end{macrocode}
% \end{macro}
%
% \begin{macro}{\SB@submitsong}
% Submit the most recently finished song (or block of other vertical material)
% for output.
% If we're generating a partial list of songs, save it in a box instead of
% submitting it here.
% (The saved boxes will be submitted in the requested order at the end of
% the songs section.)
%    \begin{macrocode}
\newcommand\SB@submitsong{%
  \ifpartiallist\SB@submitpart\else\SB@stype\fi%
}
%    \end{macrocode}
% \end{macro}
%
% \begin{macro}{\SB@submitenv}
% Submit the |\SB@envbox| box as a page-width contribution.
%    \begin{macrocode}
\newcommand\SB@submitenv{%
  \begingroup%
    \let\SB@songbox\SB@envbox%
    \SB@styppage%
  \endgroup%
}
%    \end{macrocode}
% \end{macro}
%
% \begin{macro}{\SB@songlistbrk}
% \begin{macro}{\SB@songlistnc}
% \begin{macro}{\SB@songlistcp}
% \begin{macro}{\SB@songlistcdp}
% These macros define the words that, when placed in a |\songlist|,
% force a column break at that point.
% Using |brk| produces a soft break (like |\brk|) that won't leave
% whitespace at the bottom of the broken column in lyric books.
% Using |nextcol| produces a hard break (like |\nextcol|) that may
% insert whitespace to finish the column.
% Using |sclearpage| moves to the next page if the current page is
% nonempty.
% Using |scleardpage| moves to the next double-page if the current
% double-page is nonempty.
%    \begin{macrocode}
\newcommand*\SB@songlistbrk{brk}
\newcommand*\SB@songlistnc{nextcol}
\newcommand*\SB@songlistcp{sclearpage}
\newcommand*\SB@songlistcdp{scleardpage}
%    \end{macrocode}
% \end{macro}
% \end{macro}
% \end{macro}
% \end{macro}
%
% \begin{macro}{\commitsongs}\MainImpl{commitsongs}
% If we're generating only a partial list, then wait until the end of the
% section and then output all the songs we saved in boxes in the order
% specified.
%    \begin{macrocode}
\newcommand\commitsongs{%
  \ifpartiallist%
    \ifnum\SB@numcols>\z@%
      \@for\SB@temp:=\songlist\do{%
        \ifx\SB@temp\SB@songlistnc\SB@nextcol\@ne\@flushglue\else%
        \ifx\SB@temp\SB@songlistbrk\SB@nextcol\@ne\colbotglue\else%
        \ifx\SB@temp\SB@songlistcp\SB@clearpage\else%
        \ifx\SB@temp\SB@songlistcdp\SB@cleardpage\else%
          \SB@groupcnt\m@ne\SB@finloop%
          \SB@groupcnt\z@\SB@finloop%
        \fi\fi\fi\fi%
      }%
    \else%
      \@for\SB@temp:=\songlist\do{%
        \ifx\SB@temp\SB@songlistnc\vfil\break\else%
        \ifx\SB@temp\SB@songlistbrk\break\else%
        \ifx\SB@temp\SB@songlistcp\clearpage\else%
        \ifx\SB@temp\SB@songlistcdp%
          \clearpage%
          \ifodd\c@page\null\newpage\fi%
        \else%
          \SB@groupcnt\m@ne\SB@finloop%
          \SB@groupcnt\z@\SB@finloop%
        \fi\fi\fi\fi%
      }%
    \fi%
    \SB@clearpboxes%
  \fi%
  \SB@clearpage%
}
%    \end{macrocode}
% \end{macro}
%
% \begin{macro}{\SB@finloop}
% While contributing saved material included by |\includeonlysongs|,
% this macro contributes each series of boxes grouped together as part of a
% |songgroup| environment.
%    \begin{macrocode}
\newcommand\SB@finloop{%
  \loop\edef\SB@tempii{\SB@temp @\the\SB@groupcnt}%
       \expandafter\ifx%
         \csname songbox@\SB@tempii\endcsname\relax\else%
    \setbox\SB@songbox\expandafter\copy%
        \csname songbox@\SB@tempii\endcsname%
    \expandafter\ifx\csname chbox@\SB@tempii\endcsname\relax%
      \repchorusfalse%
    \else%
      \repchorustrue%
      \setbox\SB@chorusbox\expandafter\copy%
        \csname chbox@\SB@tempii\endcsname%
    \fi%
    \csname stype@\SB@tempii\endcsname%
    \advance\SB@groupcnt\ifnum\SB@groupcnt<\z@\m@ne\else\@ne\fi%
  \repeat%
}
%    \end{macrocode}
% \end{macro}
%
% \begin{macro}{\SB@insertchorus}
% Insert a chorus into the first marked spot in the box given
% in the first argument.
% This is usually achieved by splitting the box at the first valid
% breakpoint after the first |\SB@cmark| in the box.
% The box is globally modified.
%    \begin{macrocode}
\newcommand\SB@insertchorus[1]{{%
  \vbadness\@M\vfuzz\maxdimen%
  \setbox\SB@box\copy#1%
  \setbox\SB@box\vsplit\SB@box to\maxdimen%
  \edef\SB@temp{\splitfirstmarks\SB@nocmarkclass}%
  \ifx\SB@temp\SB@nocmark\else%
    \edef\SB@temp{\splitfirstmarks\SB@cmarkclass}%
    \ifx\SB@temp\SB@cmark%
      \SB@dimen4096\p@%
      \SB@dimenii\maxdimen%
      \SB@dimeniii\SB@dimen%
      \loop%
        \SB@dimeniii.5\SB@dimeniii%
        \setbox\SB@box\copy#1%
        \setbox\SB@box\vsplit\SB@box to\SB@dimen%
        \edef\SB@temp{\splitfirstmarks\SB@cmarkclass}%
        \ifx\SB@temp\SB@cmark%
          \SB@dimenii\SB@dimen%
          \advance\SB@dimen-\SB@dimeniii%
        \else%
          \advance\SB@dimen\SB@dimeniii%
        \fi%
      \ifdim\SB@dimeniii>2\p@\repeat%
      \setbox\SB@box\vsplit#1to\SB@dimenii%
      \global\setbox#1\vbox{%
        \unvbox\SB@box\unskip%
        \SB@inversefalse\SB@prevversetrue\SB@stanzabreak%
        \SB@putbox\unvcopy\SB@chorusbox%
        \SB@inversetrue\SB@prevversefalse\SB@stanzabreak%
        \unvbox#1%
      }%
%    \end{macrocode}
% However, if the first mark is a |\SB@lastcmark|, it means that this chorus
% should go after the last verse in the song.
% There is no valid breakpoint there, so to get a chorus into that spot, we
% have to do a rather ugly hack:
% We pull the bottom material off the box with |\unskip|, |\unpenalty|, and
% |\lastbox|, then insert the chorus, then put the bottom material back on.
% This works because the high-level structure of the bottom material should
% be static.
% Even if the user redefines |\makepostlude|, the new definition gets put
% in a single box that can be manipulated with |\lastbox|.
% However, if we ever change the high-level structure, we need to remember to
% change this code accordingly.
%    \begin{macrocode}
    \else\ifx\SB@temp\SB@lastcmark%
      \global\setbox#1\vbox{%
        \unvbox#1%
        \unskip%
        \ifdim\sbarheight>\z@%
          \setbox\SB@box\lastbox%
          \unskip\unpenalty%
        \fi%
        \setbox\SB@box\lastbox%
        \unskip\unskip%
        \SB@inversefalse\SB@prevversetrue\SB@stanzabreak%
        \marks\SB@nocmarkclass{\SB@nocmark}%
        \unvcopy\SB@chorusbox%
        \vskip\versesep\vskip\beforepostludeskip\relax%
        \nointerlineskip\box\SB@box%
        \ifdim\sbarheight>\z@%
          \nobreak\vskip2\p@\@plus\p@%
          \hrule\@height\sbarheight\@width\SB@colwidth%
        \fi%
      }%
    \fi\fi%
  \fi%
}}
%    \end{macrocode}
% \end{macro}
%
% \begin{macro}{\nextcol}\MainImpl{nextcol}
% End the current column (inserting vertical space as needed).
% This differs from column breaks produced with |\brk|, which does not
% introduce any empty vertical space.
%    \begin{macrocode}
\newcommand\nextcol{%
  \@ifstar{\SB@nextcol\@ne\@flushglue}%
          {\ifpartiallist\else\SB@nextcol\@ne\@flushglue\fi}%
}
%    \end{macrocode}
% \end{macro}
%
% \begin{macro}{\sclearpage}\MainImpl{sclearpage}
% Move to the next page if the current page is nonempty.
%    \begin{macrocode}
\newcommand\sclearpage{%
  \@ifstar\SB@clearpage{\ifpartiallist\else\SB@clearpage\fi}%
}
%    \end{macrocode}
% \end{macro}
%
% \begin{macro}{\scleardpage}\MainImpl{scleardpage}
% Move to the next even-numbered page if the current page is odd or nonempty.
%    \begin{macrocode}
\newcommand\scleardpage{%
  \@ifstar\SB@cleardpage{\ifpartiallist\else\SB@cleardpage\fi}%
}
%    \end{macrocode}
% \end{macro}
%
% \subsection{Songs}
%
% The following macros handle the parsing and formatting of the material that
% begins and ends each song.
%
% \begin{macro}{\SB@lop}
% \begin{macro}{\SB@@lop}
% \begin{macro}{\SB@emptylist}
% \begin{macro}{\SB@ifempty}
% The following macros were adapted from Donald Knuth's \emph{The \TeX book},
% for manipulating lists of the form
% {\it |\\|item1|\\|item2|\\|...|\\|itemN|\\|}.
%    \begin{macrocode}
\newcommand\SB@lop[1]{\expandafter\SB@@lop\the#1\SB@@lop#1}
\newcommand\SB@@lop{}
\def\SB@@lop\\#1\\#2\SB@@lop#3#4{\global#3{\\#2}\global#4{#1}}
\newcommand*\SB@emptylist{\\}
\newcommand\SB@ifempty[3]{%
  \edef\SB@temp{\the#1}%
  \ifx\SB@temp\SB@emptylist#2\else#3\fi%
}
%    \end{macrocode}
% \end{macro}
% \end{macro}
% \end{macro}
% \end{macro}
%
% \begin{macro}{\SB@titlelist}
% \begin{macro}{\SB@titletail}
% These registers hold the full list of titles for the current song and
% the tail list of titles that has not yet been iterated over.
%    \begin{macrocode}
\SB@newtoks\SB@titlelist
\SB@newtoks\SB@titletail
%    \end{macrocode}
% \end{macro}
% \end{macro}
%
% \begin{macro}{\songtitle}
% \changes{v1.15}{2005/05/26}{Added song title iterators}
% The |\songtitle| macro will initially hold the primary title of the
% current song.
% The user can iterate over titles using |\nexttitle| or |\foreachtitle|.
%    \begin{macrocode}
\newcommand\songtitle{}
%    \end{macrocode}
% \end{macro}
%
% \begin{macro}{\resettitles}\MainImpl{resettitles}
% \changes{v1.15}{2005/05/26}{Added.}
% Initialize the title list iterator.
%    \begin{macrocode}
\newcommand\resettitles{%
  \global\SB@titletail\SB@titlelist%
  \nexttitle%
}
%    \end{macrocode}
% \end{macro}
%
% \begin{macro}{\nexttitle}\MainImpl{nexttitle}
% \changes{v1.15}{2005/05/26}{Added.}
% Advance the title list iterator to the next title.
%    \begin{macrocode}
\newcommand\nexttitle{%
  \SB@ifempty\SB@titletail{%
    \global\let\songtitle\relax%
  }{%
    \SB@lop\SB@titletail\SB@toks%
    \edef\songtitle{\the\SB@toks}%
  }%
}
%    \end{macrocode}
% \end{macro}
%
% \begin{macro}{\foreachtitle}\MainImpl{foreachtitle}
% \changes{v1.15}{2005/05/26}{Added.}
% Execute a block of code for each remaining title in the title list.
%    \begin{macrocode}
\newcommand\foreachtitle[1]{%
  \ifx\songtitle\relax\else%
    \loop#1\nexttitle\ifx\songtitle\relax\else\repeat%
  \fi%
}
%    \end{macrocode}
% \end{macro}
%
% \begin{macro}{\ifSB@insong}
% \begin{macro}{\ifSB@intersong}
% \begin{macro}{\ifSB@inverse}
% \begin{macro}{\ifSB@inchorus}
% To help the user locate errors, keep track of which environments we're inside
% and immediately signal an error if someone tries to use a song command inside
% a scripture quotation, etc.
%    \begin{macrocode}
\newif\ifSB@songsenv\SB@songsenvfalse
\newif\ifSB@insong\SB@insongfalse
\newif\ifSB@intersong\SB@intersongfalse
\newif\ifSB@inverse\SB@inversefalse
\newif\ifSB@inchorus\SB@inchorusfalse
%    \end{macrocode}
% \end{macro}
% \end{macro}
% \end{macro}
% \end{macro}
%
% \begin{macro}{\SB@closeall}
% If an error is detected using one of the above, the following macro will
% contain a macro sequence sufficient to end the unclosed environment,
% hopefully allowing processing to continue.
%    \begin{macrocode}
\newcommand\SB@closeall{}
%    \end{macrocode}
% \end{macro}
%
% \begin{macro}{\SB@rawrefs}
% \begin{macro}{\songauthors}\MainImpl{songauthors}
% \begin{macro}{\songcopyright}\MainImpl{songcopyright}
% \begin{macro}{\songlicense}\MainImpl{songlicense}
% The current song's scripture references, authors, copyright info, and
% copyright license information are stored in these macros.
%    \begin{macrocode}
\newcommand\SB@rawrefs{}
\newcommand\songauthors{}
\newcommand\songcopyright{}
\newcommand\songlicense{}
%    \end{macrocode}
% \end{macro}
% \end{macro}
% \end{macro}
% \end{macro}
%
% \begin{macro}{\songrefs}\MainImpl{songrefs}
% When the user asks for the song's scripture references, rather than give
% them the raw token list that the author entered, we return a prettier
% version in which spaces, dashes, and penalties have been adjusted.
% The prettier version is stored in the following control sequence.
%    \begin{macrocode}
\newcommand\songrefs{}
%    \end{macrocode}
% \end{macro}
%
% \begin{macro}{\setlicense}\MainImpl{setlicense}
% The user sets the licensing info for the current song with this command.
%    \begin{macrocode}
\newcommand\setlicense{\gdef\songlicense}
%    \end{macrocode}
% \end{macro}
%
% \begin{macro}{\newsongkey}\MainImpl{newsongkey}
% \begin{macro}{\SB@clearbskeys}
% \changes{v2.0}{2007/06/18}{Added.}
% Defining a new key for |\beginsong| is just like the |keyval| package's
% |\define@key| macro except that we must also define some initializer code
% for each key.
% This provides an opportunity to clear registers before each song.
% (Otherwise when a key wasn't specified, we'd inherit the old values from
% the previous song.)
%    \begin{macrocode}
\newcommand\SB@clearbskeys{}
\newcommand\newsongkey[2]{%
  \SB@app\gdef\SB@clearbskeys{#2}%
  \define@key{beginsong}{#1}%
}
%    \end{macrocode}
% \end{macro}
% \end{macro}
%
% Define keys |sr|, |by|, |cr|, |li|, |index|, and |ititle| for scripture
% references, authors, copyright info, licensing info, lyric index entries,
% and alternate title index entries, respectively.
%    \begin{macrocode}
\newsongkey{sr}{\def\SB@rawrefs{}\gdef\songrefs{}}
               {\def\SB@rawrefs{#1}\SB@parsesrefs{#1}}
\newsongkey{by}{\def\songauthors{}}{\def\songauthors{#1}}
\newsongkey{cr}{\def\songcopyright{}}{\def\songcopyright{#1}}
\newsongkey{li}{\setlicense{}}{\setlicense{#1}}
\newsongkey{index}{}{\indexentry{#1}}
\newsongkey{ititle}{}{\indextitleentry{#1}}
%    \end{macrocode}
%
% \begin{environment}{song}\MainEnvImpl{song}
% \begin{macro}{\beginsong}
% \begin{macro}{\SB@@beginsong}
% \begin{macro}{\SB@bsoldfmt}
% \begin{macro}{\SB@@bskvfmt}
% Parse the arguments of a |\beginsong| macro.
% The |\beginsong| macro supports two syntaxes.
% The preferred syntax takes the song title(s) as its first argument and
% an optional keyval list in brackets as its second argument.
% A legacy syntax supports four arguments, all enclosed in braces,
% which are: the title(s), scripture references, authors, and copyright info.
%    \begin{macrocode}
\newenvironment{song}{\beginsong}{\SB@endsong}
\newcommand\beginsong[1]{%
  \ifSB@insong\SB@errboo\SB@closeall\fi%
  \ifSB@intersong\SB@errbor\SB@closeall\fi%
  \SB@insongtrue%
  \def\SB@closeall{\endsong}%
  \SB@parsetitles{#1}%
  \global\setbox\SB@songwrites\box\voidb@x%
  \SB@clearbskeys%
  \@ifnextchar[\SB@bskvfmt\SB@@beginsong%
}
\newcommand\SB@@beginsong{%
  \@ifnextchar\bgroup\SB@bsoldfmt\SB@@@beginsong%
}
\newcommand\SB@bsoldfmt[3]{%
  \SB@bskvfmt[sr={#1},by={#2},cr={#3}]%
}
\newcommand\SB@bskvfmt{}
\def\SB@bskvfmt[#1]{%
  \setkeys{beginsong}{#1}%
  \SB@@@beginsong%
}
%    \end{macrocode}
% \end{macro}
% \end{macro}
% \end{macro}
% \end{macro}
%
% \begin{macro}{\SB@@@beginsong}
% \changes{v1.12}{2005/05/10}{Redid spacing and page-breaking}
% \changes{v1.14}{2005/05/15}{Improved scripture reference line-breaking}
% \changes{v2.0}{2007/06/18}{Added keyval syntax.}
% Begin typesetting a song.
% Beginning a song involves typesetting the title and other info, adding
% entries to the indexes, and setting up the environment in which verses and
% choruses reside.
%    \begin{macrocode}
\newcommand\SB@@@beginsong{%
  \global\SB@stanzafalse%
  \setbox\SB@chorusbox\box\voidb@x%
  \SB@gotchorusfalse%
  \setbox\SB@songbox\vbox\bgroup\begingroup%
    \ifnum\SB@numcols>\z@\hsize\SB@colwidth\fi%
    \leftskip\z@skip\rightskip\z@skip%
    \parfillskip\@flushglue\parskip\z@skip%
    \SB@raggedright%
    \global\SB@transposefactor\z@%
    \global\SB@cr@{\\}%
    \protected@edef\@currentlabel{\p@songnum\thesongnum}%
    \setcounter{versenum}{1}%
    \SB@prevversetrue%
    \meter44%
    \resettitles%
    \SB@addtoindexes\songtitle\SB@rawrefs\songauthors%
    \nexttitle%
    \foreachtitle{\expandafter\SB@addtotitles\expandafter{\songtitle}}%
    \resettitles%
    \lyricfont\relax%
    \SB@setbaselineskip%
}
%    \end{macrocode}
% \end{macro}
%
% \begin{macro}{\SB@endsong}
% \changes{v1.12}{2005/05/10}{Redid spacing and page-breaking}
% \changes{v2.0}{2007/06/18}{Removed hyperref dependency}
% Ending a song involves creating the song header (with |\makeprelude|),
% creating the song footer (with |\makepostlude|), and then assembling
% everything together into the |\SB@songbox|.
% The box is then submitted to the page-builder via |\SB@submitsong|.
% We do things this way instead of just contributing material directly
% to the main vertical list because submitting material song by song allows
% for a more sophisticated page-breaking algorithm than is possible with
% \TeX's built-in algorithm.
%    \begin{macrocode}
\newcommand\SB@endsong{%
  \ifSB@insong%
      \ifSB@inverse\SB@erreov\endverse\fi%
      \ifSB@inchorus\SB@erreoc\endchorus\fi%
      \global\SB@skip\versesep%
      \unskip%
      \ifrepchorus\ifvoid\SB@chorusbox\else%
        \ifSB@prevverse\ifvnumbered%
          \marks\SB@cmarkclass{\SB@lastcmark}%
        \fi\fi%
      \fi\fi%
    \endgroup\egroup%
    \begingroup%
      \ifnum\SB@numcols>\z@%
        \hsize\ifpagepreludes\textwidth\else\SB@colwidth\fi%
      \fi%
      \leftskip\z@skip\rightskip\z@skip%
      \parfillskip\@flushglue\parskip\z@skip\parindent\z@%
      \global\setbox\SB@envbox\vbox{%
        \songmark%
        \unvbox\SB@songwrites%
        \ifpagepreludes\else\ifdim\sbarheight>\z@%
          \hrule\@height\sbarheight\@width\hsize%
          \nobreak\vskip5\p@\relax%
        \fi\fi%
        \resettitles%
        \begingroup%
          \songtarget{\ifnum\c@section=\z@1\else2\fi}%
                     {song\theSB@songsnum-\thesongnum}%
        \endgroup%
        \vbox{\makeprelude}%
        \nobreak\vskip\SB@skip%
        \vskip\afterpreludeskip\relax%
      }%
      \ifnum\SB@numcols>\z@\hsize\SB@colwidth\fi%
      \global\setbox\SB@songbox\vbox{%
        \ifpagepreludes\else\unvbox\SB@envbox\fi%
        \unvbox\SB@songbox%
        \nobreak\vskip\SB@skip%
        \vskip\beforepostludeskip\relax%
        \nointerlineskip%
        \vbox{\makepostlude}%
        \ifdim\sbarheight>\z@%
          \nobreak\vskip2\p@\@plus\p@%
          \nointerlineskip%
          \hbox{\vrule\@height\sbarheight\@width\hsize}%
        \fi%
      }%
    \endgroup%
    \SB@insongfalse%
    \edef\SB@sgroup{\thesongnum}%
    \global\SB@groupcnt\z@%
    \ifpagepreludes\SB@submitenv\fi%
    \SB@submitsong%
    \ifnum\SB@grouplvl=\z@\let\SB@sgroup\@empty\fi%
    \stepcounter{songnum}%
  \else%
    \ifSB@intersong\SB@erreor\SB@closeall%
    \else\SB@erreot\fi%
  \fi%
}
%    \end{macrocode}
% \end{macro}
% \end{environment}
%
% \begin{macro}{\SB@setbaselineskip}
% \changes{v1.22}{2007/05/15}{Added.}
% \changes{v2.1}{2007/08/02}{Fixed to scale better with large font sizes.}
% Set the |\baselineskip| to an appropriate line height.
%    \begin{macrocode}
\newcommand\SB@setbaselineskip{%
  \SB@dimen\f@size\p@%
  \baselineskip\SB@dimen\relax%
  \ifchorded%
    \setbox\SB@box\hbox{{\printchord{ABCDEFG\shrp\flt/j7}}}%
    \advance\baselineskip\ht\SB@box%
    \advance\baselineskip2\p@%
  \fi%
  \ifslides%
    \advance\baselineskip.2\SB@dimen\@plus.5\SB@dimen%
      \@minus.2\SB@dimen%
  \else%
    \advance\baselineskip\z@\@plus.1\SB@dimen\relax%
  \fi%
  \advance\baselineskip\baselineadj%
}
%    \end{macrocode}
% \end{macro}
%
% \begin{macro}{\SB@setversesep}
% Set the |\versesep| to an appropriate amount if has not already been
% explicitly set by the user.
%    \begin{macrocode}
\newcommand\SB@setversesep{%
  \SB@dimen123456789sp%
  \edef\SB@temp{\the\SB@dimen}%
  \edef\SB@tempii{\the\versesep}%
  \ifx\SB@temp\SB@tempii%
    \begingroup%
      \lyricfont\relax%
      \SB@dimen\f@size\p@%
      \ifchorded%
        \setbox\SB@box\hbox{{\printchord{ABCDEFG\shrp\flt/j7}}}%
        \advance\SB@dimen\ht\SB@box%
      \fi%
      \ifslides%
        \global\versesep1.2\SB@dimen\@plus.3\SB@dimen%
        \@minus.3\SB@dimen%
      \else%
        \global\versesep.75\SB@dimen\@plus.25\SB@dimen%
        \@minus.13\SB@dimen%
      \fi%
    \endgroup%
  \fi%
}
%    \end{macrocode}
% \end{macro}
%
% \begin{macro}{\makeprelude}\MainImpl{makeprelude}
% \changes{v1.15}{2005/05/26}{Added to make song header format customizable.}
% \changes{v2.0}{2007/06/18}{Arguments removed to support keyval syntax.}
% Generate the material that begins each song.
% This macro is invoked at |\endsong| so that its code can access song info
% defined throughout the song.
%
% Note that if you are redefining |\makeprelude|, you can probably replace
% everything below with something much simpler.
% The code below is lengthy because it accommodates all of the many different
% options that various authors may adjust to customize their books.
% If you redefine it, you can replace all of this with smaller, more
% specialized programming that just outputs the prelude format you desire. 
%    \begin{macrocode}
\newcommand\makeprelude{%
  \resettitles%
%    \end{macrocode}
% In slides mode, the title, references, and authors are simply centered on
% the page with no song number.
% Only the first of the song titles is included.
% The references and authors only span the middle 50\% of the page, since
% letting them span the whole page width stretches them out too much and makes
% their fine print too hard to read.
%    \begin{macrocode}
  \ifslides%
    \hbox to\hsize{{\hfil\stitlefont\relax\songtitle\hfil}}%
    \vskip5\p@%
    \hbox to\hsize{%
      \hfil%
      \vbox{%
        \divide\hsize\tw@\parskip\p@\relax%
        \centering\small\extendprelude%
      }%
      \hfil%
    }%
  \else%
%    \end{macrocode}
% In non-slides mode, we write the song number in a shaded box to the left
% (if |\songnumwidth| is positive) and everything else in left-justified
% paragraphs to the right of it (or centered if |\pagepreludes| is on).
% The height of the shaded box that contains the song number depends on
% which is higher: the natural height of the song number, or everything else
% that goes to the right of it.
% To find out which is higher, we start by putting the song number in its
% own box (|\SB@boxii|).
%    \begin{macrocode}
    \ifdim\songnumwidth>\z@%
      \setbox\SB@boxii\hbox{{\SB@colorbox\snumbgcolor{%
        \hbox to\songnumwidth{%
          \printsongnum{\thesongnum}\hfil%
        }%
      }}}%
    \fi%
%    \end{macrocode}
% Now we know the width $w$ of the song number box, so we typeset everything
% else in a box (|\SB@box|) of width $c-w$, where $c$ is the column width.
% (If |\pagepreludes| is on, we instead use width $c-2w$ so that the material
% stays centered on the page.)
%    \begin{macrocode}
    \setbox\SB@box\vbox{%
      \ifdim\songnumwidth>\z@%
        \SB@dimen\wd\SB@boxii%
        \advance\SB@dimen3\p@%
        \ifpagepreludes\multiply\SB@dimen\tw@\fi%
        \advance\hsize-\SB@dimen%
      \fi%
      \ifpagepreludes\centering\else\SB@raggedright\fi%
      \offinterlineskip\lineskip\p@%
      {\stitlefont\relax%
       \songtitle\par%
       \nexttitle%
       \foreachtitle{(\songtitle)\par}}%
      \ifdim\prevdepth=\z@\kern\p@\fi%
      \parskip\p@\relax\tiny%
      \extendprelude%
      \kern\z@%
    }%
%    \end{macrocode}
% If the song number is being printed (i.e., |\songnumwidth| is positive),
% and its height is greater than the height of the other material, then we
% just put |\SB@boxii| and |\SB@box| side-by-side.
% If the song number is being printed but its height is less, then we
% re-typeset it at height equal to the other material, and place the boxes
% side-by-side.
% Finally, if the song number is not being printed at all, we just unbox
% |\SB@box| onto the vertical list.
%    \begin{macrocode}
    \ifdim\songnumwidth>\z@%
      \hbox{%
        \ifdim\ht\SB@boxii>\ht\SB@box%
          \box\SB@boxii%
          \kern3\p@%
          \vtop{\box\SB@box}%
        \else%
          \SB@colorbox\snumbgcolor{\vbox to\ht\SB@box{{%
            \hbox to\songnumwidth{%
              \printsongnum{\thesongnum}\hfil%
            }\vfil%
          }}}%
          \kern3\p@%
          \box\SB@box%
        \fi%
      }%
    \else%
      \unvbox\SB@box%
    \fi%
  \fi%
}
%    \end{macrocode}
% \end{macro}
%
% \begin{macro}{\makepostlude}\MainImpl{makepostlude}
% \changes{v1.15}{2005/05/26}{Added to make song trailer format customizable.}
% \changes{v2.0}{2007/06/18}{Arguments removed to support keyval syntax.}
% Generate the material that ends each song.
% The default implementation just prints the copyright and licensing
% information (if any) as a single, left-justified, non-indentended paragraph
% in fine print.
%    \begin{macrocode}
\newcommand\makepostlude{%
  \SB@raggedright\baselineskip\z@skip\parskip\z@skip\parindent\z@%
  \tiny\extendpostlude%
}
%    \end{macrocode}
% \end{macro}
%
% \begin{macro}{\showauthors}\MainImpl{showauthors}
% Display the author information in the prelude.
% This macro is only called by |\extendprelude|, which is only called by
% |\makeprelude|; so if you redefine either of those, you don't need this.
% The default implementation prints the authors in boldface and shortens the
% spacing after periods so that they don't look like ends of sentences.
%    \begin{macrocode}
\newcommand\showauthors{%
  \setbox\SB@box\hbox{\bfseries\sfcode`.\@m\songauthors}%
  \ifdim\wd\SB@box>\z@\unhbox\SB@box\par\fi%
}
%    \end{macrocode}
% \end{macro}
%
% \begin{macro}{\showrefs}\MainImpl{showrefs}
% Display the scripture references in the prelude.
% This macro is only called by |\extendprelude|, which is only called by
% |\makeprelude|; so if you redefine either of those, you don't need this.
% The default implementation prints the scripture references in slanted
% (oblique) font.
%    \begin{macrocode}
\newcommand\showrefs{%
  \setbox\SB@box\hbox{\slshape\songrefs\vphantom,}%
  \ifdim\wd\SB@box>\z@\unhbox\SB@box\par\fi%
}
%    \end{macrocode}
% \end{macro}
%
% \begin{macro}{\SB@next}
% \begin{macro}{\SB@donext}
% \begin{macro}{\SB@dothis}
% Several macros use |\futurelet| to look ahead in the input stream, and then
% take various actions depending on what is seen.
% In these macros, |\SB@next| is assigned the token seen, |\SB@dothis| is
% assigned the action to be taken on this loop iteration, and |\SB@donext| is
% assigned the action to be taken to continue (or terminate) the loop.
%    \begin{macrocode}
\newcommand\SB@next{}
\newcommand\SB@donext{}
\newcommand\SB@dothis{}
%    \end{macrocode}
% \end{macro}
% \end{macro}
% \end{macro}
%
% \begin{macro}{\SB@nextname}
% Sometimes when scanning ahead we |\string|ify the name of the next token.
% When that happens, the name is stored in this macro for safekeeping.
%    \begin{macrocode}
\newcommand\SB@nextname{}
%    \end{macrocode}
% \end{macro}
%
% \begin{macro}{\SB@appendsp}
% Append an explicit space token (catcode 10) to a token register.
% This is a useful macro to have around because inlining this code directly
% into a larger macro is harder than it seems:
% If you write the following code but with an explicit control sequence
% instead of |#1|, then the space immediately following the name will get
% stripped by the \TeX{} parser.
% But invoking the following macro with a control sequence as an argument
% works fine, because in that case the explicit space has already been
% tokenized when this macro was first defined and won't be stripped as it
% is expanded.
%    \begin{macrocode*}
\newcommand\SB@appendsp[1]{#1\expandafter{\the#1 }}
%    \end{macrocode*}
% \end{macro}
%
% \begin{macro}{\SB@parsetitles}
% \changes{v2.1}{2007/08/02}{Added}
% Parse a list of song titles.
% This just involves removing leading and trailing spaces from around each
% title in the |\\|-separated list.
%    \begin{macrocode}
\newcommand\SB@parsetitles[1]{%
  \begingroup%
    \global\SB@titlelist{\\}%
    \SB@toks{}%
    \let\\\SB@titlesep%
    \SB@pthead#1\SB@endparse%
  \endgroup%
}
%    \end{macrocode}
% \end{macro}
%
% \begin{macro}{\SB@pthead}
% \begin{macro}{\SB@@pthead}
% \begin{macro}{\SB@@@pthead}
% While processing tokens at the head of a title, we skip over all spaces
% until we reach a non-space token.
%    \begin{macrocode}
\newcommand\SB@pthead{\futurelet\SB@next\SB@@pthead}
\newcommand\SB@@pthead{%
  \ifcat\noexpand\SB@next\@sptoken%
    \expandafter\SB@@@pthead%
  \else%
    \expandafter\SB@ptmain%
  \fi%
}
\newcommand\SB@@@pthead{%
  \afterassignment\SB@pthead%
  \let\SB@next= }
%    \end{macrocode}
% \end{macro}
% \end{macro}
% \end{macro}
%
% \begin{macro}{\SB@ptloop}
% The iterator of the title parser loop just scans the next token.
%    \begin{macrocode}
\newcommand\SB@ptloop{\futurelet\SB@next\SB@ptmain}
%    \end{macrocode}
% \end{macro}
%
% \begin{macro}{\SB@ptmain}
% Once we've reached a non-space token in the title, we consume the remainder
% of the title as-is, except that space tokens should be trimmed from the end
% of each title.
%    \begin{macrocode}
\newcommand\SB@ptmain{%
  \ifcat\noexpand\SB@next\@sptoken%
    \let\SB@donext\SB@ptsp%
  \else\ifcat\noexpand\SB@next\bgroup%
    \let\SB@donext\SB@ptbg%
  \else\ifx\SB@next\SB@endparse%
    \global\SB@titlelist\expandafter{\the\SB@titlelist\\}%
    \let\SB@donext\@gobble%
  \else\ifx\SB@next\\%
    \SB@toks{}%
    \def\SB@donext{\SB@ptstep\SB@pthead}%
  \else%
    \def\SB@donext{\SB@ptstep\SB@ptloop}%
  \fi\fi\fi\fi%
  \SB@donext}
%    \end{macrocode}
% \end{macro}
%
% \begin{macro}{\SB@ptstep}
% Consume a non-space, non-left-brace token and add it to the current song
% title.
% If any spaces preceded it, add those too.
%    \begin{macrocode}
\newcommand\SB@ptstep[2]{%
  \global\SB@titlelist\expandafter\expandafter\expandafter{%
    \expandafter\the\expandafter\SB@titlelist\the\SB@toks#2}%
  \SB@toks{}%
  #1}
%    \end{macrocode}
% \end{macro}
%
% \begin{macro}{\SB@ptbg}
% The next title token is a left-brace.
% It should be balanced, so consume the entire group and add it (along with
% its surrounding braces) as-is to the current title.
%    \begin{macrocode}
\newcommand\SB@ptbg[1]{\SB@ptstep\SB@ptloop{{#1}}}
%    \end{macrocode}
% \end{macro}
%
% \begin{macro}{\SB@ptsp}
% The next title token is a space.
% We won't know whether to include it in the title until we see what
% follows it.
% Strings of spaces followed by the |\\| title-delimiter token, or that
% conclude a title argument, should be stripped.
% So rather than add the space token to the title, we remember it in a
% token register for possible later inclusion.
%    \begin{macrocode}
\newcommand\SB@ptsp{
  \SB@appendsp\SB@toks%
  \afterassignment\SB@ptloop%
  \let\SB@next= }
%    \end{macrocode}
% \end{macro}
%
% \begin{macro}{\SB@titlesep}
% While parsing song titles, we temporarily assign |\\| a non-trivial
% top-level expansion (|\SB@titlesep|) in order to distinguish it from
% other macros.
%    \begin{macrocode}
\newcommand\SB@titlesep{SB@titlesep}
%    \end{macrocode}
% \end{macro}
%
% \begin{macro}{\SB@endparse}
% The |\SB@endparse| token marks the end of a token sequence being parsed.
% If parsing works as intended, the macro should never be expanded, so
% produce an error if it is.
%    \begin{macrocode}
\newcommand\SB@endparse{%
  \SB@Error{Title parsing failed}{This error should not occur.}%
}
%    \end{macrocode}
% \end{macro}
%
% \begin{macro}{\SB@parsesrefs}
% \changes{v1.14}{2005/05/15}{Added}
% Assign the |\songrefs| macro a processed version of a scripture reference in
% which the following adjustments have been made:
% (1)~Spaces not preceded by a comma or semicolon are made non-breaking.
% For example, |2 John 1:1| and |Song of Solomon 1:1| become |2~John~1:1| and
% |Song~of~Solomon~1:1|, respectively.
% (2)~Spaces between a semicolon and a book name are lengthened to en-spaces.
% (3)~Single hyphens are lengthened to en-dashes (|--|).
% (4)~Non-breaking, thin spaces are appended to commas not followed by a
% space. For example |John 3:16,17| becomes |John~3:16,\nobreak\thinspace17|.
% (5)~Everything within an explicit group is left unchanged, allowing the
% user to suppress all of the above as desired.
%
% To achieve this, we must change all commas, hyphens, and spaces
% in the scripture reference into active characters.
% Unfortunately, the catcodes of everything in the text were set back when
% the full keyval list was digested as an argument to |\beginsong|, so we
% must unset and reset the catcodes.
% One obvious solution is to use |\scantokens| from $\varepsilon$-\TeX{} to
% do this, but that doesn't allow us to suppress the re-catcoding process
% within groups, and we'd like to avoid intoducing features that require
% $\varepsilon$-\TeX{} anyway for compatibility reasons.
% Therefore, we build the following small scanner instead.
%
% The scanner walks through the text token by token, replacing each important
% token by its active equivalent.
% No character codes are modified during this process and no tokens are
% inserted because some of these tokens might end up being arguments to
% multi-byte unicode character macros rather than being expanded directly.
% The |inputenc| package only cares about the character codes, not the
% category codes, so modifying only the category codes should be safe.
%    \begin{macrocode}
\newcommand\SB@parsesrefs[1]{%
  \begingroup%
    \SB@toks{\begingroup\SB@sractives}%
    \SB@prloop#1\SB@endparse%
    \xdef\songrefs{\the\SB@toks\endgroup}%
  \endgroup%
}
%    \end{macrocode}
% \end{macro}
%
% \begin{macro}{\SB@prloop}
% \begin{macro}{\SB@prstep}
% \begin{macro}{\SB@@prstep}
% The main loop of the scripture reference scanner identifies each space,
% hyphen, and comma for special treatment.
%    \begin{macrocode}
\newcommand\SB@prloop{\futurelet\SB@next\SB@prstep}
\newcommand\SB@prstep{%
  \ifcat\noexpand\SB@next A%
    \expandafter\SB@prcpy%
  \else%
    \expandafter\SB@@prstep%
  \fi%
}
\newcommand\SB@@prstep{%
  \ifcat\noexpand\SB@next\@sptoken%
    \let\SB@donext\SB@prspace%
  \else\ifx\SB@next-%
    \let\SB@donext\SB@prhyphen%
  \else\ifx\SB@next,%
    \let\SB@donext\SB@prcomma%
  \else\ifx\SB@next\SB@endparse%
    \let\SB@donext\@gobble%
  \else\ifcat\noexpand\SB@next\bgroup%
    \let\SB@donext\SB@prgr%
  \else%
    \let\SB@donext\SB@prcpy%
  \fi\fi\fi\fi\fi%
  \SB@donext%
}
%    \end{macrocode}
% \end{macro}
% \end{macro}
% \end{macro}
%
% \begin{macro}{\SB@prcpy}
% \begin{macro}{\SB@prgr}
% Anything that isn't one of the special tokens above, and anything in a
% group, is copied without modification.
%    \begin{macrocode}
\newcommand\SB@prcpy[1]{\SB@toks\expandafter{\the\SB@toks#1}\SB@prloop}
\newcommand\SB@prgr[1]{\SB@toks\expandafter{\the\SB@toks{#1}}\SB@prloop}
%    \end{macrocode}
% \end{macro}
% \end{macro}

% \begin{macro}{\SB@prcomma}
% \begin{macro}{\SB@prhyphen}
% Commas and hyphens are replaced with active equivalents.
%    \begin{macrocode} 
\newcommand\SB@prcomma[1]{}
{\catcode`,\active
 \gdef\SB@prcomma#1{\SB@toks\expandafter{\the\SB@toks,}\SB@prloop}}
\newcommand\SB@prhyphen[1]{}
{\catcode`-\active
 \gdef\SB@prhyphen#1{\SB@toks\expandafter{\the\SB@toks-}\SB@prloop}}
%    \end{macrocode}
% \end{macro}
% \end{macro}
%
% \begin{macro}{\SB@prspace}
% \begin{macro}{\SB@@prspace}
% Spaces are made active as well, but doing so requires some
% specialized code since they cannot be consumed as implicit macro arguments.
%    \begin{macrocode*}
\newcommand\SB@prspace[1]{}
{\obeyspaces
\gdef\SB@prspace{\SB@toks\expandafter{\the\SB@toks }\SB@@prspace}}
%    \end{macrocode*}
%    \begin{macrocode}
\newcommand\SB@@prspace{\afterassignment\SB@prloop\let\SB@temp= }
%    \end{macrocode}
% \end{macro}
% \end{macro}

% \begin{macro}{\SB@sractives}
% Assign macro definitions to active commas, hyphens, spaces, and returns
% when the token list generated by |\SB@parsesrefs| is used to typeset a
% scripture reference list.
%    \begin{macrocode*}
\newcommand\SB@sractives{}
{\catcode`,\active\catcode`-\active\obeyspaces%
\gdef\SB@sractives{%
\let,\SB@srcomma\let-\SB@srhyphen\let \SB@srspace%
\SB@srspacing}%
}
%    \end{macrocode*}
% \end{macro}
%
% \begin{macro}{\SB@srspacing}
% The space factors of semicolons and commas are what the active spaces
% within a scripture reference text use to decide what came before.
% The following sets them to their default values in case they have been
% changed, but sets all other space factors to 1000.
%    \begin{macrocode}
\newcommand\SB@srspacing{%
  \nonfrenchspacing\sfcode`\;=1500\sfcode`\,=1250\relax%
}
%    \end{macrocode}
% \end{macro}
%
% \begin{macro}{\SB@srcomma}
% \begin{macro}{\SB@@srcomma}
% Commas not already followed by whitespace are appended with a thin,
% non-breaking space.
%    \begin{macrocode}
\newcommand\SB@srcomma{,\futurelet\SB@next\SB@@srcomma}
\newcommand\SB@@srcomma{%
  \ifx\SB@next\SB@srspace\else%
    \nobreak\thinspace%
  \fi%
}
%    \end{macrocode}
% \end{macro}
% \end{macro}
%
% \begin{macro}{\SB@srhyphen}
% \begin{macro}{\SB@@srhyphen}
% \begin{macro}{\SB@srdash}
% \begin{macro}{\SB@@srdash}
% Hyphens that are not already part of a ligature (an en- or em-dash)
% become en-dashes.
%    \begin{macrocode}
\newcommand\SB@srhyphen{\futurelet\SB@next\SB@@srhyphen}
\newcommand\SB@@srhyphen{%
  \ifx\SB@next\SB@srhyphen\expandafter\SB@srdash\else--\fi%
}
\newcommand\SB@srdash[1]{\futurelet\SB@next\SB@@srdash}
\newcommand\SB@@srdash{%
  \ifx\SB@next\SB@srhyphen---\expandafter\@gobble\else--\fi%
}
%    \end{macrocode}
% \end{macro}
% \end{macro}
% \end{macro}
% \end{macro}
%
% \begin{macro}{\SB@srspace}
% \begin{macro}{\SB@@srspace}
% To compress consecutive whitespace, we ignore spaces
% immediately followed by more whitespace.
% Spaces not preceded by a semicolon or comma become non-breaking.
% Most spaces following a semicolon become en-spaces with favorable
% breakpoints, but a special case arises for spaces between a semicolon
% and a digit (see |\SB@srcso| below).
%    \begin{macrocode}
\newcommand\SB@srspace{\futurelet\SB@next\SB@@srspace}
\newcommand\SB@@srspace{%
  \let\SB@donext\relax%
  \ifx\SB@next\SB@srspace\else%
    \ifnum\spacefactor>\@m%
      \ifnum\spacefactor>1499 %
        \ifcat\noexpand\SB@next0%
          \let\SB@donext\SB@srcso%
        \else%
          \penalty-5\enskip%
        \fi%
      \else%
        \space%
      \fi%
    \else%
      \nobreak\space%
    \fi%
  \fi%
  \SB@donext%
}
%    \end{macrocode}
% \end{macro}
% \end{macro}
%
% \begin{macro}{\SB@srcso}
% \begin{macro}{\SB@@srcso}
% A space between a semicolon and a digit could be within a list of
% verse references for a common book (e.g., |Job 1:1; 2:2|);
% or it could separate the previous book from a new book whose name
% starts with a number (e.g., |Job 1:1; 1 John 1:1|).
% In the former case, we should just use a regular space;
% but in the latter case we should be using an en-space with a
% favorable breakpoint.
% To distinguish between the two, we peek ahead at the next two tokens.
% If the second one is a space, assume the latter; otherwise assume the
% former.
%    \begin{macrocode}
\newcommand\SB@srcso[1]{\futurelet\SB@temp\SB@@srcso}
\newcommand\SB@@srcso{%
  \ifx\SB@temp\SB@srspace%
    \penalty-5\enskip%
  \else%
    \space%
  \fi%
  \SB@next%
}
%    \end{macrocode}
% \end{macro}
% \end{macro}
%
% \subsection{Verses and Choruses}
%
% The following programming typesets song contents, including verses, choruses,
% and textual notes.
%
% \begin{macro}{\ifSB@stanza}
% The following conditional remembers if we've seen any stanzas yet in the
% current song.
%    \begin{macrocode}
\newif\ifSB@stanza
%    \end{macrocode}
% \end{macro}
%
% \begin{macro}{\SB@stanzabreak}
% \changes{v1.12}{2005/05/10}{Fixed stanza counting code and improved spacing}
% End this song stanza and start a new one.
%    \begin{macrocode}
\newcommand\SB@stanzabreak{%
  \ifhmode\par\fi%
  \ifSB@stanza%
    \SB@breakpoint{%
      \ifSB@inverse%
        \ifSB@prevverse\vvpenalty\else\cvpenalty\fi%
      \else%
        \ifSB@prevverse\vcpenalty\else\ccpenalty\fi%
      \fi%
    }%
    \vskip\versesep%
  \fi%
}
%    \end{macrocode}
% \end{macro}
%
% \begin{macro}{\SB@breakpoint}
% Insert a valid breakpoint into the vertical list comprising a song.
%    \begin{macrocode}
\newcommand\SB@breakpoint[1]{%
  \begingroup%
    \ifnum#1<\@M%
      \SB@skip\colbotglue\relax%
      \SB@skip-\SB@skip%
    \else%
      \SB@skip\z@skip%
    \fi%
    \advance\SB@skip\lastskip%
    \unskip%
    \nobreak%
    \ifnum#1<\@M%
      \vskip\colbotglue\relax%
      \penalty#1%
    \fi%
    \vskip\SB@skip%
  \endgroup%
}
%    \end{macrocode}
% \end{macro}
%
% \begin{macro}{\SB@putbox}
% Unbox a vbox and follow it by vertical glue if its depth is unusually
% shallow.
% This ensures that verses and choruses will look equally spaced even if
% one of them has a final line with no descenders.
%    \begin{macrocode}
\newcommand\SB@putbox[2]{%
  \begingroup%
    \SB@dimen\dp#2%
    #1#2%
    \setbox\SB@box\hbox{{\lyricfont\relax p}}%
    \ifdim\SB@dimen<\dp\SB@box%
      \advance\SB@dimen-\dp\SB@box%
      \vskip-\SB@dimen%
    \fi%
    \setbox\SB@box\box\voidb@x%
  \endgroup%
}
%    \end{macrocode}
% \end{macro}
%
% \begin{macro}{\SB@obeylines}
% Within verses and choruses we would like to use |\obeylines| so that each
% \Meta{return} in the source file ends a paragraph without having to say
% |\par| explicitly.
% The \LaTeX{} base code establishes the convention that short-term changes to
% |\par| will restore |\par| by setting it equal to |\@par|.
% Long-term (i.e., environment-long) changes to |\par| should therefore
% redefine |\@par| to restore the desired long-term definition.
% The following code starts a long-term redefinition of |\par| adhering to
% these conventions, and extends that definition to \Meta{return} as well.
%    \begin{macrocode}
\newcommand\SB@obeylines{%
  \let\par\SB@par%
  \obeylines%
  \let\@par\SB@@par%
}
%    \end{macrocode}
% \end{macro}
%
% \begin{macro}{\SB@par}
% The following replacement definition of |\par| constructs paragraphs in
% which page-breaks are disallowed, since no wrapped line in a song should
% span a page- or column-break.
% It then inserts an interlinepenalty after the paragraph so that such
% penalties will appear between consecutive lines in each verse.
% (Note: The |\endgraf| macro must not be uttered within a local group
% since this prevents parameters like |\hangindent| from being
% reset at the conclusion of each paragraph.)
%    \begin{macrocode}
\newcommand\SB@par{%
  \ifhmode%
    \SB@cnt\interlinepenalty%
    \interlinepenalty\@M%
    \endgraf%
    \interlinepenalty\SB@cnt%
    \ifSB@inchorus%
      \ifdim\cbarwidth>\z@\nobreak\else\SB@ilpenalty\fi%
    \else%
      \SB@ilpenalty%
    \fi%
  \fi%
}
%    \end{macrocode}
% \end{macro}
%
% \begin{macro}{\SB@ilpenalty}
% By default, breaking a vertical list between paragraphs incurs a penalty
% of zero.
% Thus, we only insert an explicit penalty between lines if
% |\interlinepenalty| is non-zero.
% This avoids cluttering the vertical list with superfluous zero penalties.
%    \begin{macrocode}
\newcommand\SB@ilpenalty{%
  \ifnum\interlinepenalty=\z@\else%
    \penalty\interlinepenalty%
  \fi%
}
%    \end{macrocode}
% \end{macro}
%
% \begin{macro}{\SB@@par}
% This replacement definition of |\@par| restores the |\SB@par| definition of
% |\par| and then ends the paragraph.
%    \begin{macrocode}
\newcommand\SB@@par{\let\par\SB@par\par}
%    \end{macrocode}
% \end{macro}
%
% \begin{macro}{\SB@parindent}
% \changes{v1.12}{2005/05/10}{Added}
% Reserve a length to remember the current |\parindent|.
%    \begin{macrocode}
\SB@newdimen\SB@parindent
%    \end{macrocode}
% \end{macro}
%
% \begin{macro}{\SB@everypar}
% Reserve a control sequence to hold short-term changes to |\everypar|.
%    \begin{macrocode}
\newcommand\SB@everypar{}
%    \end{macrocode}
% \end{macro}
%
% \begin{macro}{\SB@raggedright}
% \changes{v1.12}{2005/05/10}{Added}
% Perform |\raggedright| except don't nuke the |\parindent|.
%    \begin{macrocode}
\newcommand\SB@raggedright{%
  \SB@parindent\parindent%
  \raggedright%
  \parindent\SB@parindent%
}
%    \end{macrocode}
% \end{macro}
%
% \begin{macro}{\vnumbered}
% \changes{v2.1}{2007/08/02}{Renamed.}
% The following conditional remembers whether this verse is being numbered
% or not (i.e., it distinguishes between |\beginverse| and |\beginverse*|).
%    \begin{macrocode}
\newif\ifvnumbered
%    \end{macrocode}
% \end{macro}
%
% \begin{macro}{\ifSB@prevverse}
% Reserve a conditional to remember if the previous block in this song was
% a verse.
%    \begin{macrocode}
\newif\ifSB@prevverse
%    \end{macrocode}
% \end{macro}
%
% Before replacing the little-used |verse| environment with a new one,
% issue a warning if the current definition of |\verse| is not the
% \LaTeX-default one.
% This may indicate a package clash.
%    \begin{macrocode}
\CheckCommand\verse{%
  \let\\\@centercr%
  \list{}{%
    \itemsep\z@%
    \itemindent-1.5em%
    \listparindent\itemindent%
    \rightmargin\leftmargin%
    \advance\leftmargin1.5em%
  }%
  \item\relax%
}
%    \end{macrocode}
%
% \begin{environment}{verse}\MainEnvImpl{verse}
% \begin{environment}{verse*}
% \begin{macro}{\beginverse}
% Begin a new verse.
% This can be done by beginning a |verse| environment or by using the
% |\beginverse| macro.
% The latter must check for a trailing star to determine whether this
% verse should be numbered.
% We use |\@ifstar| to scan ahead for the star, but this needs to be done
% carefully because while scanning we might encounter tokens that
% should be assigned different catcodes once the verse really begins.
% Thus, we temporarily invoke |\SB@loadactives| for the duration of
% |\@ifstar| so that everything gets the right catcode.
%    \begin{macrocode}
\renewenvironment{verse}
  {\vnumberedtrue\SB@beginverse}
  {\SB@endverse}
\newenvironment{verse*}
  {\vnumberedfalse\SB@beginverse}
  {\SB@endverse}
\newcommand\beginverse{%
  \begingroup%
    \SB@loadactives%
    \@ifstar{\endgroup\vnumberedfalse\SB@beginverse}%
            {\endgroup\vnumberedtrue\SB@beginverse}%
}
%    \end{macrocode}
% \end{macro}
% \end{environment}
% \end{environment}
%
% \begin{macro}{\SB@beginverse}
% \changes{v1.12}{2005/05/10}{Shifted to using \cs{parindent} instead of \cs{everypar} to do indentation}
% Start the body of a verse.
% We begin by inserting a mark if |\repchoruses| is active and this verse
% was preceded by a numbered verse (making this an eligible place to insert
% a chorus later).
%
% Verse numbering is implemented using |\everypar| so that if there is any
% vertical material between the |\beginverse| and the first line of the
% verse, that material will come before the verse number.
% Intervening horizontal material (e.g., |\textnote|) can temporarily
% clear |\everypar| to defer the verse number until later.
%    \begin{macrocode}
\newcommand\SB@beginverse{%
  \ifSB@insong%
    \ifSB@inverse\SB@errbvv\endverse\fi%
    \ifSB@inchorus\SB@errbvc\endchorus\fi%
  \else%
    \SB@errbvt\beginsong{Unknown Song}%
  \fi%
  \ifrepchorus\ifvoid\SB@chorusbox\else%
    \SB@gotchorustrue%
    \ifSB@prevverse\ifvnumbered%
      \marks\SB@cmarkclass{\SB@cmark}%
    \fi\fi%
  \fi\fi%
  \SB@inversetrue%
  \def\SB@closeall{\endverse\endsong}%
  \SB@stanzabreak%
  \versemark\nobreak%
  \global\SB@stanzatrue%
  \SB@ifempty\SB@cr@\memorize{\replay[]}%
  \setbox\SB@box\vbox\bgroup\begingroup%
    \ifvnumbered%
      \protected@edef\@currentlabel{\p@versenum\theversenum}%
      \def\SB@everypar{%
        \setbox\SB@box\hbox{{\printversenum{\theversenum}}}%
        \ifdim\wd\SB@box<\versenumwidth%
          \setbox\SB@box%
          \hbox to\versenumwidth{\unhbox\SB@box\hfil}%
        \fi%
        \ifchorded\vrule\@height\baselineskip\@width\z@\@depth\z@\fi%
        \placeversenum\SB@box%
        \gdef\SB@everypar{}%
      }%
    \else%
      \def\SB@everypar{%
        \ifchorded\vrule\@height\baselineskip\@width\z@\@depth\z@\fi%
        \gdef\SB@everypar{}%
      }%
    \fi%
    \everypar{\SB@everypar\everypar{}}%
    \versefont\relax\SB@setbaselineskip\versejustify%
    \SB@loadactives%
    \SB@obeylines%
    \penalty12345 %
    \everyverse\relax%
}
%    \end{macrocode}
% \end{macro}
%
% \begin{macro}{\SB@endverse}
% End a verse.
% This involves unboxing the verse material with |\SB@putbox|, which
% corrects for last lines that are unusually shallow.
%    \begin{macrocode}
\newcommand\SB@endverse{%
  \ifSB@insong%
    \ifSB@inverse%
        \unpenalty%
      \endgroup\egroup%
      \SB@putbox\unvbox\SB@box%
      \SB@inversefalse%
      \def\SB@closeall{\endsong}%
      \ifvnumbered\stepcounter{versenum}\fi%
      \SB@prevversetrue%
    \else\ifSB@inchorus\SB@errevc\endchorus%
    \else\SB@errevo\fi\fi%
  \else%
    \SB@errevt%
  \fi%
}
%    \end{macrocode}
% \end{macro}
%
% \begin{macro}{\ifSB@chorustop}
% When a chorus is broken in to several pieces by column-breaks (via |\brk|),
% the following conditional remembers whether the current piece is the
% topmost one for this chorus.
%    \begin{macrocode}
\newif\ifSB@chorustop
%    \end{macrocode}
% \end{macro}
%
% \begin{macro}{\SB@chorusbox}
% When |\repchoruses| is used, the first sequence of consecutive choruses
% is remembered in the following box register.
%    \begin{macrocode}
\SB@newbox\SB@chorusbox
%    \end{macrocode}
% \end{macro}
%
% \begin{macro}{\ifSB@gotchorus}
% The following conditional remembers whether we've completed storing the
% first block of consecutive choruses.
%    \begin{macrocode}
\newif\ifSB@gotchorus
%    \end{macrocode}
% \end{macro}
%
% \begin{macro}{\SB@cmarkclass}
% \begin{macro}{\SB@nocmarkclass}
% \changes{v2.6}{2008/02/14}{Added safe allocation of extended mark registers}
% The |\repeatchoruses| feature requires the use of two extended mark
% classes provided by $\varepsilon$-\TeX.
% We use the |\newmarks| macro to allocate these classes, if it's
% available.
% If |\newmarks| doesn't exist, then that means the user has an
% $\varepsilon$-\TeX{} compatible version of \LaTeX, but no |etex| style
% file to go with it;
% we just have to pick two mark classes and hope that nobody else is
% using them.
%    \begin{macrocode}
\ifSB@etex
  \@ifundefined{newmarks}{
    \@ifundefined{newmark}{
      \mathchardef\SB@cmarkclass83
      \mathchardef\SB@nocmarkclass84
    }{
      \newmark\SB@cmarkclass
      \newmark\SB@nocmarkclass
    }
  }{
    \newmarks\SB@cmarkclass
    \newmarks\SB@nocmarkclass
  }
\fi
%    \end{macrocode}
% \end{macro}
% \end{macro}
%
% \begin{macro}{\SB@cmark}
% \begin{macro}{\SB@lastcmark}
% \begin{macro}{\SB@nocmark}
% To determine where choruses should be inserted when |\repchoruses| is
% active, three kinds of marks are inserted into song boxes:
% |\SB@cmark| is used to mark places where a chorus might be inserted between
% verses, and |\SB@lastcmark| marks a place where a chorus might be inserted
% after the last verse of the song.
% Both marks are $\varepsilon$-\TeX{} marks of class |\SB@cmarkclass|,
% to avoid disrupting the use of standard \TeX{} marks.
% Each time a chorus is automatically inserted, |\SB@nocmark| is inserted
% with mark class |\SB@nocmarkclass| just above it (and at the top of each
% additional page it spans).
% This inhibits future chorus inserts until the already-inserted chorus has
% been fully committed to the output file.
% Otherwise some choruses could get auto-inserted multiple times at the same
% spot, possibly even leading to an infinite loop!
%    \begin{macrocode}
\newcommand*\SB@cmark{SB@cmark}
\newcommand*\SB@lastcmark{SB@lastcmark}
\newcommand*\SB@nocmark{SB@nocmark}
%    \end{macrocode}
% \end{macro}
% \end{macro}
% \end{macro}
%
% \begin{environment}{chorus}\MainEnvImpl{chorus}
% \begin{macro}{\beginchorus}
% \changes{v1.12}{2005/05/10}{Shifted to using \cs{parindent} instead of \cs{everypar} to do indentation}
% \changes{v1.14}{2005/05/15}{Choruses now stretch like the verses}
% Start a new chorus.
% If |\repchoruses| is active and this is part of the first set of consecutive
% choruses in the song, then include it and its preceding vertical material
% in the |\SB@chorusbox| for possible later duplication elsewhere.
%    \begin{macrocode}
\newenvironment{chorus}{\beginchorus}{\SB@endchorus}
\newcommand\beginchorus{%
  \ifSB@insong
    \ifSB@inverse\SB@errbcv\endverse\fi%
    \ifSB@inchorus\SB@errbcc\endchorus\fi%
  \else%
    \SB@errbct\beginsong{Unknown Song}%
  \fi%
  \SB@inchorustrue%
  \def\SB@closeall{\endchorus\endsong}%
  \SB@chorustoptrue%
  \vnumberedfalse%
  \SB@stanzabreak%
  \chorusmark%
  \ifrepchorus%
    \ifSB@gotchorus\else\ifSB@prevverse\else%
      \global\setbox\SB@chorusbox\vbox{%
        \unvbox\SB@chorusbox%
        \SB@stanzabreak%
        \chorusmark%
      }%
    \fi\fi%
  \fi%
  \global\SB@stanzatrue%
  \replay[]%
  \SB@@beginchorus%
  \everychorus\relax%
}
%    \end{macrocode}
% \end{macro}
% \end{environment}
%
% \begin{macro}{\SB@@beginchorus}
% Begin the body of a chorus, or continue the body of a chorus after |\brk|
% has paused it to insert a valid breakpoint.
% We insert an empty class-|\SB@cmarkclass| mark here so that this chorus
% will not be duplicated elsewhere on the same page(s) where it initially
% appears.
%    \begin{macrocode}
\newcommand\SB@@beginchorus{%
  \ifrepchorus\marks\SB@cmarkclass{}\fi%
  \setbox\SB@box\vbox\bgroup\begingroup%
    \ifchorded%
      \def\SB@everypar{%
        \vrule\@height\baselineskip\@width\z@\@depth\z@%
        \gdef\SB@everypar{}%
      }%
      \everypar{\SB@everypar\everypar{}}%
    \fi%
    \chorusfont\relax\SB@setbaselineskip\chorusjustify%
    \SB@loadactives%
    \SB@obeylines%
    \penalty12345 %
}
%    \end{macrocode}
% \end{macro}
%
% \begin{macro}{\SB@endchorus}
% End a chorus.
% This involves creating the vertical line to the left of the chorus and then
% unboxing the chorus material that was previously accumulated.
%    \begin{macrocode}
\newcommand\SB@endchorus{%
  \ifSB@insong%
    \ifSB@inchorus%
        \unpenalty%
      \endgroup\egroup%
      \SB@inchorusfalse%
      \def\SB@closeall{\endsong}%
      \setbox\SB@box\vbox{%
        \SB@chorusbar\SB@box%
        \SB@putbox\unvbox\SB@box%
      }
      \ifrepchorus\ifSB@gotchorus\else%
        \global\setbox\SB@chorusbox\vbox{%
          \unvbox\SB@chorusbox%
          \unvcopy\SB@box%
        }%
      \fi\fi%
      \unvbox\SB@box%
      \SB@prevversefalse%
    \else\ifSB@inverse\SB@errecv\endverse%
    \else\SB@erreco\fi\fi%
  \else%
    \SB@errect%
  \fi%
}
%    \end{macrocode}
% \end{macro}
%
% \begin{macro}{\SB@cbarshift}
% Increase |\leftskip| to accommodate the chorus bar, if any.
%    \begin{macrocode}
\newcommand\SB@cbarshift{%
  \ifSB@inchorus\ifdim\cbarwidth>\z@%
    \advance\leftskip\cbarwidth%
    \advance\leftskip5\p@\relax%
  \fi\fi%
}
%    \end{macrocode}
% \end{macro}
%
% \begin{macro}{\SB@chorusbar}
% Create the vertical bar that goes to the left of a chorus.
% Rather than boxing up the chorus in order to put the bar to the left,
% the bar is introduced as leaders directly into the vertical list of the
% main song box.
% This allows it to stretch and shrink when a column is typeset by the
% page-builder.
%    \begin{macrocode}
\newcommand\SB@chorusbar[1]{%
  \ifdim\cbarwidth>\z@%
    \SB@dimen\ht#1%
    \SB@dimenii\dp#1%
    \advance\SB@dimen%
      \ifSB@chorustop\ifchorded\else2\fi\fi\SB@dimenii%
    \SB@skip\SB@dimen\relax%
    \SB@computess\SB@skip1\@plus#1%
    \SB@computess\SB@skip{-1}\@minus#1%
    \nointerlineskip\null\nobreak%
    \leaders\vrule\@width\cbarwidth\vskip\SB@skip%
    \ifSB@chorustop\ifchorded\else%
      \advance\SB@skip-\SB@dimenii%
    \fi\fi%
    \nobreak\vskip-\SB@skip%
  \fi%
}
%    \end{macrocode}
% \end{macro}
%
% \begin{macro}{\SB@computess}
% \changes{v1.14}{2005/05/15}{Added}
% This computes the stretchability or shrinkability of a vbox and stores
% the result in the skip register given by \argp{1}.
% If $\argp{2}=1$ and \argp{3} is ``\texttt{plus}'', then the stretchability
% of box \argp{4} is added to the plus component of \argp{1}.
% If $\argp{2}=-1$ and \argp{3} is ``\texttt{minus}'', then the shrinkability
% of the box is added to the minus component of \argp{1}.
% If the stretchability or shrinkability is infinite, then we guess 1fil
% for that component.
%    \begin{macrocode}
\newcommand\SB@computess[4]{%
  \begingroup%
    \vbadness\@M\vfuzz\maxdimen%
    \SB@dimen4096\p@%
    \setbox\SB@box\vbox spread#2\SB@dimen{\unvcopy#4}%
    \ifnum\badness=\z@%
      \global\advance#1\z@#31fil\relax%
    \else%
      \SB@dimenii\SB@dimen%
      \loop%
        \SB@dimenii.5\SB@dimenii%
        \ifnum\badness<100 %
          \advance\SB@dimen\SB@dimenii%
        \else
          \advance\SB@dimen-\SB@dimenii%
        \fi%
        \setbox\SB@box\vbox spread#2\SB@dimen{\unvcopy#4}%
        \ifnum\badness=100 \SB@dimenii\z@\fi%
      \ifdim\SB@dimenii>.1\p@\repeat%
      \ifdim\SB@dimen<.1\p@\SB@dimen\z@\fi%
      \global\advance#1\z@#3\SB@dimen\relax%
    \fi%
  \endgroup%
}
%    \end{macrocode}
% \end{macro}
%
% \begin{macro}{\brk}\MainImpl{brk}
% Placing |\brk| within a line in a verse or chorus tells \TeX{} to break the
% line at that point (if it needs to be broken at all).
%
% Placing |\brk| on a line by itself within a chorus stops the chorus (and its
% vertical bar), inserts a valid breakpoint, and then restarts the chorus
% with no intervening space so that if the breakpoint isn't used, there will
% be no visible effect.
% Placing it on a line by itself within a verse just inserts a breakpoint.
%
% Placing |\brk| between songs forces a column- or page-break, but only if
% generating a non-partial list of songs.
% When generating a partial list, |\brk| between songs is ignored.
%    \begin{macrocode}
\newcommand\brk{%
  \ifSB@insong%
    \ifhmode\penalty-5 \else%
      \unpenalty%
      \ifSB@inchorus%
        \ifdim\cbarwidth=\z@%
          \ifrepchorus\marks\SB@cmarkclass{}\fi%
          \SB@breakpoint\brkpenalty%
        \else%
          \endgroup\egroup%
          \ifrepchorus\ifSB@gotchorus\else%
            \global\setbox\SB@chorusbox\vbox{%
              \unvbox\SB@chorusbox%
              \SB@chorusbar\SB@box%
              \unvcopy\SB@box%
              \SB@breakpoint\brkpenalty%
            }%
          \fi\fi%
          \SB@chorusbar\SB@box%
          \unvbox\SB@box%
          \SB@breakpoint\brkpenalty%
          \SB@chorustopfalse%
          \SB@@beginchorus%
        \fi%
      \else%
        \SB@breakpoint\brkpenalty%
      \fi%
    \fi%
  \else%
    \ifpartiallist\else\SB@nextcol\@ne\colbotglue\fi%
  \fi%
}
%    \end{macrocode}
% \end{macro}
%
% \begin{macro}{\SB@boxup}
% Typeset a shaded box containing a textual note to singers or musicians.
% We first try typesetting the note on a single line.
% If it's too big, then we try again in paragraph mode with full
% justification.
%    \begin{macrocode}
\newcommand\SB@boxup[1]{%
  \setbox\SB@box\hbox{{\notefont\relax#1}}%
  \SB@dimen\wd\SB@box%
  \advance\SB@dimen6\p@%
  \advance\SB@dimen\leftskip%
  \advance\SB@dimen\rightskip%
  \ifdim\SB@dimen>\hsize%
    \vbox{{%
      \advance\hsize-6\p@%
      \advance\hsize-\leftskip%
      \advance\hsize-\rightskip%
      \notejustify%
      \unhbox\SB@box\par%
      \kern\z@%
    }}%
  \else%
    \vbox{\box\SB@box\kern\z@}%
  \fi%
}
%    \end{macrocode}
% \end{macro}
%
% \begin{macro}{\textnote}\MainImpl{textnote}
% \changes{v1.12}{2005/05/10}{Defined unset paragraph parameters}
% Create a textual note for singers and musicians.
% If the note begins a verse or chorus, it should not be preceded by any
% spacing.
% Verses and choruses begin with the sentinel penalty 12345, so we check
% |\lastpenalty| to identify this case.
% When typesetting the note, we must be sure to temporarily clear |\everypar|
% to inhibit any verse numbering that might be pending.
% We also readjust the |\baselineskip| as if we weren't doing chords, since
% no chords go above a textual note.
%    \begin{macrocode}
\newcommand\textnote[1]{%
  \ifhmode\par\fi%
  \ifnum\lastpenalty=12345\else%
    \ifSB@inverse%
      \vskip2\p@\relax%
    \else\ifSB@inchorus%
      \vskip2\p@\relax%
    \else\ifSB@stanza%
      \nobreak\vskip\versesep%
    \fi\fi\fi%
  \fi%
  \begingroup%
    \everypar{}%
    \ifchorded\chordedfalse\SB@setbaselineskip\chordedtrue\fi%
    \placenote{\SB@colorbox\notebgcolor{\SB@boxup{#1}}}%
  \endgroup%
  \nobreak%
  \ifSB@inverse%
    \vskip2\p@\relax%
  \else\ifSB@inchorus%
    \vskip2\p@\relax%
  \else\ifSB@stanza\else%
    \nobreak\vskip\versesep%
  \fi\fi\fi%
}
%    \end{macrocode}
% \end{macro}
%
% \begin{macro}{\musicnote}\MainImpl{musicnote}
% \changes{v1.12}{2005/05/10}{Now just (conditionally) calls \cs{textnote} for consistency}
% Create a textual note for musicians.
%    \begin{macrocode}
\newcommand\musicnote[1]{\ifchorded\textnote{#1}\fi}
%    \end{macrocode}
% \end{macro}
%
% \begin{macro}{\echo}\MainImpl{echo}
% \begin{macro}{\SB@echo}
% \begin{macro}{\SB@@echo}
% \changes{v1.21}{2006/09/17}{Customized fonts now preserved.}
% \changes{v2.1}{2007/08/02}{Toggles instead of forces slanted font.}
% Typeset an echo part in the lyrics.
% Echo parts are in a user-customizable font and parenthesized.
%
% The |\echo| macro must be able to accept chords in its argument.
% This complicates the implementation because chord macros should change
% catcodes, but if we grab |\echo|'s argument in the usual way then all the
% catcodes will be set before the chord macros have a chance to change them.
% This would disallow chord name abbreviations like |#| and |&| within
% |\echo| parts.
%
% If we're using $\varepsilon$-\TeX{} then the solution is easy: we use
% |\scantokens| to re-scan the argument and thereby re-assign the catcodes.
% (One subtlety: Whenever \LaTeX{} consumes an argument to a macro, it changes
% |#| to |##| so that when the argument text is substituted into the body of
% the macro, the replacement text will not contain unsubstituted parameters
% (such as |#1|).
% If |\scantokens| is used on the replacement text and the scanned tokens
% assign a new catcode to |#|, that causes |#|'s to be doubled in the
% \emph{output}, which was not the intent.
% To avoid this problem, we use |\@sanitize| before consuming the argument to
% |\echo|, which sets the catcodes of most special tokens (including |#|) to
% 12, so that \LaTeX{} will not recognize any of them as parameters and will
% therefore not double any of them.)
%    \begin{macrocode}
\ifSB@etex
  \newcommand\echo{\begingroup\@sanitize\SB@echo}
  \newcommand\SB@echo[1]{%
    \endgroup%
    \begingroup%
      \echofont\relax%
      \endlinechar\m@ne%
      \scantokens{(#1)}%
    \endgroup%
  }
\else
%    \end{macrocode}
% If we're not using $\varepsilon$-\TeX, we must do something more complicated.
% We set up the appropriate font within a local group and finish with
% |\hbox| so that the argument to |\echo| is treated as the body of the box.
% Control is reacquired after the box using |\aftergroup|, whereupon we
% unbox the box and insert the closing parenthesis.
% This almost works except that if the last thing in an echo part is a long
% chord name atop a short lyric, the closing parenthesis will float out away
% from the lyric instead of being sucked under the chord.
% I can find no solution to this problem, so to avoid it users must find a
% version of \LaTeX{} that is $\varepsilon$-\TeX{} compatible.
%    \begin{macrocode}
  \newcommand\echo{%
    \begingroup%
      \echofont\relax%
      \afterassignment\SB@echo%
      \setbox\SB@box\hbox%
  }
  \newcommand\SB@echo{\aftergroup\SB@@echo(}
  \newcommand\SB@@echo{\unhbox\SB@box)\endgroup}
\fi
%    \end{macrocode}
% \end{macro}
% \end{macro}
% \end{macro}
%
% \begin{macro}{\rep}\MainImpl{rep}
% \changes{v1.21}{2006/09/17}{Changed to avoid math mode.}
% Place |\rep{|\Meta{n}|}| at the end of a line to indicate that it should be
% sung \Meta{n} times.
%    \begin{macrocode}
\newcommand\rep[1]{%
  (\raise.25ex\hbox{%
    \fontencoding{OMS}\fontfamily{cmsy}\selectfont\char\tw@%
   }#1)%
}
%    \end{macrocode}
% \end{macro}
%
% \subsection{Scripture Quotations}
%
% The macros in this section typeset scripture quotations and other
% between-songs environments.
%
% \begin{environment}{songgroup}\MainEnvImpl{songgroup}
% A |songgroup| environment associates all enclosed environments
% with the enclosed song.
% When generating a partial list, all the enclosed environments are
% contributed if and only if the enclosed song is contributed.
%    \begin{macrocode}
\newenvironment{songgroup}{%
  \ifnum\SB@grouplvl=\z@%
    \edef\SB@sgroup{\thesongnum}%
    \global\SB@groupcnt\m@ne%
  \fi%
  \advance\SB@grouplvl\@ne%
}{%
  \advance\SB@grouplvl\m@ne%
  \ifnum\SB@grouplvl=\z@\let\SB@sgroup\@empty\fi%
}
%    \end{macrocode}
% \end{environment}
%
% \begin{macro}{\SB@grouplvl}
% Count the |songgroup| environment nesting depth.
%    \begin{macrocode}
\SB@newcount\SB@grouplvl
%    \end{macrocode}
% \end{macro}
%
% \begin{environment}{intersong}\MainEnvImpl{intersong}
% An intersong block contributes vertical material to a column between the
% songs of a songs section.
% It is subject to the same column-breaking algorithm as real songs, but
% receives none of the other formatting applied to songs.
%    \begin{macrocode}
\newenvironment{intersong}{%
  \ifSB@insong\SB@errbro\SB@closeall\fi%
  \ifSB@intersong\SB@errbrr\SB@closeall\fi%
  \setbox\SB@chorusbox\box\voidb@x%
  \SB@intersongtrue%
  \def\SB@closeall{\end{intersong}}%
  \setbox\SB@songbox\vbox\bgroup\begingroup%
    \ifnum\SB@numcols>\z@\hsize\SB@colwidth\fi%
    \ifdim\sbarheight>\z@%
      \hrule\@height\sbarheight\@width\hsize%
      \nobreak%
    \fi%
}{%
  \ifSB@intersong
      \ifdim\sbarheight>\z@%
        \ifhmode\par\fi%
        \SB@skip\lastskip%
        \unskip\nobreak\vskip\SB@skip%
        \hbox{\vrule\@height\sbarheight\@width\hsize}%
      \fi%
    \endgroup\egroup%
    \ifSB@omitscrip%
      \setbox\SB@songbox\box\voidb@x%
    \else%
      \SB@submitsong%
    \fi%
    \SB@intersongfalse%
  \else%
    \ifSB@insong\SB@errero\SB@closeall\else\SB@errert\fi%
  \fi%
}
%    \end{macrocode}
% The starred form contributes page-spanning vertical material directly to
% the top of the nearest fresh page.
%    \begin{macrocode}
\newenvironment{intersong*}{%
  \ifSB@insong\SB@errbro\SB@closeall\fi%
  \ifSB@intersong\SB@errbrr\SB@closeall\fi%
  \setbox\SB@chorusbox\box\voidb@x%
  \SB@intersongtrue%
  \def\SB@closeall{\end{intersong*}}%
  \setbox\SB@songbox\vbox\bgroup\begingroup%
}{%
  \ifSB@intersong%
    \endgroup\egroup%
    \ifSB@omitscrip%
      \setbox\SB@songbox\box\voidb@x%
    \else%
      \def\SB@stype{\SB@styppage}%
      \SB@submitsong%
      \def\SB@stype{\SB@stypcol}%
    \fi%
    \SB@intersongfalse%
  \else%
    \ifSB@insong\SB@errero\SB@closeall\else\SB@errert\fi%
  \fi%
}
%    \end{macrocode}
% \end{environment}
%
% \begin{environment}{scripture}\MainEnvImpl{scripture}
% \begin{macro}{\beginscripture}
% Begin a scripture quotation.
% We first store the reference in a box for later use, and then set up
% a suitable environment for the quotation.
% Quotations cannot typically be reworded if line-breaking fails,
% so we set |\emergencystretch| to a relatively high value at the outset.
%    \begin{macrocode}
\newenvironment{scripture}{\beginscripture}{\SB@endscripture}
\newcommand\beginscripture[1]{%
  \begin{intersong}%
    \SB@parsesrefs{#1}%
    \setbox\SB@envbox\hbox{{\printscrcite\songrefs}}%
    \def\SB@closeall{\endscripture}%
    \nobreak\vskip5\p@%
    \SB@parindent\parindent\parindent\z@%
    \parskip\z@skip\parfillskip\@flushglue%
    \leftskip\SB@parindent\rightskip\SB@parindent\relax%
    \scripturefont\relax%
    \baselineskip\f@size\p@\@plus\p@\relax%
    \advance\baselineskip\p@\relax%
    \emergencystretch.3em%
}
%    \end{macrocode}
% \end{macro}
%
% \begin{macro}{\SB@endscripture}
% End a scripture quotation.
%    \begin{macrocode}
\newcommand\SB@endscripture{%
  \ifSB@intersong
      \scitehere%
      \ifhmode\par\fi%
      \vskip-3\p@%
    \end{intersong}%
  \fi%
}
%    \end{macrocode}
% \end{macro}
% \end{environment}
%
% \begin{macro}{\scitehere}\MainImpl{scitehere}
% \changes{v2.1}{2007/08/02}{Added}
% Usually the scripture citation should just come at the |\endscripture|
% line, but at times the user might want to invoke this macro explicitly
% at a more suitable point.
% A good example is when something near the end of the scripture quotation
% drops \TeX{} into vertical mode.
% In such cases, it is often better to issue the citation before leaving
% horizontal mode.
%
% In any case, this macro should work decently whether in horizontal or
% vertical mode.
% In horizontal mode life is easy: we just append the reference to the
% current horizontal list using the classic code from p.~106 of The \TeX book.
% However, if we're now in vertical mode, the problem is a little harder.
% We do the best we can by using |\lastbox| to remove the last line, then
% adding the reference and re-typesetting it.
% This isn't as good as the horizontal mode solution because \TeX{} only
% gets to reevaluate the last line instead of the whole paragraph, but
% usually the results are passable.
%    \begin{macrocode}
\newcommand\scitehere{%
  \ifSB@intersong%
    \ifvoid\SB@envbox\else%
      \ifvmode%
        \setbox\SB@box\lastbox%
        \nointerlineskip\noindent\hskip-\leftskip%
        \unhbox\SB@box\unskip%
      \fi%
      \unskip\nobreak\hfil\penalty50\hskip.8em\null\nobreak\hfil%
      \box\SB@envbox\kern-\SB@parindent%
      {\parfillskip\z@\finalhyphendemerits2000\par}%
    \fi%
  \else%
    \SB@errscrip\scitehere%
  \fi%
}
%    \end{macrocode}
% \end{macro}
%
% \begin{macro}{\Acolon}\MainImpl{Acolon}
% \begin{macro}{\Bcolon}\MainImpl{Bcolon}
% \changes{v1.13}{2005/05/12}{Added}
% Typeset a line of poetry in a scripture quotation.
%    \begin{macrocode}
\newcommand\Acolon{\SB@colon2\Acolon}
\newcommand\Bcolon{\SB@colon1\Bcolon}
%    \end{macrocode}
% \end{macro}
% \end{macro}
%
% \begin{macro}{\SB@colon}
% Begin a group of temporary definitions that will end at the next
% \Meta{return}.
% The \Meta{return} will end the paragraph and close the local scope.
%    \begin{macrocode}
\newcommand\SB@colon[2]{%
  \ifSB@intersong\else%
    \SB@errscrip#2%
    \beginscripture{Unknown}%
  \fi%
  \ifhmode\par\fi%
  \begingroup%
    \rightskip\SB@parindent\@plus4em%
    \advance\leftskip2\SB@parindent%
    \advance\parindent-#1\SB@parindent%
    \def\par{\endgraf\endgroup}%
    \obeylines%
}
%    \end{macrocode}
% \end{macro}
%
% \begin{macro}{\strophe}\MainImpl{strophe}
% \changes{v1.13}{2005/05/12}{Added}
% Insert blank space indicative of a strophe division in a scripture quotation.
%    \begin{macrocode}
\newcommand\strophe{%
  \ifSB@intersong\else%
    \SB@errscrip\strophe\beginscripture{Unknown}%
  \fi%
  \vskip.9ex\@plus.45ex\@minus.68ex\relax%
}
%    \end{macrocode}
% \end{macro}
%
% \begin{macro}{\scripindent}\MainImpl{scripindent}
% \begin{macro}{\scripoutdent}\MainImpl{scripoutdent}
% \begin{macro}{\SB@scripdent}
% \changes{v1.13}{2005/05/12}{Added}
% Create an indented sub-block within a scripture quotation.
%    \begin{macrocode}
\newcommand\SB@scripdent[2]{%
  \ifSB@intersong\else%
    \SB@errscrip#2\beginscripture{Unknown}%
  \fi%
  \ifhmode\par\fi%
  \advance\leftskip#1\SB@parindent\relax%
}
\newcommand\scripindent{\SB@scripdent1\scripindent}
\newcommand\scripoutdent{\SB@scripdent-\scripoutdent}
%    \end{macrocode}
% \end{macro}
% \end{macro}
% \end{macro}
%
% \begin{macro}{\shiftdblquotes}\MainImpl{shiftdblquotes}
% \changes{v1.13}{2005/05/12}{Added}
% \begin{macro}{\SB@ldqleft}
% \begin{macro}{\SB@ldqright}
% \begin{macro}{\SB@rdqleft}
% \begin{macro}{\SB@rdqright}
% \begin{macro}{\SB@scanlq}
% \begin{macro}{\SB@scanrq}
% \begin{macro}{\SB@dolq}
% \begin{macro}{\SB@dorq}
% The Zaph Chancery font used by default to typeset scripture quotations
% seems to have some kerning problems with double-quote ligatures. The
% |\shiftdblquotes| macro allows one to modify the spacing around all
% double-quotes until the current group ends.
%    \begin{macrocode}
\newcommand\SB@quotesactive{%
  \catcode`'\active%
  \catcode``\active%
}
\newcommand\shiftdblquotes[4]{}
\newcommand\SB@ldqleft{}
\newcommand\SB@ldqright{}
\newcommand\SB@rdqleft{}
\newcommand\SB@rdqright{}
\newcommand\SB@scanlq{}
\newcommand\SB@scanrq{}
\newcommand\SB@dolq{}
\newcommand\SB@dorq{}
{
  \SB@quotesactive
  \gdef\shiftdblquotes#1#2#3#4{%
    \def\SB@ldqleft{\kern#1}%
    \def\SB@ldqright{\kern#2}%
    \def\SB@rdqleft{\kern#3}%
    \def\SB@rdqright{\kern#4}%
    \SB@quotesactive%
    \def`{\futurelet\SB@next\SB@scanlq}%
    \def'{\futurelet\SB@next\SB@scanrq}%
  }
  \gdef\SB@scanlq{%
    \ifx\SB@next`%
      \expandafter\SB@dolq%
    \else%
      \expandafter\lq%
    \fi%
  }
  \gdef\SB@scanrq{%
    \ifx\SB@next'%
      \expandafter\SB@dorq%
    \else%
      \expandafter\rq%
    \fi%
  }
  \gdef\SB@dolq`{%
    \ifvmode\leavevmode\else\/\fi%
    \vadjust{}%
    \SB@ldqleft\lq\lq\SB@ldqright%
    \vadjust{}%
  }
  \gdef\SB@dorq'{%
    \ifvmode\leavevmode\else\/\fi%
    \vadjust{}%
    \SB@rdqleft\rq\rq\SB@rdqright%
    \vadjust{}%
  }
}
%    \end{macrocode}
% \end{macro}
% \end{macro}
% \end{macro}
% \end{macro}
% \end{macro}
% \end{macro}
% \end{macro}
% \end{macro}
% \end{macro}
%
% \subsection{Transposition}
%
% The macros that transpose chords are contained in this section.
%
% \begin{macro}{\SB@transposefactor}
% This counter identifies the requested number of halfsteps by which chords are
% to be transposed (from $-11$ to $+11$).
%    \begin{macrocode}
\SB@newcount\SB@transposefactor
%    \end{macrocode}
% \end{macro}
%
% \begin{macro}{\ifSB@convertnotes}
% Even when transposition is not requested, the transposition logic can be
% used to automatically convert note names to another form.
% The following conditional turns that feature on or off.
%    \begin{macrocode}
\newif\ifSB@convertnotes
%    \end{macrocode}
% \end{macro}
%
% \begin{macro}{\notenameA}
% \begin{macro}{\notenameB}
% \begin{macro}{\notenameC}
% \begin{macro}{\notenameD}
% \begin{macro}{\notenameE}
% \begin{macro}{\notenameF}
% \begin{macro}{\notenameG}
% Reserve a control sequence for each note of the diatonic scale.
% These will be used to identify which token sequences the input file uses
% to denote the seven scale degrees.
% Their eventual definitions \emph{must} consist entirely of uppercase
% letters, and they must be assigned using |\def|, but that comes later.
%    \begin{macrocode}
\newcommand\notenameA{}
\newcommand\notenameB{}
\newcommand\notenameC{}
\newcommand\notenameD{}
\newcommand\notenameE{}
\newcommand\notenameF{}
\newcommand\notenameG{}
%    \end{macrocode}
% \end{macro}
% \end{macro}
% \end{macro}
% \end{macro}
% \end{macro}
% \end{macro}
% \end{macro}
%
% \begin{macro}{\printnoteA}
% \begin{macro}{\printnoteB}
% \begin{macro}{\printnoteC}
% \begin{macro}{\printnoteD}
% \begin{macro}{\printnoteE}
% \begin{macro}{\printnoteF}
% \begin{macro}{\printnoteG}
% These control sequences are what the transposition logic actually
% outputs to denote each scale degree.
% They can include any \LaTeX{} code that is legal in horizontal mode.
%    \begin{macrocode}
\newcommand\printnoteA{}
\newcommand\printnoteB{}
\newcommand\printnoteC{}
\newcommand\printnoteD{}
\newcommand\printnoteE{}
\newcommand\printnoteF{}
\newcommand\printnoteG{}
%    \end{macrocode}
% \end{macro}
% \end{macro}
% \end{macro}
% \end{macro}
% \end{macro}
% \end{macro}
% \end{macro}
%
% \begin{macro}{\notenamesin}\MainImpl{notenamesin}
% Set the note names used by the input file.
%    \begin{macrocode}
\newcommand\notenamesin[7]{%
  \def\notenameA{#1}%
  \def\notenameB{#2}%
  \def\notenameC{#3}%
  \def\notenameD{#4}%
  \def\notenameE{#5}%
  \def\notenameF{#6}%
  \def\notenameG{#7}%
  \SB@convertnotestrue%
}
%    \end{macrocode}
% \end{macro}
%
% \begin{macro}{\notenamesout}\MainImpl{notenamesout}
% Set the note names that are output by the transposition logic.
%    \begin{macrocode}
\newcommand\notenamesout[7]{%
  \def\printnoteA{#1}%
  \def\printnoteB{#2}%
  \def\printnoteC{#3}%
  \def\printnoteD{#4}%
  \def\printnoteE{#5}%
  \def\printnoteF{#6}%
  \def\printnoteG{#7}%
  \SB@convertnotestrue%
}
%    \end{macrocode}
% \end{macro}
%
% \begin{macro}{\notenames}\MainImpl{notenames}
% Set an identical input name and output name for each scale degree.
%    \begin{macrocode}
\newcommand\notenames[7]{%
  \notenamesin{#1}{#2}{#3}{#4}{#5}{#6}{#7}%
  \notenamesout{#1}{#2}{#3}{#4}{#5}{#6}{#7}%
  \SB@convertnotesfalse%
}
%    \end{macrocode}
% \end{macro}
%
% \begin{macro}{\alphascale}\MainImpl{alphascale}
% \begin{macro}{\solfedge}\MainImpl{solfedge}
% Predefine scales for alphabetic names and solfedge names, and
% set alphabetic scales to be the default.
%    \begin{macrocode}
\newcommand\alphascale{\notenames ABCDEFG}
\newcommand\solfedge{\notenames{LA}{SI}{DO}{RE}{MI}{FA}{SOL}}
\alphascale
%    \end{macrocode}
% \end{macro}
% \end{macro}
%
% \begin{macro}{\ifSB@prefshrps}
% When a transposed chord falls on a black key, the code must choose which
% enharmonically equivalent name to give the new chord.
% (For example, should C transposed by +1 be named C$\#$ or D$\flat$?)
% A heuristic is used to guess which name is most appropriate.
% The following conditional records whether the current key signature is
% sharped or flatted according to this heuristic guess.
%    \begin{macrocode}
\newif\ifSB@prefshrps
%    \end{macrocode}
% \end{macro}
%
% \begin{macro}{\ifSB@needkey}
% The first chord seen is usually the best indicator of the key of the song.
% (Even when the first chord isn't the tonic, it will often be the dominant
% or subdominant, which usually has the same kind of accidental in its key
% signatures as the actual key.) This conditional remembers whether the current
% chord is the first one seen in the song, and should therefore be used to
% guess the key of the song.
%    \begin{macrocode}
\newif\ifSB@needkey
%    \end{macrocode}
% \end{macro}
%
% \begin{macro}{\transpose}\MainImpl{transpose}
% The |\transpose| macro sets the transposition adjustment factor and
% informs the transposition logic that the next chord seen will be the first
% one in the new key.
%    \begin{macrocode}
\newcommand\transpose[1]{%
  \advance\SB@transposefactor by#1\relax%
  \SB@cnt\SB@transposefactor%
  \divide\SB@cnt12 %
  \multiply\SB@cnt12 %
  \advance\SB@transposefactor-\SB@cnt%
  \SB@needkeytrue%
}
%    \end{macrocode}
% \end{macro}
%
% \begin{macro}{\capo}\MainImpl{capo}
% Specifying a |\capo| normally just causes a textual note to musicians to be
% typeset, but if the |transposecapos| option is active, it activates
% transposition of the chords.
%    \begin{macrocode}
\newcommand\capo[1]{%
  \iftranscapos\transpose{#1}\else\musicnote{capo #1}\fi%
}
%    \end{macrocode}
% \end{macro}
%
% \begin{macro}{\prefersharps}\MainImpl{prefersharps}
% \begin{macro}{\preferflats}\MainImpl{preferflats}
% One of these macros is called after the first chord has been seen to
% register that we're transposing to a key with a sharped or flatted key
% signature.
%    \begin{macrocode}
\newcommand\prefersharps{\SB@prefshrpstrue\SB@needkeyfalse}
\newcommand\preferflats{\SB@prefshrpsfalse\SB@needkeyfalse}
%    \end{macrocode}
% \end{macro}
% \end{macro}
%
% \begin{macro}{\transposehere}\MainImpl{transposehere}
% If automatic transposition has been requested, yield the given chord
% transposed by the requested amount.
% Otherwise return the given chord verbatim.
%    \begin{macrocode}
\newcommand\transposehere[1]{%
  \ifnum\SB@transposefactor=\z@%
    \ifSB@convertnotes%
      \SB@dotranspose{#1}%
      \the\SB@toks%
    \else%
      #1%
    \fi%
  \else%
    \ifSB@convertnotes%
      {\SB@transposefactor\z@%
       \SB@dotranspose{#1}%
       \xdef\SB@tempv{\the\SB@toks}}%
    \else%
      \def\SB@tempv{#1}%
    \fi%
    \SB@dotranspose{#1}%
    \expandafter\trchordformat\expandafter{\SB@tempv}{\the\SB@toks}%
  \fi%
}
%    \end{macrocode}
% \end{macro}
%
% \begin{macro}{\notrans}\MainImpl{notrans}
% Suppress chord transposition without suppressing note name conversion.
% When a |\notrans{|\Meta{text}|}| macro appears within text undergoing
% transposition, the |\notrans| macro and the group will be preserved
% verbatim by the transposition parser.
% When it is then expanded after parsing, we must therefore re-invoke
% the transposition logic on the argument, but in an environment where
% the transposition factor has been temporarily set to zero.
% This causes note name conversion to occur without actually transposing.
%    \begin{macrocode}
\newcommand\notrans[1]{%
  \begingroup%
    \SB@transposefactor\z@%
    \transposehere{#1}%
  \endgroup%
}
%    \end{macrocode}
% \end{macro}
%
% \begin{macro}{\SB@dotranspose}
% Parse the argument to a chord macro, yielding the transposed equivalent in
% the |\SB@toks| token register.
%    \begin{macrocode}
\newcommand\SB@dotranspose[1]{%
  \SB@toks{}%
  \let\SB@dothis\SB@trmain%
  \SB@trscan#1\SB@trend%
}
%    \end{macrocode}
% \end{macro}
%
% \begin{macro}{\trchordformat}\MainImpl{trchordformat}
% By default, transposing means replacing old chords with new chords in the
% new key. However, sometimes the user may want to typeset something more
% sophisticated, like old chords followed by new chords in parentheses so
% that musicians who use capos and those who don't can play from the same
% piece of music. Such typesetting is possible by redefining the following
% macro to something like |#1 (#2)| instead of |#2|.
%    \begin{macrocode}
\newcommand\trchordformat[2]{#2}
%    \end{macrocode}
% \end{macro}
%
% \begin{macro}{\SB@trscan}
% This is the entrypoint to the code that scans over the list of tokens
% comprising a chord and transposes note names as it goes.
% Start by peeking ahead at the next symbol without consuming it.
%    \begin{macrocode}
\newcommand\SB@trscan{\futurelet\SB@next\SB@dothis}
%    \end{macrocode}
% \end{macro}
%
% \begin{macro}{\SB@trmain}
% Test to see whether the token was a begin-brace, end-brace, or space.
% These tokens require special treatment because they cannot be
% accepted as implicit arguments to macros.
%    \begin{macrocode}
\newcommand\SB@trmain{%
  \ifx\SB@next\bgroup%
    \let\SB@donext\SB@trgroup%
  \else\ifx\SB@next\egroup%
    \SB@toks\expandafter{\the\SB@toks\egroup}%
    \let\SB@donext\SB@trskip%
  \else\ifcat\noexpand\SB@next\@sptoken%
    \SB@appendsp\SB@toks%
    \let\SB@donext\SB@trskip%
  \else%
    \let\SB@donext\SB@trstep%
  \fi\fi\fi%
  \SB@donext%
}
%    \end{macrocode}
% \end{macro}
%
% \begin{macro}{\SB@trgroup}
% A begin-group brace lies next in the input stream.
% Consume the entire group as an argument to this macro, and append it,
% including the begin- and end-group tokens, to the list of tokens processed
% so far.
% No transposition takes place within a group; they are copied verbatim
% because they probably contain macro code.
%    \begin{macrocode}
\newcommand\SB@trgroup[1]{%
  \SB@toks\expandafter{\the\SB@toks{#1}}%
  \SB@trscan%
}
%    \end{macrocode}
% \end{macro}
%
% \begin{macro}{\SB@trskip}
% A space or end-brace lies next in the input stream.
% It has already been added to the token list, so skip over it.
%    \begin{macrocode}
\newcommand\SB@trskip{%
  \afterassignment\SB@trscan%
  \let\SB@next= }
%    \end{macrocode}
% \end{macro}
%
% \begin{macro}{\SB@trstep}
% A non-grouping token lies next in the input stream.
% Consume it as an argument to this macro, and then test it to see whether
% it's a note letter or some other recognized item.
% If so, process it; otherwise just append it to the token list and continue
% scanning.
%    \begin{macrocode}
\newcommand\SB@trstep[1]{%
  \let\SB@donext\SB@trscan%
  \ifcat\noexpand\SB@next A%
    \ifnum\uccode`#1=`#1%
      \def\SB@temp{#1}%
      \let\SB@dothis\SB@trnote%
    \else%
      \SB@toks\expandafter{\the\SB@toks#1}%
    \fi%
  \else\ifx\SB@next\SB@trend
    \let\SB@donext\relax%
  \else%
    \SB@toks\expandafter{\the\SB@toks#1}%
  \fi\fi%
  \SB@donext%
}
%    \end{macrocode}
% \end{macro}
%
% \begin{macro}{\SB@trnote}
% We're in the midst of processing a sequence of uppercase letters that
% might comprise a note name.
% Check to see whether the next token is an accidental (sharp or flat),
% or yet another letter.
%    \begin{macrocode}
\newcommand\SB@trnote{%
  \ifcat\noexpand\SB@next A%
    \let\SB@donext\SB@trnotestep%
  \else\ifnum\SB@transposefactor=\z@%
    \SB@cnt\z@%
    \let\SB@donext\SB@trtrans%
  \else\ifx\SB@next\flt%
    \SB@cnt\m@ne%
    \let\SB@donext\SB@tracc%
  \else\ifx\SB@next\shrp%
    \SB@cnt\@ne%
    \let\SB@donext\SB@tracc%
  \else%
    \SB@cnt\z@%
    \let\SB@donext\SB@trtrans%
  \fi\fi\fi\fi%
  \SB@donext%
}
%    \end{macrocode}
% \end{macro}
%
% \begin{macro}{\SB@trnotestep}
% The next token is a letter.
% Consume it and test to see whether it is an uppercase letter.
% If so, add it to the note name being assembled; otherwise reinsert it into
% the input stream and jump directly to the transposition logic.
%    \begin{macrocode}
\newcommand\SB@trnotestep[1]{%
  \ifnum\uccode`#1=`#1%
    \SB@app\def\SB@temp{#1}%
    \expandafter\SB@trscan%
  \else%
    \SB@cnt\z@%
    \expandafter\SB@trtrans\expandafter#1%
  \fi%
}
%    \end{macrocode}
% \end{macro}
%
% \begin{macro}{\SB@tracc}
% We've encountered an accidental (sharp or flat) immediately following a
% note name.
% Peek ahead at the next token without consuming it, and then jump to the
% transposition logic.
% This is done because the transposition logic might need to infer the key
% signature of the song, and if the next token is an m (for minor), then
% that information can help.
%    \begin{macrocode}
\newcommand\SB@tracc[1]{\futurelet\SB@next\SB@trtrans}
%    \end{macrocode}
% \end{macro}
%
% \begin{macro}{\SB@trtrans}
% We've assembled a sequence of capital letters (in |\SB@temp|) that might
% comprise a note name to be transposed.
% If the letters were followed by a |\shrp| then |\SB@cnt| is 1; if they were
% followed by a |\flt| then it is $-1$; otherwise it is 0.
% If the assembled letters turn out to not match any valid note name, then
% do nothing and return to scanning.
% Otherwise compute a new transposed name.
%    \begin{macrocode}
\newcommand\SB@trtrans{%
  \advance\SB@cnt%
    \ifx\SB@temp\notenameA\z@%
    \else\ifx\SB@temp\notenameB\tw@%
    \else\ifx\SB@temp\notenameC\thr@@%
    \else\ifx\SB@temp\notenameD5 %
    \else\ifx\SB@temp\notenameE7 %
    \else\ifx\SB@temp\notenameF8 %
    \else\ifx\SB@temp\notenameG10 %
    \else-99 \fi\fi\fi\fi\fi\fi\fi%
  \ifnum\SB@cnt<\m@ne%
    \SB@toks\expandafter\expandafter\expandafter{%
      \expandafter\the\expandafter\SB@toks\SB@temp}%
  \else%
    \advance\SB@cnt\SB@transposefactor%
    \ifnum\SB@cnt<\z@\advance\SB@cnt12 \fi%
    \ifnum\SB@cnt>11 \advance\SB@cnt-12 \fi%
    \ifSB@needkey\ifnum\SB@transposefactor=\z@\else\SB@setkeysig\fi\fi%
    \edef\SB@temp{%
      \the\SB@toks%
      \ifSB@prefshrps%
        \ifcase\SB@cnt\printnoteA\or\printnoteA\noexpand\shrp\or%
          \printnoteB\or\printnoteC\or\printnoteC\noexpand\shrp\or%
          \printnoteD\or\printnoteD\noexpand\shrp\or\printnoteE\or%
          \printnoteF\or\printnoteF\noexpand\shrp\or\printnoteG\or%
          \printnoteG\noexpand\shrp\fi%
      \else%
        \ifcase\SB@cnt\printnoteA\or\printnoteB\noexpand\flt\or%
          \printnoteB\or\printnoteC\or\printnoteD\noexpand\flt\or%
          \printnoteD\or\printnoteE\noexpand\flt\or\printnoteE\or%
          \printnoteF\or\printnoteG\noexpand\flt\or\printnoteG\or%
          \printnoteA\noexpand\flt\fi%
      \fi}%
    \SB@toks\expandafter{\SB@temp}%
  \fi%
  \let\SB@dothis\SB@trmain%
  \SB@trscan%
}
%    \end{macrocode}
% \end{macro}
%
% \begin{macro}{\SB@setkeysig}
% If this is the first chord of the song, assume that this is the tonic of the
% key, and select whether to use a sharped or flatted key signature for the
% rest of the song based on that.
% Even if this isn't the tonic, it's probably the dominant or sub-dominant,
% which almost always has a number of sharps or flats similar to the tonic.
% If the bottom note of the chord turns out to be a black key, we choose the
% enharmonic equivalent that is closest to C on the circle of fifths
% (i.e., the one that has fewest sharps or flats).
%    \begin{macrocode}
\newcommand\SB@setkeysig{%
  \global\SB@needkeyfalse%
  \ifcase\SB@cnt%
    \global\SB@prefshrpstrue\or% A
    \global\SB@prefshrpsfalse\or% Bb
    \global\SB@prefshrpstrue\or% B
    \ifx\SB@next m% C
      \global\SB@prefshrpsfalse%
    \else%
      \global\SB@prefshrpstrue%
    \fi\or%
    \global\SB@prefshrpstrue\or% C#
    \ifx\SB@next m% D
      \global\SB@prefshrpsfalse%
    \else%
      \global\SB@prefshrpstrue%
    \fi\or%
    \global\SB@prefshrpsfalse\or% Eb
    \global\SB@prefshrpstrue\or% E
    \global\SB@prefshrpsfalse\or% F
    \global\SB@prefshrpstrue\or% F#
    \ifx\SB@next m% G
      \global\SB@prefshrpsfalse%
    \else%
      \global\SB@prefshrpstrue%
    \fi\or%
    \global\SB@prefshrpsfalse\else% Ab
    \global\SB@needkeytrue% non-chord
  \fi%
}
%    \end{macrocode}
% \end{macro}
%
% \begin{macro}{\SB@trend}
% The following macro marks the end of chord text to be processed.
% It should always be consumed and discarded by the chord-scanning
% logic above, so generate an error if it is ever expanded.
%    \begin{macrocode}
\newcommand\SB@trend{%
  \SB@Error{Internal Error: Transposition failed}%
           {This error should not occur.}%
}
%    \end{macrocode}
% \end{macro}
%
% \subsection{Measure Bars}
%
% The following code handles the typesetting of measure bars.
%
% \begin{macro}{\SB@metertop}
% \begin{macro}{\SB@meterbot}
% These macros remember the current numerator and denominator of the meter.
%    \begin{macrocode}
\newcommand\SB@metertop{}
\newcommand\SB@meterbot{}
%    \end{macrocode}
% \end{macro}
% \end{macro}
%
% \begin{macro}{\meter}\MainImpl{meter}
% Set the current meter without producing an actual measure bar yet.
%    \begin{macrocode}
\newcommand\meter[2]{\gdef\SB@metertop{#1}\gdef\SB@meterbot{#2}}
%    \end{macrocode}
% \end{macro}
%
% \begin{macro}{\SB@measuremark}
% Normally measure bar boxes should be as thin as possible so that they can be
% slipped into lyrics without making them hard to read. But when two measure
% bars appear consecutively, they need to be spaced apart more so that they
% look like two separate lines instead of one thick line. To achieve this,
% there needs to be a way to pull a vbox off the current list and determine
% whether or not it is a box that contains a measure bar. The solution is to
% insert a mark (|\SB@measuremark|) at the top of each measure bar vbox.
% We can then see whether this measure bar immediately follows another measure
% bar by using |\vsplit| on |\lastbox|.
%    \begin{macrocode}
\newcommand\SB@measuremark{SB@IsMeasure}
%    \end{macrocode}
% \end{macro}
%
% \begin{macro}{\SB@makembar}\MainImpl{mbar}
% Typeset a measure bar. If provided, \argp{1} is the numerator and \argp{2} is
% the denominator of the meter to be rendered above the bar. If those arguments
% are left blank, render a measure bar without a meter marking.
%    \begin{macrocode}
\newcommand\SB@makembar[2]{%
  \ifSB@inverse\else%
    \ifSB@inchorus\else\SB@errmbar\fi%
  \fi%
  \ifhmode%
    \SB@skip\lastskip\unskip%
    \setbox\SB@box\lastbox%
    \copy\SB@box%
    \ifvbox\SB@box%
      \begingroup%
        \setbox\SB@boxii\copy\SB@box%
        \vbadness\@M\vfuzz\maxdimen%
        \setbox\SB@boxii%
          \vsplit\SB@boxii to\maxdimen%
      \endgroup%
      \long\edef\SB@temp{\splitfirstmark}%
      \ifx\SB@temp\SB@measuremark%
        \penalty100\hskip1em%
      \else%
        \penalty100\hskip\SB@skip%
      \fi%
    \else%
      \penalty100\hskip\SB@skip%
    \fi%
  \fi%
  \ifvmode\leavevmode\fi%
  \setbox\SB@box\hbox{{\meterfont\relax#1}}%
  \setbox\SB@boxii\hbox{{\meterfont\relax#2}}%
  \SB@dimen\wd\ifdim\wd\SB@box>\wd\SB@boxii\SB@box\else\SB@boxii\fi%
  \SB@dimenii\baselineskip%
  \advance\SB@dimenii-2\p@%
  \advance\SB@dimenii-\ht\SB@box%
  \advance\SB@dimenii-\dp\SB@box%
  \advance\SB@dimenii-\ht\SB@boxii%
  \advance\SB@dimenii-\dp\SB@boxii%
  \let\SB@temp\relax%
  \ifdim\SB@dimen>\z@%
    \advance\SB@dimenii-.75\p@%
    \def\SB@temp{\kern.75\p@}%
  \fi%
  \SB@maxmin\SB@dimen<{.5\p@}%
  \SB@maxmin\SB@dimenii<\z@%
  \vbox{%
    \mark{\SB@measuremark}%
    \hbox to\SB@dimen{%
      \hfil%
      \box\SB@box%
      \hfil%
    }%
    \nointerlineskip%
    \hbox to\SB@dimen{%
      \hfil%
      \box\SB@boxii%
      \hfil%
    }%
    \SB@temp%
    \nointerlineskip%
    \hbox to\SB@dimen{%
      \hfil%
      \vrule\@width.5\p@\@height\SB@dimenii%
      \hfil%
    }%
  }%
  \meter{}{}%
}
%    \end{macrocode}
% \end{macro}
%
% \begin{macro}{\mbar}
% The |\mbar| macro invokes |\SB@mbar|, which gets redefined by macros and
% options that turn measure bars on and off.
%    \begin{macrocode}
\newcommand\mbar{\SB@mbar}
%    \end{macrocode}
% \end{macro}
%
% \begin{macro}{\measurebar}\MainImpl{measurebar}
% Make a measure bar using the most recently defined meter.
% Then set the meter to nothing so that the next measure bar will not
% display any meter unless the meter changes.
%    \begin{macrocode}
\newcommand\measurebar{%
  \mbar\SB@metertop\SB@meterbot%
}
%    \end{macrocode}
% \end{macro}
%
% \begin{macro}{\SB@repcolon}
% Create the colon that preceeds or follows a repeat sign.
%    \begin{macrocode}
\newcommand\SB@repcolon{{%
  \usefont{OT1}{cmss}{m}{n}\selectfont%
  \ifchorded%
    \baselineskip.5\SB@dimen%
    \vbox{\hbox{:}\hbox{:}\kern.5\p@}%
  \else%
    \raise.5\p@\hbox{:}%
  \fi%
}}
%    \end{macrocode}
% \end{macro}
%
% \begin{macro}{\lrep}\MainImpl{lrep}
% Create a begin-repeat sign.
%    \begin{macrocode}
\newcommand\lrep{%
  \SB@dimen\baselineskip%
  \advance\SB@dimen-2\p@%
  \vrule\@width1.5\p@\@height\SB@dimen\@depth\p@%
  \kern1.5\p@%
  \vrule\@width.5\p@\@height\SB@dimen\@depth\p@%
  \SB@repcolon%
}
%    \end{macrocode}
% \end{macro}
%
% \begin{macro}{\rrep}\MainImpl{rrep}
% Create an end-repeat sign.
%    \begin{macrocode}
\newcommand\rrep{%
  \SB@dimen\baselineskip%
  \advance\SB@dimen-2\p@%
  \SB@repcolon%
  \vrule\@width.5\p@\@height\SB@dimen\@depth\p@%
  \kern1.5\p@%
  \vrule\@width1.5\p@\@height\SB@dimen\@depth\p@%
}
%    \end{macrocode}
% \end{macro}
%
% \subsection{Lyric Scanning}\label{sec:lyricscan}
%
% The obvious way to create a chord macro is as a normal macro with
% two arguments, one for the chord name and one for the lyrics to go
% under the chord---e.g.~|\chord{|\Meta{chordname}|}{|\Meta{lyric}|}|.
% However, in practice such a macro is extremely cumbersome and
% difficult to use.
% The problem is that in order to use such a macro properly, the user
% must remember a bunch of complex style rules that govern what
% part of the lyric text needs to go in the \Meta{lyric} parameter and
% what part should be typed after the closing brace.
% To avoid separating a word from its trailing punctuation, the
% \Meta{lyric} parameter must often include punctuation but not certain
% special punctuation like hyphens, should include the rest of the
% word but not if there's another chord in the word, should omit
% measure bars but only if measure bars are being shown, etc.
% This is way too difficult for the average user.
%
% To avoid this problem, we define chords using a one-argument macro
% (the argument is the chord name), but with no explicit argument for
% the lyric part.
% Instead, the macro scans ahead in the input stream, automatically
% determining what portion of the lyric text that follows should be
% sucked in as an implicit second argument.
% The following code does this look-ahead scanning.
%
% \begin{macro}{\ifSB@wordends}
% \begin{macro}{\ifSB@brokenword}
% Chord macros must look ahead in the input stream to see whether this chord
% is immediately followed by whitespace or the remainder of a word.
% If the latter, hyphenation might need to be introduced.
% These macros keep track of the need for hyphenation, if any.
%    \begin{macrocode}
\newif\ifSB@wordends
\newif\ifSB@brokenword
%    \end{macrocode}
% \end{macro}
% \end{macro}
%
% \begin{macro}{\SB@lyric}
% Lyrics appearing after a chord are scanned into the following token list
% register.
%    \begin{macrocode}
\SB@newtoks\SB@lyric
%    \end{macrocode}
% \end{macro}
%
% \begin{macro}{\SB@numhyps}
% Hyphens appearing in lyrics require special treatment.
% The following counter counts the number of explicit hyphens ending
% the lyric syllable that follows the current chord.
%    \begin{macrocode}
\SB@newcount\SB@numhyps
%    \end{macrocode}
% \end{macro}
%
% \begin{macro}{\SB@lyricnohyp}
% When a lyric syllable under a chord ends in exactly one hyphen, the
% following token register is set to be the syllable without the hyphen.
%    \begin{macrocode}
\SB@newtoks\SB@lyricnohyp
%    \end{macrocode}
% \end{macro}
%
% \begin{macro}{\SB@lyricbox}
% \begin{macro}{\SB@chordbox}
% The following two boxes hold the part of the lyric text that is to be
% typeset under the chord, and the chord text that is to be typeset above.
%    \begin{macrocode}
\SB@newbox\SB@lyricbox
\SB@newbox\SB@chordbox
%    \end{macrocode}
% \end{macro}
% \end{macro}
%
% \begin{macro}{\SB@chbstok}
% \changes{v1.22}{2007/05/15}{Added.}
% When |\MultiwordChords| is active, the following reserved control
% sequence remembers the first (space) token not yet included into the
% |\SB@lyricbox| box.
%    \begin{macrocode}
\newcommand\SB@chbstok{}
%    \end{macrocode}
% \end{macro}
%
% \begin{macro}{\SB@setchord}
% \changes{v2.3}{2007/09/23}{Support replayed chords over ligatures}
% \changes{v2.7}{2009/01/08}{Extend rather than replace chordbox}
% \changes{v3.0}{2017/04/19}{Fix transposition of replayed chord over ligature}
% The following macro typesets its argument as a chord and stores the
% result in box |\SB@chordbox| for later placement into the document.
% The hat token (|^|) is redefined so that outside of math mode it
% suppresses chord memorization, but inside of math mode it retains
% its usual superscript meaning.
% If memorization is active, the chord's token sequence is stored in
% the current replay register.
% If |\SB@chordbox| is non-empty, the new chord is appended to it
% rather than replacing it.
% This allows consecutive chords not separated by whitespace to be
% typeset as a single chord sequence atop a common lyric.
%    \begin{macrocode}
\newcommand\SB@setchord{}
{
  \catcode`^\active
  \gdef\SB@setchord#1{%
    \SB@gettabindtrue\SB@nohattrue%
    \setbox\SB@chordbox\hbox{%
      \unhbox\SB@chordbox%
      \begingroup%
        \ifSB@trackch%
          \let\SB@activehat\SB@hat@tr%
        \else%
          \let\SB@activehat\SB@hat@notr%
        \fi%
        \let^\SB@activehat%
        \printchord{%
          \ifSB@firstchord\else\kern.15em\fi%
          \vphantom/%
          \transposehere{#1}%
          \kern.2em%
        }%
      \endgroup%
    }%
    \SB@gettabindfalse%
    \ifSB@trackch\ifSB@nohat%
      \global\SB@creg\expandafter{\the\SB@creg#1\\}%
    \fi\fi%
    \let\SB@noreplay\@firstofone%
  }
}
%    \end{macrocode}
% \end{macro}
%
% \begin{macro}{\SB@outertest}
% \begin{macro}{\SB@otesta}
% \begin{macro}{\SB@otestb}
% The lyric-scanning code must preemptively determine whether the next token
% is a macro declared |\outer| before it tries to accept that token as an
% argument.
% Otherwise \TeX{} will abort with a parsing error.
% Macros declared |\outer| are not allowed in arguments, so determining
% whether a token is |\outer| is a delicate process.
% The following does so by consulting |\meaning|.
% A macro can be identified as |\outer| if its meaning has the
% word ``|\outer|'' before the first colon.
%    \begin{macrocode}
\newcommand\SB@outertest{%
  \expandafter\SB@otesta\meaning\SB@next:\SB@otesta%
}
\newcommand\SB@otesta{}
\edef\SB@otesta#1:#2\SB@otesta{%
  \noexpand\SB@otestb%
  #1\string\outer%
  \noexpand\SB@otestb%
}
\newcommand\SB@otestb{}
\expandafter\def\expandafter\SB@otestb%
\expandafter#\expandafter1\string\outer#2\SB@otestb{%
  \def\SB@temp{#2}%
  \ifx\SB@temp\@empty\SB@testfalse\else\SB@testtrue\fi%
}
%    \end{macrocode}
% \end{macro}
% \end{macro}
% \end{macro}
%
% \begin{macro}{\SB@UTFtest}
% \begin{macro}{\SB@U@two}
% \begin{macro}{\SB@U@three}
% \begin{macro}{\SB@U@four}
% \begin{macro}{\SB@@UTFtest}
% \changes{v1.22}{2007/05/15}{Added.}
% To support UTF-8 encoded \LaTeX{} source files, we need to be able to
% identify multibyte characters during the lyric scanning process.
% Alas, the |utf8.def| file provides no clean way of identifying the
% macros it defines for this purpose.
% The best solution seems to be to look for any token named
% |\UTFviii@|$\ldots$|@octets| in the top-level expansion of the macro.
%    \begin{macrocode}
\newcommand\SB@UTFtest{}
\edef\SB@UTFtest#1{%
  \noexpand\expandafter%
  \noexpand\SB@@UTFtest%
  \noexpand\meaning#1%
  \string\UTFviii@zero@octets%
  \noexpand\SB@@UTFtest%
}
\newcommand\SB@U@two{\global\SB@cnt\tw@}
\newcommand\SB@U@three{\global\SB@cnt\thr@@}
\newcommand\SB@U@four{\global\SB@cnt4\relax}
\newcommand\SB@@UTFtest{}
{\escapechar\m@ne
 \xdef\SB@temp{\string\@octets}}
\edef\SB@temp{##1\string\UTFviii@##2\SB@temp}
\expandafter\def\expandafter\SB@@UTFtest\SB@temp#3\SB@@UTFtest{%
  \SB@cnt\z@%
  {\csname SB@U@#2\endcsname}%
}
%    \end{macrocode}
% \end{macro}
% \end{macro}
% \end{macro}
% \end{macro}
% \end{macro}
%
% \begin{macro}{\DeclareLyricChar}\MainImpl{DeclareLyricChar}
% \begin{macro}{\DeclareNonLyric}\MainImpl{DeclareNonLyric}
% \begin{macro}{\DeclareNoHyphen}\MainImpl{DeclareNoHyphen}
% \begin{macro}{\SB@declare}
% \changes{v1.22}{2007/05/15}{Added \cs{DeclareLyricChar}.}
% \changes{v2.1}{2007/08/02}{Added \cs{DeclareNonLyric} and \cs{DeclareNoHyphen}.}
% \changes{v2.6}{2008/03/27}{Macro tests made name-based instead of def-based}
% When scanning the lyric text that follows a chord, it is necessary to
% distinguish accents and other intra-word macros (which should be included
% in the under-chord lyric text) from other macros (which should be pushed
% out away from the text).
% The following macros allow users to declare a token to be lyric-continuing
% or lyric-ending.
%    \begin{macrocode}
\newcommand\SB@declare[3]{%
  \afterassignment\iffalse\let\SB@next= #3\relax\fi%
  \SB@UTFtest\SB@next%
  \ifcase\SB@cnt%
    \ifcat\noexpand#3\relax%
      \SB@addNtest\SB@macrotests#1#2#3%
    \else\ifcat\noexpand#3.%
      \SB@addDtest\SB@othertests#1#2%
    \else\ifcat\noexpand#3A%
      \SB@addDtest\SB@lettertests#1#2%
    \else%
      \SB@addDtest\relax0#2%
    \fi\fi\fi%
  \or%
    \SB@addNtest\SB@macrotests#1#2#3%
  \else%
    \SB@addMtest\SB@multitests#1#2#3\relax\relax\relax%
  \fi%
}
\newcommand\DeclareLyricChar{\SB@declare\SB@testtrue0}
\newcommand\DeclareNonLyric{\SB@declare\SB@testfalse\SB@testfalse}
\newcommand\DeclareNoHyphen{\SB@declare\SB@testfalse\SB@testtrue}
%    \end{macrocode}
% \end{macro}
% \end{macro}
% \end{macro}
% \end{macro}
%
% \begin{macro}{\SB@lettertests}
% \begin{macro}{\SB@macrotests}
% \begin{macro}{\SB@multitests}
% \begin{macro}{\SB@othertests}
% For speed, token tests introduced by |\DeclareLyricChar| and friends
% are broken out into separate macros based on category codes.
%    \begin{macrocode}
\newcommand\SB@lettertests{}
\newcommand\SB@macrotests{}
\newcommand\SB@multitests{}
\newcommand\SB@othertests{}
%    \end{macrocode}
% \end{macro}
% \end{macro}
% \end{macro}
% \end{macro}
%
% The following macros add tests to the test macros defined above.
% In each, \argp{1} is the test macro to which the test should be added,
% \argp{2} and \argp{3} is the code to be executed at scanning-time and
% at hyphenation-time if the test succeeds (or ``0'' if no action is to
% be performed), and \argp{4} is the token to which the currently scanned
% token should be compared to determine whether it matches.
%
% \begin{macro}{\SB@addDtest}
% A definition-test: The test succeeds if the next lyric token has the same
% meaning (at test-time) of the non-macro, non-active character token that
% was given to the |\Declare| macro.
%    \begin{macrocode}
\newcommand\SB@addDtest[3]{%
  \ifx0#2\else%
    \def#1{{\csname SB@!\meaning\SB@next\endcsname}}%
    \expandafter\def\csname SB@!\meaning\SB@next\endcsname{\global#2}%
  \fi%
  \ifx0#3\else%
    \expandafter\def\csname SB@HT@\meaning\SB@next\endcsname{\global#3}%
  \fi%
}
%    \end{macrocode}
% \end{macro}
%
% \begin{macro}{\SB@addNtest}
% A name-test:  The test succeeds if the next token is a non-|\outer|
% macro or active character and its |\string|ified name matches the
% |\string|ified name of the control sequence that was given to the
% |\Declare| macro.
%    \begin{macrocode}
\newcommand\SB@addNtest[4]{%
  \ifx0#2\else%
    \def#1{{\csname SB@!\SB@nextname\endcsname}}%
    \expandafter\def\csname SB@!\string#4\endcsname{\global#2}%
  \fi%
  \ifx0#3\else%
    \expandafter\def\csname SB@HT@\string#4\endcsname{\global#3}%
  \fi%
}
%    \end{macrocode}
% \end{macro}
%
% \begin{macro}{\SB@addMtest}
% A multibyte-test:  The test succeeds if the next lyric token is the
% beginning of a UTF-8 encoded multibyte character sequence that matches
% the multibyte sequence given to the |\Declare| macro.
%    \begin{macrocode}
\newcommand\SB@addMtest[7]{%
  \edef\SB@temp{%
    \string#4%
    \ifx\relax#5\else\string#5\fi%
    \ifx\relax#6\else\string#6\fi%
    \ifx\relax#7\else\string#7\fi%
  }%
  \ifx0#2\else%
    \def#1{{\csname SB@!\SB@nextname\endcsname}}%
    \expandafter\def\csname SB@!\SB@temp\endcsname{\global#2}%
  \fi%
  \ifx0#3\else%
    \expandafter\def\csname SB@HT@\SB@temp\endcsname{\global#3}%
  \fi%
}
%    \end{macrocode}
% \end{macro}
%
% The following code declares the common intra-word macros provided by
% \TeX{} (as listed on p.~52 of The \TeX book) to be lyric-continuing.
%    \begin{macrocode}
\DeclareLyricChar\`
\DeclareLyricChar\'
\DeclareLyricChar\^
\DeclareLyricChar\"
\DeclareLyricChar\~
\DeclareLyricChar\=
\DeclareLyricChar\.
\DeclareLyricChar\u
\DeclareLyricChar\v
\DeclareLyricChar\H
\DeclareLyricChar\t
\DeclareLyricChar\c
\DeclareLyricChar\d
\DeclareLyricChar\b
\DeclareLyricChar\oe
\DeclareLyricChar\OE
\DeclareLyricChar\ae
\DeclareLyricChar\AE
\DeclareLyricChar\aa
\DeclareLyricChar\AA
\DeclareLyricChar\o
\DeclareLyricChar\O
\DeclareLyricChar\l
\DeclareLyricChar\L
\DeclareLyricChar\ss
\DeclareLyricChar\i
\DeclareLyricChar\j
\DeclareLyricChar\/
\DeclareLyricChar\-
\DeclareLyricChar\discretionary
%    \end{macrocode}
%
% We declare |\par| to be lyric-ending without introducing hyphenation.
% The |\par| macro doesn't actually appear in most verses because we use
% |\obeylines|, but we include a check for it in case the user says |\par|
% explicitly somewhere.
%    \begin{macrocode}
\DeclareNoHyphen\par
%    \end{macrocode}
%
% \begin{macro}{\SB@bracket}
% This macro gets invoked by the |\[|\eat\] macro whenever a chord begins.
% It gets redefined by code that turns chords on and off, so its initial
% definition doesn't matter.
%    \begin{macrocode}
\newcommand\SB@bracket{}
%    \end{macrocode}
% \end{macro}
%
% \begin{macro}{\SB@chord}
% Begin parsing a chord macro.
% While parsing the chord name argument, we set some special catcodes so
% that chord names can use |#| and |&| for sharps and flats.
%    \begin{macrocode}
\newcommand\SB@chord{\SB@begincname\SB@@chord}
%    \end{macrocode}
% \end{macro}
%
% \begin{macro}{\SB@begincname}
% \begin{macro}{\SB@endcname}
% While parsing a chord name, certain characters such as |#| and |&| are
% temporarily set active so that they can be used as abbreviations for
% sharps and flats.
% To accomplish this, |\SB@begincname| must always be invoked before any
% macro whose argument is a chord name, and |\SB@endcname| must be invoked
% at the start of the body of any macro whose argument is a chord name.
% To aid in debugging, we also temporarily set \Meta{return} characters and
% chord macros |\outer|.
% This will cause \TeX{} to halt with a runaway argument error on the correct
% source line if the user forgets to type a closing end-brace (a common typo).
% Colon characters are also set non-active to avoid a conflict between the
% \textsf{Babel} French package and the |\gtab| macro.
%    \begin{macrocode}
\newcommand\SB@begincname{}
{\catcode`\^^M\active
 \gdef\SB@begincname{%
   \begingroup%
     \catcode`##\active\catcode`&\active%
     \catcode`:12\relax%
     \catcode`\^^M\active\SB@outer\def^^M{}%
     \SB@outer\def\[{}%
     \chordlocals\relax%
  }
}
\newcommand\SB@endcname{}
\let\SB@endcname\endgroup
%    \end{macrocode}
% \end{macro}
% \end{macro}
%
% \begin{macro}{\SB@nbsp}
% Non-breaking spaces (|~|) should be treated as spaces by the lyric-scanner
% code that follows.
% Although |~| is usually an active character that creates a non-breaking
% space, some packages (e.g., the \textsf{Babel} package) redefine it to
% produce accents, which are typically not lyric-ending.
% To distinguish the real |~| from redefined |~|, we need to create a macro
% whose definition is the non-breaking space definition normally assigned to
% |~|.
%    \begin{macrocode}
\newcommand*\SB@nbsp{\nobreakspace{}}
%    \end{macrocode}
% \end{macro}
%
% \begin{macro}{\SB@firstchord}
% The following conditional is true when the current chord is the first
% chord in a sequence of one or more chord macros.
%    \begin{macrocode}
\newif\ifSB@firstchord\SB@firstchordtrue
%    \end{macrocode}
% \end{macro}
%
% \begin{macro}{\SB@@chord}
% Finish processing the chord name and then begin scanning the implicit
% lyric argument that follows it.
% This is the main entrypoint to the lyric-scanner code.
%    \begin{macrocode}
\newcommand*\SB@@chord{}
\def\SB@@chord#1]{%
  \SB@endcname%
  \ifSB@firstchord%
    \setbox\SB@lyricbox\hbox{\kern\SB@tabindent}%
    \global\SB@tabindent\z@%
    \SB@lyric{}%
    \SB@numhyps\z@%
    \SB@spcinit%
    \setbox\SB@chordbox\box\voidb@x%
  \fi%
  \SB@setchord{#1}%
  \SB@firstchordfalse%
  \let\SB@dothis\SB@chstart%
  \SB@chscan%
}
%    \end{macrocode}
% \end{macro}
%
% \begin{macro}{\MultiwordChords}\MainImpl{MultiwordChords}
% \begin{macro}{\SB@spcinit}
% \changes{v1.22}{2007/05/15}{Added.}
% The |\SB@spcinit| macro is invoked at the beginning of the lyric
% scanning process.
% By default it does nothing, but if |\MultiwordChords| is invoked,
% it initializes the lyric-scanner state to process spaces as part of
% lyrics.
%    \begin{macrocode}
\newcommand\SB@spcinit{}
\newcommand\MultiwordChords{%
  \def\SB@spcinit{%
    \let\SB@chdone\SB@chlyrdone%
    \let\SB@chimpspace\SB@chnxtdone%
    \let\SB@chexpspace\SB@chnxtdone%
    \let\SB@chespace\SB@chendspace%
  }%
}
%    \end{macrocode}
% \end{macro}
% \end{macro}
%
% \begin{macro}{\SB@chscan}
% \begin{macro}{\SB@chmain}
% This is the main loop of the lyric-scanner.
% Peek ahead at the next token without consuming it, then execute
% a loop body based on the current state (|\SB@dothis|), and finally
% go to the next iteration (|\SB@donext|).
%    \begin{macrocode}
\newcommand\SB@chscan{%
  \let\SB@nextname\relax%
  \futurelet\SB@next\SB@chmain%
}
\newcommand\SB@chmain{\SB@dothis\SB@donext}
%    \end{macrocode}
% \end{macro}
% \end{macro}
%
% \begin{macro}{\SB@chnxtrelax}
% \begin{macro}{\SB@chnxtstep}
% \begin{macro}{\SB@chnxtdone}
% To shorten the lyric parser macros that follow and thereby improve their
% speed, we here define some abbreviations for common logic in untaken
% branches.
%    \begin{macrocode}
\newcommand\SB@chnxtrelax{\let\SB@donext\relax}
\newcommand\SB@chnxtstep{\let\SB@donext\SB@chstep}
\newcommand\SB@chnxtdone{\let\SB@donext\SB@chdone}
%    \end{macrocode}
% \end{macro}
% \end{macro}
% \end{macro}
% 
% Warning: In the lyric-scanner macros that follow, |\SB@next|
% might be a macro declared |\outer|.
% This means that it must \emph{never} be passed as an argument to
% a macro and it must never explicitly appear in any untaken branch
% of a conditional.
% If it does, the \TeX{} parser will complain of a runaway argument
% when it tries to skip over an |\outer| macro while consuming tokens
% at high speed.
%
% \begin{macro}{\SB@chstart}
% We begin lyric-scanning with two special cases:
% (1) If the chord macro is immediately followed by another chord macro with
% no intervening whitespace, drop out of the lyric scanner and reenter it when
% the second macro is parsed.
% The chord texts will get concatenated together above the lyric that follows.
% (2) If the chord macro is immediately followed by one or more quote
% tokens, then consume them all and output them \emph{before} the chord.
% This causes the chord to sit above the actual word instead of the
% left-quote or left-double-quote symbol, which looks better.
%    \begin{macrocode}
\newcommand\SB@chstart{%
  \ifx\SB@next\[\SB@chnxtrelax%
  \else\ifx\SB@next\SB@activehat\SB@chnxtrelax%
  \else\ifx\SB@next\ch\SB@chnxtrelax%
  \else\ifx\SB@next\mch\SB@chnxtrelax%
  \else\ifx\SB@next`\SB@chnxtstep%
  \else\ifx\SB@next'\SB@chnxtstep%
  \else\ifx\SB@next"\SB@chnxtstep%
  \else%
    \the\SB@lyric%
    \SB@lyric{}%
    \SB@firstchordtrue%
    \let\SB@dothis\SB@chnorm%
    \SB@chnorm%
  \fi\fi\fi\fi\fi\fi\fi%
}
%    \end{macrocode}
% \end{macro}
% \eat\]
%
% \begin{macro}{\SB@chnorm}
% \changes{v2.0}{2007/06/18}{Rewritten for speed}
% First, check to see whether the lyric token is a letter.
% Since that's the most common case, we do this check first for speed.
%    \begin{macrocode}
\newcommand\SB@chnorm{%
  \ifcat\noexpand\SB@next A%
    \SB@testtrue\SB@lettertests%
    \ifSB@test%
      \SB@chespace\SB@chnxtstep%
    \else%
      \SB@chnxtdone%
    \fi%
  \else%
    \SB@chtrymacro%
  \fi%
}
%    \end{macrocode}
% \end{macro}
%
% \begin{macro}{\SB@chtrymacro}
% Next, check to see whether it's a macro or active character.
% We do these checks next because these are the only cases when the
% token might be |\outer|.
% Once we eliminate that ugly possibility, we can write the rest of
% the code without having to worry about putting |\SB@next| in
% places where |\outer| tokens are illegal.
%    \begin{macrocode}
\newcommand\SB@chtrymacro{%
  \ifcat\noexpand\SB@next\relax%
    \SB@chmacro%
  \else%
    \SB@chother%
  \fi%
}
%    \end{macrocode}
% \end{macro}
%
% \begin{macro}{\SB@chother}
% The token is not a letter, macro, or active character.
% The only other cases of interest are spaces, braces, and hyphens.
% If it's one of those, take the appropriate action; otherwise end the
% lyric here.
% Since we've eliminated the possibility of macros and active characters,
% we can be sure that the token isn't |\outer| at this point.
%    \begin{macrocode}
\newcommand\SB@chother{%
  \ifcat\noexpand\SB@next\@sptoken%
    \SB@chexpspace%
  \else\ifcat\noexpand\SB@next\bgroup%
    \SB@chespace\let\SB@donext\SB@chbgroup%
  \else\ifcat\noexpand\SB@next\egroup%
    \SB@chespace\let\SB@donext\SB@chegroup%
  \else\ifx\SB@next-%
    \SB@numhyps\@ne\relax%
    \SB@lyricnohyp\expandafter{\the\SB@lyric}%
    \let\SB@dothis\SB@chhyph%
    \SB@chespace\SB@chnxtstep%
  \else\ifcat\noexpand\SB@next.%
    \SB@testtrue\SB@othertests%
    \ifSB@test%
      \SB@chespace\SB@chnxtstep%
    \else%
      \SB@chnxtdone%
    \fi%
  \else%
    \SB@chespace\SB@chnxtstep%
  \fi\fi\fi\fi\fi%
}
%    \end{macrocode}
% \end{macro}
%
% \begin{macro}{\SB@chmacro}
% \changes{v1.22}{2007/05/15}{Added support for UTF-8.}
% The lyric-scanner has encountered a macro or active character.
% If it's |\outer|, it should never be used in an argument, so stop here.
%    \begin{macrocode}
\newcommand\SB@chmacro{%
  \SB@outertest%
  \ifSB@test%
    \SB@chnxtdone%
  \else%
    \let\SB@donext\SB@chgetname%
  \fi%
}
%    \end{macrocode}
% \end{macro}
%
% \begin{macro}{\SB@chgetname}
% We've encountered a non-|\outer| macro or active character.
% Use |\string| to get its name, but insert the token back into the
% input stream since we haven't decided whether to consume it yet.
%    \begin{macrocode}
\newcommand\SB@chgetname[1]{%
  \edef\SB@nextname{\string#1}%
  \SB@@chmacro\SB@donext#1%
}
%    \end{macrocode}
% \end{macro}
%
% \begin{macro}{\SB@@chmacro}
% The lyric-scanner has encountered a non-|\outer| macro or active character.
% Its |\string|ified name has been stored in |\SB@nextname|.
% Test to see whether it's a known macro or the beginning of a
% multibyte-encoded international character.
% If the former, dispatch some macro-specific code to handle it.
% If the latter, grab the full multibyte sequence and include it in the lyric.
%    \begin{macrocode}
\newcommand\SB@@chmacro{%
  \ifx\SB@next\SB@activehat%
    \SB@chnxtdone%
  \else\ifx\SB@next\SB@par%
    \SB@chnxtdone%
  \else\ifx\SB@next\measurebar%
    \SB@chmbar%
  \else\ifx\SB@next\mbar%
    \SB@chmbar%
  \else\ifx\SB@next\ch%
    \SB@chespace\let\SB@donext\SB@chlig%
  \else\ifx\SB@next\mch%
    \SB@chespace\let\SB@donext\SB@mchlig%
  \else\ifx\SB@next\ %
    \SB@chimpspace%
  \else\ifx\SB@next\SB@nbsp%
    \SB@chimpspace%
  \else%
    \SB@UTFtest\SB@next%
    \ifcase\SB@cnt\SB@chothermac%
    \or\or\SB@chespace\let\SB@donext\SB@chsteptwo%
    \or\SB@chespace\let\SB@donext\SB@chstepthree%
    \or\SB@chespace\let\SB@donext\SB@chstepfour\fi%
  \fi\fi\fi\fi\fi\fi\fi\fi%
}
%    \end{macrocode}
% \end{macro}
%
% \begin{macro}{\SB@chothermac}
% \changes{v2.6}{2008/03/26}{All active chars now included in lyrics by default.}
% The lyric-scanner has encountered a macro or active character that is
% not |\outer|, not a known macro that requires special treatment,
% and not a multibyte international character.
% First, check the macro's name (stored in |\SB@nextname|) to see whether it
% begins with a non-escape character.
% If so, it's probably an accenting or punctuation character made active
% by the |inputenc| or |babel| packages.
% Most such characters should be included in the lyric, so include it by
% default; otherwise exclude it by default.
% The user can override the defaults using |\DeclareLyricChar| and friends.
%    \begin{macrocode}
\newcommand\SB@chothermac{%
  \SB@testfalse%
  \afterassignment\iffalse%
  \SB@cnt\expandafter`\SB@nextname x\fi%
  \ifnum\the\catcode\SB@cnt=\z@\else\SB@testtrue\fi%
  \SB@macrotests%
  \ifSB@test%
    \SB@chespace\SB@chnxtstep%
  \else%
    \SB@chnxtdone%
  \fi%
}
%    \end{macrocode}
% \end{macro}
%
% \begin{macro}{\SB@chstep}
% \begin{macro}{\SB@chsteptwo}
% \begin{macro}{\SB@chstepthree}
% \begin{macro}{\SB@chstepfour}
% \begin{macro}{\SB@chmulti}
% \begin{macro}{\SB@chmstop}
% We've encountered one or more tokens that should be included in the
% lyric text.
% (More than one means we've encountered a multibyte encoding of an
% international character.)
% Consume them (as arguments to this macro) and add them to the list
% of tokens we've already consumed.
%    \begin{macrocode}
\newcommand\SB@chstep[1]{%
  \SB@lyric\expandafter{\the\SB@lyric#1}%
  \SB@chscan%
}
\newcommand\SB@chsteptwo[2]{\SB@chmulti{#1#2}{\string#1\string#2}}
\newcommand\SB@chstepthree[3]{%
  \SB@chmulti{#1#2#3}{\string#1\string#2\string#3}%
}
\newcommand\SB@chstepfour[4]{%
  \SB@chmulti{#1#2#3#4}{\string#1\string#2\string#3\string#4}%
}
\newcommand\SB@chmulti[2]{%
  \def\SB@next{#1}%
  \edef\SB@nextname{#2}%
  \SB@testtrue\SB@multitests%
  \ifSB@test%
    \SB@lyric\expandafter{\the\SB@lyric#1}%
    \expandafter\SB@chscan%
  \else%
    \expandafter\SB@chmstop%
  \fi%
}
\newcommand\SB@chmstop{\expandafter\SB@chdone\SB@next}
%    \end{macrocode}
% \end{macro}
% \end{macro}
% \end{macro}
% \end{macro}
% \end{macro}
% \end{macro}
%
% \begin{macro}{\SB@chhyph}
% We've encountered a hyphen.
% Continue to digest hyphens, but terminate as soon as we see anything
% else.
%    \begin{macrocode}
\newcommand\SB@chhyph{%
  \ifx\SB@next-%
    \advance\SB@numhyps\@ne\relax%
    \SB@chnxtstep%
  \else%
    \SB@chnxtdone%
  \fi%
}
%    \end{macrocode}
% \end{macro}
%
% \begin{macro}{\SB@chimpspace}
% \begin{macro}{\SB@chexpspace}
% We've encountered an implicit or explicit space.
% Normally this just ends the lyric, but if |\MultiwordChords| is
% active, these macros both get redefined to process the space.
%    \begin{macrocode}
\newcommand\SB@chimpspace{}
\let\SB@chimpspace\SB@chnxtdone
\newcommand\SB@chexpspace{}
\let\SB@chexpspace\SB@chnxtdone
%    \end{macrocode}
% \end{macro}
% \end{macro}
%
% \begin{macro}{\SB@chespace}
% \begin{macro}{\SB@chendspace}
% \changes{v1.22}{2007/05/15}{Added.}
% The |\SB@chespace| macro gets invoked by the lyric-scanner just before a
% non-space token is about to be accepted as part of an under-chord lyric.
% Normally it does nothing; however, if |\MultiwordChords| is active, it
% gets redefined to do one of three things:
% (1) Initially it is set equal to |\SB@chendspace| so that if the very
% first token following the chord macro is not a space, the lyric-scanner
% macros are redefined to process any future spaces encountered.
% Otherwise the very first token is a space, and the lyric ends immediately.
% (2) While scanning non-space lyric tokens, it is set to nothing, since no
% special action needs to be taken until we encounter a sequence of one or
% more spaces.
% (3) When a space token is encountered (but not the very first token after
% the chord macro), it is set equal to |\SB@chendspace| again so that
% |\SB@chendspace| is invoked once the sequence of one or more space tokens
% is finished.
%    \begin{macrocode}
\newcommand\SB@chespace{}
\newcommand\SB@chendspace{%
  \let\SB@chdone\SB@chlyrdone%
  \def\SB@chexpspace{\SB@chbspace\SB@chexpspace}%
  \def\SB@chimpspace{\SB@chbspace\SB@chimpspace}%
  \def\SB@chespace{}%
}
%    \end{macrocode}
% \end{macro}
% \end{macro}
%
% \begin{macro}{\SB@chbspace}
% \begin{macro}{\SB@chgetspace}
% \changes{v1.22}{2007/05/15}{Added.}
% The |\SB@chbspace| macro gets invoked when |\MultiwordChords| is active
% and the lyric-scanner has encountered a space token that was immediately
% preceded by a non-space token.
% Before processing the space, we add all lyrics seen so far to the
% |\SB@lyricbox| and check its width.
% If we've seen enough lyrics to match or exceed the width of the chord,
% a space stops the lyric-scanning process.
% (This is important because it minimizes the size of the chord box,
% providing as many line breakpoints as possible to the paragraph-formatter.)
%
% Otherwise we begin scanning space tokens without adding them to the
% lyric until we see what the next non-space token is.
% If the next non-space token would have ended the lyric anyway, roll back
% and end the lyric here, reinserting the space tokens back into the token
% stream.
% If the next non-space token would have been included in the lyric,
% the lyric-scanner proceeds as normal.
%    \begin{macrocode}
\newcommand\SB@chbspace{%
  \setbox\SB@lyricbox\hbox{%
    \unhbox\SB@lyricbox%
    \the\SB@lyric%
  }%
  \SB@lyric{}%
  \ifdim\wd\SB@lyricbox<\wd\SB@chordbox%
    \let\SB@chbstok= \SB@next%
    \def\SB@chexpspace{\let\SB@donext\SB@chgetspace}%
    \let\SB@chimpspace\SB@chnxtstep%
    \let\SB@chespace\SB@chendspace%
    \let\SB@chdone\SB@chspcdone%
  \else%
    \let\SB@chimpspace\SB@chnxtdone%
    \let\SB@chexpspace\SB@chnxtdone%
  \fi%
}
\newcommand\SB@chgetspace{%
  \SB@appendsp\SB@lyric%
  \let\SB@nextname\relax%
  \afterassignment\SB@chscan%
  \let\SB@next= }
%    \end{macrocode}
% \end{macro}
% \end{macro}
%
% \begin{macro}{\SB@chmbar}
% We've encountered a measure bar.
% Either ignore it or end the lyric text, depending on whether
% measure bars are being displayed.
%    \begin{macrocode}
\newcommand\SB@chmbar{%
  \ifmeasures%
    \SB@chnxtdone%
  \else%
    \SB@chespace\SB@chnxtstep%
  \fi%
}
%    \end{macrocode}
% \end{macro}
%
% \begin{macro}{\SB@chbgroup}
% We've encountered a begin-group brace.
% Consume the entire group that it begins, and add it to the list
% of tokens including the begin and end group tokens.
%    \begin{macrocode}
\newcommand\SB@chbgroup[1]{%
  \SB@lyric\expandafter{\the\SB@lyric{#1}}%
  \SB@chscan%
}
%    \end{macrocode}
% \end{macro}
%
% \begin{macro}{\SB@chegroup}
% \begin{macro}{\SB@chegrpscan}
% \begin{macro}{\SB@chegrpmacro}
% \begin{macro}{\SB@chegrpouter}
% \begin{macro}{\SB@chegrpname}
% \begin{macro}{\SB@chegrpdone}
% We've encountered an end-group brace whose matching begin-group brace
% must have come before the chord macro itself.
% This forcibly ends the lyric text.
% Before stopping, we must set |\SB@next| to the token following the brace
% and |\SB@nextname| to its |\string|ified name so that |\SB@emitchord| will
% know whether to add hyphenation.
% Therefore, we temporarily consume the end-group brace, then scan the
% next token without consuming it, and finally reinsert the end-group brace
% and stop.
%    \begin{macrocode}
\newcommand\SB@chegroup{%
  \let\SB@nextname\relax%
  \afterassignment\SB@chegrpscan%
  \let\SB@next= }
\newcommand\SB@chegrpscan{%
  \futurelet\SB@next\SB@chegrpmacro%
}
\newcommand\SB@chegrpmacro{%
  \ifcat\noexpand\SB@next\relax%
    \expandafter\SB@chegrpouter%
  \else%
    \expandafter\SB@chegrpdone%
  \fi%
}
\newcommand\SB@chegrpouter{%
  \SB@outertest%
  \ifSB@test%
    \expandafter\SB@chegrpdone%
  \else%
    \expandafter\SB@chegrpname%
  \fi%
}
\newcommand\SB@chegrpname[1]{%
  \edef\SB@nextname{\string#1}%
  \SB@chegrpdone#1%
}
\newcommand\SB@chegrpdone{\SB@chdone\egroup}
%    \end{macrocode}
% \end{macro}
% \end{macro}
% \end{macro}
% \end{macro}
% \end{macro}
% \end{macro}
%
% \begin{macro}{\SB@chlig}
% \begin{macro}{\SB@mchlig}
% We've encountered a |\ch| chord-over-ligature macro, or an
% |\mch| measurebar-and-chord-over-ligature macro.
% Consume it and all of its arguments, and load them into some
% registers for future processing.
% (Part of the ligature might fall into this lyric text or might
% not, depending on whether we decide to add hyphenation.)
% Then end the lyric text here.
%    \begin{macrocode}
\newcommand\SB@chlig[5]{%
  \gdef\SB@ligpre{{#3}}%
  \gdef\SB@ligpost{\[#2]{#4}}%
  \gdef\SB@ligfull{%
    \[\SB@noreplay{\hphantom{{\lyricfont\relax#3}}}#2]{#5}%
  }%
  \SB@chdone%
}
\newcommand\SB@mchlig[5]{%
  \SB@lyric\expandafter{\the\SB@lyric#3}%
  \let\SB@next\measurebar%
  \edef\SB@nextname{\string\measurebar}%
  \gdef\SB@ligpost{\measurebar\[#2]{#4}}%
  \gdef\SB@ligfull{\measurebar\[#2]{#4}}%
  \SB@chdone%
}
%    \end{macrocode}
% \eat\]
% \end{macro}
% \end{macro}
%
% \begin{macro}{\SB@chdone}
% \begin{macro}{\SB@chlyrdone}
% \begin{macro}{\SB@chspcdone}
% The |\SB@chdone| macro is invoked when we've decided to end the lyric
% text (usually because we've encountered a non-lyric token).
% Normally this expands to |\SB@chlyrdone|, which adds any uncontributed
% lyric material to the |\SB@lyricbox| and jumps to the main chord
% formatting macro.
% However, if |\MultiwordChords| is active and if the lyric ended with
% a sequence of one or more space tokens, then we instead reinsert the
% space tokens into the token stream without contributing them to the
% |\SB@lyricbox|.
%    \begin{macrocode}
\newcommand\SB@chlyrdone{%
  \setbox\SB@lyricbox\hbox{%
    \unhbox\SB@lyricbox%
    \ifnum\SB@numhyps=\@ne%
      \the\SB@lyricnohyp%
    \else%
      \the\SB@lyric%
    \fi%
  }%
  \SB@emitchord%
}
\newcommand\SB@chspcdone{%
  \let\SB@nextname\relax%
  \let\SB@next= \SB@chbstok%
  \expandafter\SB@emitchord\the\SB@lyric%
}
\newcommand\SB@chdone{}
\let\SB@chdone\SB@chlyrdone
%    \end{macrocode}
% \end{macro}
% \end{macro}
% \end{macro}
%
% \begin{macro}{\SB@ligpre}
% \begin{macro}{\SB@ligpost}
% \begin{macro}{\SB@ligfull}
% The following three macros record arguments passed to a |\ch| macro that
% concludes the lyric text of the |\[]|\eat\] macro currently being processed.
%    \begin{macrocode}
\newcommand\SB@ligpre{}
\newcommand\SB@ligpost{}
\newcommand\SB@ligfull{}
%    \end{macrocode}
% \end{macro}
% \end{macro}
% \end{macro}
%
% \begin{macro}{\SB@clearlig}
% Clear all ligature-chord registers.
%    \begin{macrocode}
\newcommand\SB@clearlig{%
  \gdef\SB@ligpre{}%
  \gdef\SB@ligpost{}%
  \gdef\SB@ligfull{}%
}
%    \end{macrocode}
% \end{macro}
%
% \subsection{Chords}
%
% \begin{macro}{\SB@emitchord}\MainImpl{[}
% \changes{v1.12}{2005/05/10}{Inhibited hyphenation of trailing punctuation}
% \changes{v1.13}{2005/05/12}{Added code to preserve the spacefactor}
% \changes{v1.16}{2005/07/23}{Chord macros massively reorganized to take lyrics as implicit rather than explicit arguments}
% The |\SB@emitchord| macro does the actual work of typesetting chord text
% over lyric text, introducing appropriate hyphenation when necessary.
% We begin by consulting |\SB@next|, which should have been set by the
% lyric-scanning code in \S\ref{sec:lyricscan} to the token that immediately
% follows the lyric under this chord, to determine whether the lyric text
% ends on a word boundary.
%    \begin{macrocode}
\newcommand\SB@emitchord{%
  \ifSB@inverse\else\ifSB@inchorus\else\SB@errchord\fi\fi%
  \SB@testfalse%
  \ifcat\noexpand\SB@next\@sptoken\SB@testtrue\fi%
  \ifcat\noexpand\SB@next.\SB@testtrue\fi%
  \ifx\SB@next\SB@par\SB@testtrue\fi%
  \ifx\SB@next\egroup\SB@testtrue\fi%
  \ifx\SB@next\endgroup\SB@testtrue\fi%
  {\csname%
     SB@HT@\ifx\SB@nextname\relax\meaning\SB@next\else\SB@nextname\fi%
   \endcsname}%
  \ifSB@test\SB@wordendstrue\else\SB@wordendsfalse\fi%
%    \end{macrocode}
% Next, compare the width of the lyric to the width of the chord to
% determine whether hyphenation might be necessary.
% The original lyric text might have ended in a string of one or more
% explicit hyphens, enumerated by |\SB@numhyps|.
% If it ended in exactly one, the lyric-scanning code suppresses that hyphen
% so that we can here add a new hyphen that floats out away from the word
% when the chord above it is long.
% If it ended in more than one (e.g., the encoding of an en- or em-dash) then
% the lyric-scanner leaves it alone; we must not add any hyphenation or float
% the dash away from the word.
%
% There is also code here to insert a penalty that discourages linebreaking
% immediately before lyricless chords.
% Beginning a wrapped line with a lyricless chord is undesirable because it
% makes it look as though the wrapped line is extra-indented (due to the
% empty lyric space below the chord).
% It should therefore happen only as a last resort.
%    \begin{macrocode}
  \SB@dimen\wd\SB@chordbox%
  \ifvmode\leavevmode\fi%
  \SB@brokenwordfalse%
  \ifdim\wd\SB@lyricbox>\z@%
    \ifdim\SB@dimen>\wd\SB@lyricbox%
      \ifSB@wordends\else\SB@brokenwordtrue\fi%
    \fi%
  \else%
    \SB@skip\lastskip%
    \unskip\penalty200\hskip\SB@skip%
  \fi%
  \ifnum\SB@numhyps>\z@%
    \ifnum\SB@numhyps>\@ne%
      \SB@brokenwordfalse%
    \else%
      \SB@brokenwordtrue%
    \fi%
  \fi%
%    \end{macrocode}
% If lyrics are suppressed on this line (e.g., by using |\nolyrics|), then just
% typeset the chord text on the natural baseline.
%    \begin{macrocode}
  \SB@testfalse%
  \ifnolyrics\ifdim\wd\SB@lyricbox=\z@\SB@testtrue\fi\fi%
  \ifSB@test%
    \unhbox\SB@chordbox%
    \gdef\SB@temp{\expandafter\SB@clearlig\SB@ligfull}%
  \else%
%    \end{macrocode}
% Otherwise, typeset the chord above the lyric on a double-height line.
%    \begin{macrocode}
    \vbox{\clineparams\relax%
      \ifSB@brokenword%
        \global\setbox\SB@lyricbox\hbox{%
          \unhbox\SB@lyricbox%
          \SB@ligpre%
        }%
        \SB@maxmin\SB@dimen<{\wd\SB@lyricbox}%
        \advance\SB@dimen.5em%
        \hbox to\SB@dimen{\unhbox\SB@chordbox\hfil}%
        \hbox to\SB@dimen{%
          \unhcopy\SB@lyricbox\hfil
          \ifnum\hyphenchar\font>\m@ne\char\hyphenchar\font\hfil\fi%
        }%
        \global\SB@cnt\@m%
        \gdef\SB@temp{\expandafter\SB@clearlig\SB@ligpost}%
      \else%
        \box\SB@chordbox%
        \hbox{%
          \unhcopy\SB@lyricbox%
          \global\SB@cnt\spacefactor%
          \hfil%
        }%
        \gdef\SB@temp{\expandafter\SB@clearlig\SB@ligfull}%
      \fi%
    }%
%    \end{macrocode}
% If the chord is lyricless, inhibit a linebreak immediately following it.
% This prevents sequences of lyricless chords (which often end lines) from
% being wrapped in the middle, which looks very unsightly and makes them
% difficult to read.
% If the chord has a lyric but it doesn't end on a word boundary, insert
% an appropriate penalty to prevent linebreaking without hyphenation.
% Also preserve the spacefactor in this case, which allows \LaTeX{} to
% fine-tune the spacing between consecutive characters in the word that
% contains the chord.
%    \begin{macrocode}
    \ifSB@wordends%
      \ifdim\wd\SB@lyricbox>\z@\else\nobreak\fi%
    \else%
      \penalty%
        \ifnum\SB@numhyps>\z@\exhyphenpenalty%
        \else\ifSB@brokenword\hyphenpenalty%
        \else\@M\fi\fi%
      \spacefactor\SB@cnt%
    \fi%
  \fi%
%    \end{macrocode}
% Finally, end the macro with some code that handles the special case that
% this chord is immediately followed by a chord-over-ligature macro.
% The code above sets |\SB@temp| to the portion of the ligature that should
% come after this chord but before the chord that tops the ligature.
% This text must be inserted here.
%    \begin{macrocode}
  \SB@temp%
}
%    \end{macrocode}
% \end{macro}
%
% \begin{macro}{\SB@accidental}
% Typeset an accidental symbol as a superscript within a chord.
% Since chord names are often in italics but math symbols like sharp and
% flat are not, we need to do some kerning adjustments before and after the
% accidental to position it as if it were italicized.
% The pre-adjustment is just a simple italic correction using |\/|.
% The post-adjustment is based on the current font's slant-per-point metric.
%    \begin{macrocode}
\newcommand\SB@accidental[1]{{%
  \/%
  \m@th#1%
  \SB@dimen-\fontdimen\@ne\font%
  \advance\SB@dimen.088142\p@%
  \ifdim\SB@dimen<\z@%
    \kern\f@size\SB@dimen%
  \fi%
}}
%    \end{macrocode}
% \end{macro}
%
% \begin{macro}{\sharpsymbol}\MainImpl{sharpsymbol}
% \begin{macro}{\flatsymbol}\MainImpl{flatsymbol}
% When changing the sharp or flat symbol, change these macros rather than
% changing |\shrp| or |\flt|.
% This will ensure that other shortcuts like |#| and |&| will reflect your
% change.
%    \begin{macrocode}
\newcommand\sharpsymbol{\ensuremath{^\#}}
\newcommand\flatsymbol{\raise.5ex\hbox{{\SB@flatsize$\flat$}}}
%    \end{macrocode}
% \end{macro}
% \end{macro}
%
% \begin{macro}{\shrp}\MainImpl{shrp}
% \begin{macro}{\flt}\MainImpl{flt}
% \changes{v2.10}{2009/08/18}{Font size made relative}
% These macros typeset sharp and flat symbols.
%    \begin{macrocode}
\newcommand\shrp{\SB@accidental\sharpsymbol}
\newcommand\flt{\SB@accidental\flatsymbol}
%    \end{macrocode}
% \end{macro}
% \end{macro}
%
% \begin{macro}{\DeclareFlatSize}
% The |\flat| math symbol is too small for properly typesetting
% chord names.
% (Its size was designed for staff notation not textual chord names.)
% The correct size for the symbol should be approximately 30\% larger
% than the current superscript size, or 90\% of the base font size $b$.
% However, simply computing $0.9b$ does not work well because most fonts
% do not render well in arbitrary sizes.
% To solve the problem, we must therefore choose an appropriate size
% individually for each possible base font size $b$.
% This is the solution adopted by the rest of \LaTeX{} for such things.
% For example, \LaTeX's |\DeclareMathSizes| macro defines an appropriate
% superscript size for each possible base font size.
% The macro below creates a similar macro that that defines an appropriate
% flat-symbol size for each possible base font size.
%    \begin{macrocode}
\newcommand\DeclareFlatSize[2]{%
  \expandafter\xdef\csname SB@flatsize@#1\endcsname{#2}%
}
\DeclareFlatSize\@vpt\@vpt
\DeclareFlatSize\@vipt\@vipt
\DeclareFlatSize\@viipt\@vipt
\DeclareFlatSize\@viiipt\@viipt
\DeclareFlatSize\@ixpt\@viiipt
\DeclareFlatSize\@xpt\@ixpt
\DeclareFlatSize\@xipt\@xpt
\DeclareFlatSize\@xiipt\@xipt
\DeclareFlatSize\@xivpt\@xiipt
\DeclareFlatSize\@xviipt\@xivpt
\DeclareFlatSize\@xxpt\@xviipt
\DeclareFlatSize\@xxvpt\@xxpt
%    \end{macrocode}
% \end{macro}
%
% \begin{macro}{\SB@flatsize}
% Select the correct flat symbol size based on the current font size.
%    \begin{macrocode}
\newcommand\SB@flatsize{%
  \@ifundefined{SB@flatsize@\f@size}{}{%
    \expandafter\fontsize%
      \csname SB@flatsize@\f@size\endcsname\f@baselineskip%
    \selectfont%
  }%
}    
%    \end{macrocode}
% \end{macro}
%
% In the following code, the |\ch|, |\mch|, |\[|\eat\], and |^| macros are
% each defined to be a single macro that then expands to the real definition.
% This is necessary because the top-level definitions of each must stay the
% same in order to allow the lyric-scanning code to uniquely identify them,
% yet their internal definitions must be redefined by code that turns
% chords and/or measure bars on and off.
% Such code redefines |\SB@ch|, |\SB@mch|, |\SB@bracket|, and |\SB@rechord|
% to effect a change of mode without touching the top-level definitions.
%
% \begin{macro}{\ch}\MainImpl{ch}
% \begin{macro}{\SB@ch}
% \begin{macro}{\SB@ch@on}
% \begin{macro}{\SB@@ch}
% \begin{macro}{\SB@@@ch}
% \begin{macro}{\SB@ch@off}
% The |\ch| macro puts a chord atop a ligature without breaking the ligature.
% Normally this just means placing the chord midway over the unbroken
% ligature (ignoring the third argument completely).
% However, when a previous chord macro encounters it while scanning ahead in
% the input stream to parse its lyric, the |\ch| macro itself is not actually
% expanded at all.
% Instead, the chord macro scans ahead, spots the |\ch| macro, gobbles it,
% and then steals its arguments, breaking the ligature with hyphenation.
% Thus, the |\ch| macro is only actually expanded when the ligature
% shouldn't be broken.
%    \begin{macrocode}
\newcommand\ch{\SB@ch}
\newcommand\SB@ch{}
\newcommand\SB@ch@on{\SB@begincname\SB@@ch}
\newcommand*\SB@@ch[1]{\SB@endcname\SB@@@ch{#1}}
\newcommand*\SB@@@ch[4]{\[\SB@noreplay{\hphantom{#2}}#1]#4}
\newcommand*\SB@ch@off[4]{#4}
%    \end{macrocode}
% \end{macro}
% \end{macro}
% \end{macro}
% \end{macro}
% \end{macro}
% \end{macro}
% \eat\]
%
% \begin{macro}{\mch}\MainImpl{mch}
% \begin{macro}{\SB@mch}
% \begin{macro}{\SB@mch@m}
% \begin{macro}{\SB@mch@on}
% \begin{macro}{\SB@@mch}
% \begin{macro}{\SB@@@mch}
% The |\mch| macro is like |\ch| except that it also introduces a measure
% bar.
%    \begin{macrocode}
\newcommand\mch{\SB@mch}
\newcommand\SB@mch{}
\newcommand*\SB@mch@m[4]{#2\measurebar#3}
\newcommand\SB@mch@on{\SB@begincname\SB@@mch}
\newcommand*\SB@@mch[1]{\SB@endcname\SB@@@mch{#1}}
\newcommand*\SB@@@mch[4]{#2\measurebar\[#1]#3}
%    \end{macrocode}
% \end{macro}
% \end{macro}
% \end{macro}
% \end{macro}
% \end{macro}
% \end{macro}
% \eat\]
%
% \begin{macro}{\SB@activehat}
% This macro must always contain the current definition of the |^|
% chord-replay active character, in order for the lyric scanner to properly
% identify it and insert proper hyphenation when necessary.
%    \begin{macrocode}
\newcommand\SB@activehat{%
  \ifmmode^\else\expandafter\SB@rechord\fi%
}
%    \end{macrocode}
% \end{macro}
%
% \begin{macro}{\SB@hat@tr}
% In verses/choruses where chords are being memorized, |\SB@activehat|
% gets set to this definition, which marks the current chord as immune to
% memorization.
%    \begin{macrocode}
\newcommand\SB@hat@tr{%
  \ifmmode^\else\global\SB@nohatfalse\fi%
}
%    \end{macrocode}
% \end{macro}
%
% \begin{macro}{\SB@hat@notr}
% In verses/choruses where chords are being replayed, |\SB@activehat|
% get set to the following, which replays the next memorized chord and
% subjects it to any required transposition and/or note conversion.
%    \begin{macrocode}
\newcommand\SB@hat@notr{%
  \ifmmode^\else%
    \SB@lop\SB@ctail\SB@toks%
    \expandafter\transposehere\expandafter{\the\SB@toks}%
  \fi%
}
%    \end{macrocode}
% \end{macro}
%
% \begin{macro}{\SB@loadactives}
% It's cumbersome to have to type |\shrp|, |\flt|, and |\mbar| every time you
% want a sharp, flat, or measure bar, so within verses and choruses we allow
% the hash, ampersand, and pipe symbols to perform the those functions too.
% It's also cumbersome to have to type something like |\chord{Am}{lyric}| to
% produce each chord.
% As an easier alternative, we here define |\[Am]|\eat\]
% to typeset chords.
%    \begin{macrocode}
\newcommand\SB@loadactives{}
{
  \catcode`&\active
  \catcode`#\active
  \catcode`|\active
  \catcode`^\active
  \global\let&\flt
  \global\let#\shrp
  \global\let|\measurebar
  \global\let^\SB@activehat
  \gdef\SB@loadactives{%
    \catcode`^\ifchorded\active\else9 \fi%
    \catcode`|\ifmeasures\active\else9 \fi%
    \def\[{\SB@bracket}%
  }
}
%    \end{macrocode}
% \end{macro}
%
% \subsection{Chord Replaying}
%
% \begin{macro}{\SB@trackch}
% While inside a verse where the chord history is being remembered for future
% verses, |\SB@trackch| is true.
%    \begin{macrocode}
\newif\ifSB@trackch
%    \end{macrocode}
% \end{macro}
%
% \begin{macro}{\SB@cr@}
% Reserve token registers to record a history of the chords seen in a verse.
%    \begin{macrocode}
\SB@newtoks\SB@cr@
\SB@newtoks\SB@ctail
%    \end{macrocode}
% \end{macro}
%
% \begin{macro}{\SB@creg}
% The following control sequence equals the token register being memorized
% into or replayed from.
%    \begin{macrocode}
\newcommand\SB@creg{}
%    \end{macrocode}
% \end{macro}
%
% \begin{macro}{\newchords}\MainImpl{newchords}
% \changes{v2.6}{2008/02/23}{Added}
% Allocate a new chord-replay register to hold memorized chords.
%    \begin{macrocode}
\newcommand\newchords[1]{%
  \@ifundefined{SB@cr@#1}{%
    \expandafter\SB@newtoks\csname SB@cr@#1\endcsname%
    \global\csname SB@cr@#1\endcsname{\\}%
  }{\SB@errdup{#1}}%
}
%    \end{macrocode}
% \end{macro}
%
% \begin{macro}{\memorize}\MainImpl{memorize}
% \begin{macro}{\SB@memorize}
% \changes{v2.6}{2008/02/23}{Optional argument added}
% Saying |\memorize| throws out any previously memorized list of chords and
% starts memorizing chords until the end of the current verse or chorus.
%    \begin{macrocode}
\newcommand\memorize{%
  \@ifnextchar[\SB@memorize{\SB@memorize[]}%
}
\newcommand\SB@memorize{}
\def\SB@memorize[#1]{%
  \@ifundefined{SB@cr@#1}{\SB@errreg{#1}}{%
    \SB@trackchtrue%
    \global\expandafter\let\expandafter\SB@creg%
      \csname SB@cr@#1\endcsname%
    \global\SB@creg{\\}%
  }%
}
%    \end{macrocode}
% \end{macro}
% \end{macro}
%
% \begin{macro}{\replay}\MainImpl{replay}
% \begin{macro}{\SB@replay}
% \begin{macro}{\SB@@replay}
% \changes{v2.6}{2008/02/23}{Added}
% Saying |\replay| stops any memorization and begins replaying memorized
% chords.
%    \begin{macrocode}
\newcommand\replay{\@ifnextchar[\SB@replay\SB@@replay}
\newcommand\SB@replay{}
\def\SB@replay[#1]{%
  \@ifundefined{SB@cr@#1}{\SB@errreg{#1}}{%
    \SB@trackchfalse%
    \global\expandafter\let\expandafter\SB@creg%
      \csname SB@cr@#1\endcsname%
    \global\SB@ctail\SB@creg%
  }%
}
\newcommand\SB@@replay{%
  \SB@trackchfalse%
  \global\SB@ctail\SB@creg%
}
%    \end{macrocode}
% \end{macro}
% \end{macro}
% \end{macro}
%
% \begin{macro}{\SB@rechord}
% \begin{macro}{\SB@@rechord}
% Replay the same chord that was in a previous verse.
%    \begin{macrocode}
\newcommand\SB@rechord{}
\newcommand\SB@@rechord{%
  \SB@ifempty\SB@ctail{%
    \SB@errreplay%
    \SB@toks{}%
    \let\SB@donext\@gobble%
  }{%
    \SB@lop\SB@ctail\SB@toks%
    \let\SB@donext\SB@chord%
    \let\SB@noreplay\@gobble%
  }%
  \expandafter\SB@donext\the\SB@toks]%
}
%    \end{macrocode}
% \end{macro}
% \end{macro}
%
% \begin{macro}{\ifSB@nohat}
% The |\ifSB@nohat| conditional is set to false when a chord macro contains
% a |^| in its argument.
% This suppresses the recording mechanism momentarily so that replays will
% skip this chord.
%    \begin{macrocode}
\newif\ifSB@nohat
%    \end{macrocode}
% \end{macro}
%
% \begin{macro}{\SB@noreplay}
% Sometimes material must be added to a chord but omitted when the chord is
% replayed.
% We accomplish this by enclosing such material in |\SB@noreplay| macros,
% which are set to |\@gobble| just before a replay and reset to
% |\@firstofone| at other times.
%    \begin{macrocode}
\newcommand\SB@noreplay{}
\let\SB@noreplay\@firstofone
%    \end{macrocode}
% \end{macro}
%
% \subsection{Guitar Tablatures}
%
% The song book software not only supports chord names alone, but can also
% typeset guitar tablature diagrams. The macros for producing these diagrams
% are found here.
%
% \begin{macro}{\SB@fretwidth}
% Set the width of each vertical string in the tablature diagram.
%    \begin{macrocode}
\newlength\SB@fretwidth
\setlength\SB@fretwidth{6\p@}
%    \end{macrocode}
% \end{macro}
%
% \begin{macro}{\SB@fretnum}
% Typeset a fret number to appear to the left of the diagram.
%    \begin{macrocode}
\newcommand\SB@fretnum[1]{{%
  \sffamily\fontsize\@xpt\@xpt\selectfont#1%
}}
%    \end{macrocode}
% \end{macro}
%
% \begin{macro}{\SB@onfret}
% Typeset one string of one fret with \argp{1} typeset overtop of it (usually
% a dot or nothing at all).
%    \begin{macrocode}
\newcommand\SB@onfret[1]{%
  \kern.5\SB@fretwidth\kern-.2\p@%
  \vrule\@height6\p@%
  \kern-.2\p@\kern-.5\SB@fretwidth%
  \hbox to\SB@fretwidth{\hfil#1\hfil}%
}
%    \end{macrocode}
% \end{macro}
%
% \begin{macro}{\SB@atopfret}
% Typeset material (given by \argp{1}) to be placed above a string in the
% tablature diagram.
%    \begin{macrocode}
\newcommand\SB@atopfret[1]{%
  \hbox to\SB@fretwidth{\hfil#1\hfil}%
}
%    \end{macrocode}
% \end{macro}
%
% \begin{macro}{\SB@fretbar}
% Typeset a horizontal fret bar of width |\SB@dimen|.
%    \begin{macrocode}
\newcommand\SB@fretbar{%
  \nointerlineskip%
  \hbox to\SB@dimen{%
    \advance\SB@dimen-\SB@fretwidth%
    \advance\SB@dimen.4\p@%
    \hfil%
    \vrule\@width\SB@dimen\@height.4\p@\@depth\z@%
    \hfil%
  }%
  \nointerlineskip%
}
%    \end{macrocode}
% \end{macro}
%
% \begin{macro}{\SB@topempty}
% \begin{macro}{\SB@topX}
% \begin{macro}{\SB@topO}
% Above a string in a tablature diagram there can be nothing,
% an $\times$, or an $\circ$.
%    \begin{macrocode}
\newcommand\SB@topempty{\SB@atopfret\relax}
\newcommand\SB@topX{\SB@atopfret{%
  \hbox{%
    \kern-.2\p@%
    \fontencoding{OMS}\fontfamily{cmsy}%
    \fontseries{m}\fontshape{n}%
    \fontsize\@viipt\@viipt\selectfont\char\tw@%
    \kern-.2\p@%
  }%
}}
\newcommand\SB@topO{\SB@atopfret{%
  \vrule\@width\z@\@height4.3333\p@\@depth.8333\p@%
  \lower.74\p@\hbox{%
    \fontencoding{OMS}\fontfamily{cmsy}%
    \fontseries{m}\fontshape{n}%
    \fontsize\@xpt\@xpt\selectfont\char14%
  }%
}}
%    \end{macrocode}
% \end{macro}
% \end{macro}
% \end{macro}
%
% \begin{macro}{\SB@doify}
% \begin{macro}{\SB@@doify}
% \begin{macro}{\SB@do}
% Define the macro given in the first argument to equal the fully expanded
% content of the second argument, but with |\SB@do| inserted before each token
% or group.
%    \begin{macrocode}
\newcommand\SB@do[1]{}
\newcommand\SB@doify[2]{%
  \SB@toks{}%
  \edef#1{#2}%
  \expandafter\SB@@doify#1\SB@@doify%
  \edef#1{\the\SB@toks}%
}
\newcommand\SB@@doify[1]{%
  \ifx#1\SB@@doify\else%
    \SB@toks\expandafter{\the\SB@toks\SB@do{#1}}%
    \expandafter\SB@@doify%
  \fi%
}
%    \end{macrocode}
% \end{macro}
% \end{macro}
% \end{macro}
%
% \begin{macro}{\SB@allbarres}
% \begin{macro}{\SB@dobarre}
% Reserve a control sequence to remember all the stacks, start control
% sequences, and end control sequences associated with barre delimiter pairs;
% and a control sequence to perform an arbitrary action on them.
%    \begin{macrocode}
\newcommand\SB@allbarres{}
\newcommand\SB@dobarre{}
%    \end{macrocode}
% \end{macro}
% \end{macro}
%
% \begin{macro}{\SB@barreI}
% \begin{macro}{\SB@barreN}
% \begin{macro}{\SB@barreY}
% As we process strings in order, barres in progress can be in one of three
% states: initial (|\SB@barreI|), deactivated (|\SB@barreN|), or
% tentatively activated (|\SB@barreY|).
%    \begin{macrocode}
\newcommand\SB@barreI{\noexpand\SB@barreI}
\newcommand\SB@barreN{\noexpand\SB@barreN}
\newcommand\SB@barreY{\noexpand\SB@barreY}
%    \end{macrocode}
% \end{macro}
% \end{macro}
% \end{macro}
%
% \begin{macro}{\SB@lowfret}
% \begin{macro}{\SB@@lowfret}
% If we see a lower numbered fret than the current fret within a barre,
% deactivate the barre.
% (It has already been shown on an earlier fret.)
%    \begin{macrocode}
\newcommand\SB@lowfret{%
  \let\SB@dobarre\SB@@lowfret\SB@allbarres%
  \SB@fretempty%
}
\newcommand\SB@@lowfret[3]{{%
  \let\SB@barreI\SB@barreN%
  \let\SB@barreY\SB@barreN%
  \xdef#1{#1}%
}}
%    \end{macrocode}
% \end{macro}
% \end{macro}
%
% \begin{macro}{\SB@bactivate}
% If we see the current fret within a barre, tentatively activate the barre
% (unless it is already deactivated).
%    \begin{macrocode}
\newcommand\SB@bactivate[3]{{%
  \let\SB@barreI\SB@barreY%
  \xdef#1{#1}%
}}
%    \end{macrocode}
% \end{macro}
%
% \begin{macro}{\SB@bbarre}
% Starting a barre group pushes it onto its stack in the initial state.
%    \begin{macrocode}
\newcommand\SB@bbarre[1]{%
  \xdef#1{\SB@barreI{\the\SB@cntii}#1}%
}
%    \end{macrocode}
% \end{macro}
%
% \begin{macro}{\SB@ebarre}
% \begin{macro}{\SB@@ebarre}
% \begin{macro}{\SB@@@ebarre}
% Ending a barre group pops it and draws it if it's active.
%    \begin{macrocode}
\newcommand\SB@ebarre[3]{%
  \ifx#1\@empty%
    \ifnum\SB@cnt=\@ne\SB@errebar#2#3\fi%
  \else%
    \expandafter\SB@@ebarre#1\SB@@ebarre#1%
  \fi%
}
\newcommand\SB@@ebarre{}
\def\SB@@ebarre#1#2#3\SB@@ebarre#4{{%
  \gdef#4{#3}%
  \let\SB@barreI\@gobble%
  \let\SB@barreN\@gobble%
  \let\SB@barreY\SB@barre%
  #1{#2}%
}}
%    \end{macrocode}
% \end{macro}
% \end{macro}
% \end{macro}
%
% \begin{macro}{\SB@barreson}
% \begin{macro}{\SB@barresoff}
% Turn barre delimiters on or off, depending on whether we're typesetting
% the interior or upper part of the tablature diagram.
%    \begin{macrocode}
\newcommand\SB@barreson[3]{%
  \def#2{\SB@bbarre#1}%
  \def#3{\SB@ebarre#1#2#3}%
}
\newcommand\SB@barresoff[3]{\let#2\relax\let#3\relax}
%    \end{macrocode}
% \end{macro}
% \end{macro}
%
% \begin{macro}{\SB@fretempty}
% \begin{macro}{\SB@fretdot}
% \begin{macro}{\SB@@fretdot}
% On a string in a fret diagram there can be nothing or a filled circle.
%    \begin{macrocode}
\newcommand\SB@fretempty{%
  \advance\SB@cntii\@ne%
  \SB@onfret\relax%
}
\newcommand\SB@fretdot{%
  \advance\SB@cntii\@ne%
  \let\SB@dobarre\SB@bactivate\SB@allbarres%
  \SB@@fretdot%
}
\newcommand\SB@@fretdot{%
  \SB@onfret{%
    \fontencoding{OMS}\fontfamily{cmsy}%
    \fontseries{m}\fontshape{n}%
    \fontsize\@xiipt\@xiipt\selectfont\char15%
  }%
}
%    \end{macrocode}
% \end{macro}
% \end{macro}
% \end{macro}
%
% \begin{macro}{\SB@barre}
% Draw a barre.
%    \begin{macrocode}
\newcommand\SB@barre[1]{{%
  \SB@dimen\SB@fretwidth%
  \multiply\SB@dimen\SB@cntii%
  \advance\SB@dimen-#1\SB@fretwidth%
  \kern-\SB@dimen%
  \SB@@fretdot%
  \kern-.5\SB@fretwidth%
  \advance\SB@dimen-\SB@fretwidth%
  \raise.7pt\hbox{\vrule\@height4.6\p@\@width\SB@dimen}%
  \kern-.5\SB@fretwidth%
  \SB@@fretdot%
}}
%    \end{macrocode}
% \end{macro}
%
% \begin{macro}{\SB@fretend}
% At the end of a barred row in a tablature diagram, we auto-finish any
% activated barres that weren't explicitly closed by the user.
%    \begin{macrocode}
\newcommand\SB@fretend{{%
  \let\SB@barreI\@gobble%
  \let\SB@barreN\@gobble%
  \let\SB@barreY\SB@barre%
  \def\SB@dobarre##1##2##3{##1\gdef##1{}}\SB@allbarres%
}}
%    \end{macrocode}
% \end{macro}
%
% \begin{macro}{\SB@finger}
% \begin{macro}{\SB@X}
% \begin{macro}{\SB@Z}
% \begin{macro}{\SB@O}
% If we're including fingering info in the tablature diagram, then below
% each string there might be a number.
%    \begin{macrocode}
\newcommand*\SB@X{X}
\newcommand*\SB@Z{0}
\newcommand*\SB@O{O}
\newcommand\SB@finger[1]{%
  \def\SB@temp{#1}%
  \ifx\SB@temp\SB@X\SB@topempty\else%
  \ifx\SB@temp\SB@Z\SB@topempty\else%
  \ifx\SB@temp\SB@O\SB@topempty\else%
    \SB@atopfret{\sffamily\fontsize\@vipt\@vipt\selectfont#1}%
  \fi\fi\fi%
}
%    \end{macrocode}
% \end{macro}
% \end{macro}
% \end{macro}
% \end{macro}
%
% \begin{macro}{\ifSB@gettabind}
% \begin{macro}{\SB@tabindent}
% Lyrics under tablature diagrams look odd if they aren't aligned with the
% leftmost string of the diagram.
% To accomplish this, the following two macros record the amount by which
% a lyric under this tablature diagram must be indented to position it
% properly.
%    \begin{macrocode}
\newif\ifSB@gettabind\SB@gettabindfalse
\SB@newdimen\SB@tabindent
%    \end{macrocode}
% \end{macro}
% \end{macro}
%
% \begin{macro}{\SB@targfret}
% \begin{macro}{\SB@targstr}
% \begin{macro}{\SB@targfing}
% Reserve some macro names in which to store the three pieces of the
% second argument to the |\gtab| macro.
% The first is for the fret number, the second is for the \Meta{strings}
% info, and the last is for the \Meta{fingering} info.
%    \begin{macrocode}
\newcommand\SB@targfret{}
\newcommand\SB@targstr{}
\newcommand\SB@targfing{}
%    \end{macrocode}
% \end{macro}
% \end{macro}
% \end{macro}
%
% In general |\gtab| macros often appear inside chord macros, which means
% that their arguments have already been scanned by the time the
% |\gtab| macro itself is expanded.
% This means that catcodes cannot be reassigned (without resorting to
% $\varepsilon$-\TeX).
%
% We therefore adopt the alternative strategy of converting each token
% in the \Meta{strings} and \Meta{fingering} arguments of a |\gtab| macro
% into a control sequence (using |\csname|).
% We can then temporarily assign meanings to those control sequences and
% replay the arguments to achieve various effects.
%
% \begin{macro}{\SB@gtinit}
% \begin{macro}{\SB@gtinc}
% Different meanings are assigned to digits, |X|'s, and |O|'s
% as we typeset each row of the interior of the diagram.
% These meanings are set by |\SB@gtinit| and |\SB@gtinc|.
%    \begin{macrocode}
\newcommand\SB@gtinit{%
  \def\SB@do##1{\csname##1\endcsname}%
  \let\O\0%
  \let\3\2\let\4\2\let\5\2\let\6\2%
  \let\7\2\let\8\2\let\9\2%
}
\newcommand\SB@gtinc{%
  \advance\SB@cnt\@ne%
  \let\9\8\let\8\7\let\7\6\let\6\5\let\5\4%
  \let\4\3\let\3\2\let\2\1\let\1\SB@lowfret%
}
%    \end{macrocode}
% \end{macro}
% \end{macro}
%
% \begin{macro}{\BarreDelims}
% \begin{macro}{\SB@bdelims}
% Each pair of barre delimiters reserves a stack and augments the
% initialization state to recognize those delimiters.
%    \begin{macrocode}
\newcommand\BarreDelims[2]{%
  \expandafter\SB@bdelims\csname SB@bs@#1#2\expandafter\endcsname%
    \csname#1\expandafter\endcsname\csname#2\endcsname%
}
\newcommand\SB@bdelims[3]{%
  \newcommand*#1{}%
  \SB@app\def\SB@allbarres{\SB@dobarre#1#2#3}%
}
\BarreDelims()
\BarreDelims[]
%    \end{macrocode}
% \end{macro}
% \end{macro}
%
% \begin{macro}{\gtab}
% \begin{macro}{\SB@gtab}
% \changes{v2.9}{2009/03/27}{Fixed compatibility issue with Babel French}
% A |\gtab| macro begins by setting catcodes suitable for parsing a chord
% name as its first argument.
% This allows tokens like |#| and |&| to be used for sharp and flat even
% when |\gtab| is used outside a chord macro.
% Colon is reset to a non-active character while processing the second
% argument to avoid a potential conflict with \textsf{Babel} French.
%    \begin{macrocode}
\newcommand\gtab{\SB@begincname\SB@gtab}
\newcommand*\SB@gtab[1]{%
  \SB@endcname%
  \begingroup%
    \catcode`:12\relax%
    \SB@@gtab{#1}%
}
%    \end{macrocode}
% \end{macro}
% \end{macro}
%
% \begin{macro}{\SB@@gtab}
% If transposition is currently taking place, allow the user to customize
% the behavior by redefining |\gtabtrans|.
% Using |\gtab| within |\gtabtrans| should go directly to |\SB@@@gtab|
% (otherwise an infinite loop would result!).
%    \begin{macrocode}
\newcommand*\SB@@gtab[2]{%
  \endgroup%
  \ifnum\SB@transposefactor=\z@%
    \SB@@@gtab{#1}{#2}%
  \else%
    \begingroup%
      \let\gtab\SB@@@gtab%
      \gtabtrans{#1}{#2}%
    \endgroup%
  \fi%
}
%    \end{macrocode}
% \end{macro}
%
% \begin{macro}{\gtabtrans}\MainImpl{gtabtrans}
% By default, transposed guitar tablatures just display the transposed
% chord name and omit the diagram.
% Transposing a tablature diagram requires manual judgment calls for most
% stringed instruments, so we can't do it automatically.
%    \begin{macrocode}
\newcommand\gtabtrans[2]{\transposehere{#1}}
%    \end{macrocode}
% \end{macro}
%
% \begin{macro}{\SB@@@gtab}\MainImpl{gtab}
% \changes{v2.13}{2011/04/16}{Added transposition for chord names}
% Typeset a full tablature diagram.
% Text \argp{1} is a chord name placed above the diagram.
% Text \argp{2} consists of a colon-separated list of:
% (1) an optional fret number placed to the left of the diagram;
% (2) a sequence of tokens, each of which can be
% |X| (to place an $\times$ above the string),
% |0| or |O| (to place an $\circ$ above the string), or
% one of |1| through |9| (to place a filled circle on that string at the
% fret of the given number); and
% (3) an optional sequence of tokens, each of which is either |0|
% (no fingering information for that string),
% or one of |1| through |4| (to place the given number under that string).
%    \begin{macrocode}
\newcommand\SB@@@gtab[2]{%
  \let\SB@targfret\@empty%
  \let\SB@targstr\@empty%
  \let\SB@targfing\@empty%
  \SB@tabargs#2:::\SB@tabargs%
  \ifx\SB@targstr\@empty%
    \def\SB@targstr{\0\0\0\0\0\0}%
  \fi%
  \ifvmode\leavevmode\fi%
  \vbox{%
    \normalfont\normalsize%
    \setbox\SB@box\hbox{%
      \thinspace{\printchord{\transposehere{#1}\strut}}\thinspace%
    }%
    \setbox\SB@boxii\hbox{\SB@fretnum{\SB@targfret}}%
    \setbox\SB@boxiii\hbox{{%
      \let\X\SB@topX\let\0\SB@topO%
      \let\1\SB@topempty\let\2\1%
      \SB@gtinit%
      \let\SB@dobarre\SB@barresoff\SB@allbarres%
      \SB@targstr%
    }}%
    \hsize\wd\SB@box%
    \ifSB@gettabind%
      \global\SB@tabindent\wd\SB@boxii%
      \global\advance\SB@tabindent.5\SB@fretwidth%
      \global\advance\SB@tabindent-.5\p@%
    \fi%
    \SB@dimen\wd\SB@boxii%
    \advance\SB@dimen\wd\SB@boxiii%
    \ifdim\hsize<\SB@dimen%
      \hsize\SB@dimen%
    \else\ifSB@gettabind%
      \SB@dimenii\hsize%
      \advance\SB@dimenii-\SB@dimen%
      \divide\SB@dimenii\tw@%
      \global\advance\SB@tabindent\SB@dimenii%
    \fi\fi%
    \hbox to\hsize{\hfil\unhbox\SB@box\hfil}%
    \kern-\p@\nointerlineskip%
    \hbox to\hsize{%
      \hfil%
      \vtop{\kern\p@\kern2\p@\box\SB@boxii}%
      \vtop{%
        \SB@dimen\wd\SB@boxiii%
        \box\SB@boxiii%
        \let\X\SB@fretempty\let\0\X%
        \let\1\SB@fretdot\def\2{\SB@fretempty\global\SB@testtrue}%
        \SB@gtinit%
        \let\SB@dobarre\SB@barreson\SB@allbarres%
        \SB@cnt\@ne%
        \loop%
          \SB@testfalse%
          \SB@fretbar\hbox{\SB@cntii\z@\SB@targstr\SB@fretend}%
          \ifnum\SB@cnt<\minfrets\SB@testtrue\fi%
        \ifSB@test\SB@gtinc\repeat%
        \SB@fretbar%
        \ifx\SB@targsfing\@empty\else%
          \kern1.5\p@%
          \hbox{\let\SB@do\SB@finger\SB@targfing}%
        \fi%
      }%
      \hfil%
    }%
    \kern3\p@%
  }%
  \SB@gettabindfalse%
}
%    \end{macrocode}
% \end{macro}
%
% \begin{macro}{\SB@tabargs}
% \begin{macro}{\SB@@tabargs}
% \begin{macro}{\SB@ctoken}
% \changes{v3.1}{2017/06/23}{Allow macros within 2nd gtab arg}
% Break the second argument to a |\gtab| macro into three sub-arguments.
% The possible forms are:
% (a) \Meta{strings},
% (b) \Meta{fret}|:|\Meta{strings},
% (c) \Meta{strings}|:|\Meta{fingering}, or
% (d) \Meta{fret}|:|\Meta{strings}|:|\Meta{fingering}.
% To distinguish forms (b) and (c), we count the number of tokens before
% the first colon.
% If there is only one token or group, we assume it must be form (b),
% since frets larger than 9 and 1-stringed instruments are both rare.
% Otherwise we assume form (c).
%    \begin{macrocode}
\newcommand\SB@ctoken{} \def\SB@ctoken{:}
\newcommand\SB@tabargs{}
\def\SB@tabargs#1:#2:#3:#4\SB@tabargs{%
  \def\SB@temp{#4}%
  \ifx\SB@temp\@empty%
    \SB@doify\SB@targstr{#1}%
  \else\ifx\SB@temp\SB@ctoken%
    \SB@@tabargs#1\SB@@tabargs%
    \ifx\SB@temp\@empty%
      \def\SB@targfret{#1}%
      \SB@doify\SB@targstr{#2}%
    \else%
      \SB@doify\SB@targfing{#2}%
      \SB@doify\SB@targstr{#1}%
    \fi%
  \else%
    \def\SB@targfret{#1}%
    \SB@doify\SB@targfing{#3}%
    \SB@doify\SB@targstr{#2}%
  \fi\fi%
}
\newcommand\SB@@tabargs{}
\def\SB@@tabargs#1#2\SB@@tabargs{\def\SB@temp{#2}}
%    \end{macrocode}
% \end{macro}
% \end{macro}
% \end{macro}
%
% \subsection{Book Sectioning}
%
% The following macros divide the song book into distinct sections, each with
% different headers, different song numbering styles, different indexes, etc.
%
% \begin{macro}{\songchapter}\MainImpl{songchapter}
% \changes{v1.19}{2005/10/24}{Added}
% Format the chapter header for a chapter in a song book.
% By default, chapter headers on a song book omit the chapter number, but do
% include an entry in the pdf index or table of contents.
% Thus, the chapter has a number; it's just not displayed at the start of
% the chapter.
%    \begin{macrocode}
\newcommand\songchapter{%
  \let\SB@temp\@seccntformat%
  \def\@seccntformat##1{}%
  \@startsection{chapter}{0}{\z@}%
    {3.5ex\@plus1ex\@minus.2ex}%
    {.4ex\let\@seccntformat\SB@temp}%
    {\sffamily\bfseries\LARGE\centering}%
}
%    \end{macrocode}
% \end{macro}
%
% \begin{macro}{\songsection}\MainImpl{songsection}
% \changes{v1.19}{2005/10/24}{Section headers changed to omit numbers}
% Format the section header for a section in a song book.
% This is the same as for chapter headers except at the section level.
%    \begin{macrocode}
\newcommand\songsection{%
  \let\SB@temp\@seccntformat%
  \def\@seccntformat##1{}%
  \@startsection{section}{1}{\z@}%
    {3.5ex\@plus1ex\@minus.2ex}%
    {.4ex\let\@seccntformat\SB@temp}%
    {\sffamily\bfseries\LARGE\centering}%
}
%    \end{macrocode}
% \end{macro}
%
% \begin{environment}{songs}\MainEnvImpl{songs}
% \changes{v1.19}{2005/10/24}{Song numbers now starts at one instead of zero}
% Begin and end a book section.
% The argument is a list of indexes with which to associate songs in this
% section. 
%    \begin{macrocode}
\newenvironment{songs}[1]{%
  \ifSB@songsenv\SB@errnse\fi%
  \gdef\SB@indexlist{#1}%
  \SB@chkidxlst%
  \stepcounter{SB@songsnum}%
  \setcounter{songnum}{1}%
  \let\SB@sgroup\@empty%
  \ifinner\else\ifdim\pagetotal>\z@%
    \null\nointerlineskip%
  \fi\fi%
  \songcolumns\SB@numcols%
  \SB@songsenvtrue%
}{%
  \commitsongs%
  \global\let\SB@indexlist\@empty%
  \ifinner\else\clearpage\fi%
  \SB@songsenvfalse%
}
%    \end{macrocode}
% \end{environment}
%
% Each |songs| section needs a unique number to aid in hyperlinking.
%    \begin{macrocode}
\newcounter{SB@songsnum}
%    \end{macrocode}
%
% \subsection{Index Generation}
% \label{sec:indexgen}
%
% The following macros generate the various types of indexes. At present there
% are four types:
% \begin{enumerate}
% \item A ``large'' index has a separate section for each capital letter and
% is printed in two columns.
% \item A ``small'' index has only a single column, centered, and has no
% sections.
% \item A ``scripture'' index has three columns and each entry has a
% comma-separated list of references.
% \item An ``author'' index is like a large index except in bold and without
% the sectioning.
% \end{enumerate}
% ``Large'' and ``small'' indexes will be chosen automatically based on the
% number of index entries when building a song index. The other two types are
% designated by the user.
%
% As is typical of \LaTeX{} indexes, generation of song book indexes requires
% two passes of document compilation. During the first pass, data files are
% generated with song titles, authors, and scripture references. An external
% program is then used to produce \LaTeX{} source files from those data files.
% During the second pass of document compilation, those source files are
% imported to typeset all the indexes and display them in the document.
%
% Internally, this package code uses a \emph{four} step process to move
% the index data from the source |.tex| file to the |.sxd| data files.
% \begin{enumerate}
% \item While the current song box is in the midst of construction,
% the data is stored in a box of non-immediate write whatsit nodes.
% \item The whatsits are migrated out to the top of the song box when
% it is finalized at |\endsong|.
% \item When the song box is shipped out to the output file, \TeX{} expands
% the whatsits, causing the data to be written to the |.sxc| auxiliary file.
% \item At the |\end{document}| line, the |.sxc| is processed multiple
% times---once for each index---to split the data into the respective
% |.sxd| files.
% \end{enumerate}
% The first and second steps allow index references to point to the
% beginning of the song no matter where the indexing commands appear
% within the song.
% The third step allows \TeX{} to drop index entries that refer to
% songs that do not actually appear in the output (e.g., because of
% |\includeonlysongs|).
% It also allows index entries to refer to information that is only decided
% at shipout time, such as page numbers.
% The fourth step allows all indexing to be accomplished with at most one
% write register.
% \LaTeX{} provides extremely few write registers, so using as
% few as possible is essential for supporting books with many indexes.
%
% \begin{macro}{\songtarget}\MainImpl{songtarget}
% This macro is invoked by each \mac{beginsong} environment with two arguments:
% (1) a suggested pdf bookmark index level, and
% (2) a target name to which hyperlinks for this song in the index will refer.
% The macro is expected to produce a suitable pdf bookmark entry and/or
% link target.
% The default definition tries to use |\pdfbookmark| if generating a PDF,
% and resorts to |\hypertarget| (if it exists) otherwise.
% The user can redefine the macro to customize how and whether bookmarks
% and/or links are created.
%    \begin{macrocode}
\newcommand\songtarget[2]{%
  \ifnum\@ne=0\ifSB@pdf\ifx\pdfbookmark\undefined\else%
                       \ifx\pdfbookmark\relax\else1\fi\fi\fi\relax%
    \pdfbookmark[#1]{\thesongnum. \songtitle}{#2}%
  \else\ifx\hypertarget\undefined%
  \else\ifx\hypertarget\relax\else%
    \hypertarget{#2}{\relax}%
  \fi\fi\fi%
}
%    \end{macrocode}
% \end{macro}
%
% \begin{macro}{\songlink}\MainImpl{songlink}
% This macro is invoked by the index code to produce a link to a song target
% created by \mac{songtarget}.
% Its two arguments are:
% (1) the target name (same as the second argument to \mac{songtarget}, and
% (2) the text that is to be linked.
% The default implementation uses |\hyperlink| if it exists; otherwise it
% just leaves the text unlinked.
%    \begin{macrocode}
\newcommand\songlink{%
  \ifnum\@ne=0\ifx\hyperlink\undefined\else%
              \ifx\hyperlink\relax\else1\fi\fi\relax%
    \expandafter\hyperlink%
  \else%
    \expandafter\@gobble%
  \fi%
}
%    \end{macrocode}
% \end{macro}
%
% \begin{macro}{\SB@indexlist}
% This macro records the comma-separated list of the identifiers of indexes
% associated with the current book section.
%    \begin{macrocode}
\newcommand\SB@indexlist{}
%    \end{macrocode}
% \end{macro}
%
% \begin{macro}{\SB@allindexes}
% This macro records a comma-separated list of all the index identifiers
% for the entire document.
%    \begin{macrocode}
\newcommand\SB@allindexes{}
\let\SB@allindexes\@empty
%    \end{macrocode}
% \end{macro}
%
% \begin{macro}{\SB@out}
% The |\SB@out| control sequence is reserved for the write register allocated
% by the package code, if one is needed.
% (It is allocated at the first index declaration.)
%    \begin{macrocode}
\newcommand\SB@out{}
\let\SB@out\relax
%    \end{macrocode}
% \end{macro}
%
% \begin{macro}{\SB@newindex}
% Initialize a new title, author, or scripture index.
%    \begin{macrocode}
\newcommand\SB@newindex[4]{%
  \expandafter\newcommand\csname SB@idxfilename@#3\endcsname{#4}%
  \expandafter\newcommand\csname SB@idxsel@#3\endcsname[3]{###1}%
  \expandafter\newcommand\csname SB@idxref@#3\endcsname{\thesongnum}%
  \xdef\SB@allindexes{%
    \ifx\SB@allindexes\@empty\else\SB@allindexes,\fi#3%
  }%
  \if@filesw%
    \ifx\SB@out\relax%
      \SB@newwrite\SB@out%
      \immediate\openout\SB@out=\jobname.sxc\relax%
    \fi%
    \immediate\write\SB@out{\noexpand\SB@iwrite{#3}{#2}}%
  \fi%
}
%    \end{macrocode}
% \end{macro}
%
% \begin{macro}{\newindex}\MainImpl{newindex}
% Define a new title index.
% The first argument is an identifier for the index (used in constructing
% index-specific control sequence names).
% The second argument is a filename root; auxiliary file \argp{2}|.sxd| is
% where the index data is stored at the end of processing.
%    \begin{macrocode}
\newcommand\newindex{\SB@newindex1{TITLE INDEX DATA FILE}}
\@onlypreamble\newindex
%    \end{macrocode}
% \end{macro}
%
% \begin{macro}{\newscripindex}\MainImpl{newscripindex}
% Define a new scripture index. This is exactly like |\newindex| except that
% scripture references are added to the auxiliary file instead of titles.
%    \begin{macrocode}
\newcommand\newscripindex{\SB@newindex2{SCRIPTURE INDEX DATA FILE}}
\@onlypreamble\newscripindex
%    \end{macrocode}
% \end{macro}
%
% \begin{macro}{\newauthorindex}\MainImpl{newauthorindex}
% Define a new author index. This is exactly like |\newindex| except that
% author info is added to the auxiliary file instead of titles.
%    \begin{macrocode}
\newcommand\newauthorindex{\SB@newindex3{AUTHOR INDEX DATA FILE}}
\@onlypreamble\newauthorindex
%    \end{macrocode}
% \end{macro}
%
% \begin{macro}{\SB@cwrite}
% Write index data to a Song indeX Combined (|.sxc|) auxiliary file.
% The first argument is the identifier for the index to which the data
% ultimately belongs.
% The second argument is the data itself.
% The write is non-immediate so that it is only output if its enclosing
% song is ultimately shipped to the output file.
%    \begin{macrocode}
\newcommand\SB@cwrite[2]{%
  \ifx\SB@out\relax\else%
    \protected@write\SB@out\SB@keepactive{\protect\SB@iwrite{#1}{#2}}%
  \fi%
}
%    \end{macrocode}
% \end{macro}
%
% \begin{macro}{\SB@keepactive}
% By default, the |inputenc| package expands Unicode characters into macro
% names when writing them to files.
% This behavior must be inhibited when writing to the |.sxc| file, since
% |songidx| needs the original Unicode characters for sorting.
% To achieve this, we temporarily redefine most active characters so that
% they expand to an unexpandable string version of themselves.
%    \begin{macrocode}
\newcommand\SB@keepactive{}
{\catcode`\~\active
 \catcode`\.12
 \def\\#1#2{%
   \endgroup
   \SB@app\gdef\SB@keepactive{\def#1{#2}}%
 }
 \def\SB@temp#1#2{%
   \SB@cnt#1\relax
   \loop
     \begingroup
       \uccode`\~\SB@cnt
       \uccode`\.\SB@cnt
     \uppercase{\\~.}
   \ifnum\SB@cnt<#2\relax
     \advance\SB@cnt\@ne
   \repeat
 }
 \SB@temp{1}{8}
 \SB@temp{11}{11}
 \SB@temp{14}{91}
 \SB@temp{93}{255}
}
%    \end{macrocode}
% \end{macro}
%
% \begin{macro}{\SB@iwrite}
% The line contributed by |\SB@cwrite| to the |.sxc| file is an |\SB@iwrite|
% macro that re-outputs the data to an appropriate |.sxd| file.
%    \begin{macrocode}
\newcommand\SB@iwrite[2]{%
  \def\SB@tempii{#1}%
  \ifx\SB@temp\SB@tempii%
    \SB@toks{#2}%
    \immediate\write\SB@out{\the\SB@toks}%
  \fi%
}
%    \end{macrocode}
% \end{macro}
%
% \begin{macro}{\SB@uncombine}
% At the end of the document, the |.sxc| file can be processed multiple
% times to produce all the |.sxd| files without resorting to multiple write
% registers.
% Each pass activates the subset of the |\SB@iwrite| commands that apply to
% one index.
%    \begin{macrocode}
\newcommand\SB@uncombine{%
  \ifx\SB@out\relax\else%
    \immediate\closeout\SB@out%
    \ifsongindexes%
      \@for\SB@temp:=\SB@allindexes\do{%
        \immediate\openout\SB@out=%
          \csname SB@idxfilename@\SB@temp\endcsname.sxd\relax%
        \begingroup\makeatletter\catcode`\%12\relax%
                   \input{\jobname.sxc}\endgroup%
        \immediate\closeout\SB@out%
      }%
    \fi%
  \fi%
}
\AtEndDocument{\SB@uncombine}
%    \end{macrocode}
% \end{macro}
%
% \begin{macro}{\SB@songwrites}
% The following box register stores index data until it can be migrated to
% the top of the song box currently under construction.
%    \begin{macrocode}
\SB@newbox\SB@songwrites
%    \end{macrocode}
% \end{macro}
%
% \begin{macro}{\SB@addtoindex}
% \changes{v2.8}{2009/02/03}{Writes made non-immediate}
% Queue data \argp{2} associated with the current song for eventual writing
% to the index whose identifier is given by \argp{1}.
%    \begin{macrocode}
\newcommand\SB@addtoindex[2]{%
  \protected@edef\SB@tempii{#2}%
  \ifx\SB@tempii\@empty\else%
    \global\setbox\SB@songwrites\vbox{%
      \unvbox\SB@songwrites%
      \SB@cwrite{#1}{#2}%
      \SB@cwrite{#1}{\csname SB@idxref@#1\endcsname}%
      \SB@cwrite{#1}{song\theSB@songsnum-\thesongnum.%
                     \ifnum\c@section=\z@1\else2\fi}%
    }%
  \fi%
}
%    \end{macrocode}
% \end{macro}
%
% \begin{macro}{\SB@addtoindexes}
% Add \argp{1} to all title indexes, \argp{2} to all scripture indexes, and
% \argp{3} to all author indexes.
%    \begin{macrocode}
\newcommand\SB@addtoindexes[3]{%
  \@for\SB@temp:=\SB@indexlist\do{%
    \SB@addtoindex\SB@temp%
      {\csname SB@idxsel@\SB@temp\endcsname{#1}{#2}{#3}}%
  }%
}
%    \end{macrocode}
% \end{macro}
%
% \begin{macro}{\SB@addtotitles}
% Add \argp{1} to all title indexes, but leave other indexes unaffected.
%    \begin{macrocode}
\newcommand\SB@addtotitles[1]{%
  \@for\SB@temp:=\SB@indexlist\do{%
    \csname SB@idxsel@\SB@temp\endcsname%
      {\SB@addtoindex\SB@temp{#1}}{}{}%
  }%
}
%    \end{macrocode}
% \end{macro}
%
% \begin{macro}{\SB@chkidxlst}
% \changes{v2.3}{2007/09/23}{Added.}
% Check the current list of indexes and flag an error if any are undefined.
%    \begin{macrocode}
\newcommand\SB@chkidxlst{%
  \let\SB@temp\SB@indexlist%
  \let\SB@indexlist\@empty%
  \@for\SB@tempii:=\SB@temp\do{%
    \@ifundefined{SB@idxsel@\SB@tempii}{\SB@errnoidx\SB@tempii}{%
      \ifx\SB@indexlist\@empty%
        \SB@toks\expandafter{\SB@tempii}%
      \else%
        \SB@toks\expandafter\expandafter\expandafter{%
          \expandafter\SB@indexlist\expandafter,\SB@tempii}%
      \fi%
      \edef\SB@indexlist{\the\SB@toks}%
    }%
  }%
}
%    \end{macrocode}
% \end{macro}
%
% \begin{macro}{\indexentry}\MainImpl{indexentry}
% \changes{v2.3}{2007/09/23}{Optional argument added}
% \begin{macro}{\SB@idxentry}
% \begin{macro}{\SB@@idxentry}
% |\SB@addtoindexes| will be called automatically for each song in a section.
% However, |\indexentry| may be called by the user in order to add an
% alternative index entry for the given song.
% Usually this is done to index the song by its first line or some other
% memorable line in a chorus or verse somewhere.
%    \begin{macrocode}
\newcommand\indexentry{\@ifnextchar[{\SB@idxentry*}{\SB@@idxentry*}}
\newcommand\SB@idxentry{}
\def\SB@idxentry#1[#2]#3{{%
  \def\SB@indexlist{#2}%
  \SB@chkidxlst%
  \SB@addtoindexes{#1#3}{#3}{#3}%
}}
\newcommand\SB@@idxentry[2]{\SB@addtotitles{#1#2}}
%    \end{macrocode}
% \end{macro}
% \end{macro}
% \end{macro}
%
% \begin{macro}{\indextitleentry}\MainImpl{indextitleentry}
% \changes{v2.3}{2007/09/23}{Optional argument added}
% |\indextitleentry| may be used to add an alternate title for the song to
% the index.
% (The only difference between the effects of |\indexentry| and
% |\indextitleentry| is that the latter are italicized in the rendered index
% and the former are not.)
%    \begin{macrocode}
\newcommand\indextitleentry{%
  \@ifnextchar[{\SB@idxentry{}}{\SB@@idxentry{}}%
}
%    \end{macrocode}
% \end{macro}
%
% \begin{macro}{\indexsongsas}\MainImpl{indexsongsas}
% \changes{v2.8}{2009/02/03}{Added.}
% The following macro allows the user to change how songs are indexed on the
% right side of index entries.
% By default, the song's number is listed.
%    \begin{macrocode}
\newcommand\indexsongsas[1]{%
  \@ifundefined{SB@idxref@#1}%
    {\SB@errnoidx{#1}\@gobble}%
    {\expandafter\renewcommand\csname SB@idxref@#1\endcsname}%
}
%    \end{macrocode}
% \end{macro}
%
% \begin{macro}{\SB@idxcmd}
% \begin{macro}{\SB@@idxcmd}
% \begin{macro}{\authsepword}\MainImpl{authsepword}
% \begin{macro}{\authbyword}\MainImpl{authbyword}
% \begin{macro}{\authignoreword}\MainImpl{authignoreword}
% \begin{macro}{\titleprefixword}\MainImpl{titleprefixword}
% \changes{v2.0}{2007/06/18}{Added.}
% The |songidx| index-generation script understands several different
% directives that each dictate various aspects of how index entries are
% parsed, sorted, and displayed.
% Such directives should typically appear at the start of the |.sxd|
% file just after the header line that identifies the type of index.
%    \begin{macrocode}
\newcommand\SB@idxcmd[3]{%
  \ifx\SB@allindexes\@empty%
    \SB@warnnoidx%
  \else\ifx\SB@out\relax\else%
    \@for\SB@temp:=\SB@allindexes\do{%
      \csname SB@idxsel@\SB@temp\endcsname%
        {\SB@@idxcmd{#1}}{\SB@@idxcmd{#2}}{\SB@@idxcmd{#3}}%
    }%
  \fi\fi%
}
\newcommand\SB@@idxcmd[1]{%
  \def\SB@tempii{#1}%
  \ifx\SB@tempii\@empty\else%
    \immediate\write\SB@out{%
      \noexpand\SB@iwrite{\SB@temp}{#1}%
    }%
  \fi%
}
\newcommand\authsepword[1]{}
\newcommand\authbyword[1]{}
\newcommand\authignoreword[1]{}
\newcommand\titleprefixword[1]{}
{\catcode`\%=12
 \gdef\authsepword#1{\SB@idxcmd{}{}{%sep #1}}
 \gdef\authbyword#1{\SB@idxcmd{}{}{%after #1}}
 \gdef\authignoreword#1{\SB@idxcmd{}{}{%ignore #1}}
 \gdef\titleprefixword#1{\SB@idxcmd{%prefix #1}{}{}}}
\@onlypreamble\authsepword
\@onlypreamble\authbyword
\@onlypreamble\authignoreword
\@onlypreamble\titleprefixword
%    \end{macrocode}
% \end{macro}
% \end{macro}
% \end{macro}
% \end{macro}
% \end{macro}
% \end{macro}
%
% \begin{macro}{\SB@idxlineskip}
% Set the spacing between lines in an index.
%    \begin{macrocode}
\newcommand\SB@idxlineskip[1]{%
  \vskip#1\p@\@plus#1\p@\@minus#1\p@%
}
%    \end{macrocode}
% \end{macro}
%
% When rendering an index entry $X\ldots Y$ that is too long to fit on one
% physical line, we must break text $X$ and/or $Y$ up into multiple lines.
% Text $X$ should be typeset as a left-justified paragraph with a right
% margin of about 2em; however, its final line must not be so long that it
% cannot fit even the first item of list $Y$.
% Text $Y$ should be typeset as a right-justified paragraph whose first line
% begins on the last line of $X$.
% However, breaking $Y$ up the way paragraphs are normally broken up doesn't
% work well because that causes most of $Y$ to be crammed into the first few
% lines, leaving the last line very short.
% This looks strange and is hard to read.
% It looks much better to instead break $Y$ up in such a way that the portion
% of $Y$ that is placed on each line is of approximately equal width (subject
% to the constraint that we don't want to introduce any more lines than are
% necessary).
% This makes it visually clear that all of these lines are associated with $X$.
% The following code performs the width computations that do this
% horizontal-balancing of text.
%
% \begin{macro}{\SB@ellipspread}
% Typeset an index entry of the form $X\ldots Y$.
% In the common case, the entire entry fits on one line so we just typeset
% it in the usual way.
% If it doesn't fit on one line, we call |\SB@balancerows| for a more
% sophisticated treatment.
%    \begin{macrocode}
\newcommand\SB@ellipspread[2]{%
  \begingroup%
    \SB@dimen\z@%
    \def\SB@temp{#1}%
    \SB@toks{#2}%
    \setbox\SB@box\hbox{{%
      \SB@temp%
      \leaders\hbox to.5em{\hss.\hss}\hskip2em\@plus1fil%
      {\the\SB@toks}%
    }}%
    \ifdim\wd\SB@box>\hsize%
      \SB@balancerows%
    \else%
      \hbox to\hsize{\unhbox\SB@box}\par%
    \fi%
  \endgroup%
}
%    \end{macrocode}
% \end{macro}
%
% \begin{macro}{\SB@balancerows}
% Typeset an index entry of the form $X\ldots Y$ that doesn't fit on one line,
% where $X$ is the content of macro |\SB@temp| and
% $Y$ is the content of token register |\SB@toks|.
%
% First, we must pre-compute the width $w_1$ of the final line of $X$ when
% $X$ is typeset as a left-justified paragraph, storing it in |\SB@dimenii|.
% This is necessary because in order to force \TeX{} to typeset the first
% line of $Y$ at some chosen width $w_2$, we must insert leaders of width
% $c-w_1-w_2$ into the paragraph between $X$ and $Y$, where $c$ is the column
% width.
%
% Computing this width $w_1$ is a bit tricky.
% We must tell \TeX{} that the last line of $X$ must not be so long that it
% does not even have room for the first item of $Y$.
% Thus, we must strip off the first item of $Y$ and add it (or a non-breaking
% space of equivalent width) to the end of $X$ to typeset the paragraph.
% Then we use |\lastbox| to pull off the final line and check its width.
%    \begin{macrocode}
\newcommand\SB@balancerows{%
  \edef\SB@tempii{\the\SB@toks}%
  \setbox\SB@box\vbox{%
    \SB@toks\expandafter{\expandafter\\\the\SB@toks\\}%
    \SB@lop\SB@toks\SB@toks%
    \settowidth\SB@dimen{\the\SB@toks}%
    \advance\SB@dimen-.5em%
    \leftskip.5cm%
    {\hbadness\@M\hfuzz\maxdimen%
     \hskip-.5cm\relax\SB@temp\unskip\nobreak%
     \hskip\SB@dimen\nobreak%
     \rightskip2em\@plus1fil\par}%
    \setbox\SB@box\lastbox%
    \setbox\SB@box\hbox{%
      \unhbox\SB@box%
      \unskip\unskip\unpenalty%
      \unpenalty\unskip\unpenalty%
    }%
    \expandafter%
  }%
  \expandafter\SB@dimenii\the\wd\SB@box\relax%
%    \end{macrocode}
% Next, compute the smallest width $w_2$ such that the index entry text
% produced by |\SB@multiline| with |\SB@dimen|=$w_2$ has no more lines than
% with |\SB@dimen| set to the maximum available width for the right-hand side.
% This effectively horizontal-balances the right-hand side of the index entry
% text, making all lines of $Y$ roughly equal in width without introducing
% any extra lines.
%    \begin{macrocode}
  \SB@dimen\hsize%
  \advance\SB@dimen-.5cm%
  \setbox\SB@box\vbox{%
    \SB@multiline{\hbadness\@M\hfuzz\maxdimen}%
  }%
  \SB@dimeniii.5\SB@dimen%
  \SB@dimeniv\SB@dimeniii%
  \loop%
    \SB@dimeniv.5\SB@dimeniv%
    \setbox\SB@boxii\vbox{%
      \SB@dimen\SB@dimeniii%
      \SB@multiline{\hbadness\@M\hfuzz\maxdimen}%
    }%
    \ifnum\SB@cnt<\@M%
      \ifdim\ht\SB@boxii>\ht\SB@box%
        \advance\SB@dimeniii\SB@dimeniv%
      \else%
        \SB@dimen\SB@dimeniii%
        \advance\SB@dimeniii-\SB@dimeniv%
      \fi%
    \else%
      \advance\SB@dimeniii\SB@dimeniv%
    \fi%
  \ifdim\SB@dimeniv>2\p@\repeat%
  \setbox\SB@box\box\voidb@x%
  \setbox\SB@boxii\box\voidb@x%
%    \end{macrocode}
% Finally, typeset the results based on the quantities computed above.
%    \begin{macrocode}
  \SB@multiline\relax%
}
%    \end{macrocode}
% \end{macro}
%
% \begin{macro}{\SB@multiline}
% Create a paragraph containing text $X\ldots Y$
% where $X$ is the content of |\SB@temp|, $Y$ is the content of |\SB@tempii|,
% and $Y$ is restricted to width |\SB@dimen| (but may span multiple
% lines of that width).
% Dimen register |\SB@dimenii| must be set with the expected width of the
% final line of $X$.
% The first argument contains any parameter definitions that should be in
% effect when $X$ is processed.
%
% Note that the expansion of |\SB@tempii|, which may contain |\SB@idxitemsep|,
% depends on |\SB@dimen|.
% Therefore, the redefinition of |\SB@dimen| at the start of this macro must
% not be removed!
%    \begin{macrocode}
\newcommand\SB@multiline[1]{%
  \begingroup%
    \SB@dimen-\SB@dimen%
    \advance\SB@dimen\hsize%
    \SB@dimenii-\SB@dimenii%
    \advance\SB@dimenii\SB@dimen%
    {#1\hskip-.5cm\relax\SB@temp\unskip\nobreak%
     \SB@maxmin\SB@dimenii<{1.5em}%
     \leftskip.5cm\rightskip2em\@plus1fil%
     \interlinepenalty\@M%
     \leaders\hbox to.5em{\hss.\hss}\hskip\SB@dimenii\@plus1fill%
     \nobreak{\SB@tempii\kern-2em}%
     \par\global\SB@cnt\badness}%
  \endgroup%
}%
%    \end{macrocode}
% \end{macro}
%
% \begin{macro}{\SB@idxitemsep}
% \changes{v1.11}{2005/04/21}{Changed macro name to avoid a name clash}
% If text $Y$ in index entry $X\ldots Y$ has multiple items in a list, those
% items should be separated by |\\| macros instead of by commas.
% The |\\| macro will be assigned the definition of |\SB@idxitemsep| during
% index generation, which produces the comma along with the complex spacing
% required if $Y$ ends up being broken into multiple lines.
% In particular, it forces each wrapped line of $Y$ to be right-justified
% with left margin at least |\SB@dimen|.
%    \begin{macrocode}
\newcommand\SB@idxitemsep{%
  ,\kern-2em\penalty-8\hskip2.33em\@minus.11em%
  \hskip-\SB@dimen\@plus-1fill%
  \vadjust{}\nobreak%
  \hskip\SB@dimen\@plus1fill\relax%
}
%    \end{macrocode}
% \end{macro}
%
% The following set of macros and environments are intended for use in the
% |.sbx| files that are automatically generated by an index-generating
% program; they shouldn't normally appear in the user's |.tex| or |.sbd|
% files directly.
% However, they are named as exported macros (no |@| symbols) since they are
% used outside the package code and are therefore not stricly internal.
%
% \begin{environment}{idxblock}
% Some indexes are divided into blocks (e.g., one for each letter of the
% alphabet or one for each book of the bible).
% Each such block should be enclosed between |\begin{idxblock}{X}| and
% |\end{idxblock}| lines, where X is the title of the block. The actual
% definition of the |idxblock| environment is set within the initialization
% code for each type of index (below).
%    \begin{macrocode}
\newenvironment{idxblock}[1]{}{}
%    \end{macrocode}
% \end{environment}
%
% \begin{macro}{\idxentry}
% \begin{macro}{\idxaltentry}
% Within each |idxblock| environment there should be a series of |\idxentry|
% and/or |\idxaltentry| macros, one for each line of the index. Again, the
% exact definitions of these macros will vary between index types.
%    \begin{macrocode}
\newcommand\idxentry[2]{}
\newcommand\idxaltentry[2]{}
%    \end{macrocode}
% \end{macro}
% \end{macro}
%
% \begin{environment}{SB@lgidx}
% \begin{environment}{SB@smidx}
% Some indexes actually have two definitions for each |idxblock|
% environment---one for use when there are few enough entries to permit a
% small style index, and another for use in a large style index. These macros
% will be redefined appropriately within the initialization code for each
% type of index.
%    \begin{macrocode}
\newenvironment{SB@lgidx}[1]{}{}
\newenvironment{SB@smidx}[1]{}{}
%    \end{macrocode}
% \end{environment}
% \end{environment}
%
% \begin{macro}{\SB@idxsetup}
% Set various parameters for a column of an index environment.
%    \begin{macrocode}
\newcommand\SB@idxsetup{%
  \hsize\SB@colwidth%
  \parskip\z@skip\parfillskip\z@skip\parindent\z@%
  \baselineskip\f@size\p@\@plus\p@\@minus\p@%
  \lineskiplimit\z@\lineskip\p@\@plus\p@\@minus\p@%
  \hyphenpenalty\@M\exhyphenpenalty\@M%
}
%    \end{macrocode}
% \end{macro}
%
% \begin{macro}{\SB@makeidxcolumn}
% Break off enough material from |\SB@box| to create one column of the
% index.
%    \begin{macrocode}
\newcommand\SB@makeidxcolumn{%
  \ifdim\ht\SB@box=\z@%
    \hskip\hsize\relax%
  \else%
    \splittopskip\z@skip\splitmaxdepth\maxdepth%
    \vsplit\SB@box to\SB@dimen%
    \global\setbox\SB@box\vbox{%
      \SB@idxsetup%
      \splitbotmark%
      \unvbox\SB@box%
    }%
  \fi%
}
%    \end{macrocode}
% \end{macro}
%
% \begin{macro}{\SB@oneidxpage}
% Construct one full page of the index.
% The definition of |\SB@oneidxpage| is generated dynamically based on the
% type of index and number of columns.
%    \begin{macrocode}
\newcommand\SB@oneidxpage{}
%    \end{macrocode}
% \end{macro}
%
% \begin{macro}{\SB@displayindex}
% \changes{v1.11}{2005/04/21}{Item separator macro now localized in scope to the index file}
% \changes{v2.0}{2007/06/18}{Removed hyperref dependency}
% \changes{v2.6}{2008/02/16}{Balance columns on final page}
% \changes{v2.8}{2009/03/06}{Changed argument order}
% Create an index with title \argp{2} and with \argp{1} columns (must be a
% literal constant). Input the index contents from external file \argp{3},
% which is expected to be a \TeX{} file.
%    \begin{macrocode}
\newcommand\SB@displayindex[3]{%
  \ifsongindexes\begingroup%
    \SB@colwidth\hsize%
    \advance\SB@colwidth-#1\columnsep%
    \advance\SB@colwidth\columnsep%
    \divide\SB@colwidth#1%
    \setbox\SB@envbox\vbox{%
      \let\SB@temp\songsection%
      \ifx\chapter\undefined\else%
        \ifx\chapter\relax\else%
          \let\SB@temp\songchapter%
        \fi%
      \fi%
      \SB@temp{#2}%
    }%
%    \end{macrocode}
% The |.sbx| index file might not exist (e.g., if this is the first pass
% through the \TeX{} compiler).
% If it exists, first try typesetting its content as a small index
% (one column, centered, with no divisions).
%    \begin{macrocode}
    \IfFileExists{\csname SB@idxfilename@#3\endcsname.sbx}{%
      \ifsepindexes%
        \global\setbox\SB@box\vbox{%
          \null%
          \vfil%
          \unvcopy\SB@envbox%
          \vskip.5in\@minus.3in\relax%
          \hbox to\hsize{%
            \hfil%
            \vbox{%
              \SB@idxsetup%
              \renewenvironment{idxblock}[1]%
                {\begin{SB@smidx}{####1}}{\end{SB@smidx}}%
              \let\\\SB@idxitemsep%
              \input{\csname SB@idxfilename@#3\endcsname.sbx}%
            }%
            \hfil%
          }%
          \vskip\z@\@plus2fil\relax%
        }%
%    \end{macrocode}
% Test whether the resulting small index fits within one page.
% If not, re-typeset it as a large index.
%    \begin{macrocode}
        {\vbadness\@M\vfuzz\maxdimen%
         \splitmaxdepth\maxdepth\splittopskip\z@skip%
         \global\setbox\SB@boxii\vsplit\SB@box to\textheight}%
        \ifvoid\SB@box%
          \box\SB@boxii%
        \else%
          \SB@lgindex{#1}{#3}%
        \fi%
      \else%
        \SB@lgindex{#1}{#3}%
      \fi%
    }%
%    \end{macrocode}
% If the |.sbx| file doesn't exist, then instead typeset a page with a
% message on it indicating that the document must be compiled a second
% time in order to generate the index.
%    \begin{macrocode}
    {%
      \ifsepindexes%
        \vbox to\textheight{%
          \vfil%
          \unvbox\SB@envbox%
          \vskip1em\relax%
          \hbox to\hsize{\hfil[Index not yet generated.]\hfil}%
          \vskip\z@\@plus2fil\relax%
        }%
      \else%
        \unvbox\SB@envbox%
        \hbox to\hsize{\hfil[Index not yet generated.]\hfil}%
      \fi%
    }%
    \ifsepindexes\clearpage\fi%
  \endgroup\fi%
}
%    \end{macrocode}
% \end{macro}
%
% \begin{macro}{\SB@lgindex}
% Typeset a large-style index.
% We begin by typesetting the entire index into a box.
%    \begin{macrocode}
\newcommand\SB@lgindex[2]{%
  \global\setbox\SB@box\vbox{%
    \renewenvironment{idxblock}[1]%
      {\begin{SB@lgidx}{##1}}{\end{SB@lgidx}}%
    \let\\\SB@idxitemsep%
    \SB@idxsetup%
    \input{\csname SB@idxfilename@#2\endcsname.sbx}%
    \unskip%
  }%
%    \end{macrocode}
% Next, we split the box into columns and pages until the last page is reached.
%    \begin{macrocode}
  \SB@toks{\SB@makeidxcolumn}%
  \SB@cnt#1\relax%
  \loop\ifnum\SB@cnt>\@ne%
    \SB@toks\expandafter{\the\SB@toks%
      \kern\columnsep\SB@makeidxcolumn}%
    \advance\SB@cnt\m@ne%
  \repeat%
  \edef\SB@oneidxpage{\the\SB@toks}%
  \unvbox\SB@envbox%
  \vskip.2in\relax%
  \nointerlineskip%
  \null%
  \nointerlineskip%
  \SB@cnt\vbadness\vbadness\@M%
  \SB@dimenii\vfuzz\vfuzz\maxdimen%
  \loop%
    \SB@dimen\textheight%
    \ifinner\else\kern\z@\advance\SB@dimen-\pagetotal\fi%
    \global\setbox\SB@boxii\copy\SB@box%
    \global\setbox\SB@boxiii\hbox{\SB@oneidxpage}%
    \ifdim\ht\SB@box>\z@%
      \box\SB@boxiii%
      \vfil\break%
  \repeat%
%    \end{macrocode}
% The final page of the index should have all equal-height columns instead
% of a few full columns followed by some short or empty columns at the end.
% To achieve this, we re-typeset the final page, trying different column
% heights until we find one that causes the material to span an equal
% percentage of all the columns on the page.
%    \begin{macrocode}
  \SB@dimenii\ht\SB@boxii%
  \divide\SB@dimenii#1\relax%
  \SB@maxmin\SB@dimen>\SB@dimenii%
  \loop%
    \global\setbox\SB@box\copy\SB@boxii%
    \global\setbox\SB@boxiii\hbox{\SB@oneidxpage}%
    \ifdim\ht\SB@box>\z@%
      \advance\SB@dimen\p@%
  \repeat%
  \box\SB@boxiii%
  \global\setbox\SB@boxii\box\voidb@x%
  \vbadness\SB@cnt\vfuzz\SB@dimenii%
}
%    \end{macrocode}
% \end{macro}
%
% \begin{macro}{\showindex}\MainImpl{showindex}
% \changes{v2.8}{2009/03/06}{Added optional argument}
% Create an index with title \argp{2} based on the data associated with index
% identifier \argp{3} (which was passed to |\newindex|).
% Optional argument \argp{1} specifies the number of columns.
% This macro calls the appropriate index-creation macro depending on the type
% of index that \argp{3} was declared to be.
%    \begin{macrocode}
\newcommand\showindex[3][0]{%
  \@ifundefined{SB@idxsel@#3}{\SB@errnoidx{#3}}{%
    \expandafter\let\expandafter\SB@temp\csname SB@idxsel@#3\endcsname%
    \SB@cnt#1\relax%
    \ifnum\SB@cnt<\@ne\SB@cnt\SB@temp232\relax\fi%
    \expandafter\SB@temp%
    \expandafter\SB@maketitleindex%
    \expandafter\SB@makescripindex%
    \expandafter\SB@makeauthorindex%
    \expandafter{\the\SB@cnt}%
    {#2}{#3}%
  }%
}
%    \end{macrocode}
% \end{macro}
%
% \begin{macro}{\SB@maketitleindex}
% \changes{v2.8}{2009/03/06}{Added columns argument}
% Create a song title index.
% \argp{1} is a column count,
% \argp{2} is the title, and
% \argp{3} is the index identifier (which was passed to |\newindex|).
%    \begin{macrocode}
\newcommand\SB@maketitleindex{%
  \ifnum\idxheadwidth>\z@%
    \renewenvironment{SB@lgidx}[1]{
      \hbox{\SB@colorbox\idxbgcolor{\vbox{%
        \hbox to\idxheadwidth{{\idxheadfont\relax##1}\hfil}%
      }}}%
      \nobreak\vskip3\p@\@plus2\p@\@minus2\p@\nointerlineskip%
    }{\penalty-50\vskip5\p@\@plus5\p@\@minus4\p@}%
  \else%
    \renewenvironment{SB@lgidx}[1]{}{}%
  \fi%
  \renewenvironment{SB@smidx}[1]{}{}%
  \renewcommand\idxentry[2]{%
    \SB@ellipspread{\idxtitlefont\relax\ignorespaces##1\unskip}%
                   {{\idxrefsfont\relax##2}}%
  }%
  \renewcommand\idxaltentry[2]{%
    \SB@ellipspread{\idxlyricfont\relax\ignorespaces##1\unskip}%
                   {{\idxrefsfont\relax##2}}%
  }%
  \SB@displayindex%
}
%    \end{macrocode}
% \end{macro}
%
% \begin{macro}{\SB@idxcolhead}
% In a scripture index, this macro remembers the current book of the bible
% we're in so that new columns can be headed with ``Bookname (continued)''.
%    \begin{macrocode}
\newcommand\SB@idxcolhead{}
%    \end{macrocode}
% \end{macro}
%
% \begin{macro}{\SB@idxheadsep}
% Add vertical space following the header line that begins (or continues) a
% section of a scripture index.
%    \begin{macrocode}
\newcommand\SB@idxheadsep{{%
  \SB@dimen4\p@%
  \advance\SB@dimen-\prevdepth%
  \SB@maxmin\SB@dimen<\z@%
  \SB@dimenii\SB@dimen%
  \SB@maxmin\SB@dimenii>\p@%
  \vskip\SB@dimen\@plus\p@\@minus\SB@dimenii%
}}
%    \end{macrocode}
% \end{macro}
%
% \begin{macro}{\SB@idxcont}
% Typeset the ``Bookname (continued)'' line that continues a scripture
% index section when it spans a column break.
%    \begin{macrocode}
\newcommand\SB@idxcont[1]{%
  \hbox to\hsize{{\idxcont{#1}}\hfil}%
  \nobreak%
  \SB@idxheadsep\nointerlineskip%
}
%    \end{macrocode}
% \end{macro}
%
% \begin{macro}{\SB@makescripindex}
% \changes{v2.4}{2007/10/08}{Scripture index spacing made more uniform}
% \changes{v2.8}{2009/03/06}{Added columns argument}
% Create a scripture index.
% \argp{1} is a column count,
% \argp{1} is the title, and
% \argp{2} is the index identifier (which was passed to |\newscripindex|).
%    \begin{macrocode}
\newcommand\SB@makescripindex{%
  \renewenvironment{SB@lgidx}[1]{%
    \gdef\SB@idxcolhead{##1}%
    \hbox to\hsize{{\idxbook{##1}}\hfil}%
    \nobreak%
    \SB@idxheadsep\nointerlineskip%
  }{%
    \mark{\noexpand\relax}%
    \penalty-20\vskip3\p@\@plus3\p@\relax%
  }%
  \renewenvironment{SB@smidx}[1]
    {\begin{SB@lgidx}{##1}}{\end{SB@lgidx}}%
  \renewcommand\idxentry[2]{%
    \SB@ellipspread{\hskip.25cm\idxscripfont\relax##1}%
                   {{\idxrefsfont\relax##2}}%
    \SB@toks\expandafter{\SB@idxcolhead}%
    \mark{\noexpand\SB@idxcont{\the\SB@toks}}%
  }%
  \renewcommand\idxaltentry[2]{\SB@erridx{a scripture}}%
  \SB@displayindex%
}
%    \end{macrocode}
% \end{macro}
%
% \begin{macro}{\SB@makeauthorindex}
% \changes{v2.8}{2009/03/06}{Added columns argument}
% Create an author index.
% \argp{1} is a column count,
% \argp{2} is the title, and
% \argp{2} is the index identifier (which was passed to |\newauthindex|).
%    \begin{macrocode}
\newcommand\SB@makeauthorindex{%
  \renewenvironment{SB@lgidx}[1]{}{}%
  \renewenvironment{SB@smidx}[1]{}{}%
  \renewcommand\idxentry[2]{%
    \SB@ellipspread{{\idxauthfont\relax\sfcode`.\@m##1}}%
                   {{\idxrefsfont##2}}%
  }%
  \renewcommand\idxaltentry[2]{\SB@erridx{an author}}%
  \SB@displayindex%
}
%    \end{macrocode}
% \end{macro}
%
% \subsection{Error Messages}
%
% We break error messages out into separate macros here in order to reduce the
% length (in tokens) of the more frequently used macros that do actual work.
% This can result in a small speed improvement on slower machines.
%
% \begin{macro}{\SB@Error}
% \begin{macro}{\SB@Warn}
% All errors and warnings will be reported as coming from package ``songs''.
%    \begin{macrocode}
\newcommand\SB@Error{\PackageError{songs}}
\newcommand\SB@Warn{\PackageWarning{songs}}
%    \end{macrocode}
% \end{macro}
% \end{macro}
%
% \begin{macro}{\SB@errspos}
%    \begin{macrocode}
\newcommand\SB@errspos{%
  \SB@Error{Illegal \protect\songpos\space argument}{The argume%
  nt to \protect\songpos\space must be a number from 0 to 3.}%
}
%    \end{macrocode}
% \end{macro}
%
% \begin{macro}{\SB@errnse}
%    \begin{macrocode}
\newcommand\SB@errnse{%
  \SB@Error{Nested songs environments are not supported}{End th%
  e previous songs environment before beginning the next one.}%
}
%    \end{macrocode}
% \end{macro}
%
% \begin{macro}{\SB@errpl}
%    \begin{macrocode}
\newcommand\SB@errpl{%
  \SB@Error{\protect\includeonlysongs\space not permitted with%
  in a songs environment}{\protect\includeonlysongs\space can o%
  nly be used in the document preamble or between songs environ%
  ments in the document body.}%
}
%    \end{macrocode}
% \end{macro}
%
% \begin{macro}{\SB@errrtopt}
%    \begin{macrocode}
\newcommand\SB@errrtopt{%
  \SB@Error{Cannot display chords in a rawtext dump}{You have u%
  sed the rawtext option in the \protect\usepackage\space lin%
  e and have either used the chorded option as well or have use%
  d the \protect\chordson\space macro subsequently.}%
}
%    \end{macrocode}
% \end{macro}
%
% \begin{macro}{\SB@warnrc}
%    \begin{macrocode}
\newcommand\SB@warnrc{%
  \SB@Warn{The \protect\repchoruses\space feature will not wor%
  k when the number of columns is set to zero}%
}
%    \end{macrocode}
% \end{macro}
%
% \begin{macro}{\SB@warnnoidx}
%    \begin{macrocode}
\newcommand\SB@warnnoidx{%
  \SB@Warn{Index command has no effect since no indexes are ye%
  t declared}%
}
%    \end{macrocode}
% \end{macro}
%
% \begin{macro}{\SB@errboo}
%    \begin{macrocode}
\newcommand\SB@errboo{%
  \SB@Error{Encountered \protect\beginsong\space without seein%
  g an \protect\endsong\space for the previous song}%
  {Song \thesongnum\space might be missing a%
  n \protect\endsong\space line.}%
}
%    \end{macrocode}
% \end{macro}
%
% \begin{macro}{\SB@errbor}
%    \begin{macrocode}
\newcommand\SB@errbor{%
  \SB@Error{Encountered \protect\beginsong\space without seein%
  g an \protect\endscripture\space for the preceding scriptur%
  e quotation}{A scripture quotation appearing after son%
  g \thesongnum\space might be missing a%
  n \protect\endscripture\space line.}%
}
%    \end{macrocode}
% \end{macro}
%
% \begin{macro}{\SB@erreov}
%    \begin{macrocode}
\newcommand\SB@erreov{%
  \SB@Error{Encountered \protect\endsong\space without seein%
  g an \protect\endverse\space for the preceding verse}{Son%
  g \thesongnum\space has a \protect\beginverse\space%
  line with no matching \protect\endverse\space line.}%
}
%    \end{macrocode}
% \end{macro}
%
% \begin{macro}{\SB@erreoc}
%    \begin{macrocode}
\newcommand\SB@erreoc{%
  \SB@Error{Encountered \protect\endsong\space without seein%
  g an \protect\endchorus\space for the preceding chorus}{Son%
  g \thesongnum\space has a \protect\beginchorus\space%
  line with no matching \protect\endchorus\space line.}%
}
%    \end{macrocode}
% \end{macro}
%
% \begin{macro}{\SB@erreor}
%    \begin{macrocode}
\newcommand\SB@erreor{%
  \SB@Error{Encountered \protect\endsong\space without seein%
  g an \protect\endscripture for the preceding scripture quot%
  e}{A scripture quote appearing before song \thesongnum\space%
  ended with \protect\endsong\space instead of wit%
  h \protect\endscripture.}%
}
%    \end{macrocode}
% \end{macro}
%
% \begin{macro}{\SB@erreot}
%    \begin{macrocode}
\newcommand\SB@erreot{%
  \SB@Error{Encountered \protect\endsong\space with no matchin%
  g \protect\beginsong}{Before song \thesongnum\space there wa%
  s an \protect\endsong\space with no matchin%
  g \protect\beginsong.}%
}
%    \end{macrocode}
% \end{macro}
%
% \begin{macro}{\SB@errbvv}
%    \begin{macrocode}
\newcommand\SB@errbvv{%
  \SB@Error{Encountered \protect\beginverse\space without seein%
  g an \protect\endverse\space for the preceding verse}{Son%
  g \thesongnum\space might have a verse that has n%
  o \protect\endendverse\space line.}%
}
%    \end{macrocode}
% \end{macro}
%
% \begin{macro}{\SB@errbvc}
%    \begin{macrocode}
\newcommand\SB@errbvc{%
  \SB@Error{Encountered \protect\beginverse\space without seein%
  g an \protect\endchorus\space for the preceding chorus}{Son%
  g \thesongnum\space might have a chorus that has n%
  o \protect\endchorus\space line.}%
}
%    \end{macrocode}
% \end{macro}
%
% \begin{macro}{\SB@errbvt}
%    \begin{macrocode}
\newcommand\SB@errbvt{%
  \SB@Error{Encountered \protect\beginverse\space without firs%
  t seeing a \protect\beginsong\space line}{Before son%
  g \thesongnum, there is a \protect\beginverse\space line no%
  t contained in any song.}%
}
%    \end{macrocode}
% \end{macro}
%
% \begin{macro}{\SB@errevc}
%    \begin{macrocode}
\newcommand\SB@errevc{%
  \SB@Error{Encountered \protect\endverse\space while process%
  ing a chorus}{Song \thesongnum\space might hav%
  e a \protect\beginchorus\space concluded by a%
  n \protect\endverse\space instead of an \protect\endchorus.}%
}
%    \end{macrocode}
% \end{macro}
%
% \begin{macro}{\SB@errevo}
%    \begin{macrocode}
\newcommand\SB@errevo{%
  \SB@Error{Encountered \protect\endverse\space without firs%
  t seeing a \protect\beginverse}{Song \thesongnum\space m%
  ight have an \protect\endverse\space with no matchin%
  g \protect\beginverse.}%
}
%    \end{macrocode}
% \end{macro}
%
% \begin{macro}{\SB@errevt}
%    \begin{macrocode}
\newcommand\SB@errevt{%
  \SB@Error{Encountered an \protect\endverse\space outside o%
  f any song}{Before song \thesongnum, there is a%
  n \protect\endverse\space line not preceded b%
  y a \protect\beginsong\space line.}%
}
%    \end{macrocode}
% \end{macro}
%
% \begin{macro}{\SB@erretex}
%    \begin{macrocode}
\newcommand\SB@erretex{%
  \SB@Error{The \protect\repchoruses\space feature requires e-%
  TeX compatibility}{Your version of LaTeX2e does not appear t%
  o be e-TeX compatible. Find a distribution that includes e-T%
  eX support in order to use this feature.}%
}
%    \end{macrocode}
% \end{macro}
%
% \begin{macro}{\SB@errbcv}
%    \begin{macrocode}
\newcommand\SB@errbcv{%
  \SB@Error{Encountered \protect\beginchorus\space without see%
  ing an \protect\endverse\space for the preceding verse}{Son%
  g \thesongnum\space might hav%
  e a \protect\beginverse\space with no match%
  ing \protect\endverse.}%
}
%    \end{macrocode}
% \end{macro}
%
% \begin{macro}{\SB@errbcc}
%    \begin{macrocode}
\newcommand\SB@errbcc{%
  \SB@Error{Encountered \protect\beginchorus\space without see%
  ing an \protect\endchorus\space for the preceding chorus}%
  {Song \thesongnum\space might have a \protect\beginchorus%
  \space with no matching \protect\endchorus.}%
}
%    \end{macrocode}
% \end{macro}
%
% \begin{macro}{\SB@errbct}
%    \begin{macrocode}
\newcommand\SB@errbct{%
  \SB@Error{Encountered \protect\beginchorus\space without see%
  ing a \protect\beginsong\space line first}{After son%
  g \thesongnum\space there is a \protect\beginchorus\space%
  line outside of any song.}%
}
%    \end{macrocode}
% \end{macro}
%
% \begin{macro}{\SB@errecv}
%    \begin{macrocode}
\newcommand\SB@errecv{%
  \SB@Error{Encountered an \protect\endchorus\space while proc%
  essing a verse}{Song \thesongnum\space might hav%
  e a \protect\beginverse\space concluded by \protect\endchorus%
  \space instead of \protect\endverse.}%
}
%    \end{macrocode}
% \end{macro}
%
% \begin{macro}{\SB@erreco}
%    \begin{macrocode}
\newcommand\SB@erreco{%
  \SB@Error{Encountered \protect\endchorus\space without firs%
  t seeing a \protect\beginchorus}{Song \thesongnum\space m%
  ight have an \protect\endchorus\space with no match%
  ing \protect\beginchorus.}%
}
%    \end{macrocode}
% \end{macro}
%
% \begin{macro}{\SB@errect}
%    \begin{macrocode}
\newcommand\SB@errect{%
  \SB@Error{Encountered an \protect\endchorus\space outside o%
  f any song}{Before song \thesongnum, there is a%
  n \protect\endchorus\space line not preceded b%
  y a \protect\beginsong\space line.}%
}
%    \end{macrocode}
% \end{macro}
%
% \begin{macro}{\SB@errbro}
%    \begin{macrocode}
\newcommand\SB@errbro{%
  \SB@Error{Missing \protect\endsong}%
  {Nested song and intersong environments are not supported%
  . Song \thesongnum\space might be missing a%
  n \protect\endsong\space line.}%
}
%    \end{macrocode}
% \end{macro}
%
% \begin{macro}{\SB@errbrr}
%    \begin{macrocode}
\newcommand\SB@errbrr{%
  \SB@Error{Nested intersong environments are not supported}%
  {A scripture quote or other intersong environment before s%
  ong \thesongnum\space is missing its ending line.}%
}
%    \end{macrocode}
% \end{macro}
%
% \begin{macro}{\SB@errero}
%    \begin{macrocode}
\newcommand\SB@errero{%
  \SB@Error{Encountered an \protect\endscripture\space whil%
  e processing a song}{Song \thesongnum\space ends wit%
  h \protect\endscripture\space when it should end wit%
  h \protect\endsong.}%
}
%    \end{macrocode}
% \end{macro}
%
% \begin{macro}{\SB@errert}
%    \begin{macrocode}
\newcommand\SB@errert{%
  \SB@Error{Encountered an \protect\endscripture\space with%
  out first seeing a \protect\beginscripture}{Before son%
  g \thesongnum, there is an \protect\endscripture\space w%
  ith no matching \protect\beginscripture.}%
}
%    \end{macrocode}
% \end{macro}
%
% \begin{macro}{\SB@errscrip}
%    \begin{macrocode}
\newcommand\SB@errscrip[1]{%
  \SB@Error{Encountered a \protect#1\space outside a scriptu%
  re quote}{\protect#1\space can only appear betwee%
  n \protect\beginscripture\space an%
  d \protect\endscripture\space lines.}%
}
%    \end{macrocode}
% \end{macro}
%
% \begin{macro}{\SB@errchord}
%    \begin{macrocode}
\newcommand\SB@errchord{%
  \SB@Error{Song \thesongnum\space seems to have chord%
  s that appear outside of any verse or chorus}{All chords a%
  nd lyrics should appear between \protect\beginverse\space%
  and \protect\endverse, or between \protect\beginchorus\space%
  and \protect\endchorus.}%
}
%    \end{macrocode}
% \end{macro}
%
% \begin{macro}{\SB@errreplay}
%    \begin{macrocode}
\newcommand\SB@errreplay{%
  \SB@Error{Replayed chord has no matching chord}{Son%
  g \thesongnum\space uses \protect^ more times than the%
  re are chords in the previously memorized verse.}%
}
%    \end{macrocode}
% \end{macro}
%
% \begin{macro}{\SB@errreg}
%    \begin{macrocode}
\newcommand\SB@errreg[1]{%
  \SB@Error{Unknown chord-replay register name: #1}{Chord-re%
  play registers must be declared with \protect\newchords.}%
}
%    \end{macrocode}
% \end{macro}
%
% \begin{macro}{\SB@errdup}
%    \begin{macrocode}
\newcommand\SB@errdup[1]{%
  \SB@Error{Duplicate definition of chord-replay register%
  : #1}{\protect\newchords\space was used to declare the sa%
  me chord-replay register twice.}%
}
%    \end{macrocode}
% \end{macro}
%
% \begin{macro}{\SB@errmbar}
%    \begin{macrocode}
\newcommand\SB@errmbar{%
  \SB@Error{Song \thesongnum\space seems to have measur%
  e bars that appear outside of any verse or chorus}{All mea%
  sure bars (produced with \protect\mbar\space or |) must ap%
  pear between \protect\beginverse\space an%
  d \protect\endverse, or between \protect\beginchorus\space%
  and \protect\endchorus.}%
}
%    \end{macrocode}
% \end{macro}
%
% \begin{macro}{\SB@errebar}
%    \begin{macrocode}
\newcommand\SB@errebar[2]{%
  \SB@Error{Ignoring unbalanced \expandafter\@gobble\string#2 i%
  n \protect\gtab}{Found no \expandafter\@gobble\string#1 to ma%
  tch the \expandafter\@gobble\string#2.}%
}
%    \end{macrocode}
% \end{macro}
%
% \begin{macro}{\SB@errnoidx}
%    \begin{macrocode}
\newcommand\SB@errnoidx[1]{%
  \SB@Error{Unknown index identifier: #1}{This index identifie%
  r was never declared using \protect\newindex.}%
}
%    \end{macrocode}
% \end{macro}
%
% \begin{macro}{\SB@erridx}
%    \begin{macrocode}
\newcommand\SB@erridx[1]{%
  \SB@Error{\protect\idxaltentry\space not allowed in #1 index}%
  {This error should not occur. The index generation routines ha%
  ve malfunctioned. Try deleting all temporary files and then re%
  compiling.}%
}
%    \end{macrocode}
% \end{macro}
%
% \subsection{Option Processing}\label{sec:optproc}
%
% \begin{macro}{\ifchorded}
% \begin{macro}{\iflyric}
% \begin{macro}{\ifslides}
% \begin{macro}{\ifmeasures}
% \begin{macro}{\ifpartiallist}
% \begin{macro}{\ifrepchorus}
% \begin{macro}{\iftranscapos}
% \begin{macro}{\ifnolyrics}
% \begin{macro}{\ifrawtext}
% \begin{macro}{\ifsongindexes}
% \begin{macro}{\ifsepindexes}
% \begin{macro}{\ifpagepreludes}
% \begin{macro}{\ifSB@colorboxes}
% \begin{macro}{\ifSB@omitscrip}
% Reserve conditionals for all of the various option settings.
% We wait to define these since if any are used earlier than this, it is
% an error in the package code, and we'd rather get an error than continue.
%    \begin{macrocode}
\newif\ifchorded
\newif\iflyric\lyrictrue
\newif\ifslides
\newif\ifmeasures
\newif\ifpartiallist
\newif\ifrepchorus
\newif\iftranscapos
\newif\ifnolyrics
\newif\ifrawtext
\newif\ifsongindexes\songindexestrue
\newif\ifsepindexes\sepindexestrue
\newif\ifpagepreludes
\newif\ifSB@colorboxes
\IfFileExists{color.sty}\SB@colorboxestrue\SB@colorboxesfalse
\newif\ifSB@omitscrip
%    \end{macrocode}
% \end{macro}
% \end{macro}
% \end{macro}
% \end{macro}
% \end{macro}
% \end{macro}
% \end{macro}
% \end{macro}
% \end{macro}
% \end{macro}
% \end{macro}
% \end{macro}
% \end{macro}
% \end{macro}
%
% \begin{macro}{\nolyrics}
% \begin{macro}{\pagepreludes}
% The |\nolyrics| and |\pagepreludes| macros are just shorthand for
% |\nolyricstrue| and |\pagepreludestrue|, respectively.
%    \begin{macrocode}
\newcommand\nolyrics{}
\let\nolyrics\nolyricstrue
\newcommand\pagepreludes{\pagepreludestrue\songpos0}
%    \end{macrocode}
% \end{macro}
% \end{macro}
%
% Finally we're ready to process all of the package options.
% This is delayed until near the end because the option processing code
% needs to execute various macros found in the previous sections.
%    \begin{macrocode}
\SB@chordson
\ProcessOptions\relax
%    \end{macrocode}
%
% \begin{macro}{\SB@colorbox}
% Include the colors package and define colors, if requested.
%    \begin{macrocode}
\ifSB@colorboxes
  \RequirePackage{color}
  \definecolor{SongbookShade}{gray}{.80}
  \newcommand\SB@colorbox[2]{%
    \ifx\@empty#1%
      \vbox{%
        \kern3\p@%
        \hbox{\kern3\p@{#2}\kern3\p@}%
        \kern3\p@%
      }%
    \else%
      \colorbox{#1}{#2}%
    \fi%
  }
\else
  \newcommand\SB@colorbox[2]{\vbox{%
    \kern3\p@%
    \hbox{\kern3\p@{#2}\kern3\p@}%
    \kern3\p@%
  }}
\fi
%    \end{macrocode}
% \end{macro}
%
% \subsection{Rawtext Mode}
%
% If generating raw text, most of what has been defined previously is ignored
% in favor of some very specialized macros that write all the song lyrics to
% a text file.
%    \begin{macrocode}
\ifrawtext
  \SB@newwrite\SB@txtout
  \immediate\openout\SB@txtout=\jobname.txt
  \newif\ifSB@doEOL
  {\catcode`\^^M12 %
   \catcode`\^^J12 %
   \gdef\SB@printEOL{\ifSB@doEOL^^M^^J\fi}}
  {\catcode`#12\gdef\SB@hash{#}}
  {\catcode`&12\gdef\SB@amp{&}}
  \renewcommand\SB@@@beginsong{%
    \begingroup%
      \def\'{}\def\`{}\def\v{}\def\u{}\def\={}\def\^{}%
      \def\.{}\def\H{}\def\~{}\def\"{}\def\t{}%
      \def\copyright{(c)}%
      \let~\space%
      \let\par\SB@printEOL%
      \let\#\SB@hash%
      \let\&\SB@amp%
      \catcode`|9 %
      \catcode`*9 %
      \catcode`^9 %
      \def\[##1]{}%
      \resettitles%
      \immediate\write\SB@txtout{\thesongnum. \songtitle}%
      \nexttitle%
      \foreachtitle{\immediate\write\SB@txtout{(\songtitle)}}%
      \ifx\songauthors\@empty\else%
         \immediate\write\SB@txtout{\songauthors}%
      \fi%
      \ifx\SB@rawrefs\@empty\else%
         \immediate\write\SB@txtout{\SB@rawrefs}%
      \fi%
      \immediate\write\SB@txtout{}%
      \SB@doEOLfalse%
      \obeylines%
  }
  \renewcommand\SB@endsong{%
      \SB@doEOLtrue%
      \immediate\write\SB@txtout{\songcopyright\space%
        \songlicense\SB@printEOL}%
    \endgroup%
    \SB@insongfalse%
    \stepcounter{songnum}%
  }
  \def\SB@parsesrefs#1{\def\songrefs{#1}}
  \long\def\beginverse#1#2\endverse{%
    \SB@doEOLtrue\begingroup%
      \def\textnote##1{##1}%
      \def\SB@temp{#1}%
      \def\SB@star{*}%
      \ifx\SB@temp\SB@star%
        \immediate\write\SB@txtout{\@gobble#2}%
      \else%
        \immediate\write\SB@txtout{#2}%
      \fi%
    \endgroup\SB@doEOLfalse}
  \long\def\beginchorus#1\endchorus{%
    \SB@doEOLtrue\begingroup%
      \def\textnote##1{##1}%
      \immediate\write\SB@txtout{Chorus:#1}%
    \endgroup\SB@doEOLfalse}
  \long\def\beginscripture#1\endscripture{}
  \def\musicnote#1{}
  \def\textnote#1{%
    \SB@doEOLtrue%
    \immediate\write\SB@txtout{#1\SB@printEOL}%
    \SB@doEOLfalse}
  \def\brk{}
  \def\rep#1{(x#1)}
  \def\echo#1{(#1)}
  \def\mbar#1#2{}
  \def\lrep{}
  \def\rrep{}
  \def\nolyrics{}
  \renewcommand\memorize[1][]{}
  \renewcommand\replay[1][]{}
\fi
%    \end{macrocode}
%
% \Finale
\endinput

